\documentclass[12pt,letterpaper]{article}
\usepackage[utf8]{inputenc}
\usepackage{amsmath}
\usepackage{amsfonts}
\usepackage{amssymb}
\begin{document}
\marginpar{p.27}not stand against a breath of ridicule, they 
could not stand against a little temptation, and 
were soon "wallowing in the mire. One act of 
sacrifice is not the same thing as a life gathered 
into one consummate sacrifice, whose value is 
that it has the whole personality put into it 
for ever. 

Third, this man could not take the full 
measure of all that he was doing, and Christ 
could. Christ did not go to His death with 
His eyes shut. He died because He willed to 
die, having counted the cost with the greatest, 
deepest moral vision in the world. 

Fourthly, the hero in the story had nothing 
to do with the moral condition of those whom 
he saved. The scoundrel and the saint in that 
train were both alike to him. 

Again, he had no quarrel with those whom he 
saved. He had nothing to complain of. He had 
nothing from them to try his heroism. They 
were not his bitter enemies. His valour was 
not the heroism of forgiveness, where lies the 
wondrous majesty of God. His act was not 
an act of grace, which is the grand glory of 
the love of Christ. Christ died for people who 
not only did not know Him, but who hated and 
despised Him. He died, not for a trainful of 
people, but for the whole organic world of 
\marginpar{p.28}
people. It was an infinite death, that of His, 
in its range and in its power. It was death 
for enemies more bitter than anything that 
man can feel against man, for such haters as 
only holiness can produce. Here is the singular 
thing: the greater the favour that is done to 
us, the more fiercely we resent it if it does not 
break us down and make us grateful. The 
greater the favour, if we do not respond in its 
own spirit, so much the more resentful and 
antagonistic it makes us. I have already said 
that we speak too often as though the effect 
of Christ's death upon human nature must be 
gratitude as soon as it is understood. It is 
not always gratitude. Unless it is received in 
the Holy Ghost, the effect may just be the 
other way. It is judgment. It is a death unto 
death. 

I conclude by saying what I have often said, 
and what often needs saying, that it is not 
possible to hear the gospel and to go away just 
as you came. I wish that were more realised. 
We should not have so many sermon-hunters. 
If people felt that every time they heard the 
gospel they were either better or worse for it, 
they would be more careful about hearing. 
They would not go so often, possibly; better they 
\marginpar{p.29}
should not, perhaps. I am not speaking about 
hearing of sermons. That is neither here nor 
there. A man may hear sermons and be neither 
the better nor the worse. But a man cannot 
hear the gospel without being either better or 
worse, whether he knows it or not. When you 
come to face the last issues, it is either unto 
salvation or unto condemnation. The great 
central, decisive thing, the last judgment of the 
world, is the Cross of Christ. The reason why 
so many sermons are found uninteresting is not 
always due to the dullness of the preacher. God 
knows how often that is the case, but it is not 
always. It is because the sermons so often turn, 
or ought to turn, upon the miracle of the grace 
of God, which is so great a miracle that it is 
strange, remote, and alien to our natural ways 
of thinking and feeling. It seems foreign to us. 
It is like reading a guide-book if you have never 
been in the country. I take down my Baedeker 
in the winter and read it with the greatest 
delight, because I know the country. If I had 
not been there I should find it the dreariest read- 
ing. Why do not people read the Bible more? 
Because they have not been in that country. 
There is no experience for it to stir and develop. 
The Cross of Christ, the infinite wonder of it 
wo have got to learn that. We have got to 
\marginpar{p.30}
learn the deep meaning of that by having been 
there, by the evangelical experience whose lack 
is the cause of all the religious vagrancy of the 
hour. We have got to learn that it was not 
simply magnificent heroism, but that it was God 
in Christ reconciling the world. It was God 
that did that work in Christ. And Christ was 
the living God working upon man, and working 
out the Kingdom of God. 

\end{document}