\documentclass[12pt,a5paper,twoside]{book}
\usepackage{geometry}
\usepackage{palatino}
\begin{document}

% Source PDF on archive.org - https://archive.org/details/romereformreacti00fors


ROME, REFORM 
AND REACTION 



Four Lectures on/the Religious Situ-TK
ation; by P. T. FORSYTH, M.A., D.D. 
Cambridge tJMK OMK cMKS 



LONDON: HODDER AND 
STOUGHTON MK MK 27 
PATERNOSTER ROW 899 



Butler and Tanner, The Schvood Printing Works, Frome, and London 



To 
Rev. Dr. AMORY H. BRADFORD 

MoNTCLAiR, New Jersey 
My dear Bradford,---

It has been my happiness to know quite a 
group of friends with a special genius---"the genius 
of being beloved." Some are on your side, some on 
ours; some are now jcnseitsMK some are diesseits still; 
one at least I have at home; and one, so near and 
yet so far, in you. Other men well praise your gifts, 
or your lineage; have I leave to adorn with you the 
front of this little book? 

It appears just after a first and memorable visit to 
your most hospitable shores, in circumstances which 
move many besides me to say, " Now in Christ 
Jesus we who sometimes were far off are made nigh 
by the blood of Christ." So for these discourses, as 
for all the Churches, be this the common and hos-TK
pitable text, however understood in detail. 

Always yours, 

P. T. FORSYTH 

Cambridge, NovemberMK 1899 



\chapter*{Preface} 
\markboth{Preface}{Preface}

\textsc{This} book is not a treatise, but the publication 
of a Series of Lectures; which may account 
for the lack of references, some repetition, 
and some conversational or dogmatic 
symptoms of direct address. At the same time 
additions have been made, or parts retained 
which were omitted in delivery as being in 
a style adapted for the reader rather than for 
the hearer. 

I ought also to say that, as I do not aim 
at any contribution of scientific value, I have 
abstained from reading the recent works of 
Dr. Fairbairn and Dr. Brown till I am now 
set free for that pleasure. That they are 
inimitable would not prevent their being too 
contagious for the individuality of a comparative 
amateur on the same topic. And any 
\marginpar{viii}
coincidences that occur may thus mean the 
more. 

Page 170 is based on a passage in Eucken; 
and the series on page 45 is a reminiscence 
from Bunsen. 

I am under much obligation to Rev. J. A. 
Hamilton, of Penzance, for amendments in 
proof. 

\chapter*{Contents} 

PAGE 

LECTURE I 

What is the Real Nature of the Pre-TK
sent Issue? . . . . • 13 

LECTURE II 

Where do We Really Go when We Go 

Behind the Reformation? . . -75 

LECTURE III 
What did Luther Really Do? . • 113 

LECTURE IV 
Part I. The Real Nature of the Priest-TK
hood . . . . . . ' MK11 

Part II. Some Real Sources of the 

Priest's Welcome . . . .212 

ix 



\chapter{The Real Nature of the Present Issue}
\markboth{Rome, Reform, and Reaction}{Real Nature of the Present Issue}

\section*{I}

\textsc{I do} not share the repugnance felt by a large 
number of people, especially in the present day, to 
religious controversy. The ruling spirit of the great 
classic ages and figures of faith has been one of controversy 
however rarefied. It is true that it rested 
always on a deep certainty and peace, but it was a security 
that did not allow them to become quietists, but 
thrust them into the front of battle. We have room 
and need for the men of peace, but one should protest 
against a tendency to erect them into the ideal figures 
of Christianity. They deserve love and honour, but in 
critical times it is other men and other helps that we 
chiefly need. In a crisis the man of peace may be 
counted in the main as a friend of the established side. 
The prophets of the Old Testament were men of war. 
The whole mission of Israel was to be fulfilled in the 
face of a gainsaying world. The work of Christ 
\marginpar{14}
was incessant controversy---the Lord's controversy. 
The life of Paul was ceaseless warfare. The Epistles 
of John proceed from the thick of a battle which 
Christian faith was waging for its life, when 
Christian love knew how to hate and fear. The 
great figures of Church history have been those 
whose words and deeds come down to us from the 
midst of campaigns. Athanasius, who saved the 
Church's life, was set against a world in arms. And 
the Reformation age meant war. Luther lived and 
breathed in it. If the Reformation is not yet done 
we must dread war less. It is easy for us to talk 
against controversy when it is the life-toil of the 
great controversialists that has given us the ease we 
propose only to enjoy. We need to revive some or 
the heroic features of faith. We are in danger from 
its feminine and sympathetic side, from its restful and 
acquiescent mood. We are apt to treat religion as 
the region of ease as well as the secret of peace. We 
deprecate the opening of its questions, or, when they 
are opened, the pursuit of them. We do not reflect 
that no frame of ours could be better than that for the 
enemies of faith. They are quite willing that we 
should cultivate a quietist's peace, minimise differences, 
and dwell on the common stock of belief, so long as 
we leave them with the monopolies and the abuses 
they represent. There are many people who think 
\marginpar{15}
that the kingdom of heaven means first a quiet life 
and the cultivation of friendly feeling all round. 
They do not naturally like conflict, and religion is 
not strong enough in them to compel them beyond 
their natural likes. They do not reflect that conflict 
comes to the great warriors not as a sport or hobby, 
but as a painful duty and a stern obedience. Let 
them read Jeremiah, the gentle, peace-loving man 
whom the hand of God thrust into the caldron of his 
seething time. Let them note in many another how 
the trumpet broke upon their selfish peace as the 
breath of God to save them from the stagnation or 
goodness, and stir them with the tonic of the fight. 
I am sure that many a time the revival of religion 
which we pray for ought to come by a renewal of 
the heroic vein of faith, with a new crusade; and the 
baptism of the spirit should be a baptism of blood. 
One reason why controversy is deprecated at 
present is that sympathy has been growing at the 
expense of principle. Our philanthropic energies 
have, for the time, submerged our energies of righteousness. 
I do not say so in a grudging spirit. We 
move forward with one foot at a time. For the 
present it is the turn of the heart side; but the 
time is far spent, and it grows needful that, if we 
are to keep from falling, there should be a step by the 
other foot and a movement of the other side. It is 
\marginpar{16}
time that we returned with our attention to the side 
of mind and principle, that we recognised another test 
than beneficence, and that we sought to clear our 
views for action on some of the great old issues now 
in abeyance. There are whole sections of the public 
whose mawkish religion needs more than anything 
else a gospel of severity, others whose sickly charity 
is an{\ae}mic for want of the breath of justice, and others 
whose {\ae}sthetic decorum can only be roused by some 
action in sufficiently bad taste to break their idol. 

There is another reason for that distaste of controversy 
which takes so much virility from our faith and 
age. 

I do not now mean the dislike of many for the passions 
which controversy lets loose. People who habitually 
cannot control their tempers and their tongues 
should not enter controversy. They are unfit for the 
heroic and the noble side of life. But, (I may interject,)
those also are unfit for it, and are guilty of 
some cowardice, who give way to the bullies, or 
who shrink from the advocacy of the right because 
the enemy uses poisoned arrows. There is a worse 
thing than the temper and abuse of controversy, 
and that is the mawkish sweetness and maudlin piety 
of the people who are everybody's brothers and can 
stand up to none. 

But I leave that and return to the weightier reason 
\marginpar{17}
that I have hinted for the dislike of controversy. It is 
the feeling on the part of many that it is sterile, and 
leaves us at the end no farther than when we began. 

Now this is not the case. I will venture to say 
that none ever came out of a real argument other 
than the better for it, provided that they behaved 
themselves. If they did nothing else, they cleared up 
their own views to themselves. They probably suggested 
new aspects of the case to the bystanders. And 
they may even have done so to their adversary, or he 
to them. And in any case their faculties were 
stirred; their mind was the healthier for the gymnastic; 
and they escaped for a time from the women's 
quarters, and from the office and from the shop, into 
the breeze. They are not where they were at the 
outset. 

And so it is with the great controversies that mark 
and make history, and especially the history of the 
Church. They do not come upon us to-day with 
exactly the same call, the same problem, the same 
historic situation as those of our fathers. The problem 
moves. It does not present itself to us in the 
fixed formula of our predecessors. It is really a new 
problem; it is a new question set in the same rule. 
Those who handled it before renewed it in \textit{their} time. 
They added something to it. They passed it on to 
us as something different, and ready for our contribution
\marginpar{18}
through theirs. The people who enlarge most 
on the sterility of controversy are those who know 
least about it, who have gone no deeper than its 
surface, who have resented the call to think, and be 
just; or who ran away to save their nice manners 
as soon as the bad language began. It is a trait 
of the whole Agnostic habit of mind thus to belittle 
the past, to succumb to helplessness, and acquiesce 
in despair. The same habit which says we can know 
nothing about God says also that we can know nothing 
about any of those tough and fascinating matters 
which men have argued for generations. It is the 
same shallow impatience on both heads, the same 
scepticism of human effort and intelligence. The 
same quality of mind as distrusts God's effort in 
revelation, distrusts man's effort in understanding it. 
Give us the man that cannot take his mind off the 
North Pole. The great problems are not to be settled 
in a generation; they are of historic dimensions. 
They extend over many generations, as some mathematical 
problems may cover days. But each day contributes 
something to the huge chain of calculation; 
and so it is with the great controversies of the past. 
We take them up where our predecessors left them, 
not where they found them. There are questions 
that have lasted or even slept for centuries, and whose 
aspect is materially and for ever changed by the work 
\marginpar{19}
of the last fifty years. But it is the change of evo-TK
lution, not of the kaleidoscope. It moves for all that. 

Take the Protestant question. What did the 
Reformation do? Simply add one to the many efforts 
at reform already made? No. It attacked the same 
question, but in quite a new way, with new light from 
Luther's original experience and genius. But was 
Luther's experience so new? Again, no. It was the 
revival of the same controversy as engrossed the life 
of Paul. It was fighting the same battle over again. 
But is that not rather hopeless? Surely no, still. It 
was on a far wider scale, in a far more searching spirit, 
at least as far as the enemy was concerned. It was a 
war with Paganism, but it was with the more terrible 
Christian Paganism. And, besides, \textit{is} it hopeless to 
find that the great cause which had gone out of clear 
sight for 1,500 years refused to lie dead, and asserted 
itself with such amazing power? Is it hopeless to-day 
to see so much of the work of the Reformation 
still to be done? We should not find it so. The 
corruptions and abuses of fifteen centuries were not 
to be thrown off in one, and it is eighteen centuries 
that we are struggling with to-day. Was it likely 
that Europe could speedily get rid of the moral and 
spiritual malaria which had lain so long in her system 
that death grew her habit of life? Nay, was it not 
probable that there should be relapses, that the new 
\marginpar{20}
life should have a tough fight for it, that it should 
have to be nursed back through a tedious convalescence,
with much to dishearten and much to try us? 
How often the convalescence is longer than the 
disease! Again, how vast the Reformation principle 
is, the evangelical principle! It \textit{is} the Gospel. That 
is why human nature hates and resists it worse than it 
does Rome. For long the New Testament principle 
will not leaven Europe, though it has been 2,000 years 
at work on it; and the Reformation has been working 
only a few centuries. The situation is anything but 
hopeless if we will take pains to understand the 
nature of the principles at work, of the Gospel, and 
of the enemy. The hopeless people are the people 
who will not take pains, who are not in earnest. 

If, indeed, our Protestantism to-day called upon us 
to go back to the Reformers and adopt their beliefs 
and practices in a mass, we might well demur; and 
we might suspect the uses of controversy, or its progress 
in history. But controversy, the battle of truth 
and right, cannot be the one thing which does not 
progress amid all the energies of man. And we are 
not asked to adopt the theology of the Reformers nor 
their polity \textit{en bloc}. What we are asked to do is to 
take their principle and carry it out in a way they 
could not do, to develop the Reformation, to reform 
the Reformers, to take the results their principle has 
\marginpar{21}
achieved, to go back with these results upon their 
positions, to re-read their positions in the light of 
their own results, to apply these principles afresh to 
the ground that they themselves have cleared, and 
so to carry them forward to new conquests and new 
expressions. Protestantism is not resuming the entire 
theology of the Reformers, but correcting their theology, 
when necessary, by their Gospel, by their 
principle of faith. We may correct Luther's dogmas 
by Luther's thoughts, and his thoughts by his faith. 
And so, even the High Church movement of to-day, 
medieval as it is, is not a mere copy; it is not a return 
to medievalism in the sense of lifting over bodily the 
theological contents of the Middle Ages, and pressing 
them upon faith as if no water had flowed under 
the bridge from that day to this. It is recalling the 
medieval principle of the Church, or of faith, and 
reading the world of to-day in that light. The very 
Church of Rome itself, with its claim to be the \textit{living} 
Church, takes stand on a great doctrine of development; 
and it put the crown on the long series by 
the doctrine of papal infallibility, which was not 
formulated till twenty years ago. People speak and 
write of a Reformation Settlement. There was no 
such thing. For this country at least the Reformation 
was much more of an unsettlement. It was a 
beginning, not an end. It was but the thin end of 
\marginpar{23}
the evangelical idea which pierces to the dividing 
asunder or every mere Catholic institution, and must 
overturn till He come whose right it is to reign 
directly in each soul. There was nothing in the 
nature of the Reformation which promised immediate 
finality either to Church or to State. As a matter 
of fact it has brought much more ferment than 
finality, and the more outward ferment in proportion 
as it gave the soul an inward finality. The 
peace of the justified waged but the keener war 
against things unjustifiable. And if the political settlement 
had been a much more explicit thing than it 
was, it would still be at the mercy of the principle 
of spiritual power and freedom which the Reformation 
only introduced. 

It is a great thing to be involved in these noble 
old controversies. There are many worse things 
than war on those lofty planes. The object of faith 
is not to provide us with a quiet life. Little men 
may belittle any conflict, but the conflict is great. 
The issue is high. Let it be handled in a high-minded 
way. Do not let us fight as if our one foe 
were some village cleric, some rural autocrat, and 
petty priest. The conflict is one which has engrossed 
the very greatest human souls and involves the 
greatest divine destinies. It is not English, but 
ecumenical. No State question approaches in 
\marginpar{23}
moment the gravity of the question about the true 
nature of faith, and the consequent true nature of the 
Church. It is the human question. It is a war of 
angels, saints, apostles, prophets; let us wage it as 
men of the saintly and apostolic faith in Jesus Christ. 

\section*{II}

There are many who reel disheartened in the present 
religious situation because it appears to them that we 
are in danger of losing all that the last 300 years 
have gained, and of having to fight the whole 
Reformation battle over again. This is not so. 
Even if the conflict become more severe than it is, 
it is yet not the old straw that is threshed nor the 
long dead that is slain. It is indeed the old problem 
that confronts us; it is not a new one. But it is 
the old problem at quite a new stage. It is the old 
problem at a stage which has developed a new answer, 
or compels the answer in new terms. It is the old 
problem of the unfinished Reformation; but it has 
advanced to a stage at which it becomes clear as it 
never was in its history before that the first answer is 
Disestablishment. That is the social consummation of 
the spiritual necessity in the Reformation. The battle 
with the world for a free Gospel can only be won by 
a free Church; and a free Church is the inevitable 
\marginpar{24}
effect of a free Gospel, of the freedom of the spiritual 
power. At the English Reformation there were but 
the two alternatives---a royal Church or a Roman 
Church, Erastianism or Catholicism. If you resented 
the royal supremacy you could realise the freedom of 
the Church only in a Catholic form, and between 
Henry and More our heart is all with More. But 
history has developed a better way. Before the 
Reformation the freedom of the State had only been 
attained by the subjection of the Church, or the 
freedom of the Church by the subjection of the State. 
But the existence of the Free Churches has shown, 
and their prosperity points, another and a better way. 
The solution of the old problem is a free Church in 
a free State. 

It is the old problem, but it is in a stage quite 
new. And this means a new stage also in the 
development of the idea of faith, in the public idea o 
religion. 

Let me explain what I mean. And let me do so 
by referring first to the history of this country alone, 
and next to the larger history of the Christian 
Church. 

First as to this country alone. 

There have been three great junctures at which 
English religion has been brought into direct and 
critical relation with the State. The first was at the 
\marginpar{25}
Reformation, the second was at the Commonwealth, 
and we are in the midst of the third. In the first 
the spiritual power was completely subjected to the 
temporal through the passion and self-will of the 
Tudors, and especially of Henry VIII. In the second 
it was partially released through the magnificent statesmanship 
of Cromwell. In the third the release 
promises to become complete. In the first stage the 
ruling idea for the Church was still uniformity, only 
with a lay head instead of a clerical, with Henry 
where the Pope had been. In the second the ruling 
idea was toleration, or comprehension, with a tenderness 
for some form of concurrent establishment. By 
the providence of God the sects had arisen; and this, 
which is so often deplored in connection with Protestantism, 
became the means by which the idea of 
toleration was forced upon the public as a step to 
something higher---religious liberty. In the third 
stage the ruling idea has passed beyond either comprehension 
or toleration, yea, beyond liberty; for the Free 
Churches are not only let alone by the State, but 
equally respected, and not only tolerated by each other 
but owned and acknowledged as members of each 
other. And it becomes clear that this consummation 
is only possible throughout by total disestablishment. 
We regain, on a far higher and more spiritual plane, 
the freedom which the Church, had, and always 
\marginpar{26}
demands, in the Roman system. It is the old problem 
of the Church's freedom, but it is in quite a new 
stage; and it is in a new direction that we look for 
the solution. For long the only escape from a State 
uniformity seemed to be into the Roman supremacy; 
but the last 250 years have opened a new and 
living way---the way pointed by Independency and 
heralded in the Commonwealth---the way of the Free 
Churches, of federated instead of monarchical unity in 
the Church. This freedom of the Church is the 
only true completion of the Reformation on its ecclesiastical 
side. And the reason why the Low Church 
party are powerless against the priest to-day is because 
no Established Church can ever in spirit be truly 
Protestant. It is too institutional, too legal. It is 
weak against the priest because it is spiritually lamed 
by its compromise with the State. The Reformation 
faith that should fight the priest has one hand occupied 
in clinging to the State and it can do little more 
than shake the other. Among the crucial religious 
junctures I did not name the great Evangelical movement 
of a century ago, which in the Low Church 
party has now settled on its lees. And I avoided doing 
so because it neither had nor led to any direct action 
on the State. It lost the imperial interests of religion. 
The indirect public and social service of that movement, 
especially outside the Established Church, has 
\marginpar{27}
been unspeakable; but its tendency has never been to 
the larger historic issues which are most critical for 
the national life. Its strength and its weakness have 
been its individualism and pietism. The weakness has 
been especially developed in the established section of 
it. And its distaste for public and historic affairs has 
led to a mental cramp which is another aspect or 
source of its powerlessness in the present crisis. 
Whatever may be said of the High Church party, it 
cannot be said that they have disowned the public, 
social and historic mind. And nothing could furnish 
a greater contrast with the fate of the Evangelical 
party in the Church than the career which the same 
movement has followed in the Methodist bodies that 
carried it outside the State. The Reformation principle 
\textit{found} itself in them; and it moves in them still 
with growing power to its true effect of freedom. 

\section*{III}

But why has the course of the Reformation in this 
country been so slow? And why have we still to be 
working out what other lands have long settled? 
And why does the conflict spread over not only so 
many battles but so many campaigns? 

Because the Reformation, though spiritual in its 
aim and genius, was in this country only to an inferior 
\marginpar{28}
degree a religious movement. It was in the 
first place a political movement, and in its methods 
violent and coercive. It has been cursed with the 
taint of force, and it has only been slowly purified 
into a better mind. 

There is a striking analogy offered here with the 
course of spiritual progress in the history of another 
intractable people, Israel. The soul of the Reformation 
is the moral spirit of the prophet rising up 
against the canonical temper of the priest. Now in 
the history of Hebrew prophetism we have the same 
course of error and the like correction of it. We 
have the reforming prophetic spirit in Abijah, Elijah, 
and all the early prophets, protesting against the pagan 
or curial corruption of religion, but mixing itself with 
political conspiracy, and employing political methods 
even to the extent of massacre and other violent 
means, as in the case of the priests of Baal. And we 
have this violence reproducing violence through some 
centuries, till the kingdom was destroyed by the 
nemesis of its reactions in the exile. But all the time 
the prophetic spirit was disengaging itself from this 
crudity and barbarity of its early methods. It became 
by experience spiritualised into the almost Christian 
inspiration of Jeremiah and the second Isaiah. These 
may be said to have been the persecuted Nonconformists 
who both carried on the principles and refined 
\marginpar{29}
the methods of the rugged puritans who went before 
them and brought to pass the great kingdom to come. 

It was a like discipline that passed upon the Protestant 
movement, in this country at least. As it 
went on it deepened in its principle and sweetened in 
its ways. The bane of Henry's action was its violence, 
its self-will, its mere national and individual 
passion. In the matter of the divorce there is no 
doubt the Pope was right and Henry wrong. Had 
the movement been only of man Henry would have 
killed it. As it was he threw it back indefinitely, and 
entailed upon a long posterity the task of making good 
the errors of his coarse lead. If England had only 
had but one commanding religious genius to be for her 
what Luther, Calvin, and Knox were to their respective 
lands! The like masterful and violent policy 
marked Elizabeth, though to a less degree. And it 
was largely compelled, I must admit, by the fight for 
national life against the incessant political plots and 
treacheries of Rome. It was the potsherds of the 
earth striving with the potsherds of the earth by 
earthy methods on either side. 

The Anglicans insist that the Established Church is 
not a Protestant Church, and there is a sense in 
which they are right. What established the National 
Church did not establish the Reformation. That was 
done by the Puritans, whose tradition we continue. 
\marginpar{30}
The National Church was established by Henry, and 
Henry was no Protestant. The nationalism of the 
Church had been \textit{founded} before, amid the national 
aspirations which fermented in the whole of Europe 
before the Reformation, but it had striven in vain to 
\textit{establish} itself against the Ultramontanism of the Pope. 
What did establish it was Henry's act in a plea where 
Henry was wrong and the Pope was right. The 
National Church was \textit{established by} (I do not say 
\textit{founded on}) a crime of wrong and force. And of that 
crime the Free Churches with their sufferings are the 
remote expiation, as they are the perfecting of the 
true and living word of the Reformation. 

But the Reformation would have come, Henry or 
none, though it would have come otherwise and better 
without the Tudors. 

Among the despised and persecuted sects there was 
growing up a new idea of the Church and its freedom. 
And when Cromwell came to power---far more truly 
than Henry the Defender of the Faith---there emerged 
into political practice for the first time the idea of 
toleration and mutual respect between the sections of 
the Church. This idea, with its blossom of civil and 
especially religious liberty, has been the great direct 
contribution of Independency to the higher life of 
England. It was an idea that seemed to many at the 
time a political peril and a religious crime. It could 
\marginpar{31}
have come, historically speaking, by no other way than 
by the sects. They were there by the will of God 
for the service of His great Church and its freedom. 
They came to give the idea of spiritual freedom a 
new interpretation. It was by the descent of this idea 
under Henry's royal supremacy, and its disguise under 
the extravagance of the sects---it was through such 
humiliation, death and burial that the idea passed out 
of its Roman form and rose into the large liberty of 
the Spirit for which the Free Churches stand. 

But the Commonwealth only placed this idea in a 
monumental way on the political ground. It was not 
able to keep it there. The advance was too great and 
rapid to be permanent. The whole spiritual resource 
of English Protestantism was expended on this immense 
move, and there was none left to consolidate it. 
The great wave swept back; the Restoration came 
with its disastrous results to morals as well as faith. 
And it was not till 1688 that the principle of toleration 
was really incorporated with the English Constitution. 
And it has taken all the time from then till 
now to develop toleration into its true form of liberty. 
The work is not yet complete, but completion is in 
sight. And the total separation of Church and State 
becomes to an increasing number not only the solution 
of present difficulties, but the necessary consummation 
of our national and ecclesiastical past. 

\marginpar{32}
Why was it that the great spiritual triumph of the 
Commonwealth was so short-lived? Just because it 
was (through the inevitable circumstances of the time) 
to so large an extent unspiritual; because, though it 
was the triumph of England's best soul, it was the 
victory of an army. It was a victory of the stalwarts 
rather than of the saints. It was faith, but it was 
mailed faith, faith working by force and secured by 
the sword. Why has the battle of spiritual liberty 
been so slow and hard from then to now? Because of 
that memory of an army's triumph, though it was the 
godliest army that the world ever saw, the first serious 
attempt to make the Bible instead of the Church the 
ruling influence in State affairs. So bitter were the 
memories left by that victory that it is doubtful if we 
should have got even the toleration of a century later 
had Dissent not become so weak in the reaction as 
to be thought contemptible and harmless. 

Both in Henry's work and in Cromwell's the great 
triumph was really retarded by the force and haste of 
the particular victories. He that believeth should not 
make haste. It was on no national conversion or 
conviction that either movement stood; and the 
wrath of man, even of godly men, does not work out 
to its high end the spiritual righteousness of God. 
Spiritual freedom can only be secured by spiritual and 
reasonable ways. Neither man nor nation can be 
\marginpar{33}
coerced into freedom. It must sink into men's minds 
as a principle. It must convert the nation, and not 
merely the \textit{\`{e}lite} of the nation, to its faith. A godless 
king goes down before a godly army. But even the 
godly army melts before the slow growth and instinct 
of parliamentary rule. Representative conviction wins 
permanent victories and achieves beneficent revolutions 
which are refused to dictatorial conviction. A parliament 
is \textit{in its nature} a more spiritual thing than a 
despot, even a godly despot. It appeals to moral conviction 
and rational consent. It is better that a Church 
should be ruled by a parliament than by a king. It 
is better because it is more hopeful. There is more 
hope that a parliament, with its base deep and deepening 
in the national reason, should see its true relation 
to an institution like the Church, which appeals to 
spiritual conviction alone. There is hope, I say, that 
a parliament will perceive that its true relation to the 
Church lies in letting it alone. It may be brought, 
without a king's loss of \textit{amour propre} to feel that to 
sever with the Church is not to part with it or renounce 
it, but is the debt and honour due to the 
Church's holier freedom. Severance here means reverence. 
First the monarch dictates to the Church, then 
parliament patronises it. We have now come to a 
point at which both royal supremacy and parliamentary 
patronage are felt to be unspiritual things, partaking in 
\marginpar{34}
different degrees of force and earth, compared with 
the pure spiritual and rational appeal made by the 
Gospel, which is the charter of the Church's life. 

Slowly the conditions of spiritual freedom have been 
learned, both for the soul and for the Church; and 
in England most slowly of all Reformed lands. They 
seem to have been free-born, while with a great price 
she gains her freedom. While other Churches have 
been developing their Reformation, we seem only to 
have been securing it. We have been spending, on 
the effort to keep from slipping back, the strength that 
might have carried us far forward. The State with 
us has gained more from the Church than the Church 
has from the State. It is a Church whose spirit 
savours more of the throne than of the Cross, of 
English pride than Christian penitence. Our overwhelming 
political genius has brought us, along with 
untold blessing and glory, also peril and loss. It has 
yielded to the self-confidence of strength, and attacked 
questions where even an English statesman must be 
foiled if he is statesman and no more. Of these 
questions the chief is that of the Church. For its 
problems and its freedom the wisdom of this world 
is nought. The wise have not its secret, and the 
mighty have not its power, and the mere freeman has 
not its liberty. Religious liberty brings civil, but 
civil does not bring religious. And no freedom worthy 
\marginpar{35}
of the Church can rest upon any methods, political 
or social, which despise, boycott, or coerce, but only 
on those which persuade the reason and win our trust. 
We have learnt this politically; when the lesson has 
been learnt socially as well, then the true Church will 
be free in a free State, and faith will be, as the 
Reformation preached it, its own advocate, patron, 
defence, and power. As toleration took the place of 
uniformity, and as liberty grew out of toleration, 
so out of liberty grows the true fraternity of the 
Churches, their mutual need and acknowledgment of 
each other; and thus the federal fabric grows into a 
holy temple in the Lord. 

It is the old long problem, but it is in a new stage. 
That is so, I have shown, in the evolving history of 
our own land. May I now move to a wider field, 
and show that it is so on the scale of Europe and the 
Church universal? 

\section*{IV}

It will help us to realise the situation on the large 
historic European scale if we put it in this way. We 
are familiar with the part played in the history of this 
country by King, Lords, and Commons. We understand 
more or less of the way in which their conflicts 
represent the struggle of the three political 
principles---\marginpar{36}the 
monarchical, the aristocratic, and the democratic. 
We see how the interaction of prince, peer, and 
people has worked out the line of progress. We see 
how the Commons mastered the King in the fate of 
the Stuarts, how they are now pressing for a similar 
mastery of the Lords. We see how the democratic 
principle swallowed up the monarchical, how it is 
swallowing up the aristocratic, and how in our 
American daughter it disposed both of the monarch 
and the peer. I am saying nothing of the merits of 
the case. I am simply noting the facts. And I do 
so in order to mark the same conflict of ideas in the 
medieval Church, in the Catholicism of the pre-Reformation 
age. You have the same three principles 
in collision, the same struggle waged on a 
continental scale and in the spiritual realm. You 
have the Pope corresponding to the King with his 
Divine right. You have the bishops corresponding 
to the barons or peers, with some claim to constitutional 
freedom. And you have the mass of the 
laity, who ever since the thirteenth century had been 
growing in culture, wealth, and municipal freedom. 
For a long time in this country we had the quarrels 
of King and barons; and so in the medieval Church 
it was a long war between the Pope and the bishops. 
It was a question that became acute in the twelve 
years' Council of Basle in the fifteenth century, when 
\marginpar{37}
it was decided by the bishops that a general council 
was above the Pope, and had the power, on due cause, 
of deposing him. That marked a memorable stage in 
the struggle of the Church to save itself from the 
despotism of the Roman Curia, or what we now call 
the Vatican. It seemed as if the episcopal principle, 
the conciliar principle, the House of Lords' principle, 
were going to save the Church from its despot and its 
abuses, and reform it so far to the mind of the spirit. 
It looked as if it would prune the papacy as the barons 
won our popular rights in Magna Charta from King 
John. But events were too strong for the council. 
It could not carry out its principle into fact. The 
papacy was too strongly fixed for the bishops to dislodge 
it. The Pope was himself a bishop, and the 
product of the episcopal system. That system had 
not the power, the secret, of reforming and saving 
itself. A century later the Pope was as powerful 
and mischievous for the Church as ever. The whole 
Church, its morals and its doctrine, were sacrificed to 
Rome. To build St. Peter's, Europe was overrun 
with the scandal of papal indulgences. It was built 
with human sin and shapen in iniquity. 

Meantime there had been coming up in the wake 
of both Pope and bishop the Commons of the Church, 
the layman, and especially the monk. I couple 
these two because they represented for that age the 
\marginpar{38}
democratic principle. Neither layman nor monk was 
priest; both Pope and bishop were. And that was 
why the bishop could not conquer the Pope. Satan 
could not cast out Satan. Well, behind the struggle 
of Pope and council there was moving up the democracy 
(in the form of that day), with a remedy far 
more drastic than either could bring to bear on the 
state of things. Pope and bishop were exchanging 
anathemas, but how was it meanwhile with the third 
great quantity, the soul---the mad, guilty, lost soul? 
Pope and prelate were at their long duel, and they 
were so engrossed with each other that they did not 
see the crowd of hungry souls that were pressing nearer 
and nearer round them, asking to be fed with the bread 
of life, and released from the curse of guilt. Men 
had turned away from the priests to the monks for 
some centuries now. Movement after movement had 
risen to attempt for the great Church that reformation, 
that emancipation, which the curia would not, 
and the councils could not, bring about. The devout, 
the sin-torn, the humane, turned from the altar to 
the cloister. But, alas! monasticism itself fell a victim 
to the same corruption, or the same impotence, as 
paralysed the other organs of the Church. Something 
was wanting to them all. And it was the Gospel, 
dealing directly with conscience and guilt. What the 
sacraments and absolutions of the Church could not 
\marginpar{39}
do away was sin as guilt upon the conscience and not 
as a mere infection of our nature; it was sin as guilt 
and damnation. It was the removal of guilt that the 
soul cried out for, and that the Church could not give. 
It was grace as mercy and reconciliation, not as mere 
amnesty and sweetening of the soul---grace as an 
act of God on the moral soul itself---that was the only 
remedy for sin when sin came home as guilt to the 
conscience. It was grace as a gospel, and not as 
a mere influence, not as a mere sacramental infusion, 
that was the one thing needful for the tormented soul. 
Even monasticism could not supply that. 

Yet it was out of monasticism that the real saving 
word came. It was a monk that saved---I do not 
say the Church, I say the Gospel, Christianity. 
The Church is not worth saving, except for the sake 
of the Gospel. And the Gospel was just what an 
episcopal, priestly and Catholic Church could not save. 
Luther was the protagonist of the single sinful soul---
the third estate, the supreme interest, of the Church, 
the first charge on it. \textit{Salus populi suprema lex}. The 
cause of the democracy is the cause of the soul as conscience; 
and the Reformation was the moral soul, the 
conscience, reasserting its place in the Church through 
the Gospel, in a way unparalleled since the first century. 
It is hopeless for Rome, or Anglicanism either, to 
attempt what they are attempting now---to be the 
\marginpar{40}
Church of the democracy. The religious democracy 
means a moral freedom utterly foreign to Rome, or to 
any priestly Church. It means a freedom of faith, of 
conscience, and of person, to which the priesthood is 
a standing contradiction. What brought these was 
the Reformation. The Reformation was the moral 
soul of the people rising against a priestly order that 
had hopelessly abused its power and always must. 
Spiritual falsehood \textit{must} end in moral abuse. The 
Roman priesthood is a spiritual lie, and it is self-doomed 
to moral wreck and a public reaction. The 
Reformation for Europe corresponded to the Commonwealth 
in England. It was in relation to Pope and 
bishop what the Commonwealth was in relation to 
King and Lords. 

Luther was the Cromwell of the Church; Cromwell 
was the Luther of the State. The only remedy 
for the state of things in the Church was the radical 
movement which in Luther gave the Gospel back to 
the soul. It remodelled the Church after the pattern 
shown on the mount of Calvary, by way of redemption, 
of forgiveness, as a personal experience. The 
Church could only exist as a community of the forgiven, 
not merely of the absolved; as a society of priests, 
and not a priest-led society; as a congregation of the 
justified, living by their personal faith, and having 
their spiritual head in Christ alone. The great and 
\marginpar{41}
cardinal religious principle of the Reformation was 
connected with sin, and it declared that for the forgiveness 
of sin the priest was not a necessary party. 
That is the real issue still. Is the priest essential to 
forgiveness? We say no, as the Reformers said. 
The priest was by them swept aside, along with 
Pope and bishop in so far as these stood upon their 
magical priesthood as essential between God and man. 
Forgiveness meant conversion, the direct action of God 
on the soul and access by the soul to God. And 
the conversion of the soul was so radical and so 
central that it carried with it a total change in the 
constitution of the whole Church. The power that 
remade the soul was the only power that had right 
to prescribe the fashion and order of remaking the 
Church. The Church is but the social expression 
of the same principle of grace as saves and changes 
the single soul. The polity of the Church is latent 
in the principle of our saved experience. So it was 
in the beginning, in the Church's first making; and 
so now in this great re-beginning it was declared to 
be. The Church system, like the Church doctrine, 
ought by right to be the expression of personal, saving, 
experimental faith. No constitution was given the 
Church, even by Christ---no bishops, no priests. His 
apostles were not officers, but ministers; not ecclesiastics, 
but preachers. The constitution grew historically, 
\marginpar{42}
out of the needs and insights of Christian 
faith, and it became historically corrupt, by the infection 
of a pagan time. But with this corruption, in priest, 
or priestly bishop, or Pope, the primitive and germinal 
faith is always in deadly war. The Spirit leaves the 
Church where it is compelled into these channels. 
The Holy Spirit did not of course desert all Romanists, 
but it did desert the Roman Church in its official 
organs and its Jesuit policy. The home of the Holy 
Spirit, of Redemption, and the Gospel was henceforth 
to be where the word of the Reformation Gospel came 
with power and effect. Generous Romanists concede 
to us heretics some workings in individual souls of the 
uncovenanted mercies of God, but they monopolise 
for their Church the chief blessing of His perennial 
corporate guidance. That is just how we put it in 
respect to Rome. And we do so because we must 
believe with the New Testament that the Spirit goes 
with the Gospel, and not with the succession and the 
sacraments. It is not at home in a Church where we 
hear more about absolution than about redemption, 
where devotion is more than conscience, and where 
the sacraments are more than the living Word. 

\section*{V} 

But what was the effect of the Reformation on the 
old strife within the Roman Church between Pope 
\marginpar{43}
and bishop, Popes and councils? The effect was 
what is called the Counter-reformation. The old 
policy of reforming the Church by councils of 
Catholic and monarchical bishops was resumed. It 
is the refuge the Anglican Church is taking to-day. 
Anglicanism has so far sided with the Counter-reformation. 
But it was more than reform that the Church 
really needed. It was conversion, regeneration. And it 
was regeneration that it received from Luther and his 
friends. The Regeneration would be a much more 
fit name than Reformation for a movement which 
changed central ideas of Christianity like grace and 
faith, and turned religion from assent into experience, 
from assent to a Church into experience of a Saviour. 
But the policy of mere reformation, mere amendment, 
was the only one of which the debased Church was 
capable. An institutional Church never knows its 
own spiritual ineptitude. Mere reform is about all 
that bishops, or any other officials, \textit{can} do; and they 
are very slow in doing that till they are pushed on 
from behind and beneath. The old policy of conciliar 
reformation, then, was thought adequate and was 
resumed, and the Council of Trent was called. But 
councils were not now what they had been. The 
Reformation had withdrawn from the Roman Church 
the spirit, the element, that had given the Council of 
Basle the slow strength it had. The intellect and conscience \marginpar{44}
of Christianity were among the Reformers. 
What intellects these were! The Roman Church was 
left without the moral power to vindicate the Church's 
freedom against the Pope. In the Council of Trent 
there was not much done to destroy the abuses which 
caused the Reformation. Little heed was taken of 
differences within the Roman Church itself; while 
much was done to controvert the principles of the Protestant 
heresy. But the serious constitutional feature 
of the Council of Trent was this, that it surrendered 
the ground taken by its predecessor of Basle against the 
Pope. The Vatican came out of it so much stronger 
than before that the popes frequently afterwards (especially 
in dealing with Jansenism) took to settling matters 
of doctrine on their own responsibility. In 1854 Pio 
Nono raised the doctrine of the Immaculate Conception 
of Mary to a dogma in this way, without consulting 
a general council at all. What he had was only 
a conference of bishops in sympathy. And at last, in 
1870, the Council of Rome delivered the whole 
Church for ever into the hands of the Pope by the 
dogma of papal infallibility. That was the complete 
victory of Curialism, of Vaticanism. It was thus for 
ever proved that general councils are useless without 
the evangelical element which the priesthood denies 
and excludes. Councils plus the priesthood must end 
where the Council of Trent has ended---in the deification
\marginpar{45}
of the Pope. This is a lesson which Laud 
had not learned, and which the High Church Anglicans 
to-day even do not grasp. This deification of 
the Pope is the latest act of what has been called 
"the spiritual tragedy of European society." 

The first act of that tragedy was the catholicising 
of the Church in the second century through the 
power of the monarchical \textit{Bishop}. 

The second act was the consequent secularising of 
the Church in the fourth century by its association 
with the throne under Constantine---its debasement 
by the power of the \textit{Emperor}. 

The third act may be said to have been the final 
adoption at Trent of the theory of Transubstantiation 
in the mass. It was there and then that the 
mass was finally defined as a propitiatory sacrifice. 
It was thus that the awful power of the \textit{priest} was 
locked about the neck of the Church. The catechism 
of the Council of Trent describes the priests as 
gods much more than angels. "Ipsius Dei personam 
in terris gerunt---quem merito non solum angeli sed 
dii etiam, quod Dei immortalis vim et numen apud 
nos teneant, appellantur" (ii. 7, 2). 

The fourth act of that tragedy is the promulgation 
of the dogmas of the Immaculate Conception (the 
sinlessness) of Mary and of the infallibility of the 
\textit{Pontiff}. Thus the united power of bishop, emperor 
\marginpar{46}
and priest was for ever fastened on the Church in a 
Pope, and its doom and debasement sealed thrice sure.\footnote{
Pio Nono was the victim of a self-idolatry which seems hardly 
sane, and which reminds us of some phases of another career. The 
German Emperor allows himself to be referred to in an expression 
like "the Gospel of your sacred majesty." And Pio would use 
phrases like this, "Keep, my Jesus, the flock which God has 
committed to Thee and me." He would apply to himself, "I am 
the way, the truth and the life." He regarded his troubles as a 
renewal of the sufferings of Christ. One of his cardinals spoke of 
him in 1866 as the living incarnation of the authority of Christ. 
Veuillot (1866) identified the crucified of Jerusalem and the crucified 
of Rome so far as to say to both alike, "I believe in thee, I adore 
thee." In 1868 the great Catholic newspaper of Rome said, 
"When the Pope thinks, it is God thinking in him." Faber proposed 
an act of devotion to the Pope as a supreme test of Christian 
sanctity. In 1874 a Jesuit paper applied to Pio the words, "Which 
of you convinceth me of sin?" And there was a hymn sung by the 
German Catholics celebrating his priestly jubilee in 1869, "Pius, 
priest, our sinful age, wondering, finds no sin in thee."
}

What will the fifth act be? and how will the 
dreadful \textit{d\`{e}no\^{u}ment} and crisis come? 

Did I say ill when I said that the Holy Spirit of 
Christ, of the Cross, and of the Gospel, had forsaken 
the Roman Church as a Church and taken its abode 
elsewhere? 

And is there any hope when the crisis comes but 
in the evangelised people, in the monarchical democracy 
of Christ alone, when the divine right of 
bishop and priest will be slain (as that of king has 
been slain among us), and they will be there, if there at 
all, for the service of the Church and not for her rule, 
\marginpar{47}
as her ministers and not as her lords? Is there any 
hope for the great Armageddon but in the completion 
of the Reformation, in the rescue of the Gospel from 
cultured humanism on the one hand, and from traditional 
priestism on the other, for the true Church of 
a faith which is personal experience of forgiveness 
direct to the soul from Christ in every age? 

\section*{VI}

But to grasp the present situation in the Anglican 
Church let us go back. We have these two currents 
in the Church of the sixteenth century---that of the 
Gospel, by a conversion and regeneration worked 
through personal faith; and that of the Church, by a 
mere reformation, worked through bishops, as Church 
lawyers and politicians, upon the Church's creed and 
practice. The one lays all stress upon the Christian's 
universal priesthood, and consecrates no form of 
Church government as of divine right; the other lays 
all stress on the priesthood of a class and upon an 
episcopal regime. The one operates by personal faith; 
the other throws that into the rear, and is chiefly concerned 
about the reform of the Church as an institution, 
and the perfecting of its ritual, by means of 
councils---councils of men often singularly devoid of 
the holiness which is the condition of divine knowledge;
\marginpar{48}
councils no more consistent than those of 
Basle, Trent, and Rome. 

Well, is this not just the difference between the 
Free Churches and the bishops to-day? On which of 
these two currents does the Anglican Church of to-day 
embark? One of its most brilliant representatives 
tells us that the chief value of the Reformation was 
that it called forth the Council of Trent and its blessings 
to the Church. The chief value of the Gospel 
was the enhancement of the bishops. The Church of 
England is a child of the Reformation, but the clergy 
of England are children of the Counter-reformation in 
principle and spirit. The inner spirit and temper of 
Anglicanism is that of a return, more or less prudent, 
to the method of this Counter-reformation and its 
conception of truth. Canon MacColl, in his book 
\textit{The Reformation Settlement} quotes with approval words 
which have the imprimatur of Cardinal Manning---
"All that we know and believe now, the entire cycle 
of Christian doctrine in all its circumstances, was 
known and believed by the apostles on the day of 
Pentecost before the sun went down." Its faith is in 
councils of bishops as the organs of Church reform, a 
repudiation of the lay element, and a depreciation of 
its supreme and priestly power of faith, a culture of 
formal reverence and a neglect of the soul, a rehabilitation 
of the priest and a corresponding trifling with 
\marginpar{49}
human guilt. The Church is placed in the authoritative, 
not to say infallible, place which Rome gives to 
the Pope---the priest-led Church with its episcopal 
councils. 

Observe, besides, the blindness, or the affected blindness, 
of these counter-reformers. The Council of 
Trent claimed to be the consummation of a series of 
reform movements which had been working in the 
Church quite independent of Protestantism. It 
simply carried on the Catholic tradition of the Church 
(that is, the bishops) mending itself. Protestantism 
was treated as an episode to be ignored in the development 
of the Church, an excursion which refused 
to be recalled and so became an excrescence. 
It could therefore be cut off and dropped without the 
true Church losing a limb or suffering in beauty. 
This is really the line taken by Anglicanism to-day. It 
ignores Nonconformity with its history of at least two 
centuries and its possession now of more than half 
the nation. It affects to ignore any contribution 
from that quarter, though its whole attitude to the 
country has been changed by it, as Trent was to 
Europe by the Reformation. As the old counter-reformers ignored the Reformation, so to-day our 
counterfeit reformers ignore our Re-reformation. (I 
call them counterfeit reformers because, to effect their 
ends, they use a position given only to a Protestant 
\marginpar{50}
Church.) Anglicanism as a movement stands by the 
institution rather than the soul. It places itself in 
the line of Church \textit{reform} not reformation, not regeneration. 
The Holy Church never needs regeneration, 
it says---only reform. Its institutional idea of 
faith is quite satisfactory. It is self-satisfied. It heals 
lightly the wound of the daughter of the people. It 
would only reform upon Trent and recur to Basle. 
It would discard the supremacy of the Pope and restore 
that of the episcopate. It would not discard the 
priest, but only the Jesuits, who have captured the 
counter-reformation. It rejects the white Pope and 
the black---the Pontiff at Rome and the General of 
the Jesuits---but it holds to the bishop and the priest 
as essential to the Church, and would fain hold to the 
mass. It goes round the Reformation and catches up 
the middle age Catholicism, of which it claims to be 
the true continuity and successor. The Anglican 
Catholic keeps the medieval idea of religion, or faith, 
as a threefold cord of knowledge, conduct, and mystic 
sacraments. He discards that idea of faith which 
really constitutes a new religion recovered from the 
New Testament, in which faith is personal trust in a 
personal Saviour, the soul's direct, experiential, and 
priestless answer to God's grace as the forgiveness of 
sin, the destruction of guilt, and reconciliation by the 
blood of Christ alone. 

\marginpar{51}
\section*{VII} 

Our insular issue is the revival of the question of 
Laud's day, whether the Church of England is 
Catholic or Evangelical, priestly or lay, whether its 
Reformation did more than break with the Pope, 
whether it was religious or institutional, whether it 
ought not to catch up and work out its religious 
continuity with the Catholic Church of the Middle 
Ages after effecting by all the last 300 years no 
more than the rejection of the Pope, whether the 
continuity of the Church was in its bishops or in 
its faith, in what linked Laud with Cyprian or Luther 
with Paul.\footnote{
"Pusey's idea in the \textit{Eirenicon} was to make the Trent decrees 
a basis of reconciliation; if the Romanists would only confine 
themselves within Tridentine limits, he hoped to screw up Anglican 
teaching so far" (Salmon, \textit{Infallibility}, p. 202).
} 
But while it is a revival of this question 
that we see, it is not merely a threshing of old straw. 
It is being discussed inside the Established Church in 
another than the spirit of Laud's age. It is more 
free from political complications and afterthoughts. 
It is being raised by really religious men---men not, 
like Laud, of hard, formal, mechanical intelligence---
in the interests of the Church far more than of the 
State, and on Church more than on political principles. 
It is raised by men to whom the freedom of 
the Church and its autonomy are dearer than political 
\marginpar{52}
and dynastic schemes, and who are not prepared to 
pay any and every price for establishment. That is 
to say, the issue is being raised within the Erastian 
Establishment on Free Church principles, in a monarchical 
Church on democratic principles. They are in 
a false position. In point of principle we agree with 
the High Churchmen and not with Sir W. Harcourt; 
but in point of honesty we agree with Sir W. Harcourt 
and not with them. Still that purer idea of the 
Church on their part is a great gain and a great encouragement. 
History is not simply moving in a 
circle, and a small circle, of a few centuries. It is still 
the ultimate issues of the Reformation that are being 
worked out in a new form, in a real spiritual progress. 
And the same spiritual freedom that has made 
us to be outside of the Church is making for us inside 
the Church. For our great object is not the rejection 
of the Church, but the release of the Church to her 
own spiritual and autonomous rule. And the great 
blessing of that will be the restoration of the officer 
of the Church to his proper place, his elevation from 
a priest of the sacraments to be a minister of the 
Word. When the Church is free to be herself, her 
New Testament self, it will not be so very hard for 
the Holy Ghost to deal with the priest. If the priest 
will come out from behind the prestige of our 
common State we can reach him with the Spirit's 
\marginpar{53}
weapons of the Gospel. And we are implicitly committed 
to this as the completion of the \textit{national} and 
\textit{ecclesiastical} past. First, to complete the \textit{ecclesiastical} 
past. To complete the Church we must not simply 
evade the Reformation and fall back on a medieval 
Catholicism slightly adjusted to this age. We must 
complete the Reformation because it went behind even 
Catholicism and rediscovered the faith and Church 
of the New Testament. The power and right of 
the Reformation over us is not that it was a new 
invention, a new idea, but the rediscovery and revelation 
of the primitive Christian idea and of the New 
Testament. Of that I may speak later. But, secondly, 
in pressing on our line we are surely fulfilling the 
\textit{national} past, and realising the layman's power, faith, 
as the ruling power in the Church. 

Henry's Reformation in throwing off the Pope did 
more than it knew. It implicitly threw off the 
priest. The head of the English Church was now 
a layman, not a cleric. But he was the King. 
There remained, and still remains, the incubus of 
the State. The layman must be a believing man. 
And the beginning of the State's rejection was also 
through men who were doing more than they knew. 
The Nonconformists of 1662, it is true, had not our 
Free Church principles explicitly before all their minds. 
They came out in protest against the priest and his 
\marginpar{54}
associations. They were completing the work of 
those who rejected the Pope. But in throwing off 
the priest they could only do so by throwing off the 
State that sheltered the priest. And so they set the 
foundation of our Free Churchism on an Evangelical 
base, where indeed all the freedom of the Church 
lies. A Church free from the State will soon find it 
cannot be free with the priest. It cannot be free for 
Christ, and it will proceed to deal with the priest in 
the light and power of that Gospel which makes it 
amid all perversions a Church still. These old Nonconformists 
were not all that we mean by Free 
Churchmen. They builded better than they knew. 
The Scotch Free Churchmen of 1843 were not as 
clear and thorough about Free Church principles as 
the Free Church has become to-day. Nor did the 
Evangelical Methodists of last century see where their 
Evangel was going to take them ecclesiastically. 
Many of them are not sure about it now; but is 
there any doubt how it must be? Luther did not 
foresee the great and searching work he laid his hand 
to; and I am sure none of the first Christians, not 
even St. Paul, saw what the ultimate effect of their 
principles would be on the long history of the world. 

So little are these movements due to human device; 
so much are they the organs of a diviner, all-seeing 
Spirit, of a principle latent in the perennial Gospel, 
\marginpar{55}
and bound to overturn, overturn till the Cross take 
its right and reign. 

\section*{VIII}

The real issue then which is raised in the struggle 
now going on in the Church of England demands \textit{our} 
attention because it is really a struggle as to the true 
nature of Christianity, and especially English Christianity. 
It is not a question of ritual. It is a far 
deeper question than one about a more or less of 
ritual. Nor is it one as to whether Episcopacy is 
\textit{preferable}; it claims to be \textit{sole}. Nor is it a 
question of bringing back the Pope or keeping him 
out. So long as the Pope claims the temporal power, 
and asserts his place as a continental sovereign, 
England is inaccessible to him. The national independence 
of this country will always protect us 
from a Pope who is not wholly a spiritual power. It 
is not the sovereign Pope we dread, but the priest 
Pope. The issue is not between Popery and Protestantism, 
but between priestism and the Gospel, 
between sacerdotalism and evangelicalism. The 
antagonist of the Catholic is not to be described by 
a word so negative as Protestant, but by the word 
evangelical. And this drives me into some theology. 

When the great breach took place at the Reformation, 
\marginpar{56}
what was discarded in England, I have said, 
\textit{was} the supremacy of the Pope. It was Curialism, 
it was Vaticanism, it was the Italian suzerainty 
which finally conquered the Catholic Church in 
1870. But the European movement was really a 
greater. It was more than political, more than 
ecclesiastical; it was religious, it was spiritual. It 
concerned the way of forgiveness and the place of 
forgiveness in religion. It had especial reference to 
the \textit{place} of forgiveness. Was forgiveness itself the 
Gospel, or only an incident of the Gospel? Was 
faith faith in forgiveness, or was it faith in something 
else, say in the love of God, with forgiveness for a 
mere accident of the position, a clearing of the way, 
to be forgotten when the path was opened? Does 
Christian faith begin and abide in forgiveness or only 
pass through that stage? The conflict concerned the 
nature of grace and the corresponding nature of faith. 
The Catholic view of grace is sacramental, the Protestant 
is evangelical. In the Catholic idea grace is, 
as it were, a new substance infused into the soul, 
first by baptism, then by the mass (\textit{gratia infusa}). 
It is a sort of antiseptic influence made to pervade 
the spiritual system like new blood. The blood of 
Christ is understood in a material way, though in the 
way of a very refined material. It does not give a 
new righteousness, but power to please God by the 
\marginpar{57}
old. And the faith that answers it is an acceptance 
of the Church's power to convey this rarefied and 
spiritualised substance. The love or charity so produced 
is thought of in the like way as a sort of 
spiritual ether infused into the soul. But in the 
Protestant and evangelical idea grace is not an infusion, 
but an act and way of God's treatment of us. 
\textit{It is not infused, but exercised}. It deals with man 
as a will, not as a substance. It is the same as 
mercy, the mercy of God, the forgiveness of sin, 
the cancelling of guilt, the change and not the mere 
pacifying of the conscience. In a word, for Catholicism 
grace is magic, for Evangelicalism it is mercy. 
The grace of Evangelicalism is Christ, the Gospel, 
the Word. The faith that answers that is living 
faith in a living person directly in converse with 
the soul. It is a new type of religion, and not 
merely a variety of the old. It is faith changed 
from assent to trust. So the Reformation was a 
movement affecting not only the hierarchy or polity 
of the Church, but the whole nature of the Church; 
it challenged the whole Catholic view of Christianity, 
the whole Catholic view of salvation. It was not the 
Pope only that was challenged, but the Catholic and 
medieval conception of faith, of religion. 

From this great searching and fundamental movement 
even the insularity of England could not be 
\marginpar{58}
exempt. The greater spirits of the Church were 
profoundly interested in the large spiritual affairs of 
the Continent. It is only since the Reformation 
that England has become the most provincial of all 
European countries in her thought, most cut off from 
the stream of European culture, most self-satisfied in 
her isolation, and most unconscious of her ignorance. 
Her Reformers and Puritans represented the last of 
the great cosmopolitan influences on her spiritual 
culture. At the Reformation England was through 
the Church a portion of the West still. Her Church 
had not yet become sectional. It was Catholic and 
not national. And so the movement of the court 
and the politicians could not stop with them. To 
renounce the Pope meant renouncing in principle 
that Western Catholicism which had borne the Pope 
for its inevitable fruit. It was renouncing the whole 
Church of the West. There was more at issue 
even than the independence of the national Church; 
there was the independence of the individual soul 
throughout Christendom. It was a \textit{human} issue, and 
the greatest. 

Like much else in English affairs, that which was 
really the most vital issue was not the issue that held 
the foreground in historic time, in the contemporary 
mind. The movers, I would repeat, did not know 
all that they were moving. The Reformation in 
\marginpar{59}
England (not in Germany) only \textit{began} by renouncing 
the Pope; its real nature came later to light---not 
under Henry VIII., but under Elizabeth. It was the 
renunciation of the priest. In throwing off the Pope 
what was really rejected was the priest, though that 
was not realised at the time. Henry had no idea of 
such a thing. He was first a devotee of his own 
passion and self-will. He was next a Catholic nationalist, 
a Home Ruler, an Anglican Catholic, a defender 
of the Catholic faith. The vital nerve of Catholicism 
has two branches---the institutional and the priestly. 
I speak of the latter first. The former will engage us 
afterwards. The true inwardness of the Reformation 
was the rejection of priest and mass. And it was a 
rejection caused by the return to the Bible and the 
rediscovery of the Gospel. What dislodged the priest 
was the Gospel. It was the faith that made every 
Christian man his own priest in Jesus Christ. The 
true anti-Catholic movement is not protesting against 
the Pope, but preaching the Gospel that kills the 
priest. It is evangelical. The Elizabethan Puritans 
were the champions of this true and ultimate reformation. 
It was they who made it what it had been 
in Luther---a reformation of \textit{religion}. For Henry and 
his satellites it was but a readjustment of the Church, 
the change from an Italian to an English head. The 
Church had been the ecclesiastical counterpart of 
\marginpar{60}
European civilization; it was now the ecclesiastical 
counterpart of the English nation. 

The controversy then was really (though not consciously) 
this: Is a Catholic Church with the priest 
the true Church of England, or an evangelical Church 
with the minister who is a layman? Was the ruling 
power in the Church lay or cleric? It was no question 
\textit{then} which of two separate \textit{bodies} was the true 
Church, for there were not two to choose from. 
The Puritans were still members and believers in a 
National Church; and there was but the one body, 
with two tendencies in mighty conflict in its soul. 
The two bodies came later with the Nonconformists. 
After this separation caused by the Act of Uniformity 
there was a long peace of a sort. With the Revolution 
the Church of England was declared to be Protestant, 
and the issue raised by the Puritans won both 
inside and outside the Established Church. But the 
victory within the Establishment was due chiefly perhaps 
to political causes, and therefore it was never 
settled on a religious or final principle. The real 
religious principle of Protestantism has been in the 
care of the Nonconformists. The English Prayer 
Book is a half-baked compromise which is Catholic in 
its services and Protestant, nay Calvinist, in its articles. 
The Prayer Book is a cake half turned which deranges 
the digestion of its Church. 

\marginpar{61}
I am ready to grant that in the wisest of the Anglicans 
there is a modification of the medieval ideas even 
of the priesthood. They may be willing to regard the 
priest as the representative of the Church rather than 
its ruler, as an expression, a projection, of the universal 
priesthood rather than its creator. Yet so long as they 
hold to the theory of baptismal regeneration they do 
practically make the priest the creator of the Church. 
So long as they cling to the apostolic succession they 
fatally sever the government of the Church from the 
living soul of the Church.\footnote{See concluding Lecture}
And the fact that they 
cling to the name priest, in spite of its studied rejection 
by the New Testament, shows that they do so 
because they give the authority and tradition of the 
Church a place too nearly abreast of Scripture. It is 
not altogether because priest expresses the sacrificial 
idea (which is essential to Christianity) in a way that 
minister does not. But the cardinal defect of their 
position is still in their conception of religion, that is 
to say in their ideas of faith and grace. Faith is still 
for them primarily an institutional thing. It is inseparable 
from faith in an empirical institution, the 
Church and its officers. It is not the direct and simple 
response to grace as an act and mercy of God in Christ 
through the Gospel to the believing soul. Their 
Church and priest are absolutely necessary to salvation. 
\marginpar{62}
Their faith is institutional, it is not evangelical. The 
Gospel for them is not where it should be in price 
and power above the Church. They do not realise 
the full force of the fact that it was the Gospel that 
made the Church, and always must make it. The 
Church was before the Bible, but it was not before the 
Gospel, it was not before faith, which is the answer to 
the Gospel and not to the Church. Such Churchmen 
have another than the Christian idea of faith, which is 
evangelical, an answer with the heart to the word of 
reconciliation, with the conscience to the act of redemption. 
They do not grasp the fact that the 
reformation of the Church means at its centre the 
reformation of faith, the change of the soul; that the 
Church needs deliverance not only from errors, nor 
from abuses, but from burdens. And they do not see 
that their conception of the bishop and of the priest 
as of the essence of the true Church lays a burden on 
the Church which it was the business of the Reformation 
once for all to cast off, not only as an impediment 
to the Church's working, but as a load of 
suffocation upon faith itself. This institutionalism 
lowers the temperature of faith, and it lowers the 
sense of sin. It sits with a frosty weight of tradition, 
convention, and worldliness upon the ideals of Christian 
people. Why is the Church so much less 
worldly than it should be? It is the place which has 
\marginpar{63}
been held in the religion of this country by a Church 
more institutional than evangelical that is responsible 
for most of the crude and childish moral sense of the 
Christian public, the lack of Christian as distinct 
from ecclesiastical enthusiasm, and the want of sensibility 
in the Christian conscience. Nowhere are these 
things more deplored than by some Churchmen who 
fail to see the cause I name. A morality merely conventional 
and social has blocked the way of a morality 
inspired and tested by the Cross; and the conscience 
of thousands has been stunted by the sealing of pagan 
ethics learned at school with the seal of a Church 
which for them had replaced the Gospel as the moral 
authority. The establishment of a Church more institutional 
than evangelical weights faith too heavily 
for its purpose in the kingdom of God. 

To demand the bishop and the priest in the name 
of the Gospel is to ask in Christ's name for what 
Christ never named. It is to load the Gospel with 
something that neither Christ nor Paul put into it, 
and to empty it of much that they thought of its 
essence. It is throwing the weight of the Church into 
the wrong scale in the age-long issue between Romism 
and Evangelism. It is to re-introduce upon the 
Gospel the spirit of the law. It is a perpetuated Judaism. 
It is the spiritual and fatal restoration of the 
Jews. It is a Christian anachronism. It is to cherish 
\marginpar{64}
an idea of faith that the Gospel of the Cross left 
behind, a Pelagian and Synergistic idea of faith which 
is fatal to faith in its absolute and imperial sense in 
the Gospel. It dissolves the work of Paul. It restores 
the Gospel to the law. A Church established 
by law can only be a law Church, a statutory Church, 
a branch of the public service, rather than the conservatory 
of the public conscience and the home of 
the godly soul. It becomes another Gospel. The 
struggle is one for the very nature of Christianity as 
Gospel and not law. The beginning of all Christian 
truth, said the Reformers, is to grasp the distinction 
between law and Gospel. If we could but see that, 
and fight the battle on that sharp issue, the conflict 
might be honester and shorter. The battle of the 
Evangelical Free Churches is for the New Testament 
idea, the true Christian idea of grace and faith. It is 
war between faith in an institution and faith in a 
Gospel, faith to which priest or bishop is essential and 
faith which is perfect without either. It is for a forgiveness 
which is complete without the priest, and 
damaged by him. What we Free Churchmen are 
committed to is the Reformation in the sense of a rebirth 
of religion, and not a mere readjustment of the 
Church. Luther never began with the idea of reconstructing 
the Church, but with the experience in 
him of a new conscience, of a new conception of 
\marginpar{65}
religion, of grace and faith. He was converted from 
the Church, not from the world. It is not our form 
of government, our view of history, that is so different 
from that of our adversaries, but our view of faith, of 
religion itself, of the soul's relation to Christ, of the 
meaning of Christ for the soul, of the will and nature 
of the object of faith---God. Oh, it is a very deep 
and serious issue that is raised, and so long as the 
Gospel Word endures it can never be stilled till the 
Gospel principle come to its own in the Church. 
And I will freely say that it means a regeneration of 
faith in the Free Churches no less than in the unfree. 
For the state of faith and the idea of faith 
which make Catholicism and its priest possible and 
fashionable are \textit{our} misfortune and \textit{our} defection as 
well. We too are infected with that poverty of personal 
faith and New Testament knowledge which the 
priesthood thrives on. And I much agree with the 
fine saying of a great Roman Catholic writer: 
"When the day of reconciliation comes to the 
Churches it will start from the confession of our 
common guilt." 

What the Church needs as the condition of reformation 
is a regeneration of the idea of faith and 
the consequent humiliation. 

\marginpar{66}
\section*{IX}
 
To sum up, there are three ways in which the 
work of the Reformation may be viewed and is viewed 
to-day. First, there are those who have nothing for 
it but antagonism and abuse. They think it was a 
huge historic mistake and calamity for the Church. 
And they believe that the chief and permanent reform 
was one which the Reformation brought about against 
its will and in correction of its work, viz., the reformation 
which the Catholic Church effected on itself 
in the Council of Trent. This view is not confined 
to the Romanists, but is the view of the High Anglican 
party as expressed in such writings as those of the 
late Aubrey Moore. Yet it will be noted that there 
was nothing more central to the Council of Trent 
than the authority of the Pope. 

Secondly, there are those who regard the work of 
the Reformation as called for in its day but now spent 
in its effects. The protest was made when it was 
needful, and it was a real contribution to the history of 
the Church. But its work is done; the Catholic 
Church received a lesson it will not easily forget. 
Many of its abuses were rectified, and with the correction 
of the abuses the necessity ceased to maintain the 
protest. The Reformers are now chiefly interesting 
to the historian, and have no direct or vital meaning 
\marginpar{67}
for the religious life of the Church. That life, over-leaping the Reformation and the abuses that created 
it, must go back and connect itself with the greater 
and earlier ages of medieval Catholicism, which again 
continued the true patristic tradition. The Reformation 
was thus a temporary expedient for the cure of 
certain abuses. It may be thanked and pensioned off 
for its services at a certain juncture, but it must never 
be allowed to take the reins of Christian progress or 
turn its course out of Catholic grooves. Perhaps this 
is the most distinctive Anglican view in so far as it 
can be reached. 

Thirdly, there are those who regard the Reformation 
not as a temporary movement for the correction 
of certain abuses, but as a really new point of departure 
in the history of the Church and a profounder 
return to the mind of Christ and the New Testament. 
It was not simply a crisis or Church history, 
but it was a rediscovery of the vital element in Christianity, 
which the Church had lost for much more 
than 1,000 years. It was a return to a more than 
patristic antiquity---to the New Testament, and in the 
New Testament to St. Paul. It was more like a 
Regeneration than a Reformation. Its authors did not 
intend more than a Reformation of the Church, but 
what God meant with them was a Regeneration of 
Christianity. It was upon this line that the true continuity
\marginpar{68}
of the Church should for the future succeed. 
The Reformation was not a loop line bringing us back 
to Catholicism on a higher plane. It was now to 
be the main line; and the spiritual traffic of the 
world was to be diverted from Rome on the old route 
and sent chiefly by the new track. The real spiritual 
continuity was from St. Paul by Luther, and not 
from the fathers by the schoolmen. It was not an 
ecclesiastical revolt, but a religious crisis, a spiritual 
new birth of the Church. It is not to be undone, 
it is not to be antiquated; but it is to be developed. 
Its principle is to be the vital principle of Christian 
progress and the most powerful agent on earth of the 
Kingdom of God. Our Christian business to-day is 
to complete for the Church that which was given in 
principle in the creative moment of the Reformation. 
We have to disentangle it from the relics of Catholicism 
which it inherited, but which are really incompatible 
with its principle. And we have to work it 
clear of the confusions and alloys that clung to its 
first stage from the state of culture, politics, and 
society on which it emerged. We have to insist that 
there is but one object of faith, which is not the 
Church, and not truth, but the cross of Christ; one 
mediator between God and man, and one confessor, 
the priest Christ Jesus; one seat of revelation, which 
is the Bible; and one principle of revelation, which is 
\marginpar{69}
the Gospel. We have to go back to the Bible and 
interpret it by its own inner light of the Gospel, and 
not by the Church. It is the Bible that interprets the 
Church, however the Church may expound the Bible. 

I need not say that this third view of the Reformation 
is ours. Christ should be master in His own 
house. The government of the Church should be in 
the hands of the Church---not of a dead and gone 
Church, but of a living Church, as Rome truly says. 
But, as Rome does not say, it should be in the hands 
of faith, not of a priesthood---of a believing people. 
We do not disown the Church, but we reduce the 
Church to its proper place for the Kingdom as prescribed 
by the nature of the Gospel. Its authority is 
merely the authority of a witness, not of a judge; or 
an expert, not of a despot. We would carry the rejection 
of the Pope onward to the rejection of his 
two assessors, the Priest and the Emperor. We 
would sweep from the headship of the Church both 
the priest and the State. We are evangelical; we 
find Christianity not in the Church but in the Gospel. 
We are Churchmen; and we find in the Gospel alone 
the true charter and freedom of the Church. We 
are evangelical Free Churchmen. If we follow the 
Reformers by going to the Bible before the Church, 
we have no room for the priest because the New 
Testament has none. And if we go to the Gospel 
\marginpar{70}
even before the Bible, we have also no room for the 
priest because the whole spiritual world is preoccupied 
and filled by the sole priesthood of Christ. If we 
go to the Gospel, which is the grand Reformation 
principle, we go to that which created both Church 
and Bible, and we have the secret of both. We live 
by faith in the grace of the cross and not of the mass. 
And we interpret all sacraments and give them their 
place by the Sacrament of the Word. 

The real nature of the struggle to-day is a battle 
for the New Testament quality of English Christianity. 
What has to be done is to save the Church 
\textit{from} the Church \textit{for} the Bible and so for Christ. 
The Church has to be saved from its medieval self and 
from its patristic self for its New Testament self. 
We perpetuate the Reformation as the grand and 
crucial movement by which Christianity was saved 
\textit{for religion}; and saved from mere culture, which is 
pagan, and from the priest, who is a Jew in soul. I 
do not say that that is a work left solely to us. That 
would be impertinent. It is partly ours because we 
are part of the Church and cannot see any section of 
it hampered with indifference, or being released without 
sympathy. It is partly ours also because the Parliament 
which at present controls the Established 
Church is as much ours as theirs. In so far as we 
are represented in Parliament we are masters of the 
\marginpar{71}
ecclesiastical situation and responsible for it. But it 
is a work, it is a struggle, that is going on within the 
Established Church itself. So long as the finality of 
Scripture is held and fairly applied to the situation we 
need not fear the issue. And there are whole schools 
in the Established Church, like the great Cambridge 
school of New Testament scholarship, determined 
that the historic and scientific interpretation of Scripture 
shall be carried through at any cost to ecclesiastical tradition, seeing that the Bible has more to say 
to the Church than the Church to the Bible, and 
that the Bible can explain the Church as the Church 
can never explain the Bible. 


\chapter{Where do we Really Go When we Go Behind the Reformation?}
\markboth{Rome, Reform, and Reaction}{What is Behind the Reformation?}

% CONTINUE HERE

\textsc{We} are invited by the Catholic party in the Anglican 
Church now to abate our arid ardours about the 
Reformation, to leave the dreary negations of Protes-TK
tantism, to abandon its hard, inhuman, and immoral 
theology, to turn from its dogmatic contentions and 
sterilities, to escape from its bare, cold, and irreverent 
ritual, to treat it as a movement that has long ago 
done its work, if ever it had any work to do worth the 
convulsion it caused; and we are bidden to go behind 
it for our new point of departure, and to start afresh 
from the beautiful and glorious, true and tender 
medieval Church, with certain modern adaptations 
and new social sympathies. A reformed Catholicism 
isMKwhat the time needs and the Spirit prescribes to the 
Churches, a mere reformation on the lines of Trent, a 
readjustment of the Church, and not a regeneration 
of the Church's soul, or a fundamental change in the 

75 



J6 ROME, REFORM, AND REACTION 

religious idea, or in the nature of Christianity, and so 
of faith. Catholicism and Christianity, we are told, 
are identical; and Catholicism is the true principle 
of progress. 

I venture to accept the invitation so far as to 
examine that identity. I will go behind the Refor-TK
mation. I will go to Catholicism. I will ask about 
its history and especially its origin, and with the best 
help that recent scholarship can afford us, I will 
inquire whether it is identical, or even coeval, with 
Christianity; and I hope to point out that it is 
neither identical nor cognate with Christianity, that 
it is due to an intrusion upon Christianity of the 
world, of the natural man and pagan culture. I hope 
to show that if we go behind the Reformation we 
cannot stop till we are landed inside the first century, 
inside the New Testament, inside the Gospels and 
the Epistles, inside the Cross, as St. Paul understood 
it, and faith as its direct and priestless answer. I 
think the greatest of all commentaries on the Gospels 
is the Epistles, that St. Peter and St. Paul are the 
interpreters, as they were the instructors, of Mark and 
Luke, and that the disposition to take the Gospels 
without the Epistles is one of those many tendencies 
in Protestantism which are in their genius Catholic, 
and make for Catholicism, and prepare for it in the 
public mind a congenial soil. There is a spiritual 



WHAT IS BEHIND THE REFORMATION? 77 

connection, subtle but powerful, between the Catholic 
movement and the movement which isolates the 
Gospels in the New Testament, and detaches their 
Christ from the Christ of Paul. The Church will 
be Catholic or Evangelical according as we dismiss 
Paul from his primacy among the apostles or keep 
him there. It is a conflict not between Paul and 
Peter, but between the New Testament Paul and 
Peter on the one side, and the ecclesiastical Peter on 
the other. If we go behind the Reformation, there 
is no stopping till we stop in Christ as interpreted by 
the faith of Paul. 

The Reformation was not a new religion, but the 
rediscovery of the old. Therefore it did not break 
with the first like Christianity, but it went back on it, 
only farther back than Rome did. At the time of 
the Reformation there was a general consent, wher-TK
ever its effect was owned at all, to go back upon the 
previous course of the Church, and seek its correction 
at some decisive era in the history of its past. There 
was a point, it was held, where the true course of the 
Spirit had been left, and the voice of the true Pilot 
had ceased to speak with commanding power. The 
Reformation Churches which remained Episcopal 
(especially the English) rejected the line of the popes, 
and took the older, higher line of the councils. Going 
back along this line, the English Church in the main 



78 ROME, REFORM, AND REACTION 

fixed on the first six councils of the Church, covering ' 
the age of the fathers, the first four centuries. It 
found the authority for all the subsequent Church to 
lie with these councils, their canons and creeds, and 
with the Bible. But the continental Reformers went 
farther back---they went back to the first century, 
and to the New Testament alone. As Paul over-TK
leaped the later centuries of historic Judaism which had 
created the institutional part of the Old Testament, 
and went back to its origins in faithful Abraham and 
in God's promise there, so Luther passed over fifteen 
centuries of Catholicism, and took his stand upon Paul, 
and on the cross of Christ as it was interpreted by 
Paul to the direct and living faith ot the sinful, grate-TK
ful soul redeemed. The Christianity of the Reforma-TK
tion is the real Church of antiquity. 

But to-day, while this process has been continued, 
there have been two great changes introduced into it, 
and it is the renewed study of the Bible, the Protes-TK
tant treatment of it, that has caused both. It has 
been found, first, that the New Testament itself 
embodies various points of view; it is not absolutely 
homogeneous in respect of doctrine, and it carries in 
it some views which were more true for the circum-TK
stances of the first century than for those of the nine-TK
teenth. So the Protestant movement has been forced 
down upon a small section even of the first century 



WHAT IS BEHIND THE REFORMATION? 79 

for its object and standard of faith---upon the few years 
covering the life, and especially the public life, and 
work of Christ. " Back to Christ " has been the cry. 
And we cannot stop there. We have not a biography 
of Christ in the modern sense to admit us to the 
very centre of His character and motive. And Christ 
shared on some points the views of His contem-TK
poraries; without prejudice to His saving work. He 
may be held to have claimed no final authority on 
matters of scientific knowledge---say, of the origin of 
the Old Testament, or the causes of disease. Con-TK
sequently within Christ Himself the final authority 
is located less in His teaching than in His person and 
work---especially His work on the cross. From the 
Gospel about Christ we penetrate to Christ as the 
Gospel, as the grace of God in action, as the living 
grace of God, the acting, dying, rising, redeeming, 
reconciling, effectual, conquering grace of God. The 
standard and authority is the Gospel m Christ---the 
cross. And rrom the first century our classic and 
commanding time is narrowed down almost to a point, 
but an infinite point---like a man in the universe---
and all is staked and focussed on the cross. The 
Gospel takes the place as our standard which used 
to be taken by the Bible. That is the change for 
Protestantism; the authority of the Gospel is the 
standard of authority for the Bible. We do not ask 



80 ROME, REFORM, AND REACTION 

if a truth is in the Bible, but in the Gospel, in 
Christ's person and work. 

But there is another change which here concerns 
us more. It presses upon those who canonise, not the 
New Testament, but the first four centuries ot the 
Church. It is caused by the discovery that the 
difference between the Church even of the second 
century and that of the first is an immense one, and 
one which grows as scientific and impartial research 
goes on. At the end of the apostolic age the history or 
the Church, it has been said, enters a tunnel for about 
a generation. We have almost no data for it. And 
when it comes out the Church has undergone a 
change whose tremendous importance is even now 
the largest influence on the Church in the world. It 
has begun to be the Catholic Church, and it has 
ceased in some essential points to be the Church of 
the New Testament. The train has crossed some 
frontier, and the guards, drivers, uniforms, even 
engines, are changed. That which was destroyed by 
the Reformation was this Catholicism by which the 
second century swamped the first. It was not a 
system which had broken away from the first four 
centuries, but one which broke away from the first 
century, from the New Testament. There are two 
senses of the word Church in the New Testament, 
either the Church local or the Church ideal. The 



WHAT IS BEHIND THE REFORMATION? 8 1 

third sense, or the Catholic Church, the huge reh'gious 
state, is not in the New Testament at all. That is 
to say, that conception of the Church on which they 
stand who unchurch the Free Churches does not exist 
in the New Testament. It is not earlier than the 
second century. It is Catholicism. It was Catho-TK
licism that the Reformation broke. Catholicism was the 
perversion which settled down on the Church in the 
second century and identified itself with the first. 
Many of us have been taught to regard the Church's 
first great apostasy from the New Testament ideal as 
connected with Constantine and his patronage or 
Christianity as the State religion about 320 a.d. But 
we must go farther back to seek the real apostasy, 
the deflection of faith, which prepared the Church for 
succumbing to the patronage of the empire. It was in 
the second century, when the Church became mastered 
by the imperial idea, when it aspired to be one uni-TK
versal Church, a grand hierarchy, a spiritual imperium 
in imperio. It was then that the Catholic Church 
arose, as distinct from the Church of the New Testa-TK
ment. In the New Testament Church the unity was 
wholly unseen, and its invisible reality acted through 
a visible multitude of independent communities. 

Catholicism, then, be it clearly understood, arose 
in the second century, and it was historically bound 
to arise, I admit. It arose out of historic necessities, 

6 



82 ROME, REFORM, AND REACTION 

which we can partly but poorly trace. The chief of 
these necessities was the rise of Gnosticism, of which 
we see the first beginnings in the Epistle to the 
Colossians and the Epistles of John. This was the 
Christian rationalism of the first Church. But it was 
a mystical rationalism, with great spiritual pretensions. 
It was like the theosophies we hear of in our own 
time. It was a vague combination of mystical vision 
with the speculative science of the day. With masses 
of people it was very plausible. The popular Christian 
mind of that day was as ignorant and gullible as west-TK
end women are with " Christian science "; and the 
movement threatened to run away with the whole of 
Christianity. If it had, Christianity would, humanly 
speaking, have been as surely destroyed as it would 
have been earlier if Paul had not risen to save it from 
Judaism. So now it had to be saved from paganism 
in this insidious form. How? By organization. 
That was the power which an imperial age best 
understood. The salvation was effected by organiz-TK
ing the Church into a rigid unity, with the bishops 
at its head. You might call this in a way a Refor-TK
mation. It saved the Church from a pagan corruption, 
as Luther's did. But its means were very different. 
This first Reformation was by the bishop, the second 
was by the Gospel. The first was by machinery, 
the second was by faith. The Catholic Church is 



WHAT IS BEHIND THE REFORMATION? 83 

machine-made, the Evangelical is soul-made, made by 
faith. The Catholic Church is a work of skill, the 
Evangelical is a work of genius. The Church was 
saved from the Gnostics by becoming an institution; 
it was saved a millennium and a half later from the 
Catholics by becoming an inspiration. 

In this episcopal reformation of the second century 
the old congregational system of the New Testament 
was left behind. The public need forced the local 
Churches ifirst into great provincial Churches with 
a bishop, and these again into a great universal or 
Catholic Church, with a bishop of bishops at the head 
of all. Episcopacy began what the papacy completed. 
I noticed in the Pan-Anglican Synod of a year or two 
ago a proposal (which was not allowed to go very far) 
that the Archbishop of Canterbury should be invested 
with a certain primacy over all the bishops of the 
Episcopal Church in England or in America. This 
was, of course, resisted by the American bishops. 
But it was an exact repetition of what took place in 
the beginning of the papacy in the third century. It 
was English imperialism trying to force itself on the 
Church, just as Roman imperialism did in its time. 
And the same appetite for rule would force itself 
upon us of the Free Churches but for the men of 
controversy and vigilance. 

Now I would repeat that Catholicism, with the 



84 ROME, REFORM, AND REACTION 

supremacy of the monarchical bishop, may have been a 
historical necessity, and may have saved the Church 
at a great crisis. Organization has that use to a 
certain degree. There are junctures that call for 
centralization and even for a dictator. But the peril 
is that the dictator may stay on as emperor, that the 
protector may prolong himself into a dynasty, that 
the ally invited in to repel an enemy may remain as 
a conqueror. And this is what happened with the 
bishop. He soon became not only useful, but indis-TK
pensable, permanent. He developed a theory of 
himself. He discovered that he was involved in the 
absolute and eternal constitution of the Church. He 
found, and declared, that there could be no Church 
and no salvation without a bishop. The Church as 
organized became: canonized. It was no longer true 
that Christ was where two or three were gathered in 
His name, but only where there was a legitimate 
bishop. Where the bishop was, and there alone, was 
a Church. 

There is a remarkable parallel to all this in Old 
Testament history. Israel, in its most spiritual time, 
was captured by a Catholicism of centralization, to 
which in the end it succumbed. No sooner had pro-TK
phetism taken its noblest and holiest form in Jeremiah 
than it was seized by a close religion of the priest and 
temple. The Judaism v/hich took the place of the old 



WHAT IS BEHIND THE REFORMATION? 85 

prophetic Hebraism, and grew into the damning right-TK
eousness of scribe and Pharisee, arose in the priestly 
centralization of all worship at Jerusalem, which won 
the assent even of the prophets of the time. This was 
a measure which took effect in connection with the 
discovery of the book of Deuteronomy in the temple 
in the reign of King Josiah, about 620 B.C. It was, 
perhaps, a necessary step in the circumstances. The 
local shrines of Jehovah, in the midst of a rude and 
half-pagan population, were scenes of abuse which 
threatened to swamp in heathen syncretism the purer 
faith. But it was a policy of emergency which, 
through the anarchy of the exile shortly after, became 
normal and perpetual, like a war-tax which lives on as 
the main element in a peace revenue, or like a presi-TK
dent or consul who comes out of a convulsion an 
emperor. It was a step with the gravest consequences. 
It developed into a policy in which the free, prophetic, 
nonconformist spirit died, and all life came to be 
organized by a hierarchy into the most minute and 
terrible ritual of conduct ever seen. It was the 
beginning of the policy which turned the State into a 
Church, and into a Church which committed the crime 
of the world. That was Hebrew Catholicism, the 
sacerdotalised community of Israel. And that is what 
Christian Catholicism would do if it had its way, and 
did do so far as it had. Many of its advocates say it 



86 ROME, REFORM, AND REACTION 

would not, and they are honest enough. But there is 
a spiritual logic which over-rules individual ideas in 
such matters. And the end I indicate is the irresis-TK
tible conclusion to which Catholic principles work out 
on the field of history. Hebrew Catholicism, in spite 
of many profound ideas and symbols, yet killed the 
religion of Israel, and made a new religion necessary. 
Christianity was a new religion, and not a develop-TK
ment of the old. It was the child of the old, and not 
its manhood. And what that terrible Judaism was to 
the spirit of prophecy Catholicism is to Christianity. 
And the Reformation was not a new religion only 
because it was the rediscovery of the old, the earliest 
Christianity, which Catholicism so early as the second 
century had lost. 

And there is this further analogy. The law books 
born in the Judaic age, like Leviticus, carried back 
the whole organization of the worship to the time of 
Moses, and were themselves believed to belong to that 
time and that author. The new cultus imported 
itself into the original institution of the religion, and 
identified itself with it. It was in like fashion that 
episcopacy and the priesthood, through mere historic 
growths, referred themselves very early to the original 
foundation of Christ. They imported into New 
Testament words which had no such meaning the 
monarchical bishop, the apostolic succession, and the 



WHAT IS BEHIND THE REFORMATION? 87 

priestly prerogative of sacrifice and absolution. It is 
not strange that the Catholic system should make great 
use of the Old Testament for the Christian Church. 
And it was only working the same vein in a richer 
lode in its employment of the spurious Ignatian letters 
and the foro;ed donation of Constantine. 

This tremendous change from Scriptural to 
Catholic Christianity took place with amazing rapidity. 
We are surprised. But we must not forget two 
things. First, the age. It was an age when the 
Roman, organized, imperial idea of society was un-TK
contested. Our modern democratic ideas did not 
as yet exist. The monarch stood for the multi-TK
tude. Peoples followed the religion of their rulers. 
There was no communal right. The people, in our 
modern sense, did not exist. But, it is said, there was 
the model of the first Churches, the New Testament 
Churches, which have made our modern Free Churches. 
Yes, but that suggests the next fact which must not 
be forgotten. The memory of the first century was 
preserved chiefly in the unwritten tradition of those 
very officers whose position was to be enhanced. 
The NewMK Testament had then hardly an existence. 
Its various books were there among others, and were 
read, but they were not authoritative; there was 
no canon. That selection was not yet made. The 
canon and place of Scripture was one of the great gifts, 



88 ROME, REFORM, AND REACTION 

probably the greatest gift, of Catholicism to the future. 
But as yet the canon was not there to keep Catho-TK
licism in its place. It was put there by Catholicism 
for its own support, as one of the greatest 
engines with which to fight the Gnostic heresy, and 
set up a standard against its speculative extrava-TK
gance. 

Catholicism with the bishop, then, arose to meet a 
historic need. It was created not by divine fiat, but 
by a historic necessity. It is not there with an abso-TK
lute and sole divine right, as unique as Christ or the 
Church. But having come into existence, it went on 
to ascribe itself to divine fiat. It traced itself to the 
Apostles. It claimed to carry down what could not 
be carried down---the unique privilege of Apostles 
who had seen the Lord. It gave itself an absolute 
right in the Church, instead of a relative and historic. 
It was good for the situation, but it became megalo-TK
maniac and said it was essential for ever. And in 
the comparative absence of a canon of scripture, it 
had little difficulty in making this claim good. I 
cannot stop to trace this process, but it can be traced. 
It persuaded the whole Church that without it there 
was no Church, therefore no salvation. To reject 
the bishop was to reject Christ and court perdition. 
The great apostle of this false gospel was Cyprian, in 
the third century, v/ho was for the early Church what 



WHAT IS BEHIND THE REFORMATION? 89 

Laud was for the Anglican. Laud, indeed, is called 
by his biographer, Heylin, Cyprianus Angl'icanus. 

Now let it be well understood that there is this huge 
gulf between the Church of the second century and 
that of the first. It is a truth which never was so 
clearly realised till the critical scholarship of this cen-TK
tury dealt with the records both in and out of the 
Bible. It is a truth that is bound to make its way in 
a learned Church. It is making its way, MKnd it is 
having its effect. It is an immense help to every con-TK
tention of ours; for it carries our principles home, 
with all the weight of ecclesiastical and historic 
science, to circles that could not be expected to have 
anything to do with a Congregational Union, or a 
Liberation Society, or even a Protestant Alliance. 

What received its death-blow at the Reformation 
was not simply the papacy, but Catholicism. We do 
not need to. regard even Roman Episcopacy as anti-TK
christ. We need only regard it as having become an 
anachronism. Episcopacy may have been historically 
necessary, as in circumstances it may be preferable to-TK
day. It does not become Catholicism till it claim to be 
sole. Catholicism is monopolist Episcopacy. It is an 
old historic necessity, which has forgotten its place 
and outstayed its time. It was a guest of the Church 
with no more taste than to linger on when the 
household groaned under its presence. It was wel-TK



90 ROME, REFORM, AND REACTION 

come when it came, and it might have been willingly 
retained in a permanent position in the household of 
faith, if it had not taken control of the establishment, 
and forbidden the other branches of the family to 
cross the door. What the Reformation did was not 
even to turn Catholicism out of the house, but to 
teach it its place in the house as one of many, and 
certainly not the firstborn. And the Reformation did 
this by restoring the original New Testament idea of 
faith, by restoring faith to its creative place in the 
government of the Church. It was the first time 
that the New Testament had been seriously and 
directly dealt with since it had become the canon. It 
was the first time it had assumed its true power and 
place. Up till then it had been entirely in the hands 
of the Catholicism that constructed the canon for its 
own purpose, and 'therefore interpreted it in its own 
sense. Now the Spirit, the gospel, took the New 
Testament out of the hands of the Church as an 
institution, and gave it into the hands of faith. It 
took the New Testament, covered with dust, out of 
the bishop's chancellory and the priest's breviary, and 
laid it open on the believer's table. 

I say the Reformation put an end to Catholicism 
by restoring the New Testament idea of faith. It 
not only took us back from the fathers of the first 
four centuries to the first century, but it forced us to 



WHAT IS BEHIND THE REFORMATION? 91 

recognise that the change from the first century to the 
second was more than a change in the ChurcKs constitution. 
It zuas a change in the Church'*s FAITH---which is a 
much more serious thing. The Church, in entering 
the social conflict of the early centuries, had lost the 
purity of its first idea. It became wide and popular, 
but at the cost of its truth and purity. It conquered 
the world by becoming more or less worldly. It 
gained the world, but it lost in its own soul. It be-TK
came an empire, but it ceased, in proportion, to be 
a communion. Becoming Catholic, it ceased, in a 
measure, to be holy and apostolic. It was secularised 
in the effort to capture the age for Christ. It was 
seduced by the world it set out to reclaim. It won 
power, but it lost in faith. A change destined to be 
fatal passed over its idea of faith, which became sacra-TK
mental, priestly, episcopal, institutional, instead of 
ethical, spiritual, and evangelical. The gospel became 
a new law; and virtue became a thing of order, 
instead of a thing of the new conscience. 

That is why I say that the present struggle is a 
struggle for the purity and permanence of Christian 
faith. It is not for reverence, or for Christian piety, or 
for Christian philanthropy; these exist as richly among 
Catholics, especially Anglo-Catholics, as among our-TK
selves. But it is for the true, original, and permanent 
nature of Christian faith, for faith's future, for the 



92 ROME, REFORM, AND REACTION 

future of Christianity. For a religion is just as its 
faith; and Christianity can only be the religion of the 
future if it retain the original idea of faith as its motive 
power and working capital. Our breach with the 
absolute and sacerdotal Episcopalians is not one which 
it can do any good to gloze over and fine down to a 
mere verbal or historic affair. The conflict will be 
much clearer, shorter, and more fruitful if we let the 
issue be as distinct as we can make it. And it is by 
way of doing so that I say, with all who are disposed 
to be thorough in this matter, that the change from 
the Christian brotherhoods and communities of the first 
century to the Christian Church of the second was 
one not merely of order, but of faith. The difference 
between the Free Churches and the Catholics, Angli-TK
can or Roman, is one of faith, and not of Church 
order merely. It is more religious than ecclesiastical. 
And whoever passes from the one to the other does 
not simply adopt another Church polity, but another, 
and in one case a less Christian, form of faith, another 
gospel, as Paul said in a similar juncture, which 
indeed is not another, because in the strict sense it is 
not a gospel. It is a return of mere law, in which 
distinctive gospel is practically lost. 

Is it not clear that it must be so? The mystery 
and the power of Christianity is faith---understood not 
merely as a religious sympathy or affection, but as 



WHAT IS BEHIND TPIE REFORMATION? 93 

direct, personal communion with Christ, based on for-TK
giveness of sins direct from Him to the conscience. 
It is not bound up absolutely with any external ordin-TK
ances or institutions. These are but functional to 
faith, not organic; historic, but not essential or 
eternal. Believe in the Lord Jesus Christ, and thou 
shalt be saved---not in the Church, not in the sacra-TK
ments, not in the priesthood. All these have their 
great worth as exhibitions and energies of the Church, 
not as conditions between Christ and the Lord. They 
are not objects of faith. But, in the change which 
made Catholicism, this communion with Christ is made 
to depend absolutely on external forms and conditions. 
That is the essence of Catholicism in one word. It 
is fixed and consecrated institutionalism, whether epis-TK
copal or sacerdotal. That is a more deep and serious 
perversion of the Church than its connection with the 
State. One of the chief reasons why it is wished to 
set the Church free from the State is this---that the 
Church of all faithful men may be more free to deal 
with that Catholic corruption which only Christian 
freedom can deal with, only Christian directness, 
originality, and vitality of evangelical faith. When 
the Church is free from the political institution, it will 
still have to deal with the clerical institution. Be-TK
cause, when a Church is in the first place institutional, 
it is only in the second place evangelical; and a 



94 ROME, REFORM, AND REACTION 

Church is in the first place institutional which refuses 
an equal recognition to any Church which is equally 
evangelical. The Church of England is to-day more 
of an institution than of a gospel. It idolises an order 
and an office, at the cost of Christian truth, faith, and 
love. " By taste are ye saved." The Church of an 
evangelical faith will have to deal with that idolatry. 
Faith will have to make the Church sit much more 
loosely to each form of polity, and especially to destroy 
the historic fallacy and delusion that episcopacy is original 
or essential to a true Church and a true faith. Such a 
belief is fatal to Christian faith at last. It is a departure 
which touches the very marrow and nature of faith as 
due to the person and work of Christ alone. I urge 
you anew to be very clear as to what the real nature of 
Catholicism is, and its incompatibility with the faith of 
the New Testament, and especially of Paul. Do not 
waste on the Pope, who is a remote danger for us, the 
breath that should be used to cool or extinguish the 
Catholic claim, which is our near foe. The enemy is 
not an institution, but a spirit---institutionalism. 
Catholicism---let it be clear---is not the use of an in-TK
stitutional Church. For then the Presbyterian would 
fall under the ban, the Methodist, and indeed every 
single church of our own order which is well and 
permanently organized. Catholicism is the idolatry of 
a particular form of institution, and its monopoly for 



WHAT IS BEHIND THE REFORMATION? 95 

salvation. It is making it an object of faith. If any 
man says, as the second century did say, that member-TK
ship of an outward organization represented by bishops 
(or we might say presbyters if they made the claim) is 
essential to a man's being a Christian; if he identify 
that episcopal organization, from Christ's institution 
downwards, with the Kingdom of God; if he thus 
rest the Church upon an office or an order, instead of 
the office on the Church---then our difference with 
him is not a difference of opinion, but a difference of 
faith. He has damaged the Christian faith. He has 
thrust between the soul and Christ an institution which 
neither Christ nor His gospel puts there. He has be-TK
come a member of the Church, more than a member 
of Christ. If a man say that only the bishop and his 
nominees are the teachers and guides appointed by 
God for His Church; that there is no Church where 
there is no bishop; that only subordination to the 
bishop gives communion with Christ in any regular 
way; that the rest may see their Saviour occasionally, 
but cannot have the indwelling spirit---if he say that, 
is it not clear that between him and us there is a great 
gulf fixed by faith itself? And we can never rest---
the Holy Spirit forbids us to rest---till that perverted 
faith is restored to the living way. This battle is 
going to end in a great clearing and uplifting of faith 
for all the Churches. We have to insist that this 



g6 ROME, REFORM, AND REACTION 

Catholic perversion is choking the Gospel with a 
specious form of the law which the Gospel came to 
set us free from. I do not say the Gospel destroyed 
law, or institutions which are in the name of law and 
order. The Gospel did not come to rid us of law, but 
to give us freedom in connection with it, to make it 
our servant when it had been our master. And so it 
enables us to use Church institutions and keep them in 
their place. But this institutionalism, which makes an 
official or an office part of the Gospel, is simply bring-TK
ing in at the window the reign of law which St. Paul, 
in Christ's name, turned out at the door. It is restor-TK
ing to the temple the business which Christ whipped 
out. It is destroying what made faith a real gospel 
and release. It is surrendering the principle of re-TK
demption. I speak of the system. I do not say it so 
appears to those who hold it. But I must say that 
the Gospel is in principle given up to the world where 
the Catholic claim is made for either bishop or priest, 
or both. Priest and bishop! We cannot fail to see 
that scientific history would enable us to deal far more 
easily with the episcopal monopoly of the Church if 
the bishop had not become merged in the priest. The 
trouble to-day is that the bishops who rule the Church 
are too much priests; and they are said by some within 
the Episcopal Church itself to appoint to the training 
colleges of the clergy heads more sacerdotal than them-TK



WHAT IS BEHIND THE REFORMATION? 97 

selves, and more narrow because more academic, more 
secluded from public life and criticism. It is the 
priestly and not the administrative functions of the 
bishop that explain the tenacity of Episcopacy. It is 
his relation to the sacraments that is the point of con-TK
flict more than his relation to polity. I have no 
objection to Episcopacy as a good polity among others 
equally good. In the Contemporary Review for last 
August (1898) you may read the scientific evidence 
for what I have been saying about the post-apostolic 
origin of the episcopate. It is an article written con-TK
jointly by two of the soundest scholars---one of them 
an Oxford clergyman, and one a tutor of Mansfield 
College. It is hard to think that there would be a 
resistance to such scientific proof were the question not 
confused by prepossessions about sacerdotal grace which 
make it a religious instead of a historical question. 
Very early---in the third century----the two streams 
met and fused, the episcopal and the sacerdotal. The 
priest and the bishop were rolled into one. The 
process can be traced. The presiding presbyter be-TK
came the sole minister of each Church. Then the 
Church came to rest on the bishop, instead of the 
bishop on the Church. The bishop was supposed to 
stand exactly in the place of an apostle, and the 
Church was said to rest on the foundation of the 
apostles. (They forgot the addition of " prophets "---

7 



98 ROME, REFORM, AND REACTION 

the preachers who stood alongside of the apostles in 
the first Church.) In virtue of this succession the 
bishop possessed an infallibility in Christian truth 
which was miraculously transmitted in his appoint-TK
ment. Moreover, as a matter of order, no sacraments 
were valid or effectual unless administered by the 
bishop of the Church or by his agents. Meanwhile, 
mystical and even magical value was ascribed to the 
sacraments and to those who dispensed them. The 
bishop not only stood for the Church, but for Christ 
in His sacrificial power. The infallibility of the 
bishop was augmented by the miraculous gift of the 
priest, and the same personMK stood for both. 

In Anglican Catholicism the infallibility of the 
bishop has virtually been dropped. I doubt if there 
are many who would now stand even for the doctrinal 
infallibility of the bishops of the first four centuries. 
They are regarded (as Canon Gore says) as " focus-TK
sing" rather than creating the faith of the Church. 
The line of episcopal infallibility was retained only 
in the Roman branch of Catholicism, and it found 
its logical conclusion in the Vatican dogma of the 
Pope's infallibility in 1870. But the sacerdotal 
miracle in the bishop as chief cleric has been retained 
by the Anglican Church, and is to-day the real nerve 
of the episcopal prerogative in those who take it most 
religiously. The Englishman cares less for truth 



WHAT IS BEHIND THE REFORMATION? 99 

than for action. So he could dispense with a bishop 
who had the gift of miraculous truth, but he kept 
a bishop who had the power of miraculous action. 
The apostolic legacy was not for him the power of 
preaching Gospel truth so as to rouse faith, but the 
power of doing priestly acts so as to mediate God. 
The ministry becomes not an office for the sake of 
order, but it becomes an order and a sacrament. 

It is this false apostolicity that we have to resist, 
and we have to resist it on the great reformation 
principle, which was twofold---a subjective appeal to 
Christian experience of sin and its forgiveness, and an 
objective appeal to the Church of the first century as 
represented by the New Testament. We can win 
this battle only by a revival of faith, by more religion 
and more Bible. I do not think we can fight this 
battle, as we are sometimes told to do, by going out to 
the public and doing more good than our opponents. 
Doing good is only understood by the public in the 
sense of philanthropy, and not in the evangelical, 
spiritual sense which alone tells in this issue. If it is 
a question of practising more philanthropy than our 
adversaries, we Congregational ists at least may give up 
the battle. We have not the organization, the wealth, 
the devotees. No minister with his voluntary staff 
can cope in this respect with the priest and his staff 
of curates (" My curate is an admirable man. He 



100 ROME, REFORM, AND REACTION 

is running about the parish the whole day ") 
and the leisured men and women to whom their 
creed and Church is a ruling and ascetic passion. 
But no Church question should really be settled by 
an appeal to philanthropy, or to the public. It is a 
piece of cant to say, as the man in the street does, 
that the priest cannot be very dangerous, because he 
is a man of such unselfish devotion, and does so much 
good among the poor. Men often admire the de-TK
votedness of others because it saves themselves trouble. 
Devotedness is not the test of truth, or else the Jesuits 
would be the true clergy. Nor is philanthropy the 
test, else the apostolic succession must run through 
St. Francis, St. Vincent, Howard, Fry, Miiller, 
and Shaftesbury. But the main point is that the 
priest himself, to do him justice, would never consent 
to rest his case upon the zeal or philanthropy of de-TK
voted priests. He unites with us in taking much 
higher ground. He appeals to the Divine order, the 
Divine will, the Divine commission, the nature of 
Christian religion. And, like us, he does not make 
his appeal to the public, but to the faithful---the really 
religious, to those who care for the will of God, and 
can be made to own it. He chafes under his parlia-TK
mentary Church. He demands that the Church rule 
the Church, that Christ be master in His own house. 
That is the principle whose defence has cost ourselves 



WHAT IS BEHIND THE REFORMATION? lOI 

SO much for so long. He objects to have the unbe-TK
liever settling Church affairs like belief, worship, 
ministry. It is that principle of the Free Churches 
that is wMKorking Disestablishment from MMKithin. Let 
us never cease preaching the principle. Let us wel-TK
come it whether it be preached by priests, friars, or 
liberationists. They are all liberationists when it 
comes to that. They believe in the Church's spiri-TK
tual freedom. And they perceive it can only come 
by Disestablishment. Set even the priest free from 
the control of the State, and from its social support; 
free to be dealt with by the Gospel, by the believer. 
If he is to be taught his place it must be by 
spiritual men and means. A spirit mightier than his 
own must eject him. It is not to the public, the 
voter, that we must look for the victory of our cause, 
but to the believer, to those who care for the will and 
word of God, to the men of faith, to those who seek 
the principle of this matter in the true nature of the 
Church, and find the true nature of the Church within 
the Bible, in the Gospel. If our cause be weak it is 
so because we are reading everything but "the Bible. 
One of the preacher's great difficulties is in dealing with 
people who are checking him not by their Bibles, but 
by their feelings, or their personal preferences. But 
the reading of the Bible is not enough. It is the 
study of the Bible that we fail in. And it is that 



102 ROME, REFORM, AND REACTION 

failure that leaves us so exposed to the ecclesiastical 
perversions and plausibilities of the hour. Is there 
that deep gulf between the Catholicism which had 
captured the Church in the second century, and the 
Christian brotherhood of faith and love which was 
the Church of the first century? You must go to your 
New Testament and see. You resent the priest in 
the name of your individual freedom. That is not 
enough. Do you do it in the name of a universal 
priesthood, which has sunk your individual freedom in 
obedience to the sacrifice of Christ? That is the 
point. You resist to the utmost the confessional and 
the intrusion of the priest into your family, between 
you and your children, you and your wife. Yes; but 
why do you so resent it? Is it simply as a sturdy, 
honest, British home-lover? or is it as a man who 
chiefly obeys the Christian principle, both of marriage 
and of the ministry? Because if the New Testa-TK
ment, if the Gospel of Christ, said that the priest had 
a Divine right in the bosom of your family, there is 
no other right strong enough to bar him out. Cer-TK
tainly the burly Briton with his house as his castle 
could not. You admit the minister to perform your 
marriage. Why? Why do you suffer the Church 
to step in and say that you shall not follow the im-TK
pulse of your mutual and private affections without 
her consent and blessing? Because you believe that 



WHAT IS BEHIND THE REFORMATION? 103 

marriage is a Christian institution. And the ground 
of that is with Christ in the New Testament. It is 
not in rules or canons of the Church. So, in the same 
way, if the New Testament said or implied that the 
priest had a right in Christ's name to set up his con-TK
fessional between you and your wife, no natural resist-TK
ance of yours would have any right. A spiritual right 
takes precedence of a natural. It would mean no 
more than the mere recalcitrance of those poor defiant 
egoists---male and female---who say that neither Church 
nor State has any right to meddle with their private 
affections, and who, therefore, live together without 
the seal of either. But how do you know that the 
minister of Christ has not this right? It can only be 
from a knowledge of the New Testament. It is your 
duty to be certain that that conception of Christ's 
minister is not there. It is your Christian duty, be-TKMK 
fore you take any extreme line on a Christian issue, to 
know what the mind of Christ on the subject is. 
And the only source of your knowledge about it is in 
the New Testament. That is faith's court of appeal. 
I do not say the New Testament is faith's statute 
book, because the New Testament is not statutory. 
It is the court of the King's bench, the seat of a 
living Lord and Judge, and the source of a Holy 
Spirit who guides us, by personal contact and practice 
and experience, into all truth. He does not so much 



104 

give us our decisions, but He gives us power, light, and 
guidance to make our decisions. But there must be 
personal contact, personal experience, personal faith. 

I lay incessant stress on that word faith---personal 
faith. The message of the Church to the world is 
not to bid men love, but to bid them believe. The 
message has come, in the refinement of our religious 
culture, to be too much, and too expressly, a call for 
love. That is not the true evangelical note. It is 
the Catholic note---the note of the Roman saint, the 
monastic community; the note of socialist piety rather 
than of the Church's faith. It is not the Reformation 
note, nor is it a true development of the Reformation 
note. Do not preach the duty of love, but the duty 
of faith. Do not begin by telling men in God's 
name that they should love one another. That is no 
more than an amiable Gospel. And it is an impossible 
Gospel till faith give the power to love. They cannot 
do it. Tell them how God has loved them. Bid 
them as sinners trust that. Preach faith as the direct 
answer to God's love. The first answer to the love of 
God is not love, but faith. Preach faith and the love 
will grow out of it of itself. Loving, as a Gospel, is 
Catholic. The Protestant and evangelical Gospel is 
believing. Believe in Christ crucified and the love will 
come. Love must come if we believe in love. But it 
has first to be believed in before it is imitated. " We 



WHAT IS BEHIND THE REFORMATION? 105 

are not saved by the love we feel, but by the love wMKe 
trust." And what we need in our preaching to-day is 
not so much Reformation truths, nor even Reforma-TK
tion enthusiasm, but the Reformation note and order of 
faith---of faith as an evangelical, personal experience, 
faith as the peace and confidence of being redeemed 
and forgiven by the death of Christ and by nothing 
else whatever. 

We are to-day in a similar position to that which 
the Church had to face in the second century---similar, 
yet with one essential difference. We are faced by a 
nineteenth-century gnosis of science fused with imagi-TK
nation, a gnosis of savant, socialist, and poet. We are 
confronted by a modern rationalism, culture, humanism, 
mysticism, half Christian, half pagan, which takes the 
Christian truths and terms and trims them down, 
under plea of filling them out, to its own sympathies, 
ideas, aspirations, principles, and morals. The peril 
of this is felt by the Church, the true Church, in all 
branches of it. And various means are taken to avert 
the danger. The means taken by the second century 
was organization. It was to close up the ranks of the 
Church, to draw the independent Churches together, 
not into a federation, but into a huge spiritual bureau-TK
cracy, to increase the episcopate, and put more power 
into the hands of the bishops. They were placed 
in a new and unique relation to the apostles. And 



I06 ROME, REFORM, AND REACTION 

they were fortified by the addition of a priestly power 
quite different from the doctrinal infallibility which 
the apostolic connection was supposed to give. Now 
that is exactly what the' ruling movement in the An-TK
glican Church to-day is doing. They are fighting a 
real and present peril with only the means most ready 
to hand in the second century. What they do not duly 
recognise is that history does not repeat itself in this 
simple way. The very work of Catholicism itself has 
essentially changed the situation. Catholicism has 
utterly changed the situation, for one thing, by giving 
us the canon of Scripture, and through that the 
Reformation. The canon did not exist in the second 
century. There was nothing objective to fall back 
on except the tradition of the apostles' teaching, sur-TK
viving in the Churches they had founded, and re-echoed 
by the leaders, bishops, and teachers of these Churches. 
It was the state of things which Wesleyanism would 
have shown if its Churches to-day had only had the 
early Methodists, the associates of Wesley, to appeal 
to, instead of his v/ritings (to say nothing of the 
Bible). No Church can exist without an objective 
authority. And the only objective authority possible 
in the second century was the bishops, as representing 
what was believed to be the apostolic tradition. They 
were unsatisfactory representatives, but they were all 
that there was. And they did give us a real successor 



WHAT IS BEHIND THE REFORMATION? 10/ 

to the original apostolate, which superseded themselves. 
It was the New Testament. The real successor of the 
apostles is the New Testament. That is now freely in 
the hands of the Church. It is an objective which 
stands while the Church may waver with the floods 
and gales of the time, or falter with the weight of 
work or years. Our modern situation is entirely 
changed by the possession of the New Testament, 
and by an understanding of it far truer, deeper, richer 
than that possessed even by the second century. 
Now, what one misses in the Anglican movement is a 
due recognition of this fact, I say a due recognition, 
for it is not without some recognition. And by a due 
recognition I mean an appeal to the Church to go 
back solely upon personal acquaintance with the New 
Testament, as earnest as the appeal on behalf of priest 
or bishop. Catholicism had really done its work 
when it gave to the Church the New Testament. The 
great gift of Catholicism to the Church was the power 
to overcome Catholicism by the Scriptures. It ought 
then to have got slowly out of the way, had it not been 
captured and prisoned by imperial and priestly ideas. 
As it was, it was destroyed in principle as soon as 
the Bible came to its own in the Reformation, and 
living faith fell at the feet of living grace. A great 
institutional Church may be a doubtful gift to the 
world. It is anything but a real gift when it claims 



I08 ROME, REFORM, AND REACTION 

a monopoly and an idolatry due only to an essential 
part of the Gospel. But there is no doubt about the 
gift that the Catholic Church gave the world in the 
Bible. And the greatness of that gift lies in the 
Bible's power to make us forget the donor in the 
author, the Church in the Redeemer. We are grate-TK
ful to the Church till we discover that its gift is 
meant as a Trojan horse, as an engine for our 
conquest. The Bible is not there to enhance the 
Church, but both Bible and Church are there to en-TK
hance the Gospel. The Bible has more to do for the 
Church than ever the Church has done or can do for 
the Bible. And the Bible never does so much for the 
Church as it does when it puts us in a position to 
judge, condemn, and reform the whole Church by 
its light. It is only that light that can reform the 
Church. It is not the light of nature, the common 
man, the worldly parliament. Set the Church free 
from these things, to be acted on by its own Bible. 
Deliver the Church from the voter for the believer. 
What the clergy say is, deliver it from the citizen for 
the priest. What the Erastians say is, deliver it from 
the priest for the citizen. What the Free Churches 
say is, deliver it from the citizen for the believer, and 
let Christ come to His own in the living faith of 
His own. 

If we go behind the Reformation, therefore, it is 



WHAT IS BEHIND THE REFORMATION? 109 

not in the medieval Church that we can stop, nor in 
the Church of the fathers and the first four centuries 
or the first six councils; nor can we stop with the 
second century and its bishops, nor even with the end 
of the first. We are carried on by the Holy Ghost 
to the source of His own action upon the world, to 
the person of Christ, to His work on the cross, to 
direct contact with the Gospel of that grace and the 
sacrifice of that one Priest. So being justified by faith, 
we have peace with God through Jesus Christ, and 
confidence unto the end. 



WHAT DID LUTHER REALLY DO? 



Ill 

WHAT DID LUTHER REALLY DO? 

I 

If we ask what the real nature of the issue is in a 
serious crisis of the Church we must always fall back 
on its conception of faith. The contention implies 
on one side or the other some serious decay and in-TK
volves some serious reform there. If there be in the 
Church long malaiseMK resulting in acute periodic dis-TK
turbance, we may be sure that the seat of the mischief 
is in some vital part; and in the severest forms of it, 
it is in the vital centre; and the vital centre of the 
Church is faith. What is then needed is not a re-TK
vival, and it is not primarily a reformation, far less 
is it mere regulation Acts. 

The present crisis is far too serious to be dealt with 
by parliamentary regulation. It is even beyond the 
reach of episcopal reformation. The priest has 
broken loose from the bishop. He claims an authority 
above the bishop; he goes behind him, as, in a sense, 

"3 8 



114 ROME, REFORM, AND REACTION 

we do. He appeals to some authority prior to the 
bishop, which made the bishop, and which the bishop 
must obey. He goes to the Church, as, in another 
sense, we do. He accuses the bishop, for the sake of 
the State connection, of betraying the Church and the 
Catholic faith. For him the Church is virtually the 
priesthood. There, of course, we are not with him; 
but we can only welcome his appeal to the Church 
from the episcopate because our own appeal is to the 
Church, and our inquiry is, what makes it? We 
hold that the Church is made by the Gospel and its 
Word of Life held forth to the world. Catholicism 
holds that it is made by something institutional, by the 
twofold institution of the bishop and the priest. But 
priest and bishop are now in antagonism. And we 
cannot but be glad that the institutional idea of the 
Church is thus proving unworkable as its essential 
idea. It all tends to place the Church on its one 
sound base of a living, personal, evangelical faith. 

We come back always to that. Every crisis drives 
us in upon our faith. And in respect of faith what 
is needed is not a revival. We have had several 
revivals during the last century, but they are only 
forcing the greater crisis. They have not brought 
peace to the Church, or solidity of conviction, energy, 
and life. We have had the Evangelical revival, the 
Oxford High Church revival, and the Broad Church 



WHAT DID LUTHER REALLY DO? II5 

revival. And they have left us where wMKe are to-day. 
What is it all moving to? What is it that the Spirit 
is striving with man, with the Church, to bring to 
pass? A new Act of Parliament, a new episcopal 
charge, the schism of an extreme priestly section from 
the Established Church? Do mountains labour of 
mice like these? What we need, what the Spirit 
moves to, is a regeneration of the central idea of faith, 
a return to the New Testament and the Reformation 
idea of it, a development of that idea which has been 
arrested both in Episcopacy and in Protestantism---in 
the one by politics, in the other by a debased ortho-TK
doxy, or by impatient social programmes.MK What is 
going on is a war against Catholicism in both its 
Roman, its Anglican, and its Protestant form. The 
Church is struggling for its spiritual life with a Catho-TK
licism of order or of doctrine which took possession 
of it in the second century, which was destroyed in 
principle at the Reformation, but which in practice 

* The laicising of religion in Protestantism had its own perils, 
and one of them was the secularising of it from which we now 
suffer. The world that faith was to leaven has captured and 
sterilised faith. What the State has done in one way society has 
done in another. This was a risk that Protestantism had to run like 
Christianity itself. The Reformation had one true note of faith in 
that it was a tremendous venture, the work of a courage super-TK
human, and either diabolic or divine. Happily hazard is not the 
badge of error, as security is not the sole effect of salvation. Nothing 
risks so much as faith, and there are few things more perilous than 
religion, especially to outward finality and peace. 



Il6 ROME, REFORM, AND REACTION 

holds it to this day. And the root of the quarrel is to 
be found in the conception of faith which distinguishes 
the Church of the New Testament, and of the 
Reformation, from that of Catholicism. 

I would be particularly understood to mean that 
the strife is not confined to the Established Church. 
The strife there can really only be settled by a change 
which affects the whole religious mind of this country, 
and by a voice which recalls the Protestant and Non-TK
conformist as well as the Catholic to the Gospel of 
Christ, from paths that are fatal at last. The war with 
Catholicism is acute in the Established Church, but it 
is waged also in the bosom of Nonconformist Protes-TK
tantism itself. Even there faith has to suffer from 
other forms of the Catholic idea, from the theological 
relics of scholastic orthodoxy on the one side, and 
especially from culture on the other; from a Pelagian, 
synergistic humanism, or from a Franciscan piety whose 
idea is a compassionate philanthropy much more fine 
than final---as far from final as monasticism was for 
the regeneration of the Church. Assisi was well, but 
It did not do what had to be done, and what was 
done at Wittenberg, Worms, and the Wartburg. 

The issue which is raised concerns the essential 
nature of Christianity. The war is for the expulsion 
of paganism by faith, or its reduction to a secondary or 
tertiary place; and by paganism is not meant the 



WHAT DID LUTHER REALLY DO? II7 

heathen cults, but the practical supremacy of the natural 
man and the supersession of revelation in the ideal of 
life and the soul. It is, therefore, clear that the war 
must be waged in Protestant communities themselves, 
to save them from the humanism and naturalism which 
are Catholic in their spirit and result. The great pagan 
of Protestantism---Goethe---was in spirit and ten-TK
dency Catholic. All art, literature, and culture gravi-TK
tate there when they become the ruling interest of life. 
And any reconstruction of Christianity upon these 
lines chiefly does more for a Catholic revival in the 
long run than it does for the New Testament faith 
of the Gospel. Our great danger is not Ultramon-TK
tanism; it is a far subtler Romanism than that, and 
one which prepares the soil on which Ultramontanism 
finds ready growth. The enemy is Catholicism, 
the worship of system in society or creed. From this 
Protestantism is but partly purged. The Reformation 
was a reformation but in part, and that part was soon 
overtaken by a resurgence of the Catholic habit of mind 
in the shape of orthodoxy. The Protestant orthodoxy 
of last century was Catholic in its spirit. It was institu-TK
tional, confessional. And no less Catholic is, on the 
one hand, the philosophic liberalism of to-day; and, 
on the other hand, our refined and literary mysticism. 
Hegelianism, with its love of system, if it remain 
positive, tends to the institutional and established 



Il8 ROME, REFORM, AND REACTION 

Churches; and the Friends who leave their Society 
and do not turn Unitarian gravitate to the aesthetic 
and sacerdotal Church. 

II 

To aid us in adjusting Faith to Humanism let us 
ask, what was the real work and permanent value of 
Luther, and in what sense does he keep his value for 
us to-day? I take Luther of course as the symbol 
and representative of the whole Reformation. 

Luther is a most heroic figure, but it is not as a 
hero that he is of perennial value to the world. As a 
hero he would be merely an aesthetic object, a colossal 
representative of modern individualism facing the old 
and corporate order of things. As a hero he is a magni-TK
ficent organ of human power and freedom, a glorious 
expression of human courage and human conscience. 
He enters into innumerable pictures and lessons upon 
human valour, stoutness of heart, and fidelity to con-TK
viction. Luther before the Diet at Worms, with his 
" Here I stand; I can do no other. So help me 
God," is one of the stock legends of moral courage 
and the lone good fight. 

But as such he is not solitary in history. He did 
nothing unique in its nature. We may be proud and 
thankful as men to think that there have been many 
men, in many ages, and in many causes, that have 



WHAT DID LUTHER REALLY DO? II9 

shown moral heroism of this kind. The present age 
does not breed them as freely as some, but they are 
not extinct even to-day. And we must recognise 
them even among those whose principles we have to 
resist. There was a grandeur in Charles V. as well 
as in Luther, a heroism of empire as well as a heroism 
of revolt. If you are to honour men for mere sin-TK
cerity, or veracity, above all things, you will have a 
pantheon as mixed as worship of that kind made 
Carlyle's to be, whose heroes ranged from Luther the 
godly to the pagan Goethe and the godless Frederick. 
If you are to make fidelity to conscience the supreme 
standard, you must pay as much respect to a narrow 
conscience as to a great, so long as it is faithfully 
followed. If it is mere fidelity to conscience that is 
the chief thing, and not the contents of the con-TK
science, or its word, then the little bigot who wrangles 
about trifles of ritual, or divides a Church upon an 
item of accounts, may be as faithful to his lean con-TK
science as the large-minded, full-souled preacher who 
risks the displeasure of the public and the loss of his 
living for startling them with a great new Gospel. 
If it is a mere case of fidelity to conscience, you must 
pay as much honour to Laud with his formal piety 
and his mechanical churchism as to Luther with his 
vast living soul and faith. When a great conflict 
is going on we cannot stop to listen to the little 



I20 

moderates who think we are too severe because Mr. 
So-and-So is a most conscientious and devoted man. 
Of course he is. We take that for granted and go 
on. It is the conscientious men that are most worth 
fighting. It is the conscientious men that do so 
much mischief, the men who are more concerned 
about being true to their conscience than about their 
conscience being true to truth and right. Many of 
the men most dangerous to mankind have been con-TK
scientious men, apostles of the canonical conscience, 
like Laud, who have sincerely believed that without 
their machinery the world would go to perdition, 
led there by the Nonconformists with a conscience 
equally supreme. I should have no difficulty in believing 
this to have been the conviction of Torquemada the in-TK
quisitor, who was so much more thorough with Laud's 
principles and his conscientious cruelty than Laud's 
time allowed him to be. The exterminators may be as 
conscientious as the persecutors---they are much more 
logical and effective. There have been heroes of the 
conscience on the Catholic side as on the Evangelical, 
and there are to-day. There may be as many, there may 
be more. We do not judge individuals or count them, 
and therefore we do not need to ask whether they were 
conscientious men or not. We deal with principles 
and gospels. 

Wherefore, the great question is what the contents 



WHAT DID LUTHER REALLY DO? 121 

of the conscience were, or are; not how the man 
lield to his conscience, but how his conscience held to 
reality, revelation, and truth. Luther's merit was not 
the heroism of his conscience, but the rediscovery of 
a new conscience beyond the natural, and beyond the 
institutional---whether canonical in the Church, or civil 
in the State. He found a conscience higher and 
deeper than the natural, the ecclesiastical, or the 
political---the individual, the canonical, or the civil; 
more royal than culture, clergy, or crown. He found 
a conscience within the conscience. He found anew 
the evangelical conscience, whose ideal is not heroism 
at all, but the humility and obedience of the con-TK
science itself, its lostness and its nothingness except 
as rescued and set on its feet by Christ, in whom no 
man is a hero, but every man a beggar for his life. 
Luther had no sense of humility in regard to the 
Church of his day, the empirical Church anywhere; 
against it he stood for the rights of the individual. 
This was where he went beyond Augustine, with 
whom he had so much in common, and did for the 
Pauline Gospel what Augustine could not do, because 
possessed with a false idea of the Church and of the 
humility due to it. Against the institutional, hierarchi-TK
cal Church of Augustine Luther stood up, with a 
colossal independence, for the individual. But the 
independence was much more than colossal, and the 



122 ROME, REFORM, AND REACTION 

courage was more than sublime. It was solemn, be-TK
cause subdued. It was not for the independent indi-TK
vidual that Luther stood, but for the humbled, broken, 
crushed individual, new-made in Jesus Christ, and 
found in Him. The whole meaning of the Reformation 
was not so much that the individual was put in a new 
relation to the institution, but that he was put in a 
new relation to God and to himself. The Church as a 
necessary mediator of that relation was pushed aside. 
The whole Church system of forms and deeds, which 
centuries had built up till the sky was opaque and God 
remote, was swept out of the way. The intention 
was not to sweep it away, but only out of the way---
to sweep it aside, and let men see Christ crucified. It 
was not to be swept out of existence, but only out of 
the path. The Reformation was not the assertion of 
the unchartered individual, but of the individual's right 
to a gracious God; the right not of the natural freeman, 
but of the freeman in Christ; not of the stalwart, but 
of the humbled and redeemed. The free conscience v/as 
a conscience bound inly and utterly to Christ alone. 

Ill 

It is well that the real and solemn nature of the 
Reformation individualism should be understood. It 
was not a thing to be snatched at, but a responsibility, 
and a heavy one. It was a calling in life---a calling 



WHAT DID LUTHER REALLY DO? 1 23 

of God. The individualism which is claimed as a 
right is common enough; we do not so often find 
it accepted as a calling, and construed as a duty. 
Luther's individualism was not the rejection of a 
burden, but its transfer from the outward Church to 
the inner soul. The vast spiritual problem of the 
medieval Church was removed from the broad arena 
of the Church, with its corporate conscience, and 
transferred to the interior of the single soul, to the 
conscience alone with judgment, and it was fought out 
in purely spiritual terms. The conflict of the soul had 
been much mitigated for the individual when it was 
conducted on his behalf by the Church at large; but 
the Reformation forced it in upon the isolated being 
whose eternal life was at stake, and he was made to 
feel it so intolerable that he rushed into the new truth 
with a breathless gratitude---born, indeed, of faith, but 
cradled in despair. The whole problem of the world 
was condensed into the experience of the single soul; 
and it was an experience in which his eternal salvation 
or damnation was at stake. In his narrow space the 
great spiritual deeps were broken up with sometimes 
volcanic force, and the eternal winds and waves roared 
in his being like the sea storm in the chimney of a 
clifF. The old helps of the Church were now useless; 
they had been found wanting. They had satisfied 
many of the best, but they were powerless even for 



124 ROME, REFORM, AND REACTION 

the common sinner now, when the issue was sharply 
placed before him. The Church system of grace had 
been the slow ascent and purification of the soul through 
sacramental stages, the gradual education of the human 
into the divine; but this no longer sufficed. A sharp, 
unsparing contrast was set up in the soul between 
God's demand and man's fulfilment; the failure was 
carried home, and with it man's impotence to mend 
his case. If salvation was to come, it must come 
direct from God; and it must come as an immediate 
and final possession, not as a slow and perilous pro-TK
cess. All natural development was here broken short; 
all spiritual culture, as mere culture, was ineffectual 
for the fiery crisis. It must be cut short by instant 
action and decisive change, and the new hope must 
flow from a miracle of God in the soul. And that 
was the nature of faith. It was a miraculous peace 
brought by God out of an intolerable war, which was 
desolating the soul in a way no Church could stay or 
cure. The battle, therefore, was no more fought on 
the broad plain of a Church's experience, nor was the 
authority for the conscience sought in a community. 
To commit the great holy war to a corporate Church 
tends to blunt the spiritual sensibilities of the man, 
and enfeeble his spiritual tone. All was now trans-TK
ferred to the narrow area of a personality, with its in-TK
finite and eternal spiritual issues. But to transfer such 



WHAT DID LUTHER REALLY DO? 1 25 

an awful conflict there, and then to leave it without 
such aid as the Church had striven to give, was more 
than the soul could bear. It would have snapped and 
perished in the strain, it would have been ground 
up in the clash and pressure, but for the fact that 
within the man's personality a new Personality stood 
with healing in His hands. In the furnace walked 
the Son of Man. Where the Church had stood like 
a baffled wizard at a magic circle beyond his spells, 
and vainly tried to enter the real area of strife, there 
stood now the Redeemer---stood suddenly in the 
midst, and said, " Peace be unto you," and there was 
peace. The individual did not become the authority 
which the Church used to be, but he did become the 
sphere in which another authority in another person-TK
ality arose to reign; the individual did become the 
area of revelation which the Church had been. In 
a religion everything turns on the nature of the revela-TK
tion. The religion is just what the revelation makes 
it, because religion is just the faith that the revelation 
evokes and that answers it. What was the Christian 
revelation? A system or an act? a theology or a re-TK
demption? a visible Church or a spiritual reformation? 
a truth or a person? grace as the capital with which 
God set up the Church in business, or grace as 
His act on the individual soul? The whole ques-TK
tion between Protestant and Catholic turns on the 



126 ROME, REFORM, AND REACTION 

nature of revelation. While to the Catholic it came 
as a system, to the Protestant it came as a salvation. 
It came as personal redemption, it became revelation 
only as redemption, and within the soul arose another 
soul to be its true King and Lord. The only truth 
for the soul was not Redemption but its Redeemer. 
What was revealed was not truth in the custody of a 
Church, but it was a spiritual act and person of salva-TK
tion in the experience of a soul. That was the nature, 
the price, the glory of the individualism of the Re-TK
formation. If it discarded a Church, it was not in 
self-will, as the mindless thought and think, but it 
was to take up an awful conflict and a solemn charge. 
The Church could never carry that charge when the 
soul really came to feel itself and its sin, but only 
the Church's Lord and Saviour could in the soul's 
own secret centre and sacred shrine. 

IV 

Luther was certainly not a champion of the Re-TK
naissance, of the new learning with its new claim for 
intellectual freedom, of the new culture with its new 
sense of human reason, human thought, human 
beauty, and human grace. The man that stood for 
that in the Christian name was Erasmus, the true type 
of the English Reformation---Erasmus, who thought 
that nothing more than mere reform was required, 



WHAT DID LUTHER REALLY DO? 12/ 

which should be in the hands of men of learning and 
position, pious scholars and gentlemen. When Luther 
appeared, these humanists rejoiced greatly, and saw in 
him a precious ally---as some of the best Pharisees 
greeted Christ Himself, and thought He might go far 
in their cause; as culture of many kinds welcomes 
Christianity for an agent of culture, of order, learning, 
art, literature, and gracious life. The universities 
always tend to treat Christianity rather as a culture 
than as a gospel, as one of the faculties than as life 
itself. But as Luther went his way, the humanists, 
the cultured people, almost all fell away from him; 
as the orderly and institutional Pharisaism had to 
renounce Paul and counterwork him. They found 
in Luther another spirit---not a reform, but what 
amounted to a new religion. And culture fell back, 
as culture always tends to do, into the institutional 
arms of some kind of Catholicism, devoid of the bold, 
perilous, original, and searching evangelical note. 
Luther was not an organ of the Renaissance, but of 
a mightier movement, in which the Renaissance itself 
was to find its true destiny, and win the victory it 
had not power with the soul to gain. 



Luther was not a champion either of the conscience 
or of the reason, but of the Gospel. He was not so 



128 ROME, REFORM, AND REACTION 

much a moral figure as a religious. His work was for 
faith more than for morality, for religion more than 
for the conscience, and for the conscience as lost 
rather than for the conscience as king. Luther did 
what Augustine had tried and failed to do for the 
reason I have named. He restored Christianity from 
the Church to religion; he made faith once more a 
religious thing. His key-word was not law and order, 
it was not even righteousness and piety; it was grace. 
And it was answered not by the mind's assent, nor by 
amendment of life, but by a new life altogether, a 
new kind and principle of life, the life of faith. He 
did free men from the letter of Scripture, from schol-TK
astic theology, from the authority of the Church; but 
that was done incidentally to the great deliverance he 
was charged with---the gospel of the soul's deliverance 
from guilt. What made him groan in his monk's cell 
was not the bondage, tyranny, narrowness, immorality 
of the Church, but the burden of his own soul, his 
own self, his own guilt. It was the load of guilt that 
was killing him, not the load of the Church. He 
turned on the Church only when he found that it could 
do nothing real and final for misery and sin. It could 
not only do nothing, but it stood in the way of any-TK
thing being done. Luther did not set out to save men 
from the tyranny of the Church, but from the guilt 
and death of sin; and he saved them from the 



WHAT DID LUTHER REALLY DO? 1 29 

Church by the way, because the Church pretended to 
save them and could not. His ideal was not emancipa-TK
tion, but redemption; what he resented at the outset 
of all, and the root of all, was not man's tyranny over 
man, but man's tyranny over himself. And the Re-TK
formation never falls into discredit but when men are 
too proud, worldly, or well ofF to feel the moral load 
of their own souls, or the need of being delivered 
from their own guilt. It is easy to agitate against an 
outward tyrant---easy by comparison. It is not easy 
— it is far too hard for any but a few---to agitate 
against the tyrant in themselves, and fight for that 
inward freedom which the Gospel alone can give. 
When you hear tell of the simplicity of the Gospel 
and the lightness of its yoke, remember these words, 
"Except your righteousness, your Christian ideal of 
righteousness, exceed the laborious righteousness of 
scribe, priest, and Pharisee, ye cannot enter the King-TK
dom of Heaven." The simplicity of Christianity is 
very searching and very severe. 

VI 

The severity of the Gospel was for Luther so great 
that it broke the soul to pieces and ground it to dust to 
make the new man. The Church was severe in a 
way, and its way to perfection was laborious; but it 
was an unsearching severity, spread thinly over a wide 

9 



130 ROME, REFORM, AND REACTION 

area of life. It was split into a multitude of demands, 
observances, mortifications, persecutions of human 
nature. It was ascetic severity. The severity of the 
Gospel is pointed; it goes to the heart; it is the 
severity of a sword; it is concentrated, intense, deadly. 
It is thorough with the old man and his sin, as it 
is thorough with the new man and his salvation. 
Luther's work, while it made faith simpler in one way, 
yet made it much more hard, exacting, and powerful 
than it had been before. It was the simplicity of con-TK
centration, which is intense and irresistible. Luther's 
work was one of concentration from functions to acts, 
from acts to the soul. He compelled religion from 
acts which were mere offices to an act which taxed 
the whole will and soul, the decisive act of faith and 
its surrender. His work was a vast concentration; 
as it withdrew religious effort from a wide range of 
detailed conduct, it made it converge upon the central 
man in such a way that the amount of his religion was 
changed into its quality. There are substances that 
under intense pressure lose their former constitution, 
as it were, and from an expansive gas become a con-TK
densed fluid. They liquefy under pressure. So with 
the work of Luther's Gospel; the soul liquefied under 
its concentrated pressure, and became as it were another 
nature. The extent of its religion, being compressed 
from works to faith, was changed into a new kind of 



WHAT DID LUTHER REALLY DO? 131 

religion. What had been a volume of outward observ-TK
ance became a drop of spiritual power. Luther called 
in the forces of the soul from the elaborate system of 
a Talmudic Church, with its penances, asceticisms, 
precepts, ordinances, and canons, and he fixed them 
on a single infinite point. He effected a huge simpli-TK
fication, while he made the one new thing far more 
searching and imperious than the variety of the old. 
It was easier to trust God than do the penances of the 
Church; yet it was harder, because the whole moral 
will must bow, and not merely the outward consent. 
Humble and sure trust in God's fatherly forgiveness 
and care in Christ was drawn forth from under a heap 
of refined and complicated regulations, like a jewel 
from the debris of a great fire. Luther took it and 
set it in the forehead of the Church, and made it the 
very eye of religion and life's ideal. He took faith, 
which had been a system of mere compliance, and 
he made it the simple but arduous act of the soul's 
penitent obedience to the Gospel of God's forgiving 
act, and deed, and promise in Christ. Grace became 
a mercy exercised by God on the soul through faith's 
act, and not an influence infused by the act of a sacra-TK
ment. Christianity becomes outwardly Christ, in-TK
wardly faith. Its key-word is not so much sanctity or 
inspiration as forgiveness. The forgiven man is the 
saint, not the consecrated monk. Not piety so much 



132 

as trust, not ecstasy but confidence, not sweetness but 
power becomes the religious ideal---power unto God, 
and power over the world---power, by the reconcilia-TK
tion with God, to be reconciled with hateful and hating 
men, and to serve, in Christ, men to whom naturally 
we would not bend an inch from our way. The 
meaning, nature, and place of faith were changed. It 
became the permanent essence of religion. It had been 
the mere assent to absolution, it was now the soul's 
response to forgiveness; from self-surrender to the 
Church it became self-committal to Christ; from com-TK
pliance with the canonical regulations it became the 
obedience of the total man to the Gospel; from opinion 
or achievement, as preliminary conditions leading to 
something greater than faith, it became the trust than 
which nothing is greater, because it trusts all the love 
in the world in the fatherly love and salvation of God. 

VII 

Let us pursue Luther's principles in more detail. 
How did he work out that new idea of faith and the 
perfect godly life? Especially, how did this idea 
of faith affect the Church? He had two things---a 
foundation which was God's Word, and a power 
which was man's faith. 

I. He believed that it was the Word of God that 
founded the ChurchMK The Church was not based on 



WHAT DID LUTHER REALLY DO? 1 33 

tradition, nor upon bishops and popes. These were 
too variable, too unsure. Yet it did rest on some-TK
thing fixed, something objective, something given to 
man and not contributed by him. It rested upon no 
invention, but on a revelation; not on an achieve-TK
ment, but on a gift. The act of Christ which 
founded the Church was, in its very nature, above all 
a gift of God to man. Christ's work was much more 
a gift of God to reconcile man than a gift of man to 
reconcile God. The Church rested on this gift of 
God---upon something which had always been there, 
though obscured and perverted---always there as the 
true reality of the Church. It was now open to all, to 
the simplest, to every Christian as Christian, in virtue 
of his faith. It was not to be opened by pope, or 
bishop, or council, or saint; nor could they close it, 
and shut out men of faith. The foundation of the 
Church was there in the Bible, when interpreted in its 
actual, original, spiritual sense, apart from allegory and 
from any outside authority. In a word, the founda-TK
tion of the Church was the Gospel, and the Church 
is the fellowship of the faithful, to whom the Gospel 
is Gospel indeed. It is easy to see how the Inde-TK
pendent idea or gathered Churches, as distinct from 
territorial, flows from this. The Gospel is thrust into 
mankind as a magnet into a heap of iron dust and 
sand; and the Church is composed of the particles 



134 ROME, REFORM, AND REACTION 

that cling, organized by the movements of the magnetic 
force. 

VIII 

2. This gospel was the true Word of God on 
which the Church was based. The Word of God, at 
the base of His Church, was not any phrase spoken 
by Christ founding a Church, nor an instruction or 
commission to the apostles. He is the Church's one 
foundation; it is no edict or commission of His. 
Christ {lid very little (some say nothing) in the way 
of founding a Church; but He was everything. The 
Church proceeded from His work and person, not from 
words He said. It stood on what He was and is, and 
not upon what He devised. It stood and stands on 
the Gospel. And by the Gospel is meant, not a book, 
or a system, or a scheme, but the very act, deed, and 
revelation of God in Christ. The Gospel is not 
truth about God's reconciliation; it is God Himself 
reconciling in Christ. The Gospel is God in Christ, 
God in His Cross, God in Redemption. The per-TK
manent Gospel is the base of the permanent Church, 
and the permanent Gospel is the eternal Christ in the 
heaven of redeemed experience. This Gospel creates 
its own answer, and that answer is faith, and so we 
come to Luther's poiver---faith. The Word of God 
has been conceived at various times to be the letter or 



WHAT DID LUTHER REALLY DO? 1 35 

the Bible, or the Bible as a whole, or the doctrines 
running through it, or the promises scattered in it. 
For Luther it was the vital principle of the Bible, the 
long act of revelation and Redemption which the Bible 
records---the Bible's heart and power; in a word, 
Jesus Christ and Him crucified. The testimony of 
Christ is the spirit of Scripture. No statement can 
save, no precept, no doctrine, no law; not the 
sweetest, comfortablest doctrine can save as doctrine, 
as mere truth; only the truth as Jesus. Only a 
person can save a person. A Church cannot, for it is 
a system, an institution. And no institution has 
saving power. It can serve salvation, but it cannot 
either save or damn. What the soul needs is Gospel, 
and an institution is law. To grasp the distinction 
between law and gospel, to grasp that with true in-TK
sight, is to grasp the real core of religion and the clear 
nature of faith. It is because Christian people do not 
grasp this difference, and do not therefore realize the 
true nature of faith, that the empirical Church is the 
formidable thing it is to-day. A Church is more of 
the Law than of the Gospel, and the more powerful it 
grows the more is it a menace to faith. What must 
control the Church, in actual practice and not mere 
theory, is the Word of God as the Gospel coming to 
the soul through faith, with the Church as a mere 
herald and medium and agent. Rob faith of its place 



136 ROME, REFORM, AND REACTION 

and power, and the Church becomes not a medium 
but a mediator, its minister becomes its priest, and its 
policy is not service but power. Faith is fatal to such 
a place for the Church. It is direct dealing of the 
soul with Christ. Christ is the object of faith, not a 
book, or a Church. Faith is taking Christ's forgive-TK
ness seriously and heartily. The devils or the wicked 
could believe in the Church (for Churchmen have 
been both); but the one thing they cannot believe in 
is the forgiveness of sin (else they would cease to be 
devil or damned); and, therefore, this is faith's dis-TK
tinction from the world and hell. The true authority 
over the soul and conscience is given through this faith. 
That authority is not the Church, but it is the effec-TK
tual Word of God in the preaching of the cross, to 
which the conscience owes its life. And doctrine is 
just the best account we can give of this living faith 
in its living community. 

IX 

This faith, then, was the new, the reformatory 
thing in Luther's position. What did it replace? 
It replaced what we find passing for religion to-day in 
the circles where the Reformation influence has not 
truly penetrated, where an institutional, episcopal, and 
priestly Church keeps the public soul under a mere 
Catholicism. What is that? What is the idea of 



WHAT DID LUTHER REALLY DO? 1 37 

religion current in the semi -reformed circles of this 
country to-day? What is the idea of religion that 
the man in the street can be made pugnacious, and 
even furious, about when it is assailed? What is the 
shape into which his vague education has cast his 
natural religiosity? It is the Catholic idea of certain 
beliefs and certain behaviours; of accepting the know-TK
ledge ot God and of the world authoritatively given 
by the historic Church of the land, along with the 
exercise of certain moral virtues to correspond; * Be-TK
lieve in the Incarnation and imitate Christ.' That is 
all very well, but it is not a Gospel, only a Church-TK
spel. Orthodoxy of creed and of behaviour is this 
ideal, rising to the idea of imitating Christ as the 
great Example, but too seldom tending to trust Him 
as a matter of direct personal experience. It is right 
knowledge on the Church's authority, and right con-TK
duct in personal relations, but less of actual and ex-TK
perienced personal relations with the divine object of 
the knowledge. Now the Reformation did not dis-TK
card either right knowledge or right conduct; but 
it cast these down, for their own sakes, to a second 
place; and it put in the first place what Catholicism 
had, for the average believer, only made second (if 
second)---the personal trust and experience of Christ 
in a real forgiveness. Out of that all right belief and 
conduct must proceed, and it was the only guarantee 



138 ROME, REFORM, AND REACTION 

for either. The first was made last and the last first. 
The whole Reformation might be defended as a 
crucial instance of that characteristic principle of Chris-TK
tian change, of divine judgment by inversion. The 
thing that was now put first is the thing that is always 
first in the spiritual order. It is the creative thing. 
Faith is the power creative both of right creed and 
right living. All the ethical world spreads away from 
the true focus of personal faith in God's forgiving grace 
in Christ. All the moral order is ruled from this 
throne. I do not say that morality does not exist 
apart from religion; it does. But I do say that 
finally it cannot; in the spiritual and ultimate nature 
of things the two are not separable, distinguish them 
as you may. The permanent ethic is Christian ethic; 
and Christian conduct dies soon after Christian faith. 

The new thing, therefore, in Luther's Christianity 
was really the religious understanding of the Gospel. 
It had been understood, theologically, ecclesiastically, 
morally before, though not properly understood. It 
was never properly understood till it was understood 
religiously, by faith alone, by the lost soul saved. 
That was Luther's starting-point and goal alike. All 
his work began in this, and it was all for the sake of 
this. It was only gradually that it was forced on him 
how incompatible with this was Catholicism, the 
Church habit of mind, the Church idea of faith, the 



WHAT DID LUTHER REALLY DO? 1 39 

Church claim on obedience, the Church's position as 
mediator between God and man, the Church custom 
canonized, the Church staff idolized, the ministry 
sacerdotalized, and administration made hierarchy. 
He did not mean a new Church. The new Church 
only arose by the resistance of Catholicism to faith, to 
religion, by the obstinacy of the canonical conscience 
to the evangelical. A new Church really arose 
because what Luther brought from the New Testa-TK
ment was a new religion. Catholicism is not so 
much another form of Christianity as another religion. 
It rests ideally though not empirically on a totally 
different idea of faith, and that is what makes a reli-TK
gion. Protestantism saved Christianity for religion, 
saved it as a religion, saved it from becoming a mere 
institution. To religion, Catholicism, Roman or 
Anglican, is at last fatal, as continental atheism shows. 
And failure to see that is due less to want of vision than 
of insight, to lack not of ability but of the intuition ot 
faith, and the witness of the Spirit. It was the Holy 
Spirit that made Protestantism, more than Luther, 
Calvin, or Melanchthon. Or, if we put it in the 
diluted language of modern thought, it was made by 
the genius of Christianity. It was Christianity re-TK
forming itself. It was the victory of the instinct 
of self-preservation in Christianity. 

The Reformation was the work of Christian faith 



140 ROME, REFORM, AND REACTION 

coming to itself, much more than the work of single 
men or groups. The faith made the men, not the 
men the faith. It was the self-assertion of the true 
Christian idea---not its assertion, but its self-assertion. 
It was not something that men spoke; it spoke in 
men. Nothing on earth could have prevented such a 
movement, amid the perversion and inversion which 
faith had undergone in the course of Catholic cen-TK
turies. Catholicism is quite incompatible with the 
New Testament idea of faith as Luther rediscovered 
it; and a decisive issue was bound to come then as 
now. The two ideas destroy each other; they cer-TK
tainly could not both be supreme in the same house. 
A mere institutional faith could not claim to be 
the saving faith in any Church which possessed, 
honoured, and understood the New Testament. The 
Catholic and the Evangelical ideas of faith are in-TK
compatible, because each claims to be absolute. The 
priest of the sacraments has no room for the minister 
of the Gospel; the ministry of the Word has no 
place for the vicarious priest. A faith in which any 
human priest is essential is utterly incompatible with 
one in which the priest is a peril and a treason. And 
this is not human self-will; it is the antipathy of two 
mutually destructive ideas, the process of a historical 
and spiritual logic. 



WHAT DID LUTHER REALLY DO? 141 

X 

You hear politicians say that the Church must 
be comprehensive, and that the High Church party-TK
has as good a right to its place within it as the 
Evangelical. How it may be as to parties I know 
not, but it is quite certain that the High Church 
idea, in so far as it is sacerdotal, can have no room 
nor tolerance for the other. If the true minister of 
Christ is a priest, then his business must be to 
remove from the Church all ministers who are not 
priests. If the Catholic idea of faith is right, it is 
supreme and sole, and there is no room over its head, 
or by its side even, for the evangelical idea of faith, 
which is bound to be equally absolute in its claims. 
Two absolutes cannot sit on the same throne or rule 
the same Church. This, of course, supposes that the 
Church is something prior to the State, higher than 
the State, existing in its own right, and living en-TK
tirely on its own faith. It supposes that the Church 
is a spiritual unity, pervaded by the one Spirit, and 
based on one consistent idea of faith, which it is 
free to give effect to, and bound both to obey and 
enforce. In such a Church, which is the true idea 
of a Church, the two orders of faith are not com-TK
patible; and it is only obscurity or insincerity that 
can lead any sacerdotalist to say that his place in the 



142 ROME, REFORM, AND REACTION 

Church allows him to concede a like right and free-TK
dom to the Low Churchman if he will let him alone. 
But if the Church be not such a free body, if it 
be a Service of the State, in which the states-TK
man's word is supreme, if it cannot give effect to its 
own principles and affinities, then I can understand 
the plea of comprehension. It is intelligible enough 
in men who know nothing whatever of the true 
genius of the Church, whose minds are incurably 
political, and who realize the spiritual situation on 
the historic scale so little that they think the same 
Church can house to-day the two ideas that rent 
Christendom in a strife that rent Europe. Two 
parties may dwell in the same realm and sit in the 
same house, and they may work well enough as 
political working goes. But the Church is not the 
State. The State has not an initial and positive 
charter, as the Church has, in the Bible. The Church 
is the sphere of revealed ideas. If its fundamental 
ideas are at feud it must be rent. The men might 
sit together, and do, and I hope always will, at dinner 
tables and philanthropies, and always in mutual re-TK
spect for Christian character, or Christian learning, 
or Christian culture and honour. But the two ideas 
cannot lie down together. Their spheres and proce-TK
dures are different. They cannot co-operate on oppo-TK
site benches of the same house. 



WHAT DID LUTHER REALLY DO? 1 43 

XI 

The sphere of the Church was for Luther the 
region of faith. Its members were the people ot 
faith. Only the believer knows the Church. Only 
the believer belongs to it; and not the believer in 
the Church, but the believer in Christ. The Church 
is net the object of faith but only its home. It 
does not produce faith, but it is the home where 
faith is born and brought up, where all things are 
ordered in the interest of faith. She is not so much 
the mother of the believer as his nurse. She holds 
the believer in her bosom, and he grows in her care. 
Faith is not faith in the Church---that is Catholicism 
— but faith through the Church. How shall they 
believe without hearing, or hear without preaching? 
and where is preaching without the Word, which is 
entrusted to the Church? Outside the Church in-TK
deed is no salvation; but it is outside the Church 
of the Word, not of the sacraments. Outside the 
Church means not so much outside its membership 
or baptism, but outside the range and influence of the 
Word that makes the Church by making Christians. 
What makes Christianity is not baptism but the work 
of the Gospel---of which Baptism is but one symbolic 
expression; it is no creative act. 

This is the Reformation principle, though there 
wMKere others of the Reformers in whom it had become 



144 ROME, REFORM, AND REACTION 

more clear than it did to Luther, especially in re-TK
spect to the sacraments. 

I say little or nothing, you may note, of the de-TK
struction ot superstition. The Reformer was there 
not to destroy superstition but to assert faith. Our 
protest to-day is too much against superstition and too 
little MKr living faith. The deeper our faith is, and the 
more adequate in its intelligence, the less likely we 
shall be to throw about charges like superstition, which 
may easily sound supercilious and certainly irritating. 

Religion is faith. The Christian religion is Chris-TK
tian faith, and Christian faith is faith in Christ alone. 
The difference between the Catholic and the Evan-TK
gelical Church, which is the great coming war, is a 
difference between the Catholic and the Evangelical 
type of faith, and therefore it is a difference as to the 
true nature of religion, of Christianity, and its practical, 
spiritual ideal. It is not a conflict of creeds in the 
sense of articles. It is a conflict of spiritual types. And 
it is not so much a conflict which shall expel the other, 
but which shall rule the other in the proper sense of 
the word rule, as influence and not domination. There 
is much in the Catholic ideal which faith would be the 
poorer to lose, so long as it is kept in its due place. 

The idea of faith in Catholicism was twofold. 
For the layman it meant assent to the Church and 
its Creed---the acceptance of these as true, and outMK 



WHAT DID LUTHER REALLY DO? I45 

ward submission to them. The state of the heart 
was a secondary matter. For the saint it was a 
mystic union with the Godhead, which had its chief 
expression in moments of insight, rapture, and ecstasy. 
Religion was regarded as a form of inspiration or 
divine indwelling, and its flower was the sanctity of 
the devotee who adopted religion as a profession. 
Lay assent and saintly mystic rapture were the two 
forms ot Catholic faith. The object of the sacra-TK
ments was to aid that fusion of human nature with 
the divine which was regarded as the core and 
crown of sanctity in the Incarnation. The ideal 
relation of God to man was an indwelling, lifting 
the soul to the height of joy and calm. Neither God 
nor man was treated supremely as a will, but rather 
as a substance, and their union was a fusion rather 
than a reconciliation. The ruling thought was not 
revelation but inspiration, not the word to the will 
but the breath to the being. Peace with God was 
rather the subjective calm of a religious mood reached 
by great and ascetic effort, and very fugitive after all. 
It was monastic, quietist, undisturbed, a state of con-TK
sciousness which was an object and end in itself. 

It was the quest for this that engaged the soul 
of Luther in his cell; and it was upon this quest 
that the new light broke which, if it had not been 
the rediscovery of the New Testament idea, would 

10 



146 ROME, REFORM, AND REACTION 

have been a new revelation to the vvMKorld, and Luther 
would have been the founder of a new religion. As 
it was, he was but the first real herald of it since 
the apostles, and especially since Paul. In Luther 
Paul came to life again. Faith was no longer to be 
the assent of the mind to certain truths, nor obedi-TK
ence to an institution, nor the enjoyment of mystic 
union and rapture; but it was trust, confidence, 
sonship with God. The foremost thing was not 
inspiration but revelation; not the indwelling of the 
divine nature but the perennial utterance of God's 
saving word in act and fact, and the whole man's 
answer to it in trust. What was that word? What 
was the revelation? It was grace, mercy, forgive-TK
ness in Jesus Christ, and in Him directly and alone. 
Faith was, as Melanchthon said, simply trust in God's 
mercy to the sinner in Christ. It was not fusion with 
God's nature even as love, it was not being sunk in 
the abyss of the divine, or filled to rapture with the 
inflowing of the Spirit. It was not the translation 
of the soul into a divine substance, man becoming 
God through God becoming man. It was not seeing 
God, or feeling Him, but trusting Him, committing 
one's self, one's sins, one's soul, one's eternity to God 
in Christ, on the strength of God's act and promise 
in Christ's redemption. It was not elation, rapture, 
ecstasy---it was confidence. It was answering a per-TK



WHAT DID LUTHER REALLY DO? I47 

son, a gospel, not a system, or a divine infusion. 
Its peace was not the calm of absorption, of losing 
ourselves in the ocean of God's love, but the peace 
of believing, of forgiveness assured and foregone in 
Christ, and trusted even amid repeated and cleaving 
sin. It was trust in God's forgiveness, and in His 
providence, for every soul. It was the peace, not of 
seeing God in rapture, but of believing amid a world 
of temptation, misgiving, and self-accusation, 

I shall my fierce accuser face. 
And tell him Thou hast died. 

It was the peace of justification rather than of com-TK
munion. It was not a state of subjective conscious-TK
ness but an assured relation of the will to a will, of 
a person to a person, of a present to a future. It 
was the peace of no condemnation rather than of no 
disturbance. It was not so much an experience as 
a standing act, attitude, and habit of the moral soul, 
the spiritual will. This faith often overcame experi-TK
ence and saved us from it; the experience might be 
troubled but the faith stood fast. It went out of 
the cloister into the world; and it proved its sanctity 
in the godly way in which it did the world's work 
rather than in the exquisite sensibility of the solitary 
to sacred things. A new type of sanctity and per-TK
fection arose, not confined to those who had the 
religious genius or religious leisure. The saint might 



148 ROME, REFORM, AND REACTION 

be something very different from the professional 
religionist, the sweet pietist, or the recipient of the 
beatific vision. That form of religion was not 
denied, but it became secondary where, for more 
than a thousand years, it had been primary. Faith 
in inspiration became second, and faith in Redemption 
and providence became first. Sanctity was approved 
in our calling, not outside it, not on Sunday, not in 
our closet. Men came into direct contact with 
the revealed God by faith. This faith became the 
acceptable, the justifying thing. It was the universal 
priesthood, and the priest and the monk fell at one 
stroke from being the idols to be the servants of 
the Church---useful, possibly, but not indispensable. 
Neither priest nor saint commanded the grace or 
forgiveness of God. Nothing human, nothing in the 
nature of an institution, must come between the soul 
and its Redeemer, whether it were the system as a 
Church or the system as a creed. The Church was 
the community of the faithful; not of the episcopal 
nor of the sacerdotal, but of the souls in direct con-TK
tact with the Saviour, and held to Him by the will's 
obedience and the heart's trust in the work of His 
Redemption. The Church was a witness, not a 
judge,---a medium, not a mediator; it might absolve 
but not forgive; it could convey a forgiveness which 
it could never effect. 



WHAT DID LUTHER REALLY DO? 149 

XII 

" Luther's central position was to Identify faith with 
the assurance of salvation." These are the words of 
the greatest of modern historians of theology whose 
further remarks I will venture freely to paraphrase.MK 

The point of Luther's breach with Catholic piety 
was this. That piety kept putting the question: How 
am I, a sinful man, to get power to do good works? I 
cannot please God unless I do them, but do them I can-TK
not to win my peace. To this question the Church 
gave Its own answer; and a long-winded answer It was. 
It constructed a tremendous apparatus of satisfaction. 
It took the sufferings of Christ, the sacraments, and 
the debris of human virtue, faith, and love; and from 
these it compounded a system through which the sin-TK
ful soul was passed, like the rags Into a paper mill, to 
come out, after a long and terrible discipline, white 
and pure at the other end. Luther began with a 
totally different question. He did not ask for power 
to do things that would commend him to God; he 
asked for such a commendation to God as would en-TK
able him to be the right man with Him, and to do the 
right things as a consequence of that. His experience 
was the soul certainty through faith, once for all, that 
in Christ he had a gracious God. He described with 

MK Harnack, DogriiengeschichtCj bk. IIL ch. iv. § 2. 



150 ROME, REFORM, AND REACTION 

mighty joy the experience which God's grace had 
made him pass through. He knew that all true life 
and blessedness, in so far as they were his, flowed from 
this certainty of faith. It was the source of his sanc-TK
tification, and all the good things he might do which 
were pleasing in God's eyes. For him the whole 
question about the relation of faith and goodness was 
simplified. He must grow in holiness. He must 
fight fearful spiritual foes with a most real and objec-TK
tive existence. And he must conquer. But when 
the battle threatened to go against him, when he felt 
he had no power in himself, when he must lay hold of 
some objective reality to withstand these real and ob-TK
jective foes, it was not at sacraments he grasped, not 
at the assurances of the Church, not at penances, and 
satisfactions, and merits of saints who had more than 
overcome. All these were not objects of faith, but 
reeds which grew on a shore he could not tread, and 
which broke in his desperate grasp as he was hurried 
on in his passionate way. When he flagged in his 
goodness, he grasped at the work and promise of his 
gracious God in Christ, and burst into the more pas-TK
sionate prayer, " Lord increase my faith." His assur-TK
ance that he was a saved man was not the sense that 
he had complied with the statutes of a Church, sent 
for the prescriptions of the priestly pharmacopoeia, and 
obeyed the advice of the Church's system of spiritual 



WHAT DID LUTHER REALLY DO? 151 

medicine. It was through his act of faith in the for-TK
giveness of God reaching him directly in the Cross of 
Jesus Christ. This was the Alpha and Omega of 
Luther's Gospel as of Paul's. The old confession of 
the Church was: where there is knowledge of God 
there is life and peace. But there was no clearness 
to the self-analysing and dim-seeing soul as to which 
knowledge of God was meant. Was it some future 
knowledge, philosophic knowledge, intuitive know-TK
ledge, mystical sacramental knowledge, knowledge by 
the Logos, knowledge by effort? On all these tracks 
men travelled and wandered, and the soul was still from 
home, weary, unsure, and desperate. Luther did not 
seek a knowledge, but found it given to his hand in 
God, in Christ, actually redeeming and reconciling him 
in his actual state of need. Where there was this for-TK
giveness and this faith there was life, and peace, and joy. 
This was the real nature of the breach with Ca-TK
tholicism that took place in the Reformation. It was 
not so much a new idea of the Church as a new idea 
and type of religion. It is the moral ideal of Protes-TK
tantism that is its grand distinction from Catholicism. 
It is not so much the theology, but the ethical quality, 
the spiritual habit, that divides them. And the moral 
quality, the spiritual habit of the English people is the 
one and not the other; to adopt the other would in-TK
volve a total change in our national characteristics, our 



152 ROME, REFORM, AND REACTION 

life ideal, and our religion and our place and function 
in the world. Catholicism is national suicide. I do 
not say political, but national. We should renounce 
not merely our prosperity, but our nature, our soul. I 
shall return to this. I would only ask here, What 
shall it profit a people if it gain the whole Church 
and lose its own soul? 

XIII 

Religion, then, is Faith. I state expressly here what 
I have often said in passing. Religion is Faith, and 
Justification by Faith is not a doctrine of Christianity, 
but its very nature and substance. The true sphere 
of religion is the sphere of faith. All that religion is 
able to do for love or hope can only be done as the 
development of what is in faith. Religion is not 
doing certain things, or obeying certain men, or lead-TK
ing a particular order of life. It is not ritual, not 
clerical, not monastic in its nature and genius. It is 
to be exercised in our natural and lawful calling in life, 
and especially in the trust of God's providence, and 
the service of our neighbour. It is the one thing 
pleasing to God and justifying to man. It was faith 
that redeemed, and it is faith that lays hold of redemp-TK
tion. It was Christ's faith that redeemed, and ours is 
but the trust of His. It is adaptable to every honest 
form of life---in marriage, the family, the state, in busi-TK



WHAT DID LUTHER REALLY DO? 1 53 

ness, in society, in affairs. The one divine service is 
faith. The one morality is trusting Christ as a life 
obedience. All morality is folded up in that and ex-TK
pands from it. Divine service is not ritual, not mystic 
contemplation, not asceticism. If, then, ceremonies 
in themselves avail nothing, either for the soul or God, 
the only sphere of faith is life. Faith is a mode of 
life, and heart, and temper, an attitude of these to-TK
wards Jesus Christ, a standing act and habit of will 
toward God. The moment you bind up with it any 
institution as an essential part of its object instead of 
a historical instrument, you have replaced Christianity 
by Catholicism, by the Church. You " bow down to 
your net and worship your own drag." You do as a 
nation does, when it worships the army, which is the 
law's instrument, above the law which should wield it 
and the people it should serve. 

The Protestant revolution was not primarily in 
Christian theology any more than it was primarily 
directed against the Church; it was a revolution in 
the religious type, in the idea of the perfect life.MK It 
was a moral and practical change. Catholicism breeds 
a different type of man from Protestantism---you might 
almost say a different type of face, certainly of con-TK
science. Luther revolutionized the Christian idea ot 

1 May I refer for detail to my little book on Chrktian Perfection? 
(Hodder \& Stoughton, 1899.) 



154 ROME, REFORM, AND REACTION 

perfection, of the perfect life, as no Christian had done 
since the apostolic age. Perhaps this was the most 
central effect of all. The new idea of faith as a life 
meant that with the supremacy of a new faith there 
came a new ideal of life. Perfection was no longer a 
thing ecclesiastical, or even saintly, but moral, religious, 
humane, worldly in the godly sense. Neither priest, 
monk, nor nun was the religious ideal, but the man 
and woman among men in Jesus Christ. It was an 
immense revolution; every new ideal of life must be. 
It reopened the world to religion, to the believer. 
The new world of America, discovered just before, 
was not so new or vast as the new world now opened 
to the human spirit. We might say the one was dis-TK
covered in order to be a refuge and a sphere for the 
other. Where would English faith have been without 
America to fly to? A vaster world dawned in all 
ways. There was more earth and more sky, a wider 
soil for a wider soul. The kingdom of God has 
something wider, humaner, more historic and pro-TK
found even than the Great Church. Nature itself took 
a new meaning and consecration. Marriage and the 
family took a new place, and ceased to be only the 
best thing for an inferior sort who were not equal to 
the altar or the cloister. Freedom took a new mean-TK
ing for the world and for nations as men were set free 
by faith and started on a new moral career. The 



WHAT DID LUTHER REALLY DO? 1 55 

future had a new light as men felt that they were 
redeemed from their past. The past itself ceased to 
be an accumulation merely or chiefly, a burden, a drag, 
a water-anchor on the race. When the kingdom of 
God and His righteousness were sought by faith in 
Christ, all else seemed added. Luther taught men and 
convinced men anew what true religion, true Chris-TK
tianity was; and in its wake came science, and the 
modern State with its civic and municipal life and 
social rights. The Church made the nation, especi-TK
ally this nation; but it was not the Church that made 
the modern State, and it would never have made it. 
Philanthropy became a public passion and a social 
duty, not the vocation of those who would be saints. 
It became an exercise of faith instead of an education 
for sanctity, the expression of the believer's love 
instead of the saint's ambition, an utterance of the 
Christian heart instead of an investment for the future 
of the soul. 

XIV 

Luther, I reiterate, rediscovered Paul and the New 
Testament. He gave back to Christianity the Gospel, 
and he restored Christianity to religion. But in giving 
us back the old he brought to pass the new age. He 
magnified the individual to himself, and so he opened 
a new world to the world. Catholicism was but half 



156 ROME, REFORM, AND REACTION 

of life. It is a maimed and unmanly thing in its type 
after all. Any creed maims and fetters humanity 
which makes personal religion but a part of it, and ties 
its religion down to an organization. The human soul 
cannot be completely organized and remain infinite 
and divine. It can use an organization, but it cannot 
be reduced to organization. It cannot be compre-TK
hended in any institution, any Church. But Catholi-TK
cism would so treat it; and the ideal is an outgrown 
Paganism, which the Reformation first broke. The 
ancient world reduced the soul to the State; the State 
was the supreme human interest. Catholicism did no 
more than apply the same Pagan and irreligious prin-TK
ciple to the Christian soul; it made the soul's supreme 
interest to be the religious state, the Church. The 
old pagan idea did not really receive its deathblow till 
the Reformation. The new age, the new human 
career, then first broke out of the old faith when 
Luther brought that faith to light. The human race 
has a treasure in the Reformation which it has never 
truly realized; how much more of treasure has it in 
the New Testament! In Catholicism the whole of 
the man was not claimed for religion, for faith, but 
only a side, a part of him. He had to be pruned 
down in order to find the one great way to God, not 
filled out. When a saint was made a man was lost. 
He had to be cloistered, monasticized, mortified. 



WHAT DID LUTHER REALLY DO? 1 57 

Whole fields of human energy had to be given up if 
mankind was to reach true holiness. But the Refor-TK
mation made the saint an active citizen of the world 
because he was so much more. Yet he was not the 
lusty natural man. His freedom was not in himself, 
but in the grace of the whole world's God, the 
Redeemer of the whole soul. What the Reforma-TK
tion brought for the new great age was not naturalism 
any more than it was monasticism. The natural man 
was broken in the cross and its faith, but the heavenly 
man that was made was free of all the world, and had 
the reversion of all its powers, and all its future. 
Modern engineering is as truly, though not as directly, 
a product of the Reformation and its moral courage 
as modern philanthropy is. The faith of the new 
movement infinitely enhanced the energy, the confi-TK
dence, the courage, the active power and joy of life. 
The world of nature became man's friend and ally 
where to the monk and his purity it had been damna-TK
tion. Man could master nature without being lost in 
it. Neither ancient Paganism nor its Christian form, 
Catholicism, ever had a principle that reconciled man 
and nature, soul and sense. Nature was either de-TK
clared by the mystic to be unreal, a mere fleeting 
show for our illusion; or it was reconciled with the 
spiritual by the priest, by a mere magical process like 
transubstantiation which carried with it no moral 



158 ROME, REFORM, AND REACTION 

power over the world. There was what is called a 
dualism in the Catholic and pagan idea of man and 
nature---an intractable, unreconciled dualism which 
meant a constant (though only half-conscious) irritation 
to the soul, and a constant leakage of its power at the 
bad joint. Marriage, for instance, was not a sacred 
thing in itself; it was only made sacred by the bless-TK
ing upon it of the Church. To separate from the 
Church was to put a stain and a ban on the continu-TK
ance of the race. An unchurched race was a cursed 
race. Nature was not hallowed by Christ's redemp-TK
tion, except in so far as that was dispensed in the 
Church's blessing. To this dualism an end was made 
by the great simplifying principle of Justification by 
Faith alone. The world is a redeemed world; and 
Nature, the redeemed servant, waits, longs to be used 
by the son of the house, the man whose manhood 
and whose mastery are made by the same redemption. 
When the great spiritual process was removed, as 
faith moves it, to the interior of the man's spiritual 
will, the new relation to God brought a new and 
trustful relation to His world. Nature was no more 
Satanic, lurking for chances to undo the soul. It was 
included with the whole creation in the same great 
final redemption to which the soul owed itself. The 
immense new strength with which the soul came out 
of its awful struggle with the ultimate powers of 



WHAT DID LUTHER REALLY DO? 1 59 

spiritual being greeted the vast powers that played in 
the natural world, and it knew itself their lord. For 
Catholicism, with its starvation of the soul's power, 
and its substitution of the calm of mere order for the 
peace of power, this was impossible. Catholicism, 
tied and galled by the absolute rule of a Church 
institution, like Gulliver by the Liliputian pack-TK
threads, could never let the human soul find its feet on 
faith. And Catholicism, if it were to return on Eng-TK
land, wMKould in course of time reduce it from the most 
free, adventurous, powerful and righteous nation on 
the earth to the timid, vainglorious, petulant, cruel, 
pleasure-loving and bankrupt race which it has made 
Spain. Catholicism would do this---not popery merely, 
but Catholicism, which hampers the soul by the wor-TK
ship of an institution, debases it by the prying of the 
priest, enfeebles it by the priest's false promise to take 
the responsibility of its fate, and prunes down energy 
by an incessant and suspicious vigilance against every 
new departure that takes the soul beyond the Church's 
right, reach, and control. Catholicism is the sacrifice 
of the soul to an institution; Protestantism is the 
soul's release for an institution. And the issue is 
this, is the soul for the Church, or the Church for the 
soul? 



l60 ROME, REFORM, AND REACTION 

XV 

If we take Catholicism as religious institutionalism, 
its most serious danger to society is the moral one. It 
affects the standard of honesty, then of honour, and it 
becomes Machiavelism. The conscience was never 
meant to have for its authority anything in the nature 
of an institution, but only a person to whom its 
relation is faith; and if for this person is substituted 
a system of any kind, not all the good and gentle-TK
manly instincts can prevent the conscience from 
ultimate perversion and decay. 

It has often been pointed out how the Catholic 
movement affects the quality of religionMK how religion 
tends to sink under its influence by ceasing to be ex-TK
perimental and personal. It might be shown how the 
very self-searching of the confessional destroys the 
real power of self-examination, and cultivates a levity 
in regard to the nature of sin by an excessive atten-TK
tion to the numbers and the penalties of sins. This 
decay of real experimental religion (which is but a 
roundabout way of saying faith) is really a decay in 
the sense of sin. It is not a decline in the notice 
sin receives, but it is a debasement of the idea of 
sin by the intrusion of a wrong standard. That is 
sin which the Church declares to be so; what is 
not so declared is not sin. Such at least is the lay 



WHAT DID LUTHER REALLY DO? l6l 

and popular inference. And it is in the lay mind 
that this religious mischief from Catholicism chiefly 
takes effect. The effect on the clergy, we shall see, 
is different. The effect on the laity is the decay of 
experimental religion. While the Broad Church 
tends to reduce sin to a mere ethical phenomenon, 
the Catholic tendency is to treat it aesthetically, or, 
what is the same thing, institutionally. It is what the 
conscience looks at rather than feels, and it measures it 
by an external standard supplied rather than realizes 
it by its own sensibility. The moral product of the 
Church system is the canonical conscience, which has 
its representative in what I have already alluded to as 
the narrow and inhuman sincerity of a man like Laud. 
To such a conscience sin is a very different thing from 
what it is to the Evangelical conscience, and far less of 
a religious thing. It becomes a social enormity. If 
an institution lace the whole sky through which God 
looks on the soul, it is inevitable that offences against 
God should be chiefly construed in an ecclesiastical or 
social way. The standard in a Catholic Church, 
especially when it is an Established Church, bound 
up with the social conventions of the country and 
its ruling class, becomes mainly conventional. The 
traditional social code becomes interwoven with the 
traditional ecclesiastical code, and both come for the 
public to form the standard of moral judgment, and 

II 



l62 ROIMKIE, REFORM, AND REACTION 

even of such self-examination as can survive so hostile 
an air. The sense of sin becomes feeble, and the 
tone of religion outward and shalloviMK. Ideals fall, 
and the existing Church becomes the best Church. 
It ceases to be thought of except as a branch of 
the national service, or a part of the social fabric. 
Genuine wonder is felt that any should regard it 
otherwise. So to view it seems an act of national 
treason, and hostility to society. This is, of course, 
the Pagan idea; and we can easily understand how 
persecution arose, not as an attack on religious views 
or practices in themselves, but as acts of self-preserva-TK
tion on the part of society against what was supposed 
to be an assault on its organic existence. Village per-TK
secution still is more social than religious in its inspira-TK
tion. And " the plea for a State Church," says Dr. 
Dale, " draws its force from the disposition of men to 
think of the Church as being nothing more than a 
great human organization for maintaining Christian 
learning and propagating religious truth, or for civilizing 
mankind and improving the morality of nations." That 
is to say, there has come to pass, through the ecclesi-TK
astical and the political Catholicizing of the Church, 
through its institutionalizing (if we may use the word), 
a fatal severance between the idea of the Church and 
the idea of Redemption. And direct faith is dissociated 
from the personality whose contact with us is the 



WHAT DID LUTHER REALLY DO? 163 

real source of the due sense of sin. It we ask indeed 
why England is not Pagan to-day, the grateful answer 
must be: Because of the Church. But if we go on 
to ask why she is but half Christian, the answer, if 
critical and honest, must still be: Because of the 
Church, and especially because of the Establishment. 
But there is another effect of the growth of 
Catholicism or religious institutionalism, which I 
said above was the more serious. It is the decay of 
the sense of honesty. And it is the form which 
most affects a clergy. By dishonesty is not, of 
course, here meant conscious turpitude, but such a 
sophistication of the moral perceptions that men 
come with elaborate sincerity to allow themselves in 
positions and practices which are open to the censure, 
not of the Christian conscience merely, but of the 
rude integrity of the world. It is not easy to resist 
this impression after the publication of Mr. Walsh's 
book. It certainly cannot be denied that a very 
great change has taken place in the conscience of 
the English clergy in the last half century. The 
Broad Church treatment of the formularies has often 
exposed the clergy to the criticisms of business men. 
But however preparatory this may have been for a 
more advanced stage of sophistication, it is nothing 
compared with the effect in that way of the prin-TK
ciples of Tract MK90. And all this is the inevitable 



164 ROME, REFORM, AND REACTION 

result of Institutionalism. It is Machiavelism. It 
is the erection of the canonical conscience in the 
place of the Evangelical. Whether it be the eccle-TK
siastical canons of Laud or the political canons of 
Bismarck that are enthroned, the eflfect of the canon-TK
izing is the same. It is moral sophistication, due to 
placing over the conscience a bureau viMKhere there should 
be a spiritual king. And the reaction against it is either 
the Reformation, or what is to-day called the Noncon-TK
formist Conscience. There have been extravagances 
in both, and, of course, to an institution-worshipper 
extravagance is the sin of sins. But none the less 
they have been, and are, the self-assertion for each 
age of that Puritanism, with its living faith, which 
is the nerve of vital godliness, and the conservator 
of moral progress in public and private life. 

The whole of Europe is suffering from this in-TK
stitutional and Machiavelian strain, this corruption 
of conscience by empire, political or ecclesiastical. 
In the modern enhancement of human force, freedom, 
and passion the need is felt for some strong outward 
authority, which the general decay of faith yet for-TK
bids to be of a truly spiritual nature. Vast organiza-TK
tions are called in to govern a human nature which yet 
was constructed and redeemed to be governed only by 
the unseen King enthroned in moral faith. De-TK
frauded of its true Sun, the conscience pines, 



WHAT DID LUTHER REALLY DO? 165 

shrivels, or dies. Its voice is silenced or warped. 
It becomes the tool of a visible organism wMKhich gives 
it its law, instead of the judge of a society whose law 
it should prescribe. And what is that but Machia-TK
velism, which justifies all things in the name of an 
Institution held supreme? It might be the army, as 
in France, where in the collapse of conscience even 
the sense of honour becomes criminal. It might be 
the State, as in Germany, where the Emperor seems 
to have no moral authority higher than his dynasty, 
and revives, in the name of a kind of theistic ortho-TK
doxy, the ancient Paganism of the worship of 
the State. The Machiavelism of Bismarck was open 
and avowed. All things were lawful which promised 
to subserve the interests of the State. Such ethic is 
more antichristian than any orthodoxy can redeem. 
It is the same thing that is expressed in the Socialism 
which is the enemy of the bureaucratic State. For 
the Socialist of the programmes all things are lawful 
which work the programme out, and the individual 
conscience has no more stand against the social State 
than the Emperor thinks it should have against his. 
Society takes the control of the conscience in the 
one case as thoroughly as the prince does in the 
other, and with less room, on the whole perhaps, 
for freedom than when the control is taken by the 
Pope. It is needless to remind you how, in the 



l66 ROME, REFORM, AND REACTION 

Roman system now developed in the hands of the 
Jesuits, the Church as the religious society or institu-TK
tion claims a divine right to the control of the con-TK
science in her oviMKn supreme interest. The Church, 
for its members, becomes the conscience of the priest, 
wMKith results vMKMKhich in this country do not clearly 
emerge because of the corrections of a healthier 
moral inspiration. Jesuit ethics are the greatest 
system of moral Machiavelism that the wMKorld has 
seen. And vMKMKherever you have the spirit of ecclesi-TK
asticism winning the upper hand you have the like 
moral results in proportion. You have crooked and 
secret methods. You find done by well-bred men, 
and men of no bad feeling, on behalf of the Church 
things that they would not do as private gentlemen. 
You have men, who claim in Ireland that law should 
be obeyed loyally while it is law, going on to School 
Boards with the avowed purpose of hampering, if not 
neutralizing, the Education Act. You have the 
highest dignitaries capturing not only the schools 
for their Church but the charities which were left 
either expressly for another communion or expressly 
for undenominational purposes. You have gentle-TK
manly men and their kindly women, whom it is a 
delight to meet in their own drawing-room, descending 
to acts of contemptuous persecution against the godliest 
of their Christian neighbours because of their crime as 



WHAT DID LUTHER REALLY DO? 167 

Dissenters. It is hateful to speak of these things ex-TK
cept as samples and as illustrations of the moral effect, 
especially on the clergy, of that institution-worship 
which is the soul of Catholicism and of Machiavelism 
alike. The constant tendency of Catholicism is 
toward Machiavelism. It is religion debased to a 
polity instead of using a polity, and being free to use 
a variety of them according to the discretion of faith. 
It is the debasement of empire infecting the great and 
sacred society which came into the world to save it 
from the condition to which the empires had brought it. 
It is the sophistication of the conscience by a system 
which came to save men from the sophistries into 
which all the systems had fallen. It is the capture 
of that inward freedom which came to be the guiding 
power of human freedom in every form. And the 
awful Armageddon which awaits Europe sooner or 
later will be due to those perversions of the conscience 
in Church and State which chiefly arise from taking it 
into the pay of an outward authority and institution; 
whose pay it must one day spurn and whose control 
it must disown if it is to remain human, progressive, 
and free. 

XVI 

Let us speak of England's national life and future 
alone. A leading statesman not long ago said that 



l68 ROME, REFORM, AND REACTION 

the secret of the British Empire lay not in the com-TK
pleteness of its constitution, nor ir) the omnipresence 
of its crown, nor in the ubiquity of its Parliament. 
For its constitution is full of illogical contradictions 
which are a working success; its crown has a very 
limited action at home, and a far more limited action 
abroad (except in the way of sentiment); and the 
arm of Parliament is a very short one when it is a 
question of action at the other side of the world. 
But the secret of the Empire is in the men whom 
Britain sends forth in their freedom, courage, mastery, 
and wisdom, in the resource and the responsibility de-TK
veloped by their having to act alone, without instruc-TK
tions, and without immediate supervision. It is not 
the English Parliament nor the English Constitution 
that is felt in the English proconsul on the skirts of 
the Himalayas, but the English man. All that is true. 
An empire like ours could not hang together for a 
century ruled simply as a magnificent and compact 
organization, and worked like a gigantic post office. 
But what does that mean? It means that our power 
is in its nature and genius Protestant and not Catholic, 
that its salvation is the development of individual 
resource and responsibility; that its doom would be to 
settle down into mere officialism, to set up the priestly 
idea of responsibility for the Protestant, and to regard 
the ideal Englishman more as a machine to obey 



WHAT DID LUTHER REALLY DO? 169 

orders than as a living moral centre of freedom, con-TK
fidence, and power. Make our religion Catholic, and 
above all things institutional, and in due time you 
reduce English enterprise to something in the nature 
of a Jesuit mission, the Englishman abroad to a political 
cleric, the merchant to a retailer, and the great firm 
to the spirit of a tied house. That wMKould in course 
of time be the result, if the type of English faith 
ceased to be Protestant and became Catholic. Our 
ideal of life would be ruled by the type which is pale, 
flat, meagre, and timid in the many, however am-TK
bitious, grasping, and domineering in the few. The 
type of pope and priest would stand out upon the 
' slavish moral complaisance of the many. For it would 
be an article of faith to bow to the priest as a part of 
the soul's homage to God, to think of the priest as a 
minor god. And to the soul's faith both in itself and 
in God that is fatal, and it has been shown by the 
atheism of Catholic Europe so to be. 

XVII 
So do not think, when we speak of Justification by 
Faith alone, that something is meant which is in-TK
telligible only to those who are interested in theology. 
If we must be theologians to be Protestant, Protest-TK
antism is not what the world needs in the way of 
religion; it is not evangelical. But Justification by 



170 ROME, REFORM, AND REACTION 

Faith is a great moral and spiritual principle. It is not 
what should be called a mere doctrine; it is a prin-TK
ciple, type, way, and ideal of life. You must live on 
this principle or on its opposite, if you live a religious 
life, or any worthy life at all. 

For Justification by Faith means three things of a 
very practical sort in our judgment of life: 

1. It means that the worth of a man is to be 
measured exclusively by his moral and spiritual quality 
of soul, by his heart and character, by his direct faith 
in a moral and spiritual God; and not by his relation 
to any institution whatever, or his correctness in any 
creed. A man is to God not what he is to any Church, 
but what he is to God's real Word, will, and presence 
in Christ. 

2. If this quality of soul, true faith, has the right 
object in a living Christ, it is bound by its very nature 
to take outward shape in hopeful and tireless moral 
energy, in righteous love and pity to other men, and 
in a Christian fellowship which is the sign and not the 
condition of faith. How can a faith which is personal 
contact with the Redeemer be any but a faith of 
practical justice, goodness, help, and blessing. 

3. The value of the highest work does not depend 
on the form it takes or the results it wins, but on the 
faith which inspires it. All the energies of life are 
justified so long as they are capable of having this faith 



WHAT DID LUTHER REALLY DO? 171 

put into them. They are not concreted by a Church 
which blesses them, or a priest who searches them 
and absolves, but by the spirit, motive, faith of the 
man who does them. In politics we are justified by 
results, in faith by motive. Law must regard actions, 
but faith regards souls. And to judge souls we must 
measure motives, and the motive of all motives is faith, 
as the test of all standards is Christ. Art itself is 
chiefly determined as great by its subject and not by 
its manner, by its content even more than by its form, 
by its faith more than by its technique, by its ideals 
even more than by its works. 

XVIII 

These are the principles of the modern man in his 
best and largest and humanest sense. They are the 
moral principles of modern civilization. The supremacy 
of faith means the supremacy of character. In Catholi-TK
cism character is there for the sake of the Church; in 
Protestantism the Church is there for the sake of char-TK
acter. In Catholicism character is trimmed down to 
one type, dominated by the saintly ascetic; in Pro-TK
testantism it is developed on individual and national 
lines, without the shadow of a universal institution 
which erases national features in its uniformity of type. 
In Catholicism we have a hu2:e International which 
levels the nations under one uniform Church; in Pro-TK



1/2 ROME, REFORM, AND REACTION 

testantism, with the flexibility of faith, we have an 
International which develops the nation's native char-TK
acter as it does the individual's. Catholicism makes 
the nations tributary to itself; Protestantism makes 
them contributory to each other. In the one Faith 
rests on the Church, in the other the Church rests on 
Faith. In the one the Church is primarily the clergy, 
in the other it is the believer. In the one Faith means 
practically faith in the Church; in the other it is faith 
in Christ. In the one it is faith in what Christ is said 
to have appointed; in the other it is faith in what 
Christ in His person was and did, is and does. In the 
one it is faith in the grace that Christ spends; in the 
other it is faith in the grace that Christ is. In the one 
the work of Christ was to make the Church possible 
among men; in the other His work was to make man 
capable of a Church. 

Luther believed in a Church, in a Church as founded 
by grace alone, in grace as mercy and not sacramental 
infusion, in grace as the Gospel, and in the Gospel as 
Christ Himself. Faith as the answer to revelation is 
the soul in direct contact with Christ crucified, not 
as the condition of grace but as Himself, the living, 
potent, omnipotent, ubiquitous, eternal grace of God. 

This is the faith of the New Testament, of the 
Gospel. It is not the faith of Catholicism, which is 
not the Gospel. In this faith let us stand. To do 



WHAT DID LUTHER REALLY DO? 173 

anything else is spiritual suicide. And indifference to 
the issue is one of the ways to this death; for you can 
kill yourself by a narcotic as surely as by a poison 
more acute. 



THE REAL NATURE OF CHRISTIAN 
PRIESTHOOD 



IV.— Part I 

THE REAL NATURE OF CHRISTIAN 
PRIESTHOOD 



There is nothing more earnestly desired by Chris-TK
tian men than the unity of Christendom, either in 
inward spirit or in outward form. And there is but 
one obstacle in chief which pushes in and forbids 
union. It is the priest. Between Catholic and 
reformed Christianity the priest is the real bone of 
contention. Between Anglican and Free Churchman 
the issue is the priest. It is a struggle on the one 
hand between the priest and the family; the priest-TK
hood means celibacy, and it means confession, and 
each of these is an assault on family life. The celibate 
priest means that marriage is on an inferior stage; 
and the confessor priest means his intrusion between 
the most intimate and sacred moral ties. It is a 
struggle, on the other hand, between the priest and 
the minister,---between the minister as a medi-TK

177 12 



1/8 ROME, REFORM, AND REACTION 

ator, and the minister as an instrument; it is a 
struggle between the minister as a man, and the 
minister as something more than a man---which is 
in effect less. Catholicism, I have said, is that form 
of Christianity which raises an institution to an object 
of faith, and makes it essential to salvation. This 
institutionalism culminates in the priest. Catholicism 
is some form of priestism. In its extreme forms it 
not only makes the priest essential to the Church, 
but identical with it. Protestantism, as one has said, 
either abolishes the priest or multiplies him. It 
makes all priests or none. The priest means Juda-TK
ism; and his reign means the relapse of Christianity 
into the religion it left behind. It is a reversion, 
which means degeneration. Something is seriously 
wrong with the principle, when we find the tendency 
with the priest, in practice, so steadily towards insolence, 
contempt, intrigue, and persecution, petty or great. 

The priest was not in the apostolic faith, and he 
did not spring up in a night. He grew upon the 
Church, " built, like one of our cathedrals, through 
generations, in pieces, at long intervals, the develop-TK
ment of a logic slow but sure from the false start." 

The whole of this conception of an outward, 
ruling, and vicarious priesthood is a corruption of 
the Christian idea. It is a later importation. It is 
not in the New Testament. So far as human priest-TK



REAL NATURE OF CHRISTIAN PRIESTHOOD 1 79 

hood goes, there is nothing in the New Testament 
but the spiritual and inward priesthood of all be-TK
lievers-TK— the universal priesthood of justification by 
faith. The justified are priests. The whole Church 
is a royal priesthood, a kingdom of priests. The word 
" priest " is deliberately avoided as a name for the 
Christian president or minister, though the air was 
full of it, and there was no religion in the world that 
the New Testament knew but called its ministers 
priests, and treated them so. Neither the name nor 
the thing is in the New Testament. It was too 
jealously monopolized for the person and work of 
Christ. The Church could have but one Priest, 
as the bride of Christ could have but one Spouse. 
Not one of the Apostles was a priest in this official 
and vicarious sense. They exercised neither mass nor 
confessional. They preached forgiveness, but they 
did not dispense it. Paul's forgiveness, in 2 Corinth-TK
ians ii. 10, follows on that of the Church, "in the 
sight of Christ," not " in His name." The absolving 
power belonged to the Church, and it was not exercised 
in an express and formal way, but by the spiritual and 
releasing action of the Church's practical influence on 
the world's soul. Christ was no priest in this Catholic 
sense. His affinities were with the old prophets of 
Israel more than with the priests. These became His 
enemies and murderers; and it was not because they 



l80 ROME, REFORM, AND REACTION 

were bad men, but because they were, before all else, 
officers of a monopolist institution---a Church. For 
some time the Church held this priestless faith of the 
common priesthood. Tertullian says (about 200 a.d.), 
" Where there are three there the Church is, if they 
be but laymen." And Augustine says, " All are 
priests as members of the one Priest." And many 
similar passages could be quoted from the Fathers of 
the Church. 

But, meantime, the Pagan influences of the Roman 
world were at work in the Church. As it mastered 
the world outwardly, the world was corrupting it 
inwardlyMK The heathen idea of priesthood returned 
on the pure Christian faith like a tide; and this tidal 
force was aided, though it was not originated, by the 
strong current setting in the same direction from the 
Old Testament. In the third century there arose a 
powerful and thoroughgoing man who gave effect 
to these influences, and fastened the magical and 
theurgic priesthood upon the neck of the Church 
from that time forward. I mean Cyprian, Bishop 
of Carthage, the Laud of the early Church. He did 
for the priest what Ignatius, in the second century, 
had done for the bishop. The official and dominating 
priest from henceforth pushed the universal priesthood 
in practice out of sight. Yet it could not be entirely 
slain; it was bound up too closely with the vital 



REAL NATURE OF CHRISTIAN PRIESTHOOD l8l 

nature of Christianity. So long as the Church re-TK
mained Christian at all this principle was bound to 
struggle for life and scope. And so late as Thomas 
Aquinas, the greatest of all the Catholic theologians in 
the thirteenth century, we have this: " A good layman 
is joined to Christ in spiritual union by faith and love, 
not by the sacramental power; so he has the spiritual 
priesthood for the offering of spiritual sacrifices." We 
shall see that, even at the Council of Trent in the 
sixteenth century, part at least of this idea remained 
and received expression. But it was only a theologi-TK
cal expression. The practice of the Church had 
hardened considerably, as we shall note. In practice 
the official priesthood had submerged the universal, as 
in the Roman Church it does at this day; and the 
only real and effectual assertion of the believer's 
priesthood in Christendom was, and is. Protestantism. 
The Reformation was the rescue of the universal 
priesthood of the Church from the official. And it 
found its only safety in doing what the New Testa-TK
ment writers had done---in banishing the name 
" priest " as the title of the Christian minister. The 
Anglican Church alone, with its want of earnestness 
and thoroughness, with its lack of spiritual " lucidity " 
(as Matthew Arnold would have called it), retained it; 
June nice lacrymce; we have the troubles of to-day. 
We have that most ominous breach in a Church---



l82 ROME, REFORM, AND REACTION 

between its clergy and its laity. We have the clergy 
in the main Catholic, the laity in the main Protestant. 
We have each side representing an order of faith which 
is incompatible with the other. This must be a far 
more serious thing than the existence of two such 
parties in the Church. And the explanation of it, 
as we shall find, is that the Anglican priest, while 
resting on a representative theory of the ministry, yet 
applies it in such a roundabout way that only some of 
the ministers themselves grasp it, and none of the 
laymen. The authority of the apostolic succession 
through bishops is dragged in, and qualifies the repre-TK
sentative nature of the priest in such a way that the 
lay priesthood cannot feel itself represented in the 
official priest at all. In our elective ministry it 
does. 

II 

But supposing we keep the name as describing the 
nature and privilege of every Christian man, the 
question I would ask is. What is the real nature of 
Christian priesthood? What is the nature and mean-TK
ing of the priest for us---for us of the Free Churches? 
Whatever is the real nature of the priesthood is some-TK
thing which belongs to the Church as Christian, and 
not merely as sectional. It belongs to all Christians. 
When we reject the Catholic priest, we do not reject 



REAL NATURE OF CHRISTIAN PRIESTHOOD 1 83 

the priesthood. How can we? It is a priest that 
we worship in Christ. The Church, as the body of 
Christ, mustMK in some sense express His priesthood. 
Priesthood, as the Roman catechism truly says, is the 
highest dignity on earth. It must be so if it was the 
great function of Christ. Priesthood, rightly under-TK
stood, is the true seat of authority among men. 

I shall begin with the admission that the true 
Church is in its nature sacerdotal. That is a truth 
which many of us have entirely lost, and we owe much 
to the present High Church movement for forcing it 
home upon all the Churches alike. To the loss 
of it is due most of our failure to reach and in-TK
fluence the world. It is priesthood that saves the 
world---the priesthood of Christ, and the real fellow-TK
ship of it by the Church which His priestly act 
founded, and in whose action its High Priest lives 
for ever. The Church which the Great High Priest 
inhabits and inspires must be a priestly Church. 

The confusion is caused when we cease to think 
that the Church is a priesthood, and begin to think 
that it has a priesthood. It is like the error the evan-TK
gelicals make (so full of practical mischief to religion) 
when they say that man has a soul, instead of saying 
that man is a soul. 

The main question is whether the essential priest-TK
hood of the Church is confined to a certain order of 



1 84 ROME, REFORM, AND REACTION 

Christians. Is the Church a priestly Church because 
it possesses this order? Is priesthood confined to the 
priesthood? Did the Church begin with priesthood 
or with the priesthood? Had the first Christians 
priests in the official sense, and did the Church 
spread outward and downward from them to an in-TK
ferior grade of laymen? Or were the first Christians 
priests only in the universal sense, and did the priest-TK
hood arise from that as a ministry, as a mere matter of 
order, agency, and convenience? Is the priesthood a 
matter of an order, or of order merely? Is the whole 
Church historically an expansion from an official priest-TK
hood, or is an official ministry a projection of the 
universal priesthood, as an organism for a particular 
purpose throws out a limb? Was the existing minis-TK
try of the Church devolved from ministers appointed 
and endowed by Christ with unique powers and 
privileges, or was it evolvedMK in a historical way, by the 
Spirit-led Church itself, to meet the successive needs of 
the hour? These are questions which the Church 
must face and solve for its life. They are not academic, 
and not antiquarian. The great Christian issue of 
the hour turns on the conception of the Christian 
ministry. The brunt of the battle does not fall on the 
pews, but on the pulpit. It is not your place in the 
Church, but ours, that is in question. The issue put 
before you is not what place you claim for yourself 



REAL NATURE OF CHRISTIAN PRIESTHOOD 1 85 

In the Church, but what place you claim for your 
minister. As your minister, has he a place and right 
to his office in Christ equal to the officer of any Church 
in the world? Is he as truly a servant of Christ's 
priesthood, and a waiter upon Christ's sacrifice, as those 
who stand by any altars in any Church? If you truly 
understand your Christian place and duty as members 
of Christ and His Church, you assert for your minis-TK
try a right to minister Christ in all the fulness of 
His blessing, which is not exceeded by the ministry 
of any Church on earth, and your minister, as minister, 
meets every other on equal terms. That is your 
claim, the very meaning of your ecclesiastical exis-TK
tence as Free Churches. Make it courteously, but 
make it plainly; and, give it to be understood that 
when your minister makes that claim for his office, it 
is not his own claim he makes, but yours. If he is 
no true minister, then you are no true Church and no 
true Christians. The minister is what the Church is. 
He is a priest only in so far as he represents the essen-TK
tial priestliness inherent in the Church; and the 
Church is priestly only in so far as it can represent 
the cross and sacrifice of Jesus Christ. 

Ill 

Is the priesthood and ministry of the Church a 
vicarious priesthood or a representative? Christ's 



1 86 

priesthood was vicarious. It did for man what he 
could never do for himself. It was not representa-TK
tive. It was not simply doing in a signal fashion what 
an ideal humanity does on the scale of the whole race. 
Of which nature is the priesthood in the Church? 
Does the minister of the Church do for the people 
with God what they cannot do for themselves? or 
does he only act on the Church's behalf, and fulfil 
conveniently a function which the Church really does 
through him? Is the priest chiefly and directly the 
organ of God to the Church, or the organ of the 
Church to God? Is he, then, to repeat the sacrifice 
of God, or to lead the sacrifice of man in Christ? Is 
he the dispenser of a sacrament or of a gospel? Is 
he a mediator or an instrument? 

These are the questions to be met; and, in dealing 
with them, do not make the mistake of thinking that 
Protestantism stands for the universal priesthood alone, 
while Catholicism stands for an official priesthood 
alone. Even Roman Catholicism recognises a uni-TK
versal priesthood of all the faithful as well as Protes-TK
tantism. Do not be puzzled if you hear a Catholic, 
while making exclusive claims for a sacerdotal order, 
insisting also on the priesthood of all believers. In 
the Catholic Catechism of the Council of Trent there 
is express mention made of the double priesthood. 
There is an inward and an outward. " All the faith-TK



REAL NATURE OF CHRISTIAN PRIESTHOOD 1 87 

ful who are baptized are called priests. Especially so 
are those good men among them who have the Spirit 
of God, and by the kindness of God's grace are made 
living members of the High Priest Jesus Christ. Such 
men, by a faith inflamed with love, offer spiritual 
sacrifices to God on the altar of their souls; and to 
these sacrifices belong all good and honourable deeds 
which tend to the glory of God." "Thus," quotes 
the Catechism, " Christ made us a kingdom and 
priests to God and His Father by washing us in His 
own blood. We are a holy priesthood, offering spirit-TK
ual sacrifices acceptable to God through Jesus Christ." 
All such texts refer to the inward priesthood. " But," 
the Catechism goes on to say, " The outward priesthood 
belongs not to all the faithful, but only to certain men, 
who are instituted and consecrated by the imposition 
of hands and the due rites of the Church to a specially 
sacred ministry. And the power of this outward 
priesthood is the power of offering to God the great 
sacrifice of the Church for the living and the dead---
the Mass." 

I have given you what is virtually a translation from 
the Latin of the Roman Catechism (ii. 7, 23). And 
I have done so that we may be quite fair to 
Catholicism, and may see that it does not deny a 
priesthood of all believers. The words in which it 
describes that priesthood are admirable; and they 



l88 ROME, REFORM, AND REACTION 

remind us well that the good and noble deeds of 
Christian men are more than noble and good---they 
are sacrificial and priestly acts offered to our spiritual 
God upon the altar of our soul. We have in our 
soul and self an altar whereto they have no access who 
merely serve the outward tabernacle of Humanity; 
and our Christian life is a most real priesthood. But 
we must recognise the following things in order also 
clearly to understand what our Protestantism means. 

1. And, first, I ask you to notice that in this state-TK
ment the priesthood of all believers is not theirs in 
virtue of their faith, but in virtue of their baptism. 
The faithful, even if devoted men, are called priests 
only after they have been baptized. The inward 
priesthood is constituted by the outward rite; and 
the outward rite is in the hands of the outward priest-TK
hood. 

2. So that, after all, it is not the inward priesthood 
which is supreme in practice, but the outward. It is 
not faith that constitutes true priesthood, but only the 
faith of the baptized, faith which has been made pos-TK
sible by a rite, and which is at the mercy of that rite 
and of those who exercise it. The priest has a power 
over the believer, which is not given by the soul's 
spiritual faith. Faith is not its own justification. We 
are not justified by faith, but by faith which is made 
possible by a rite of the Church, an ordinance, a work 



REAL NATURE OF CHRISTIAN PRIESTHOOD 1 89 

of the law. The spiritual value of faith is conditioned 
by a theurgic act in baptism; the higher gets its 
value from the lower, the inward from the outward, 
the moral from the magical. The clergy are the real 
mediators of the true priestly life, and in their priest-TK
hood the laity have no part. Moreover, that God may 
accept these good and noble acts of the lay soul, there 
is needed a propitiatory sacrifice, a sacrifice offered in 
the Mass, which is the privilege of the outward priest-TK
hood alone. And, further, that it may be believers 
who offer these lay sacrifices, their absolution is con-TK
tinually required, which absolution, again, is the func-TK
tion of the priest alone. 

3. But the most serious remark on this distinction 
of the two priesthoods is this: It is not essential that 
those who have the powers of the outward priesthood 
should have the grace of the inward. The power of 
the outward priesthood is not derived from personal 
faith, love, or sacrifice, but from ordination, from the 
due institution by the hierarchy. The priest is not 
the holiest man, but the correctly appointed man; he 
is not the truly consecrated man, but only the duly 
consecrated. The virtue and validity of the sacra-TK
ments are not affected, if it be afterwards found that 
the priest has been living in mortal sin. The most 
sacred and powerful position in the Church is not the 
holiest. Power and sanctity are disjoined. The 



190 ROME, REFORM, AND REACTION 

priesthood that gives the Church its priestly character 
is not the priesthood of sanctity, but only of function. 
This is Catholicism; this comes of making the 
essence of the Church an institution instead of a 
Gospel, a rite instead of a faith. 

The two priesthoods have, in fact, nothing in 
common except the name. They are not in essential 
and spiritual connection. The cleric is above the 
Church; he becomes the Church; he is described as 
a god. He draws his official power directly from God. 
He is the sole medium of grace for believers, who 
become and remain such only through the sacraments 
in his hands. And yet he need not be a personally 
holy man. 

The evangelical position is a very clear antagonism. 
The spiritual office is a projection of the universal 
priesthood. It is an organ of the Church, and not 
Christ's vicar in the Church. The priestly cha-TK
racter of the Church is not given by the priesthood, 
but to it. It has no mediatorial place, as the Church 
has but one mediator with God---Jesus Christ. It 
exercises no functions that do not belong by right to 
every Christian believer; only for the sake of order it 
exercises them in a definite area. It is the faith of the 
Church that acts in the minister, and it may act 
through any member as minister for the time and 
occasion if the Church so will. The ministry is a 



REAL NATURE OF CHRISTIAN PRIESTHOOD I91 

mandate from the Church to act on its behalf and in 
its presence. The minister should not baptize where 
there are not enough present to make a small congre-TK
gation; nor should he administer the Lord's Supper 
to an invalid alone without two or three in the room. 
I bid you note particularly that the minister is the ex-TK
pression not of the individual's priesthood, but of the 
priesthood as universal, of the priesthood of the Church. 
It is the commission of the Church that he holds, not 
of individual faith, not of his own. If it were the 
latter, each member might claim the right to exhort 
and rebuke the whole Church, and pray in the whole 
Church, whether the Church asked him to do so or 
not. And that means an anarchy which ruined some 
of the Independent Churches of Scotland last century, 
which were not Churches at all, but groups of in-TK
dependent individuals. The minister is the mandatory 
of the priesthood of the whole Church, and not of 
isolated believers, not of his own faith alone. He 
must preach the Church's faith, even when his own 
is low, so long as it is not dead. 

The Church conveys its rights and duties to the 
incumbent in trust, to exercise them on the whole 
Church's behalf amidst a particular community. He 
represents there the functions of the universal priest-TK
hood. What are they? The minister enters publicly the 
presence of God j but that is every Christian's right 



192 ROME, REFORM, AND REACTION 

as priest. He offers sacrifice, as it is every Christian's 
right to do, surrendering himself to God, body and soul, 
for the brethren, and bringing especially the fruit of 
the lips. The minister is the channel for others of 
God's grace in the Gospel; every Christian has the 
right and duty to be the like channel of the Gospel to 
his neighbours, whether he do it in word or in con-TK
duct, or in the special helpfulness of brotherly love to 
those who do not know how to claim their own rights 
to the same God. Christian philanthropy is a function 
of the universal priesthood. It is offering ourselves, 
our hearts and bodies, to Christ in His poor and His 
prodigals. 

If you are a real Church, then, the call you give 
your minister puts him on the same footing as the 
minister of any Church whatever. The only differ-TK
ence between the different Protestant Churches in the 
matter is this: that for some, as the Anglicans, the 
Church is the whole historic body, with episcopal con-TK
tinuity through centuries, and bound by the ordinances 
of centuries; for others, as the Presbyterians, the Church 
is the existing community composed of a number of 
separate congregations; for others, like ourselves, it is 
the single local congregation of believers. According 
to these definitions, the mandate takes various shapes, 
and is less free or more. But they all differ from the 
idea of a ministry whose mandate is not from the 



REAL NATURE OF CHRISTIAN PRIESTHOOD 193 

Church at all, but only to the Church, which is not in 
trust but in possession, which is not representative but 
vicarious. 

There is a fine and clear passage of Luther on this 
head which I will quote: 

" We take stand on this. There is no other word 
of God than that whose proclamation is enjoined on 
all Christians; there is no other baptism but that 
which any Christian may confer; there is no other 
memorial of the Lord's Supper but that which any 
Christian may make in obedience to Christ's com-TK
mand; there is no other sin than that which any 
Christian may bind or loose; there is no other sacri-TK
fice than the body \l.e. person] of any Christian. A 
Christian alone can truly pray, and Christians alone 
ought to judge of doctrine. And all these are royal 
and priestly things. 

" Every Christian has the power which pope, 
bishop, priest, or monk has to retain sins or to remit. 
We have all that power. Only the stated and public 
exercise of it should be confined to those who are 
chosen for the purpose by the Church. But this does 
not affect its private use. 

"Every Christian has the true clerical status. 
There is no difference among them, except as a matter 
of order." 

And the Smalcaldic Articles say: " If the bishops 

13 



194 

became the enemies of the Church, and refused to 
ordain proper persons, the Churches could take back 
their rights. For where the Church is, there is the 
right to administer the Gospel. It belongs to the 
Church, and no human power can take it from the 
Church." 

IV 

But our chief interest in this country is not with the 
Roman idea of the priesthood, but with the Anglican, 
and its relation to our own ministry. What is the 
Anglican idea of the ministry of the Church? I leave 
out of account those extremists in it who really take 
the Roman view; and I would go to those quarters 
where the High Anglican view is expressly put, in 
contrast with Rome on the one hand, and the Puritans 
on the other; to the Oxford High Churchism of Canon 
Gore and Dr. Moberly, as distinct from the Cam-TK
bridge Broad Churchism of Bishop Lightfoot on this 
point. 

It clears the ground by repudiating the Roman idea 
of the priest as the basis of the Church, and by him-TK
self the Church; it discards the vicarious view of the 
priesthood; and it starts from the principle, not of a 
sacerdotal order, but of a sacerdotal Church. It is the 
Church that is the priestly body---the whole Church, 
lay and cleric, as one spiritual unity. It believes in 



REAL NATURE OF CHRISTIAN PRIESTHOOD 195 

the universal priesthood of the Church; not so much 
the priesthood of every individual by himself, but the 
priesthood of a collective Church, in wMKhich all in-TK
dividuals are on the same spiritual footing. This 
Church needs officers and organs to give effect to its 
priestly quality. It needs representatives through 
whom it may act. These are its priests, strictly so 
called. They are representative. They do draw 
their authority from the univiersal priesthood of the 
whole Church; they do not draw their authority 
direct from God, and impose it on the Church. They 
do not confer on the Church its priestliness; they 
only express and represent it. It is a representative 
and not a vicarious priesthood. It is appointed by the 
whole Church. But then it is not directly appointed 
by the whole Church, not elected. It is appointed by 
the due authority in the Church. The sacerdotal MK??r-TK
ity is ideally a mandate from the Church, an exercise 
of the Church's own priestliness, but it is conferred by 
the govermnental authority of the Church. Now what 
is this governmental authority of the Church which 
has the sole right to appoint the Church's ministers as 
vehicles of the Church's inherent priestliness? It is 
the episcopate. The episcopate has, from the begin-TK
ning, been the only legitimate organ of the authority 
of the whole Church. The bishops represent the 
Church, and they rule it only because they do repre-TK



196 ROME, REFORM, AND REACTION 

sent it. Their power is constitutional, as becomes 
Englishmen, and not despotic, like Rome's. But how 
came the episcopate by this sole power for the Church 
of Christ? They received it directly from the apostles. 
And the apostles? They had it conferred on them 
by direct commission from Christ. Christ appointed 
His apostles, but He appointed them not as satraps, 
but as representatives of the whole Church; they 
were to concentrate and exercise the spiritual power 
which He really conveyed to the whole Church. 
Moreover, it is said. He gave them power to convey 
their commission and authority to the bishops; and 
the bishops, as the sole organs and administrators of 
the Church's spiritual prerogative, had the sole right 
of appointing the Church's ministers. You will 
remember that the minister, then, on the true Anglican 
theory, represents the Church, and does not rule it; 
that his priestliness is only the personalized expression 
of the priestliness of the whole Church, lay and cleric 
together; that he has nothing which the whole Church 
does not convey to him out of its own nature and pre-TK
rogative as priestly through Christ in the v/orld and 
for it. 

V 

Now if we concede the inherent priesthood of the 
whole Church everywhere (as we must), what is there 



REAL NATURE OF CHRISTIAN PRIESTHOOD 197 

to be said in criticism of this position? Why do 
we object to it? 

Is it not clear, to begin, that our first point of 
issue (granted that concession) is not so much with the 
priest as with the authority that claims to monopolize 
for the whole Christian Church the right to appoint 
him, viz. the episcopate? Both we and they, of 
course, are eager to know and do the will of Christ in 
the matter, and we both recognise the supreme priest-TK
liness of the Church under Christ. Was it the will 
and commission of Christ that the episcopate alone 
should have the sole right to appoint the ministry of 
the Church, to institute the organs of its priestliness; 
that the bishops should inherit the prerogative of the 
apostles? That is a very large question. It turns on 
the interpretation of Scripture; and opposite views 
are held about it by scholars of the Oxford school, 
and by the great New Testament scholars of the 
Cambridge school. But it is well that we should not 
allow any indignation with the Romanizing priests of 
the Anglican Church to blind us to the real location 
of the issue in that Church's most responsible speakers. 
The real conflict is on the episcopal monopoly of 
appointing oflficers, who are yet not officers of the hier-TK
archy, as in Rome, but, like our own ministers, repre-TK
sentative officers of the Church, though they can be 
ordained only by the ministry. Can the real repre-TK



198 ROME, REFORM, AND REACTION 

sentatives of the Church's priestliness be appointed by-TK
authority? An occasional and rare representative 
may be appointed by authority---as an ambassador; 
but can the standing representation of the Church's 
ultimate and characteristic power be an appointed one, 
and remain representative in any real and effectual 
sense? I venture to think it cannot. I venture to 
think that the doctrine of the apostolic succession is 
incompatible with a truly representative priesthood, 
and in practice destroys its representative quality, and 
tends to turn it into the Romish and vicarious thing. 
I think this is shown by two features in the Anglican 
clergy: first, by the relation which a vast and in-TK
creasing number of them take up to their own flock 
— shown in sympathy with the mass and the con-TK
fessional; second, by the unhappy attitude and tone 
taken to the ministers of other Churches. 

VI 

But as a theory the Anglican is really very different 
from the Roman, because it does make the priest the 
representative and projection of the priestliness of the 
whole Church. But it is not the New Testament 
theory. And, as I say, I fear that practically and 
popularly it is not easily distinguishable from the 
Roman theory; and it is constantly passing over into it. 

And the reasons are these: 



REAL NATURE OF CHRISTIAN PRIESTHOOD 1 99 

I. The representative nature of the priesthood is 
too remote from the Church's own priestly sense at a 
given time for the Church to feel represented. 

(MK) The bishop who gives the minister his validity 
in the first place is not appointed by the Church, but 
by the government of the day---by the premier of the 
day. This really takes the authorization of the min-TK
istry entirely out of the hands of the believing and 
priestly Church, and has long broken the true succes-TK
sion; for it can hardly be said that most premiers or 
most monarchs represent the Church either in its faith 
or in its priestly quality. 

{b) Even where the bishops are elected by the 
Church it is by the clergy, i.e. by those whom 
bishops had appointed, and therefore not by any 
electors representing the lay priestliness, the sacerdo-TK
t'lum laid., in the Church. It really works in a circle 
— bishops appointing priests, and priests appointing 
bishops— which makes the ministry a close body out-TK
side of the universal priestliness of the Church. 

(<:) For the chief authority of the episcopate we are 
referred to the apostles and appointment by them. 
But their procedure is very obscure. We are without 
information as to any principles of representation fol-TK
lowed by the apostles in their selection. The gap in 
their own college thej filled up by lottery. And it 
carries us a very long way round from the priestly 



200 ROME, REFORM, AND REACTION 

quality of the living Church to-day to seek its recog-TK
nition, and expression, and only valid authority in the 
apostles of two thousand years ago; an authority, too, 
based on a commission given before Pentecost, before 
there was a Church, a commission which they under-TK
stood in no sense which forbade the use of the lottery. 
Even if they represented the priestliness of the then 
Church, it places the priestliness of to-day's Church at 
a great disadvantage, and even reduces it to an in-TK
significant point, if the representatives to-day have to 
go back so far for their authority to represent it. 

2. But what we are told is that the representative 
authority of the apostles was theirs as appointed by 
Christ, and that in travelling back to them with the 
ministry we are going back to an ordination which is 
divine in the first degree; they represent the Church, 
not by the representative principle, but by Christ's will 
that they should. Well, but if that be so, have we in 
any real sense the representative character of the min-TK
istry, as the expression of the Church's priestliness? 
The priesthood then does not flow out of the univer-TK
sal priesthood of the Church conferred by an in-TK
dwelling Christ, but is parallel with it. Both priesthoods 
are the gift of Christ, and the one is not the represent-TK
ative of the other. If the Church appointed its priests, 
they might be representative. But can they represent, 
can they flow from, the Church's priestly quality, can 



REAL NATURE OF CHRISTIAN PRIESTHOOD 201 

they do more than illustrate it, if they owe their 
appointments even to the personal institution of Christ 
on earth, and not to His indwelling Spirit acting 
through the Church? Even if Christ appointed the 
apostles to represent an infant Church which was not 
yet sufficiently knit or adult to appoint its own repre-TK
sentatives, where did He tell them to keep the Church 
continually in this state of minority? where did He 
empower them to monopolize from the Church in zuhtch 
He dwelt the continuous appointment of their succes-TK
sors? The theory of an apostolic succession is incom-TK
patible with the faith of a Church made priestly by the 
indwelling Spirit of the great High Priest. The right 
of the ministry is due, we are told, to its being an 
expression and representation of the priestliness of the 
Church. The Church conveys and confers this priest-TK
liness through the authority of the bishop. But the 
authority of the bishop is not held to be derived from 
the Church, but directly from the same power to 
which the Church owes its priestliness, viz. Christ 
Himself. Therefore what the bishop conveys in ordina-TK
tion is not the priestliness of the Church, but a priestly 
character conveyed to the episcopate through the 
apostles over the head of the Church and direct from 
Christ Himself. And so we reach Rome. 

Can we wonder if this is practically indistinguish-TK
able from the Roman theory in its results? Can we 



202 ROME, REFORM, AND REACTION 

wonder if the Church has very little sense of its own 
priestliness compared with that of the priestly order, 
when the modern representative principle is overruled 
by miraculous institution and ancient prerogative, and 
when so many centuries and so many intermediaries are 
placed between its intrinsic priestliness and the priest-TK
liness of its representative staff? The living Church, 
whose priestly quality is said to be represented by its 
minister, has no real voice or action in connection 
with his appointment. Can we wonder if it do not 
feel represented, if it never think of the representative 
theory in connection with its ministry, and if it look 
upon any sanctity it may itself possess as devolved from 
the priest rather than upon the priest as evolved from 
its native sanctity and priesthood in Christ? A de-TK
volved ministry is incompatible with a representative 
ministry unless the authority which devolves is placed 
there by direct election by the living Church. If the 
Church do not elect its minister, it should at least 
elect the bishops who appoint the minister. 

The defect of the Anglican theory, therefore, is a 
practical more than a theoretical one. Its theory is 
so embarrassed and so worked as to produce a practical 
result fatal to the theory. It does not give practical 
effect to the Church's universal priesthood. It does 
not make the Church feel that priesthood by compari-TK
son with the specific priesthood. It creates a wrong 



REAL NATURE OF CHRISTIAN PRIESTHOOD 203 

emphasis, the tendency of which is constantly to turn 
the representative priesthood into a vicarious, and lose 
the sacerdotal Church in the independent priesthood 
of the successors of the apostles. 

3. But there is another and more serious reason 
why this Anglican theory of the ministry tends to 
pass over into the Roman, and its priesthood to gravi-TK
tate to the Mass. We are told by one of its finest 
and most responsible exponents (Dr. Moberly) that 
the theory is really twofold. First, the priest is what 
the Church is; second, the Church is what Christ is. 
First, the priest represents the priestly Church. " The 
priesthood of the ministry is the priesthood of the 
Church specialized and personified in certain repre-TK
sentative instruments." The priest is what the 
Church is in this respect. He cannot rise higher 
than his source and reservoir, which is the priestliness 
of the whole Church. That seems to shut out the 
Roman theory of a commissioned vicar of Christ, and 
confine us to the view of the priest as an organ of 
the Church. But I have just shown how the practical 
application of the principle tends to neutralize it. I 
come now to the next step taken, the definition of 
the Church's own priestliness, which does not arrest 
that tendency Romeward, but helps it. For the priest-TK
liness of the Church is defined thus: " What Christ 
is the Church must be." " Christ is the spirit and 



204 ROME, REFORM, AND REACTION 

principle of divine love and sacrifice in the conditions 
of human sin. He is that principle incarnate. This 
the Church must be by His indwelling, and by her 
self-identification vMKith Him." Well, that is sound 
and fine. But there is an action of Christ's sacrifice 
vMKhich goes beyond these viMKords; there is its action 
upon God and the holiness of God, as well as its ex-TK
pression of the love and sacrifice of God. There is 
the atoning action and aspect of Christ's sacrifice. 
Does the Church, by any self-identification with the 
sacrifice of Christ, share that in an active way? In 
a passive way, yes; the Church enjoys the benefits and 
blessings of that atoning sacrifice. But does the 
Church share that act, the eternal atoning act? Is its 
sacrifice in any sense propitiatory? Is its priesthood a 
share of this part of Christ's priesthood? The Church 
may offer, must offer, Christ and the sacrifice made by 
Christ. Indeed in the Church's offerings Christ in-TK
dwelling offers Himself afresh. And in heaven He 
offers perpetually to God His atoning sacrifice. But, in 
offering Himself through the act of the Church by His 
indwelling and inspiration, is it the atoning effect of His 
sacrifice that He offers? Is it in any sense an atoning 
sacrifice that the Church offers, even when it gives full 
effect to the reality of its priesthood in Him? If it is, 
are we not landed in the bosom of Rome, with its sacri-TK
fice of the Mass vere propitiatorin?n? We need not 



REAL NATURE OF CHRISTIAN PRIESTHOOD 205 

hesitate to say that the priestliness in the Church is a 
sacrificing priestliness, but is it an atoning? The 
Church shares Christ's sacrifice of love, identifies her-TK
self with it; does she share His sacrifice of grace? 
She identifies herself with Him in act as a sacrifice for 
the blessing of the world; can she identify herself with 
Him as a sacrifice for the saving of the world? She 
identifies herself with her Redeemer; yes, but as Re-TK
deemer? as the Redeemer of the world? Does she 
share in the act of redeeming, as she does in the act 
of reconciling men? 

I shall have something to say presently to indicate 
that the Church's function as the Body of Christ is 
not complete; by this metaphor alone we might even 
construe the Church in terms of a certain Christian 
pantheism; and it needs to be supplemented by the 
more fundamental conception of the Church as the 
Bride of Christ, as the object of His grace before she 
is the organ of His action in men, as a respondent be-TK
fore she is an agent, as a will confronting His before 
she is a will effecting His. And what makes the 
Church His Bride is the atoning, redeeming act 
which took her out of the world, an act which she does 
not share but only answers. If the priestliness of the 
Church mean a share, even a conferred share, of the 
atoning act, if she reproduce not only Christ's sacri-TK
fice but also His atonement, His Redemption, then it 



206 



is hard to see how we are to avoid the Roman theory 
of the Church as a prolongation of the Incarnation, and 
the priest as a demigod. It is a theory with a specula-TK
tive fascination. The chief fascination of Rome to-day 
is speculative and imaginative. But it is a theory 
with all the immense practical results that Rome's 
masterly logic (given her principles) can draw. 

But is the Anglican theory exposed to any such 
risk? Does it claim for the priestliness of the Church 
a share in the atoning aspect and effect of Christian 
sacrifice? Well says Dr. Moberly (I grieve to take 
a controversial attitude to a book so true, profound, 
and beautiful in many respects as his Ministerial 
Priesthood)MK " What Christ is the Church must be. 
She is priestly in the Eucharist, which is her cere-TK
monial identification with the atoning sacrifice." 
" The priesthood of Christ is His offering of Himself 
as a perfect sacrifice, an offering which is not more 
an outward enactment than an inward perfecting of 
holiness and of love; an offering whose outward 
enactment is but the perfect utterance of a perfect 
inwardness; an offering which, whilst, so to say, con-TK
taining Calvary in itself, is consummated eternally by 
His eternal self-presentation before the presence and 
on the throne of God. The sacrificial priesthood 
of the Church is really her identification with the 
priesthood and sacrifice of Christ." " Christ Himself 



REAL NATURE OF CHRISTIAN PRIESTHOOD 20/ 

has presented for all time an outward ceremonial, 
which is the symbolic counterpart in the Church on 
earth, not simply of Calvary, but of that eternal 
presentation of Himself in heaven in which Calvary 
is vitally contained. Through this symbolic enact-TK
ment, rightly understood---an enactment founded on, 
and intrinsically implying, as well as recalling, Calvary 
— she in her Eucharistic worship on earth is identified 
with His sacrificial self-oblation to the Father; she is 
transfigured up into the scene of the unceasing com-TK
memoration of His sacrifice in heaven, or the scene of 
His eternal offering in heaven is translated down to, 
and presented, and realized in the worship on earth." 
How much in this is admirable, but how much is 
inadequate! The writer does not seem to me to grasp 
with evangelical depth and fulness the essence of the 
redeeming act; he does not touch the main trunk of 
the evangelical nerve. The action of Christ is regarded 
too exclusively as a manifestation and presentation of 
holiness, i.e. too aesthetically, and too little as an act 
of will, a great act of struggle and conquest, a great 
transaction of some sort dealing with the divine and 
holy law. It is too apodeictic and too little pragmatic. 
If the atonement was no more than Christ's sacrificial 
self-presentation to the Father, if His holiness was 
not in its nature a unitary and compendious holy act 
pervading His life earthly and heavenly, if it was the 



208 ROME, REFORM, AND REACTION 

world-conflict with evil and its conquest; then the 
Church may be identified with it. But if it was this 
last, if it had an absolute value in regard to broken 
law and objective holiness, if there was thus a wine-TK
press which He trod and treads alone, and of the 
people there can be none with Him, then the account 
above given falls short; and falls short in the very point 
which is the focus of redeeming action. And there is 
in the atoning sacrifice and priesthood 'that which the 
priestliness of the Church can never share, that which 
Catholicism fails to realize, and which, when realized, 
is the evangelical fulcrum of the Reformation that dis-TK
placed Catholicism from the throne of the Church. 

The Catholic theory may be profound as it is 
certainly acute, but it is not the profoundest; and it 
does not keep pace with the searching of the Spirit in 
the mighty men of the Reformation. It is a theory 
more scholarly than profound, and more beautiful than 
powerful. But the point I would press here is this, 
that while its lack of profundity commends it with 
charm to many, its lack of searching precision renders 
it a too easy prey to Roman logic. And we cannot 
wonder if, when the Eucharist is described as the 
Church's identification with the Atoning Sacrifice and 
the priest is held to be what the Church is, such a 
theory of priestly function should be indistinguishable 
from the Mass except to trained and ingenious minds. 



REAL NATURE OF CHRISTIAN PRIESTHOOD 209 

VII 

But with the reserve I name, I should like to insist 
that the true nature of the Church is priestly, that the 
Church is the priest in the kingdom of God, and that 
the minister of the Church represents that priestliness. 
It is a priestliness which belongs to every member of 
the Church, not as an isolated unit, but as a member 
organized through faith into the priestliness of Christ. 
To be a priest is the power, right, and privilege of 
every member of the Christian Church in so far as 
it is a Church of believers. It is not the power or 
right of one who is a member of the Church only by 
tradition, habit, baptism, or ordination. What makes 
a priest is personal faith in the great High Priest. 
It is not the power or right of any one who is a 
member of the Church just because he is a member 
of the State whose national Church it is. Justification 
by faith is ordination to the true priesthood. But 
when we come to the public and official ministry, 
what Anglicanism says and Presbyterianism says is 
true. There should be a conveyance by the Church 
to the person concerned of whatever its priestly 
function may mean in a public way. The private 
Christian shares the Church's priestly power and 
right of access to God; but when it is a question of 
authority to speak and act in the Church's name, and to 
do so habitually, then the authority should come from 

14 



2IO ROME, REFORM, AND REACTION 

the Church by express institution. That is our own 
congregational principle. Any believing man has the 
right and power to speak the Gospel to any men he may 
get to listen. But if he is to speak and act on behalf 
of the Church, if he is to represent the Gospel commun-TK
ity, he must be appointed thereto by the community. 

They may appoint him for a particular occasion 
only, and ask him to address them, pray for them, act 
for them in a public way, only ad IwcMK as we do in 
our prayer meetings. In so doing, in so praying 
especially, the member of the Church becomes for the 
time a minister of the Church, yea, a priest, leading 
the Church's sacrifice of prayer at the spiritual altar, 
and giving outward effect to the inner priesthood of 
the Church. No man taketh this honour unto him-TK
self, but those alone who are invited to do so by the 
Church, if only through the request of its presiding 
and permanent minister in the chair. If each claim 
the authority to act for the Church on his own im-TK
pulse and initiative, then we have anarchy. 

Or the minister of the Church may be appointed 
by it for life, for standing office. He is then the 
permanent and personal representative of the priestly 
Church; but he is so only by the direct appointment 
of the Church. He is a true representative of it by 
the voice of the Spirit in its election. He receives 
authority, not to preach the Gospel but to represent 



REAL NATURE OF CHRISTIAN PRIESTHOOD 211 

the Church. There is not added to him any spiritual 
power that he did not possess before, or any Christian 
grace; but he has authority to speak in the Church's 
name as he could not before. He can speak as the 
organ of a community, and so act. Priestly he is, and 
not merely a prophet; but he is only a priest in the 
sense that he represents the priestly function of the 
collective Church within the world. As a member 
of the Church he had power and right before; what 
he receives for the Church is authority as a matter of 
convenience and order alone. And he has it from the 
Church directly, not by a circuit of centuries, nor by 
a bishop who is a creation of the State more than the 
Church. His election by the faithful communicants 
makes him a minister of the universal Church and 
the representative of whatever priestliness belongs to 
that. 

You will see that my remarks in these discourses 
are not merely a criticism of another Church system, 
but also a protest against a tone which has crept into 
our own. The very murmurs with which some may 
receive this plea for the priestly nature, the sacerdotal 
function, of the Church in the world and for it---these 
demurrers show that the preacher on this line has 
some duty to expand the attenuation of the Church 
among his own no less than to assail the exaggerations 
of it in others. 



IV.— Part II 

SOME REAL SOURCES OF THE PRIEST'S 
WELCOME 

I 

A PRIESTLY order cannot be turned to safe account 
by anything but a more priestly Church. The State 
cannot do it, the world cannot; because, after all, 
its idea is much higher than any belonging to the 
mere natural man. Even a Church, if devoid of a 
real sense of its priestliness, will be unable to cope 
with the priest who takes the priesthood in earnest, 
in however perverted a form. I may therefore, per-TK
haps, be forgiven if I repeat or dwell on this article 
of a priestly Church in the interest of our Evangelical 
faith and the reality of our Church life. 

The priesthood which the ministry represents is 
the priesthood of the Church rather than of isolated 
believers. This Church where I preside is a priest 
much more than I am, more than any member of 
it is, more than any clergyman. The great visible 
priest on earth is the Church in its various sections. 



SOURCES OF THE PRIEST'S WELCOME 21 3 

The Church is the great intermediary between God 
and man, because it is in trust of the one saving 
Gospel of the Great Mediator. The Church is the 
priest as the abode and agent on earth of the One 
Priest, the High Priest. It is priest by its unction 
of the Holy Ghost. The minister of the Church 
only represents the Church's priesthood, which conveys 
a great function of Christ's. The Church is primary, 
the office secondary. The ministry is not an order, 
but an office. The priest is what the Church is; 
it is not the Church that is what the priest is. 
The Church is the steward of the Gospel; and the 
priest's authority is only the authority of the Gospel 
committed to him. The sacraments in the minister's 
hands are only there because he himself is the hand 
of the Church; and they draw their value from the 
Word of the Church's Gospel of the One High 
Priest. They are expressions of it; and therefore 
they are in their nature not magical, but moral and 
spiritual as the effect of the Gospel is. The minister 
is in charge of the sacraments just as he is of the 
Gospel, which is the common charge of the Church. 
We cannot be brought to like the word priest for the 
minister of the Church. It was avoided in the New 
Testament, I have said, because it had become associ-TK
ated with ideas foreign to Christian office. And if 
that was the case then, it is equally the case now. 



214 ROME, REFORM, AND REACTION 

The word through its Roman use has become so 
hopelessly debased that it is mischievous to retain it. 
And the Anglican Church puts itself into a false 
position with the public by the attempt to do so. 
But, for all that, there is nothing that some of the 
Free Churches need more than a return to the idea 
of the priestly character of the Church, of the collec-TK
tive Church, whatever we may regard as its unit. 
That unit may be the Episcopal Church, or the 
Presbyterian, or the single local Church; yet if it is 
a real Church of Christ it is a priestly body in its 
nature and function in the world. The reason why 
we are not in earnest enough, and our piety is of a 
poor, flat, and unimpressive type, making too little ap-TK
peal to the public soul and imagination, is because we 
have lost the idea that our Church is, in its nature, as 
the body of Christ, a priest among men. Our indi-TK
vidualism has lost the sense of the Church as a real 
body; it is regarded as an association of people each 
having his own personal relations with God. And 
our secularity of mind has lost the idea of the Church 
as a priestly body exercising under Christ the great 
sacrificial function of the world. The name of priest, 
which we would refuse to the Church's minister, we 
should urge for the Church itself, for the sake of the 
thing it represents. The main business of the Church 
on earth is priestly; it is to show forth, so far as the 



SOURCES OF THE PRIEST'S WELCOME 21 5 

redeemed may do, the Redeemer's death in His risen 
light and power. 

Is it enough to describe the Church simply as a 
v/itness to the world of Christ's truth, as declaring to 
the world reconciliation and redemption? Is the 
Church simply a messenger from Christ to men? 
Does not Christ do more than send it? Does He not 
dwell in it? Does He not act from the midst of it? 
Is it not His chief and chosen organ on earth? And 
is not His great action based on the perpetual sacrifice 
of Himself for the world? Must the Church He in-TK
habits and uses not become in some sense the organ on 
earth of that action? Does the Church not offer sac-TK
rifice as well as proclaim truth? Does she not offer 
Christ Himself to men? does she not plead Christ for 
men before God? Is not the great sacrifice Christ, both 
to God and to man? and does the Church not offer 
this spiritual sacrifice in manifold ways continually? 

We might begin with our questions lower in the 
spiritual scale. We might ask, is not the whole 
sphere of Christian action a spiritual sacrifice? We 
present our bodily energies in duty or service, as a 
living and sacred sacrifice. If the Church sacrifice 
itself at all in the service of man, it is a priestly act. 
But we rise higher. When the Church does that 
is it showing forth its own affection for men? No. 
It is setting forth the love of God to men in action---



2l6 ROME, REFORM, AND REACTION 

not in word, but in deed. But that is what the 
sacrifice of Christ did. It is the sacrifice of Christ 
living on and working itself out through the Church. 
It is the Church doing the priestly act of expressing 
the priesthood of Christ in one aspect of it at least. 
But the Church does still more. It not only shows 
in service the love that made Christ die, but it carries 
home through this loving service the fact of the 
Reconciliation. Its service of man is not merely to 
help man, but to reveal God, to reveal by this help, the 
Redeemer, the cross. Its work and service for man is 
not only sacrificial but sacramental, both for its mem-TK
bers and for the world. By its loving service it does 
more than show forth, it conveys. It is a channel 
and agent of grace from Christ to man. It is a stand-TK
ing sacrament, a priestly minister. It administers by 
its sacrifice the Great Sacrifice. And if we turn from 
its work to its word, is its word a mere word, a mere 
declaration, such as a herald might read out at a market 
cross, or the Gazette publish in the King's name? 
The preached word of the Gospel---is it not more 
than the delivery of a report, is it not a work itself? 
When I preach a sermon am I reading a paper, an 
essay, an information? Am I getting out a fine 
composition, publishing a theory, giving a lecture, 
explaining a piece of the moral world, airing views 
and opinions? Am I speaking to critics or to be-TK



SOURCES OF THE PRIEST'S WELCOME 21/ 

lievers? There Is indeed not enough criticism of the 
right kind. There is criticism of some phrase or 
manner when it should be the ideas. But it is not 
to criticism that the preacher speaks chiefly, but to 
faith, to believers, to critics on their believing side, to 
wMKhat they and he have in common, to their Christian 
need, sympathy and hope. Take the greatest preacher 
and the truest---what is he trying to do? To exhibit 
himself, to sparkle, to please, to instruct agreeably, to 
win popular influence for God? Does he want to 
send men away saying " How well he has done! " in-TK
stead of " How well we must do! "? No; the word 
of the Gospel preached, like every divine word, is a 
work, it is a spiritual act. Why is the preacher 
exhausted as the lecturer is not? Because it is a 
spiritual struggle, the Lord's controversy. He has 
been wrestling with men---at grips with their soul, 
their fugitive, reluctant, MKrecalcitrant soul. Because 
every best sermon is a real spiritual act, an act of the 
Church moreover, of the Church which is God's 
channel and agent of grace and prayer for men, of the 
priestly Church. Every great true sermon is a great 
true sacrament, the sacrament of the word, in which 
the people participate as really as the preacher. It 
is not his message, but the Church's. It is faith 
preaching. The Church delivers itself through him. 
Every true hearer is not a hearer only, but a doer of 



2l8 ROME, REFORM, AND REACTION 

the word. To hear well is to do something actively, 
to do much. To hear as the Church should hear is 
really to preach. The preacher is but the mouth-TK
piece or the Church, with its Gospel of the great 
saving act. In every real Gospel sermon God gives 
the word and great is the company of the preachers. 
On every such occasion those who hear in faith are 
not simply present, do not simply listen, they assist 
in the service. They exercise their universal priest-TK
hood. They minister at the altar of the word of the 
cross. If that were realised it would put a new 
aspect upon church-going. Men would go there in 
the same spirit as the minister goes, and to do the 
same work in their way. They would go to some-TK
thing in which they were not passive but active, not 
a mere audience but colleagues in the ministry, and 
deacons serving the tables of the word. The pulpit 
of the true Gospel is itself an altar where the eternal 
sacrifice is offered through men by Christ the High 
Priest to men, and by men in a Church of praying 
priests to God. In the preaching of the word of the 
cross the Church is a priestly Church. It is really the 
Church that preaches, and for the Church to preach 
Christ in His eternal sacrifice is for the Church to 
offer that sacrifice in the only sense in which men 
can offer a sacrifice provided, yea, made by God. 
But we go higher still in a way. The Church not 



SOURCES OF THE PRIEST'S WELCOME 219 

only helps and serves men in the love of Christ, not 
only bears home the Gospel in a sacrificing, sacra-TK
mental way, by act or by word, but the Church 
prays for men. And this is perhaps the pricstliest 
function of all. The Church identifies itself with 
the perpetual intercession of our Eternal High Priest. 
The Church through Christ has access to God on 
man's behalf. In true prayer the Church is priestly 
in two ways. It is solid with man, for whom it 
offers intercession; and it is solid with the perpetual 
intercession of Christ, offered for Church and world 
alike. This is the greatest act of philanthropy that 
the Church can do, and at present the most neglected 
by many. I need only mention here, as I have come 
to the worship and prayer of the Church, the supreme 
act of worship in the Lord's Supper. In that act the 
Church identifies itself, within the limits I have said, 
in a ceremonial way with Christ in His sacrificial act. 
It offers Christ, the one eternal sacrifice, to God. And 
Christ dwelling in His Church body offers Himself, 
preaches Himself to the world as crucified Redeemer, 
in an act of a different nature from the spoken act of 
the pulpit. All these considerations make the func-TK
tion of the Church not only the prophet's, but the 
priest's. They make the Church in some sense under 
Christ not only the apostle of reconciliation, but also 
a reconciler. 



220 ROME, REFORM, AND REACTION 

II 

The Church is in some sense reconciling, mediat-TK
ing, and priestly. In what sense? It must be in a 
sense prescribed by the nature of Christ's priesthood, 
because that is what constituted the Church. Its 
function is determined by the priestly act and nature 
of its indwelling Christ. What is the relation of the 
Church's mediation, the Church's intercession, to 
Christ's? It can never, of course, be parallel or 
mediatorial, as if anything offered by the Church were 
in itself a true, real, and proper sacrifice, a verum et 
proprium sacr'ificiumMK with a propitiatory value of its 
own. It cannot be vere propitiator lumMK as the Roman 
Catechism says of its mass. The action of a body 
which owes its existence and nature to another con-TK
tinuous and constitutive act can never be its parallel. 
The Church's action can never be a repetition of 
Christ's redemption or intercession, nor can it be a 
mere imitation of His teaching, healing, and blessing 
of man. It cannot approach His act from the outside. 
It can only be a function of that act within us; it can 
only be the reproduction of that act working itself out 
in the Church. It is in us and through us rather 
than by us. If the Church be the medium of God's 
forgiveness to the world, it is only as the organ of the 
One Mediator. She does not produce the forgiveness, 



SOURCES OBMK THE PRIEST'S WELCOME 221 

but only reproduces it. But she does reproduce it, she 
does not only declare it. She gives it actual practical 
eflfect. She carries it home effectually and sacra-TK
mentally to men's experience. The Church cannot 
forgive---only her Lord can do that---but I do not 
think, if we had the proper views of the ministry, 
that it would be dangerous to say that the Church 
absolves. It cannot destroy guilt---God alone can do 
that in Christ---but it could, if it were its true, kind, 
holy self toward the poor soul, destroy the difficulty 
of believing that God had done so. It could de-TK
stroy the sinner's difficulty in taking forgiveness in 
earnest. The priestly Church is yet not so priestly 
that it can expiate, propitiate, atone; but it can offer 
God's own expiation both to God and man, and it 
can do so not in an external way, but by an identifi-TK
cation of itself with that expiation, Christ. The only 
propitiation it offers is Christ, who is the foregone 
offering from God Himself The Church cannot 
atone, but it can and does offer His atonement who 
could and did. It bears into God's sight, so to say, 
the foregone propitiation, the Lamb that God has 
provided for an offering. It offers this to God in a 
sacrifice of its self-righteousness and self-will. And it 
offers also to men. It offers to men this sacrifice and 
atonement of Christ. It sets Him forth as their pro-TK
pitiation. It offers it in word, in rite, and in the 



222 ROME, REFORM, AND REACTION 

humane and loving ministries in which our faith 
grows sacramental to our kind. The priestly sacri-TK
fices of the Church are only representative, and not 
vicarious. But they represent in act, not in show. 
They effect, and not only declare. They " exhibit," 
in the old and pregnant sense of the word. They 
represent, by reproducing it, the manward side of the 
sacrifice of Christ. They also represent and embody 
the sacrifices of man in grateful response to Christ's. 
But they are not instead of either God's sacrifices or 
man's, they are rather expressive and prophetic of 
these. These act through them. No priestly func-TK
tion of the Church adds anything to what Christ has 
done; it only explicates His act, actualises it variously 
in history and life. But it does explicate it. It does 
not merely either commemorate or imitate it. It is 
an act within His universal act, it is not an act con-TK
tributed to it. 

The priestly character of the Church therefore rests 
on the indwelling in it of its own Priest and Redeemer. 
It rests on a share which the Church thus has, as a 
conscious and obedient organ, in His perpetual priestly 
work. But that priestly work is twofold---it is sacri-TK
ficial, and it is atoning. It blesses through loving 
self-surrender, and it satisfies a holy broken law. 
Now in this latter function of Christ the Church does 
not share. There she is not the Body of Christ, but 



SOURCES OF THE PRIEST'S WELCOME 223 

the Bride of Christ; not the organ of His sacrificing 
love, nor the channel of His gospel of grace, but the 
recipient of His grace, the respondent to it, the heart 
that is made what it is by it. The error of Rome is 
to exalt the idea of the Church as Christ's Body at the 
cost of the idea of the Church as His Bride. It 
claims a share too intimate and organic in the priestly 
work of Christ on its atoning side. His grace is too 
much a thing infused into it, and too little a thing 
exercised towards it. The believing Church is such 
because of its practical belief in Christ's atoning work. 
It is this faith that forms its priesthood. But the object 
of faith must be something which confronts us even 
more than something we share. Therefore we cannot 
share the atoning work that we trust, but its benefits 
only. These we do share, and among them chiefly 
the spirit of sacrifice and the work of reconciling men. 
Our priesthood is a priesthood of the Reconciliation, 
not of the Redemption; of the attuning of men rather 
than of the atoning of God. But our reconciling 
sacrifices must rest on the atoning sacrifice, otherwise 
the priesthood of believers is a metaphor and a theme 
more than a principle. 

Ill 

A priestly order can only be safely used by a more 
priestly Church. But how is the real priestliness of 



224 ROME, REFORM, AND REACTION 

the Church to be found and fed? Only by the 
Church's return to a personal acquaintance both in-TK
tellectual and spiritual with the New Testament. 

We cannot settle this strife by any knowledge of 
the second or third century. That would leave its 
settlement in the hands of the scholars, not to say 
the archaeologists. We must go to the first century, 
and take the scholars with us as our assessors and 
advisers there. The English Christian public must 
become much better acquainted with the Bible if we 
are to be saved from the priest for the true priesthood. 
Without the Bible the public is powerless against the 
priest; with the Bible the priest becomes the Church's 
servant and minister for the public. It is the Bible 
that must both restore us to the Church and protect 
us from the Church. It was given for that purpose. 
It was not the product of the Church. That is a 
fallacy of which sections of the Church make great 
use. The Church gave us the canon, but it did not 
give us the books. Holy men moved by the spirit 
were the authors of these. The Church is the 
librarian more than the author. It selected the books 
and it preserved them. It has also acted as inter-TK
preter, but without finality. It is more correct to 
regard the Bible and the Church as parallel products 
of the Spirit, than to treat the Church as producing 
the Bible, and therefore in sole possession of the right 



SOURCES OF THE PRIEST'S WELCOME 225 

of interpretation. In the Bible resides a power to 
reform the Church, far higher than any power in the 
Church to reconstruct the Bible. The scholarship of 
the Church may reconstruct the Bible, but it has no 
power to reconstruct the Gospel in the Bible; and 
that Gospel has the power to reconstruct both Bible 
and Church, and especially to save us from the 
Church's perversions and corruptions of its own 
priestly power. 

The Free Churches must, one way or another, 
read and understand more of the Bible. All their 
worst misfortunes, difficulties, and inadequacies have 
arisen by the practical dropping of the Bible from 
their personal acquaintance and use. It has been 
squeezed out by other literature, much of it religious. 
The new art of printing gave the Bible at the Re-TK
formation into the hands of the Christian public, but 
to-day it is the art of printing that has thrust it out of 
their hands. It is the immense accessible mass of 
printed matter comparatively worthless that has pre-TK
occupied the reading time of most Christian people, 
till their religious taste and intelligence is of the 
lightest kind. There ought to be a system of daily 
Bible reading at work in every Church. We have 
Home Reading Unions of the most useful kind for 
other literature, and there ought to be in the 
Churches something of the kind for the Bible. Let 

15 



226 ROME, REFORM, AND REACTION 

the Christian public only become quite sure that the 
vicarious priest (in theory or in effect) is foreign to the 
New Testament, and there is enough of the Pro-TK
testant principle left as to the authority of Scripture 
to send him back to his native Rome, or confine him 
to a small and harmless sect. 

For a personal and intelligent use of the New 
Testament means personal and experiential faith, 
personal and open-eyed religion. And it is such 
personal religion that is the essence of Christianity. 
It is the fact that the priest is a religious person who 
knows his own narrow mind that gives him so much 
of his effect. Men will always trust a lucid and living 
person more than either a system or a book. The 
true priestliness of the Church is an abstraction if it 
do not work through living, convinced, and priestly 
persons. They may be official or they may be 
spiritual, but either way they go for more than a 
system and an abstraction. And we can only over-TK
come a mere official priesthood by a priestliness in 
ourselves more deeply personal, just as in philosophy 
we overcome rationalism by a deeper reasonableness. 
It is the personal effect that we give to the faith of our 
own priestliness in Christ that is its real power with 
men. We must love them for His sake, help and 
serve them, live the Gospel into them, intercede for 
them, and be a refuge to them which they do not find 



SOURCES OF THE PRIEST'S WELCOME 22/ 

in worldlings like themselves. We must exercise the 
priesthood of faith and character, of faith and conduct, 
of faith and love, of faith and mercy. The Spirit's 
action is through spiritual men. We must live into 
earth the perpetual priesthood of Christ in heaven; 
we must become sacraments to men, and not merely 
use them. We must be the sacrifices we preach---be, 
like our Lord, in some guarded sense, at once priest 
and victim, offering ourselves in the priestly com-TK
munion of a Church of blessed martyrs. For the 
priestly malady is too deep and subtle to be cured by 
anything but a priestly life in the true principle and 
power of the real active presence in us by faith of the 
High Priest of our profession, Jesus Christ. The 
sanctity of the priest can only be met and mended 
by the instructed holiness of the truly redeemed. 
The fountain of the true priesthood is not the bishop, 
nor even the institution of Christ, but the cross of 
Christ and its action on our personal sin and faith. 
The one power which the priest has to dread is the 
power of men certainly forgiven without him---broken 
by Christ and by Christ restored, bruised and healed 
by the same Saviour---men who, being justified by 
faith without works or priests of a law, have peace 
and power with God as priests indeed through Jesus 
Christ. 



228 ROME, REFORM, AND REACTION 

IV 

I have indicated those ennobling aspects of the high 
sacerdotal view vMKhich give it such footing as it has in 
the true priestly idea of faith and of the Church. 
And I should like to admit my belief that much of 
what seems the extravagance of the priesthood is due 
(like total abstinence in its direction) to a spiritual 
sense of the need for an extreme protest against, on 
the one hand, the passionate worldliness, luxury, and 
vulgarity of a wealthy and secular age, and, on the 
other, the irreverence and familiarity of a type of 
religion either too sentimental or too hard. But there 
are sides also in which the priesthood appeals to less 
spiritual instincts, and finds a soil in the very ordinary 
man. One of the features of the present day which 
imperils the Protestant position is a popular debasement 
of the sound tendency to think of the essence of re-TK
ligion as doing something; from which it is a ready 
step to think that the essential sacrifice is an act in the 
outward and usual sense of a deed---a gestum instead 
of an actum. Practical religion becomes a religion of 
performances and achievements instead of experiences 
and spiritual acts. A sacrament comes to take its 
value from being an opus operatmn instead of a phase 
of the great decisive spiritual act spread through the 
life---the act of faith. This is the soul of sacrifice, 



SOURCES OF THE PRIEST'S WELCOME 229 

the supreme sacrificial act; it is the act of self-TK
surrender, of self-committal in faith. This is the act 
which constitutes Christian priesthood. This is the 
central oblation offered by the Christian man. It is 
the community of this act of faith that is the universal 
priesthood of the Church. Every outward act is an 
expression and a detail of this act, into which is put 
the v/hole energy of the Christian soul. It is true, so 
far, that faith is doing something, though not in the 
popular sense. Religion is an act, and a sacrificial act; 
but it is an act of the inward soul, a continuous act 
of life-trust, in which the ethical rises to the spiritual 
while it remains of the will. The ethical becomes 
spiritual because its object is a person, not a lav/. It 
is the soul's act of self-committal to the sacrifice of 
Christ. It is personal faith in a personal Redeemer. 
But it is untrue to say that religion is an act in the 
sense that it is either conduct or sacraments. The 
priesthood of the Christian must be effective in some-TK
thing else before it take form in either of these. It is 
his personal relation of total surrender to the priest-TK
hood of Christ. 

Now it is the popular idea of the average man (to 
whom Christ never made His appeal), the idea of 
religion as some form of action rather than a spiritual 
quality of act---it is this idea that makes the work of 
the vicarious and operating priest so congenial to 



230 ROME, REFORM, AND REACTION 

many minds in an energetic age and race. " I like 
men who do things," says a somewhat mannish girl 
in one of Mr. Kipling's stories. And there is a 
mannish quality about the God of the period,MK about 
the religious object of the average Briton, which is 
sufficiently expressed in these words. His idea of 
faith is not an act and committal of the soul once for 
all, but a series of self-devotions. The popular hero is 
a person of exploits without a spiritual interior. Re-TK
ligion comes to mean doing certain things; and it is 
not doing the one hidden thing needful and eternal, by 
which the soul gives priestly value to all the things 
it essays. There is an obviousness about the priest's 
spectacular act at the altar which commends it much 
to this habit of mind. In this respect the mass is 
simply the ritual counterpart of the ethical tendency 
in undogmatic Christianity towards a propitiatory 
imitation of Christ in conduct. If it is likeness to 
Christ, especially in sacrifice, that commends us to 
God, instead of faith in Christ's sacrifice, then the 
difference between that and Romanism is only the 
difference between the ritual and the ethical expres-TK
sion of the same principle. Each makes really the 
same claim to God's favour through human action. 

1 2 Cor. iv. 4. (R.V. inarg.) 



SOURCES OF THE PRIEST'S WELCOME 23 1 

V 

In many directions it appears that for the hour 
the reh'gious world is more engaged with man's con-TK
tribution to God than with God's contribution to 
man. This is the large interpretation of the sec-TK
tional phenomenon of ritualism. We find it no less 
in the humanism than in the ethicism of the day. 
Erasmus, the earnest scholar, has taken the upper 
hand of Luther in the Christian tone of the pros-TK
perous educated hour. It is the spirit of Erasmus 
that rules educated society and colours the bench of 
bishops, who are scholars in Church history more than 
in the theology of Christian experience. They may 
not like the priest who takes himself thoroughly in 
earnest, but they have more sympathy with him than 
with the evano-elical minister of the Word. To take 
extreme cases, they would probably find themselves 
more at home with the meticulous Laud than with 
the mighty Jonathan Edwards. They certainly are 
not able to cope, except by the aid or fear of the State, 
with the priest who does strive to realise the despair 
of human guilt and deal seriously with it. Whoever 
is to cope with the priest must follow him to the 
roots of human sin, only he must go deeper. A 
humanist reformation is little more than reform; it 
is not regeneration. And it is regeneration that the 



232 ROME, REFORM, AND REACTION 

soul needs. But the Erasmic mind of the scholarly 
and pastoral clergyman miisses the apostolic priesthood 
and ministry of the Word. His altar is much more 
than his pulpit, his every day is a day of trivial visita-TK
tion, and he is more of a director of consciences than 
a prophet of the amazing, wrestling, living Word, 
which is hammer and fire 'upon the flinty rock of 
self-satisfaction. He tends to confessions more than 
conversions. And for the mending of the Church he 
would remove abuses, cherish a kindly, philanthropic 
Churchmanship, secure for the clergy a place midway 
between the Catholic and the Puritan with the force 
of neither, cultivate a reverence which is half aesthe-TK
tic and good taste, soften dogm.a by ethical interpreta-TK
tions, and urge moral improvement in a spirit of not 
too much zeal. He does not gauge as even the 
literary man does the great human tragedy; he 
knows not the stung soul's exceeding bitter cry, nor 
does he thrill to the world's woe or the central chord 
of expiation on the cross. He is institutional even 
more than ethical, and ethical more than s}MKmpathetic 
or enthusiastic. He is quietly devout and subduedly 
active; but he has no burthen, and he does not com-TK
pel them to come in by the native compulsion of the 
Gospel word. He has never truly reached the real 
marrow of Christian theology, the fundamental war 
of law and Gospel in the history of the soul. 



SOURCES OF THE PRIEST'S WELCOME 233 

VI 

Again, ethical preoccupation leads in large numbers 
of " quiet people " to religious indifference, and re-TK
ligious indifference is the best soil for the priest. By 
religious indifference I mean the absence of personal 
concern and experience; I do not mean the lack of all 
interest in religious truths, institutions, or activities. 
In character and in philanthropy many stand high, and 
merit much, who are yet devoid of personal experience 
in the distinctive Christian sense of the word. There 
are those, even, who are entirely evangelical in their 
convictions, but their religion has never really passed 
beyond the region of truths; contact with a truth takes 
the place of commerce with a person. It should not 
be forgotten that the vicarious priesthood grew and 
flourished during those immature ages of the Church 
when right knowledge and good living were the sum of 
Christianity. And the new element in the Reforma-TK
tion which gave Christianity back to itself was the con-TK
viction that practical Christianity is not the plain man's 
pagan combination of certain authoritative views of God 
and the world with the practice of ethical virtues; but 
that it is the religious experience of trust in God's 
grace in Christ through faith, a faith which shapes the 
whole moral realm. There is, besides the absolute 
agnosticism-TKof science, the relative and practical ag-TK



234 ROME, REFORM, AND REACTION 

nosticism of the excellent modest man who worships 
reverence more than confidence, and is a sound 
Churchman more than a true believer. It troubles him 
more to presume to know too much, than to shrink and 
trust too little. He is more afraid of pushing than he 
is of distrusting. He cherishes a vague hope of mercy, 
rather than a sure faith in grace. He hopes to be 
forgiven, rather than is sure that he is. He is bold in 
thin2;s honest, but most timid in things of faith. He 
is not so angry with the priest's claims as he is with 
the secret ways by which they are taught. There 
is that relative and even Christian agnosticism; and it 
may seem harsh to associate it with the more absolute 
and systematic---to mix up the man who knows no 
truth about God with the man who knows nothing but 
truth about God. But they are both strange to the 
real humility of Christian freedom and confidence in 
living faith. They are strange to direct and personal 
experience of God. And they are both types of 
mind too weak in the religious constitution to with-TK
stand attacks of the priest, chronic or acute. 

VII 

The temper of the hour is to a large extent priestly, 
because it is humanist, aesthetic, and Pelagian. 
Pelagianism was the temper of the medieval and 
scholastic Church which developed the priesthood, 



SOURCES OF THE PRIEST'S WELCOME 235 

and the Reformers disowned it. We believe to-day 
in human nature, and the men of genius are its pro-TK
phets. It is a liberal age, and the liberal, humane 
view of man is carried into religion till it ousts the 
soul. Like the medievals, we inhabit an aesthetic 
age. Faith is nowhere in the reading (or at least in 
the writing) world, and love is everywhere, love is 
enough. By this sentimental apparatus of the poetical 
litterateur the whole hoary world of spiritual problem 
is attacked and reduced with the masterly freshness 
of a young lady at the social board, who feels 

"The first that ever burst 
Into that silent sea." 

And this literary apotheosis of love coincides with two 
tendencies in the interior of the Church itself. First, 
as Christian faith works out into love the children of the 
men of faith are more sensitive to love's atmosphere 
than to faith's. The grandchildren of the stalwart 
believers love their Christian homes and affections 
better than they understand the principles that reared 
them. They respond to the amenities of a cultured 
society better than they do to the vigour of Christian 
faith. They are more at home in a decorous and 
kindly Church than in a true. Cultured Protestantism 
itself loses the great evangelical note, and gravitates 
either to a feeble evangelism or to a Church of 
charm. And, secondly, the literary tendency falls in 



236 ROME, REFORM, AND REACTION 

with the standing Catholic doctrine which puts love 
where the New Testament puts faith. So that 
Catholicism has the advantage and help of aesthetic 
Pelagianism on the one hand, and of the cultured piety 
of Protestantism on the other. Beautifying grace gets 
the better of justifying grace. And this is especially 
the case with women and with the young, who have 
a place in the Church they never had, at least in Pro-TK
testantism, before. The fact is one which is here 
only noted and not deplored. It is all on the way to 
the promised land if only we do not think we have 
arrived. 

VIII 

Again, we must not overlook the welcome which 
the unspiritual man alv/ays gives to a religion which 
relieves him from spiritual effort, the very human 
belief in vicarious self-sacrifice and obedience by 
deputy. The separate, thaumaturgic, and dangerous 
character of the priesthood is due as much to the 
indolence of the laity as to the ambition of the 
clergy. It is the people that make the priests more 
than the priests make themselves. The vicarious 
priest flourishes on nothing so much as on public in-TK
difference. Canon Gore well points out how in the 
early Church the lowering of the average tone due to 
the rapid extension and secularising of tlie Church 



SOURCES OF TPIE PRIEST'S WELCOME 237 

tended to throw up and isolate the ministry, to cast it 
together upon itself for sympathy, and to make it a 
spiritual aristocracy. And so to-day the claims made 
for the priest and his detachment from the layman are 
due, to no small extent, to the protest which an earnest 
spirituality must by its very existence make against 
the secularisation of religion in a Church which is at 
once the Church of the State and of the rich and 
fashionable. 

IX 

But after all other causes have been allowed for I 
continue to think that few are more favourable to 
priestly rule than that which I have first named, and 
which is all the more powerful because it is subtle 
enough to seem absurd. I allude to the popular 
passion for "doing things," which when imported 
into religion prepares a congenial welcome for the 
thaumaturgic priest. It is a temper which when 
uncurbed goes before to prepare a place for him in 
the Protestant mind itself. The real roots of the 
Roman reaction lie in the unrealised Romanism of 
Protestants. And the Protestant root of a mass 
priesthood is the idea so dear to the English mind, 
so central to a rational Broad Churchism in every 
Church, and so plausible as the ethical movement---
the idea that the best action or conduct is contribu-TK



238 ROME, REFORM, AND REACTION 

tory to salvation instead of produced by it. This 
is the Pelagian and Synergistic faith of medieval 
Catholicism reappearing in the circles of humanist 
Protestantism. Nothing is in more distinct contrast 
with the Protestant doctrine of justification by faith 
alone, nor in contrast more fatal. To adopt it is in 
principle to renounce the Reformation, whether it be 
done on agnostic or on Catholic lines. The Refor-TK
mation had to break away for its life both from the 
Catholics and from the humanists. These, as I have 
said, took up Luther, but he outgrew them, as both 
Christ and Paul outgrew the humanist rabbis. It was 
not mere sacramental works that the Reformers denied 
to have saving value, but ethical no less. It was not 
the mere ritual of worship that Paul fought when he 
led Luther's way, but that of conduct as well. Man 
can contribute nothing to his own salvation. " Work 
out your own salvation, for it is God that worketh in 
you." Yes, but God the Redeemer; what works in 
you is the redemption which you have already appre-TK
hended by faith alone. The words were spoken not to 
the natural conscience but to the redeemed. Any form 
of Synergism is fatal to justification by grace alone, 
which is the base of true Protestant priesthood.MK 

1 The most intractable of opponents are not the priests after all, 
but the ethical agnostics in the first place, and the merely ethical 
Christians in the second. The agnostic men of science at the 



SOURCES OF THE PRIEST'S WELCOME 239 

Christianity is a religion and a faith before it is an 
ethic. It is ethical because of its faith in the supreme 
and all-inclusive ethical act of God in the Redeemer. 

The public mind, through the influence of a 
literary religion like Arnold's, has become deeply im-TK
bued with the idea that religion consists in behaving 
in a certain way, in doing something palpable, in 
belonging to the Church as " a society for the promo-TK
tion of goodness," in heroic or pretty self-sacrifices, in 
morality tinged with emotion. And so the public, 
having this, as it might be called, ergistic habit of 
mind, is not startled as it should be by the vicarious 
doing of the priest; especially as there is a large class 
of people who, when a religious question is pushed 
beyond considerations of habit and decorum, being 
not so much indifferent as ignorant, give it up with 
the statement that they leave all that to their clergy-TK
man. The love of doing things becomes indifferent 
to the way they are done. And thousands prize as a 
badge of mental altitude and noble carelessness the 
shallow jingle---

*' For forms of faith let graceless bigots fight; 
His can't be wrong whose life is in the right." 

Moreover, along with the aesthetic and ethical move-TK
great centres of culture, if compelled to vote on an issue which 
involves the question of a Catholic form of Christianity or an 
evangelical, will vote for the Catholic, though it is the organization 
of all that they most deny. 



240 ROME, REFORM, AND REACTION 

ment has gone a social movement, submerging the 
individual and his responsibility in his organism, and 
accepting the acts of the society's representative or 
cleric as the unit's acts even when there is no con-TK
tinuity of faith or sentiment between them. The 
peril of specialized function, so great in these over-TK
busy days, here appears in its religious form. The 
expert not only advises but replaces. Doing some-TK
thing is the condition of salvation, and it matters not 
that the doing is gone through by another so long as 
his credentials from the religious society are valid. 
When it is a case primarily of doing things, the con-TK
dition of the doer is a minor matter, and the action 
easily passes from ethical to sacramental, and from 
that to hieratic; and the priest, ceasing to be a real 
representative through the circuitous remoteness of his 
connection with the living soul, easily becomes a sub-TK
stitute, and soon grows sole. 

My point is that the most subtle, and for us the 
most perilous, departure from the New Testament 
and the Reformation is not in the priest who is ex-TK
press and positive in his claims. He is a symptom 
rather than a source. He would rouse our suspicion 
and alarm if it were not that we are got ready for 
him by a habit of public mind which opens the door 
from the inside. Our chief danger is the view and 
temper which makes that preparation and leaves the 



SOURCES OF THE PRIEST'S WELCOME 24I 

door ajar. It is the ethicism, practicism, ergism, nom-TK
ism---call it concisely what you will---the conduct-TK
worship and love of exploit which I have spoken 
of. This first takes up the debased idea of ortho-TK
doxy, that faith is belief in truths instead of Christ, 
and that un faith, therefore, is the denial of certain 
truths; it goes on accordingly in a liberalising way 
to identify faith with the mere love of truth; it 
proceeds, very naturally and properly, to urge that 
such faith is inferior to action; it then, in its Pelagian 
and humanist fashion, replaces faith by either politi-TK
cal and distributive justice, on the one hand, or by 
love, as the mere enthusiasm of humanity, on the 
other. These are its great motives and standards of 
action. And, throughout, it follows a debased and 
institutionalised Church in totally missing the true 
nature of faith as itself the supreme act, the initial 
and final surrender of the personality to the grace of 
God, the greatest and most compendious exertion of 
will of which man is capable, with all the integrities 
and humanities in its bosom. When the process has 
gone so far faith has ceased to be a matter of mere 
assent, and yet it has not become an act of will of 
the spiritual and decisive kind just described. The 
house is swept and garnished, but unoccupied still. 
And it is left open to the most attractive forms of 
action---ritual, ethical, or imaginative; to rites, con-TK

16 



242 ROME, REFORM, AND REACTION 

duct, or heroisms in their aesthetic aspect. And 
salvation is bound to become a thing either of admir-TK
able behaviour or of impressive ceremonial, if it is not 
to sink into the matter of sentiment viMKhich it has 
become in the feebler sects. 

A religion of conduct tends to become a religion of 
ritual, because conduct is not religion and the appetite 
for " doing things " presses on to take a distinctly 
religious shape. If it do not find this in the true act 
of faith it finds it in viMKhat are called acts of faith, in 
sacraments as opera operata. If the idea of faith has 
been debased belovMKMK the level of the soul's one 
decisive and inclusive act there is only left, to fill the 
really religious soul's passion for action, the sacra-TK
mental path wMKith a vicarious priest. If the spiritual 
ansviMKer to God's sole act be not in itself our central 
act then religion asserts its active nature by be-TK
coming contributory to God, instead of responsive. 
It becomes synergistic in the outvvMKard way, the ritual 
and imposing viMKay (the ethical vMKMKay not satisfying 
religious and imaginative need). And the ethical 
Church, the society for the promotion of goodness, 
is ground up betwMKeen evangelical faith and Catholic 
sacraments, and its dust goes to the latter. 



SOURCES OF THE PRIEST'S WELCOME 243 



What we need, therefore, is a great rehabilitation 
of the idea and sense of faith among our Protestant 
selves, and not least among those sections of Pro-TK
testants that cherish a rational and liberal creed. We 
need a renewal of practical religion in the sense of a 
New Testament revised and revived, in the sense of 
a personal experience whose centre and genius is 
guilt, grace and forgiveness. Sin is not, as the Greek 
idea of it goes, infection with a moral microbe; and 
salvation is not mere imparted a(j)6apaca, or incor-TK
ruption. Nor is sin, as in the medieval idea, mere 
distance from God. It is what the Reformers de-TK
clared it to be, guilt. That idea, grasped in its ful-TK
ness and felt in its searching finality, was the great 
Reformation contribution to Christian faith. It made 
sin a religious besides a moral idea. The grace which 
saved from sin was not a sacramental infusion to 
counterwork an infused evil; it was the pure mercy 
of God exercised upon guilt and not injected against 
disease. Salvation was sine merito redimi de peccatis. 
That was the core of the Reformation. Sin became 
the idea that negatively coloured all, and prescribed the 
form of the positive faith that destroyed it. Redemp-TK
tion was the supreme humane interest, as it has now 
become, through Wagner, for humanism as art. It 



244 ROME, REFORM, AND REACTION 

was faith not in the love of God but in the justifying 
grace of God, which in Christ received believing 
sinners as if they were not sinful, yet treated them as 
if they were, and by so dealing with them made them 
saints. Faith as the response to love may be Catholic; 
evangelical faith is the response to justifying grace, 
to the central act of the moral universe as a religious 
and a priestly act. It is this faith which, in whatever 
modified form, must revive in unmodified power. It 
is the only power that can save us from the priest, 
and without which no readjustment like Disestablish-TK
ment can be of final use to religion. Agnostic 
science is a broken reed and a moral failure so long as 
it thinks the priest better for its wife and children 
and servants than the dreary negation it owns but 
cannot worship or trust at its own core. 

True faith in an act like the cross, and in a person 
like Christ, must inevitably ethicise itself. Its nature, 
because of its object, is a spiritual ethic, universal, 
nay celestial, in its range, final and fundamental in its 
penetration of the soul. But it can only ethicise itself 
by remaining above all things absolutely religious. 
That is to say, the object of trust must be the last 
reality, the final object of knowledge and thought, 
the ultimate source, power, seat, and goal of things. 
Faith must be the answer to His self-revealed nature 
and character. It must be the response to a positive. 



SOURCES OF THE PRIEST'S WELCOME 245 

final religion, and not the apotheosis of human 
religiosity. It has too readily accepted its moral cor-TK
rections from the natural conscience, and made its 
appeal to the natural mind. It has taken over ideals 
and conclusions which it had not developed, and 
which do not represent its own genius. In Catholi-TK
cism it did slowly what some Eastern nations have 
done more rapidly; it imported a civilization from the 
West ready made and full grown. As Japan takes 
Europe so young Christianity took the classical world. 
If it will return from the bondage of these ill-digested 
institutions, which it has served where it should have 
used---if it return from them to " readjust its compass 
at the cross" (as Goethe said), it will moralise and 
socialise itself at that source in a germane, distinctive 
and mighty way. But it is as dangerous for a weak 
and harassed faith to call in the literary or scientific 
ethics of the natural man as it is for a weak race to 
invite a strong one to its soil to help it against the 
perils its feebleness has caused. The mercenary force 
claims its conquest for payment. The guest stays on 
as virtual master, and the last state of that host is 
worse than the first. Faith at its own sources can 
throw off its own errors; it cannot be really cor-TK
rected or supplemented by unfaith. " Religions," 
says Harnack, " cannot be skinned; we must cause 
them to scale " (Dogmengesch., iii. 668). The fruit 



246 ROME, REFORM, AND REACTION 

of the Spirit is often an act of oblivion. " Much is 
to learn and much to forget." To be taught of God 
is to unlearn much; and to grow in grace is to cast 
off in the sun many wraps which the cold wind of 
mere criticism only blew the closer to our timid and 
benumbed souls. 



Butler \&. Tanner, The SeUvood Printing Works, Frome, and London 



\end{document}
