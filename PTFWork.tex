\documentclass[12pt,letterpaper,oneside]{book}
\usepackage[utf8]{inputenc}
\author{Peter Taylor Forsyth, D.D.}
\title{The Work of Christ}
%\publisher{Hodder and Stoughton}
\usepackage[english]{babel}
\usepackage[autostyle, english = american]{csquotes}
\MakeOuterQuote{"}


\makeindex

\begin{document}
\maketitle
%PRINCIPAL OF HACKNEY COLLEGE, HAMPSTEAD 

%LONDON: HODDER AND STOUGHTON 

%TORONTO: THE WESTMINSTER COMPANY LIMITED 


\chapter*{PREFACE} 

% Note that \textsc{} gives small caps. Still need to figure out how to do dropcaps.

\textsc{These} chapters need to have it said that 
they were given as extempore lectures 
from rough notes to a gathering, largely of 
young ministers, in connection with Rev. 
Dr. Campbell Morgan's annual conference at 
Mundesley, Norfolk. They were taken down 
in shorthand and then carefully revised. They 
took place in July, 1909, immediately after 
the delivery of my Congregational Lecture on 
the \textit{Person and Place of Christ}, which they 
supplement---especially when taken with my 
\textit{Cruciality of the Cross} a few months before. 
It will be seen from the conditions that the 
book cannot pretend to be more than a 
higher kind of popularisation, though this is 
less true of the two last chapters, which 
have been more worked over. The style approaches 
in parts a conversational familiarity 
which would have been out of place in addressing 
theological experts. And as some of the 
ideas are unfamiliar I have not been too careful 
\marginpar{vi}
to avoid repetition. My hope is to be of some 
use to those ministers who are still at a 
stage when they are seeking more footing on 
such matters than they have been provided 
with in mere Biblical or Historical Theology. 
There is no region where religion becomes so 
quickly theology as in dealing with the work 
of Christ. No doctrine takes us so straight 
to the heart of things, or so forces on us a 
discussion of the merits of the case, the dogmatic 
of it, as distinct from its scriptural or 
its ecclesiastical career. No doctrine draws so 
directly on the personal religion of sinful men, 
and none, therefore, is open to so much change 
in the course of the Church's thought upon its 
growing faith and life. Thus when we consider 
that here we are at once where the form may 
change most in time and yet the feet be most 
firmly set for eternity, we realise how difficult 
and delicate our task must be. And we are 
made to feel as if the due book on such a 
theme could only be written from behind the 
veil with the most precious blood that ever 
flowed in human veins. 

\begin{center}
\noindent\rule{4cm}{0.4pt}
\end{center}

We are in a time when a spirituality without 
positive content seems attractive to many minds. 
And the numbers may grow of those favouring 
\marginpar{vii}
an undogmatic Christianity which is without 
apostolic or evangelical substance, but cultivates 
a certain emulsion of sympathetic mysticism, 
intuitional belief, and benevolent action. Among 
lay minds of a devout and social but impatiently 
practical habit, this is not unlikely to spread; 
and particularly among those whose public interests 
get the upper hand of ethical and 
historical insight and denude their religion of 
most of the reflection it demands. 

Upon undogmatic, undenominational religion 
no Church can live. With mere spirituality the 
Church has not much directly to do; it is but 
a subjective thing; and its favour with many 
may be but another phase of the uncomprehending 
popular reverence (not to say superstition) 
for the recluse religionist, the mysterious 
ecstatic, and the ascetic pietist. What Christian 
faith and the Christian Church have to do with 
is \textit{holy} spirituality the spirituality of the Holy 
Spirit of our Redemption. The Christian revelation 
is not "God is a spirit," nor is it "God 
is love." Each of these great words is now 
much used to discredit the more positive faith 
from whose midst John wrote them down. 
Herein is love, not in affection but in propitiation 
(1 John iv. 10). Would Paul ever 
have written 1 Cor. xiii. if it had been revealed 
\marginpar{viii}
to him that it was going to be turned against 
Rom. iii. 25? And what would his language 
have been to those who abused that chapter 
so? Christian faith is neither spirituality nor 
charity. Its revelation is the holiness in judgment 
of the spiritual and loving God. Love 
if only divine as it is holy; and spirituality is 
Christian only as it meets the conditions of 
Holy Love in the way the Cross did, as the 
crisis of holy judgment and holy grace. If 
the Cross is not simply a manner of religion 
but the object of our religion and the site 
of revelation, then it stands there above all 
to effect God's holiness, and not to concentrate 
man's self-sacrifice. And except in the 
Cross we have no guarantee for the supreme 
thing, the divine thing, in God, which is the 
changeless reality and irresistible sovereignty 
of His Holy Love. 

It is upon such faith alone, given by the Cross 
alone, that a Church can live---upon the faith 
that founded it---upon a positive New Testament 
Gospel. Of that Gospel the Church is 
the trustee. And the Church betrays its trust 
and throws its life and its Lord away when it 
says, "Be beautifully spiritual and believe as 
you like," or "Do blessed good and think as you 
please." 
\marginpar{ix}
There is a timely saying of that searching 
Christian genius Kierkegaard---the great and 
melancholy Dane in whom Hamlet was mastered 
by Christ: 

"For long the tactics have been: use every 
means to move as many as you can---to move 
everybody if possible to enter Christianity. 
Do not be too curious whether what they enter 
\textit{is} Christianity. My tactics have been, with 
God's help, to use every means to make it clear 
what the demand of Christianity really is---if 
not one entered it." 

The statement is extreme; but that way lies 
the Church's salvation in its anti-Nicene relation 
to the world, its pre-Constantinian, non-established, relation to the world, and devotion 
to the Word. Society is hopeless except for the 
Church. And the Church has nothing to live on 
but the Cross that faces and overcomes the 
world. It cannot live on a cross which is on 
easy terms with the world as the apotheosis of 
all its aesthetic religion, or the classic of all its 
ethical intuition. The work of Christ, rightly 
understood, is the final spiritual condition of all 
the work we may aspire to do in converting 
society to the kingdom of God. 



\tableofcontents

\chapter{THE DIFFERENCE BETWEEN 
GOD'S SACRIFICE AND MAN'S} 
\chaptermark{The Difference Between God's Sacrifice}
%\marginpar{3}  



%I

%THE DIFFERENCE BETWEEN GOD'S 
%SACRIFICE AND MAN'S 

% STill need dropcaps and the title sorted out TK

\textsc{What} I am going to say is not directly unto 
edification, but indirectly it is so most 
certainly. Directly it is rather for that instruction 
which is a need in our Christian life as 
essential as edification. We cannot do without 
either. On the one hand instruction with no idea 
of edification at all becomes mere academical 
discourse. It may begin anywhere and it may 
end anywhere. On the other hand, edification 
without instruction very soon becomes a feeble 
and ineffective thing. I think a great many of 
us would be agreed that part of the poverty 
and weakness of the Church at the present 
moment is due to the fact that edification has 
been pursued to the neglect of instruction. We 
have been a little too prone to dwell upon the 
simple side of the gospel. All our capital is in
\marginpar{4}
small circulation. We have not put by a reserve, 
as it were. And therefore the simplicity 
itself has become unsettled and ineffectual, confused 
and confusing. 

I ask your attention to certain aspects of 
our Christian faith which perhaps do not lie 
immediately upon the surface, but which are 
yet the condition of the Church's continued 
energy and success in the world. I suppose there 
is nobody here who does not believe in the 
Church. At any rate, what I propose to say 
will be said entirely from that standpoint. We 
believe in the Holy Catholic Church. My contention 
would be that, apart from such a position 
as I desire to bring to your notice---some 
real apostolic belief in the real work of Jesus 
Christ---apart from that no Church can continue 
to exist. That is the point of view which I take 
at the outset. The Church is precious, not in 
itself, but because of God's purpose with it. It 
is there because of what God has done for it. 
It is there, more particularly, because of what 
Christ has done, and done in history. It is 
there solely to serve the Gospel. 

It is impossible not to observe at the 
present day that the Church is under a cloud. 
You cannot take any division of it, in any 
country of the world, without feeling that that 
\marginpar{5}
is so. Therefore I will begin by making quite a 
bold statement; and I should be quite prepared, 
given time and opportunity, to devote a whole 
week to making it good. The statement is 
that the Church of Christ is the greatest and 
finest product of human history. It is the 
greatest thing in the universe. That is in complete 
defiance of the general view and tendency 
of society at the present moment. I say the 
Church is the greatest and finest product of 
human history; because it is not really a 
product of human history, but the product of 
the Holy Spirit within history. It stands for the 
new creation, the New Humanity, and it has that 
in trust. The man who has a slight acquaintance 
with history is ready to bridle at a statement 
like that. He says: "Consider what the Roman 
Church has done; consider how obscurantist 
many sections of the Protestant Church are; 
consider the ineffectual position of the Church 
in modern civilisation---and what nonsense to 
talk about the Church as the greatest and 
finest product of human history!" True enough, 
the authority of the Church is failing in many 
quarters. And that does not mean only the 
external authority of what you might call a 
statutory Church, a great institutional Church, 
a great organised Church like Rome, for example.
\marginpar{6}
It means much more than that. It 
means that the authority of the whole Church is 
weakened in respect of the inward and spiritual 
matter which it contains and preaches, and 
which makes it what it is. The Church is 
there as the vehicle of the power of the Holy 
Ghost and of the authority of the saving 
God---a God, that is, who is saving not groups 
here and there, but the whole of human society. 
But a spiritual authority for man altogether 
is at a discount. Perhaps we have brought that 
in some measure upon ourselves. Perhaps, too, 
it was historically necessary. But, necessary 
or not, it is a matter of fact that our Protestantism 
has developed often into a masterless 
individualism which is as deadly to Christian 
life as an over-organised institution like Rome. 
Many spiritual people to-day find it difficult to 
make their choice between the two extremes. 
Without going into the historic causes of the 
situation, let us recognise the situation. 
Spiritual authority, especially that of the 
Church, is for the time being at a great 
discount.

\begin{center}
\S
\end{center}

The Church is valuable as the organ of 
Christian grace, and truth, and power. But 
what do we find offered us in place of the 
\marginpar{7}
Church? Those who attack the Church most 
seriously, and disbelieve in it most thoroughly, 
are not proposing simply to level the Church 
to the ground in the sense of destroying any 
religious society. What they want to do is to 
put some other kind of society in the place of 
the Church. For they say, as we all say, that it 
is impossible for religion, certainly impossible 
for Christianity, to exist without a social body 
in which it is cultivated and has its effect. 
Therefore, those who are opposed to the Church 
most bitterly are yet not prepared to make a 
total desert. But they put all kinds of organisations, 
fancy organisations and fancy religions, 
in its place. Take the great movement in the 
direction of Socialism. Take the Socialist programmes 
that you find so plentifully everywhere. 
What do these various organisations 
mean? What do all these organisations mean 
which profess to embody human brotherhood, 
and are represented by Trades Unions, Cooperation, Fraternities, Guilds, Socialisms? 
What is it they all confess? That some social 
vehicle there must be. You cannot promote 
Anarchy itself without associations for the 
purpose. So that the very existence of these 
rival organisations is a confession of the one 
fundamental principle of the Church, namely, 
\marginpar{8}
that the human ideal, that religion in the 
true sense of the word, cannot do without a 
social habitation. They put in their own way 
what we put in our way (and we think a better 
way), that there must be a Church builded 
together for a habitation of God in the Spirit. 
Our individualisms have been troubling and 
weakening us so much that everybody is looking 
away to some form of human life which 
shall have the advantages of individualism 
without its perils. The pietistic form of individualism 
did in its day great service. But it is 
out of date. Rationalistic individualism, again, 
taking shape in political radicalism, has done 
good work in its day. That also seems going 
out of date. The value of the new movement 
is its---shall I say---solidarity; which is a confession 
of that social, fraternal principle which 
finds its consummation really, and its power 
only, in the Church of Christ. 

When we look at these rival organisations 
(and they are many, and some will occur to 
you which I have not named), we can, I think, 
gather most of them under one head. In contrast 
with the Church the various social forms 
that are offered to us to-day would build society 
upon a natural basis, the basis of natural 
brotherhood, natural humanity, natural goodness\marginpar{9}---on human nature. And the issue between 
the Church and the chief rivals of the Church 
is an issue between society upon this natural 
basis, and society upon a supernatural basis. 
Our Christian belief is based upon the work 
of Christ; and we hold that human society can 
only continue to exist in final unity upon that 
same supernatural basis. It is an issue, therefore, 
between human nature deified and human 
nature saved; between mere sympathy and 
faith---faith taken in a quite positive and definite 
sense. We think that a brotherhood of 
mere sympathy, however warm it can be at a 
particular moment, has no stay in it, no eternal 
promise. The eternal promise is with supernatural 
faith. Do you ever believe otherwise? 
I hope you have been so tempted; because 
having got over it you will be very much better 
for having gone through it. I wish much more 
of our belief had gone through troubled scenes 
and come to its rest; we should make far 
greater impression upon men if we gave them 
to feel we had fought our way to the peace and 
power we have. Well, were you ever tempted 
to believe that Christianity is just human 
nature at its best? That is the most powerful 
and dangerous plea that is put forward just 
now in challenge of our Christian position 
\marginpar{10}
and Church. Is the Kingdom of God just our 
natural spirituality and altruism developed? 
Is it just the spirit of religion or self-sacrifice, 
which you often find in human nature, developed 
to its highest? Is that the Kingdom of 
God? I trust you believe not---that human 
nature is not capable, by all the finest sacrifices 
it might develop, of saving, of ensuring itself, 
and setting up the Kingdom of God. Take 
the best side of human nature, that side which 
moves men to unselfishness and sacrifice, the side 
that comes out in many a heroic battle, in the 
silent battles of our civilisation, where the 
victims get no applause and no reputation for 
their heroism whatever. Take the best side of 
human nature, illustrated in every coalpit 
accident and every such thing, in countless 
quiet homes of poverty, where lives are being 
worked down to the bone and ground to death 
toiling and slaving for others. Take the vast 
mass of fatherhood and motherhood living for 
the children only. Take that best side of human 
nature, make the most of it, and then put this 
question: "How does man's noblest work differ 
from Christ's great work?" That is the 
question to which I desire to attract your 
attention to-day. How does man's best work 
differ from Christ's great work? 
\marginpar{11}

\begin{center} \S \end{center}

Let me begin with a story which was reported 
in the Belgian papers some years ago. 

Two passenger trains were coming in opposite 
directions at full speed. As they approached the 
station, it was found the levers would not work, 
owing to the frost, and the points could not be 
set to clear the trains of each other. A catastrophe 
seemed to be inevitable; when a signalman threw himself flat between the rails, and 
with his hands held the tie-rod in such a way 
that the points were properly set and kept; and 
he remained thus while the train thundered 
over him, in great danger of having his head 
carried away by the low-hung gear of the 
Westinghouse brake. When the train had 
passed, he quietly rose and returned to his 
work. 

I offer you some reflections on this incident. 
It is the kind of incident that may be multiplied
indefinitely. I offer you certain reflections, 
first, on some of its analogies with Christ's 
work, and secondly, on some of its differences. 

\begin{center} \S \end{center} 

1. This man, in a very true sense, died and 
rose again. His soul went through what he 
would have gone through if he had never risen 
\marginpar{12}
from the track. He gave himself; and that is 
all a man can give at last. His deed had the 
moral value which it would have had if he had 
lost his life. He laid it down, but it did not 
please God to take it. Like Abraham's sacrifice 
of Isaac, it was complete and acceptable, even 
though not accepted. The man's rising from 
the ground---was it not really a resurrection 
from the dead? It was not simply a return to 
his post. He went back another man. He went 
back a heavenlier man. He had died and risen, 
just as if he had been called, and had gone, to 
God's presence could he but remain there. 
This is a death and rising again possible to us 
all. If the death and resurrection of Jesus 
Christ do not end in producing that kind of 
thing amongst us, then it is not the power of 
God unto salvation. These moral deaths and 
resurrections are what make men of us. "In 
deaths oft." That is the first point. 

\begin{center} \S \end{center} 

2. The second point is this. Not one of the 
passengers in either of those trains knew until 
they read it what had been done for them, nor 
to whom they owed their lives. It is so with 
the whole world. To-day it owes its existence, 
in a way it but poorly understands, to the 
\marginpar{13}
death and resurrection of Jesus Christ. That 
is the permanent element in Christianity---the 
Cross and resurrection of Jesus Christ. And 
yet it is nothing to all them that pass by. 
Under the feet of those travellers in Belgium 
there had taken place one of those deeds that 
are the very soul and glory of life, and they had 
no idea of it. Perhaps some of them were at 
the very moment grumbling at the staff of the 
railway for some small grievance or other. It 
is useful to remember, when we are inclined to 
grumble thus, what an amount of devotion to 
duty goes to make it possible for us to travel as 
safely as we do---far more than can be acknowledged 
by the payment of a wage. These 
people were ploughing along in safety over one 
of the railway staff lying in a living grave. I 
say it is so with the whole civilised world. Its 
progress is like that of the train; it seldom 
stops to think that its safety is owing to a 
divine death and resurrection, much more than 
heroic. The safety of that train was not due 
to the mechanism. The mechanism had gone 
wrong. It was not due to organisation, or to 
work done from fear of punishment. Heroic 
duty raised to martyrdom saved the whole 
train. And the world's progress is saved to-day 
because of a death and resurrection of which
\marginpar{14}
it knows little and mostly cares to know less. 
"\textit{Propter Jesum non qu{\ae}rimus Jesum}." The 
success of Christ hides Him. It is the death 
of Christ that is the chief condition of modern 
progress. It is not civilisation that keeps 
civilisation safe and progressive. It is that 
power which was in Jesus Christ and culminated 
in His death and resurrection. When 
people read the Bible, and get behind the 
Bible, and that principle comes home to them, 
it may sometimes be like the shock that those 
travellers would receive when they read in the 
newspaper of their risk and deliverance. 

\begin{center} \S \end{center}

3. Another point. And I am now coming on to 
the difference. This man died for people who 
would thrill with the sense of what they owed 
him as soon as they read about it. His act appeals 
to the instinct which is ready to spring to life 
in almost every breast. You felt the response 
at once when I told you the story. Some of you 
may have even felt it keenly. Do you ever feel 
as keenly about the devoted death of Christ? 
Perhaps you never have. You have believed it, 
of course, but it never came home to you and 
gripped you as the stories of the kind I instance 
do. You see the difference between Christ's 
\marginpar{15}
death and every case of human heroism. I am 
moving to answer that question I put a moment 
ago as to whether the development of the best 
in human nature would ever give us the work 
of Christ and the Kingdom of God. I have been 
illustrating one of the finest things in human 
nature, and I am asking whether, if that were 
multiplied indefinitely, we should yet have the 
effect which is produced by the death of Christ, 
or which is still to be produced by it in God's 
purpose. No, there is a difference between 
Christ's death and every case of heroism. 
Christ's was a death on behalf of people within 
whom the power of responding had to be 
created. Everybody thrills to that story I told 
you, and to every similar story. The power of 
response is lying there in the human heart 
ready it only needs to be touched. There is in 
human nature a battery charged with admiration 
for such things; you have only to put your 
knuckle to it and out comes the spark. But 
when we are dealing with the death of Christ 
we are in another position. Christ's was a 
death on behalf of people in whom the power of 
responding had to be created. We are all 
afraid of death, and rise to the man who delivers 
us from it. But we are not afraid of that worse 
thing than death from which Christ came to 
\marginpar{16}
deliver us. Christ's death was not a case of 
heroism simply, it was a case of redemption. 
It acted upon dull and dead hearts. It was 
a death which had to evoke a feeling not 
only latent but paralysed, not only asleep but 
dead. What does Paul say? "While we were 
yet without strength, Christ died for us"---without 
power, without feeling, as the full 
meaning is. 

Let me illustrate. Take a poet like Wordsworth. 
When he began to publish his poetry 
he was received, just as Browning was received 
later, with ridicule and contempt. The greatest 
critic of the time began an article in the leading 
critical organ of the day by saying, "This will 
never do." But it has done; and it has done for 
Jeffrey's critical reputation. Lord Jeffrey wrote 
himself down as one who was incapable of 
gauging the future, however much he might 
be capable of understanding the literature of 
the past. Some of you may remember---I 
remember perfectly well---the same kind of 
thing in the penny papers about Browning 
when he was fighting for recognition. I remember, 
when I was a student, reading articles 
in luminaries like \textit{The Standard} which sneered 
and jeered at Browning, just as smaller men to-day 
would sneer at men of like originality. But 
\marginpar{17}
Wordsworth and Browning have conquered. I 
take another case. Turner was assailed with 
even more ridicule when he exposed his works 
to the British public. What would have happened 
to Turner if Ruskin had not arisen to be 
his prophet I do not know. His pictures might 
not even have been mouldering in the cellars of 
the National Gallery. They might have been 
selling at little second-hand shops in back streets 
for ten shillings to any one who had eyes in his 
head. Wordsworth, Browning, and Turner were 
all people of such original and unprecedented 
genius that there was no taste and interest for 
them when they appeared; they had to create 
the very power of understanding themselves. 
A poet of less original genius, a great genius 
but less of a genius, like Tennyson, comes along, 
and he writes about the "May Queen" and 
"The Northern Farmer," and all those simple, 
elementary things which immediately fetch the 
handkerchiefs out. Now no doubt to do that 
properly takes a certain amount of genius. But 
it taps the prompt and fluent emotions; and the 
misfortune is that kind of work is easily counterfeited 
and abused by those who wish to 
exploit our feelings rather than exalt them. 
It is a more easy kind of thing than was done 
by those great geniuses I first named. Original
\marginpar{18}
poets like Wordsworth and Browning had to 
create the taste for their work. 

Now in like manner Christ had to make the soul 
which should respond to Him and understand 
Him. He had to create the very capacity for 
response. And that is where we are compelled to 
recognise the doctrine of the Holy Spirit as well 
as the doctrine of the Saviour. We are always 
told that faith is the gift of God and the work 
of the Holy Spirit. The reason why we are told 
that, and must be told it, lies in the direction 
I have indicated. The death of Christ had not 
simply to touch like heroism, but it had to 
redeem us into power of feeling its own worth. 
Christ had to save us from what we were too far 
gone to feel. Just as the man choked with 
damp in a mine, or a man going to sleep in 
arctic cold, does not realise his danger, and the 
sense of danger has to be created within him, 
so the violent action of the Spirit takes men by 
force. The death of Christ must call up more 
than a responsive feeling. It is not satisfied 
with affecting our heart. That is mere impressionism. 
It is very easy to impress an audience. 
Every preacher knows that there is nothing 
more simple than to produce tears. You have 
only to tell a certain number of stories about 
dying children, lifeboats, fire eseapes, and so on, 
\marginpar{19}
and you can make people thrill. But the thrill 
is neither here nor there. What is the thrill 
going to end in? What is the meaning of the 
thrill for life? If it is not ending as it should, 
and not ending for life, it is doing harm, not 
good, because it is sealing the springs of feeling 
and searing the power of the spiritual life. 

What the work of Christ requires is the 
tribute not of our admiration or even gratitude, 
not of our impressions or our thrills, but 
of ourselves and our shame. Now we are coming 
to the crux of the matter---the tribute of our 
shame. That death had to make new men of 
us. It had to turn us not from potential friends 
to actual, but from enemies into friends. It 
had not merely to touch a spring of slumbering 
friendship. There was a new creation. The 
love of God---I quote Paul, who did understand 
something of these things---the love of God is 
not merely evoked within us, it is "shed abroad 
in our hearts by the Holy Spirit which is given 
to us." That is a very different thing from 
simply having the reservoir of natural feeling 
tapped. The death of Christ had to do with 
our sin and not with our sluggishness. It had 
to deal with our active hostility, and not simply 
with the passive dullness of our hearts. Our 
hostility---that is what the easy-going people
\marginpar{20}
cannot be brought to recognise. That is what 
the shallow optimists, who think we can now 
dispense with emphasis on the death of Christ, 
feel themselves able to do---to ignore the fact 
that the human heart is enmity against God, 
against a God who makes demands upon it; 
who goes so far as to make demands for 
the whole, the absolute obedience of self. 
Human nature puts its back up against that. 
That is what Paul means when he speaks 
about human nature, the natural man the 
carnal man is a bad translation---being enmity 
against God. Man will cling to the last rag of 
his self-respect. He does not part with that 
when he thrills, admires, sympathises; but he
does when he has to give up his whole self in 
the obedience of faith. How much self-respect 
do you think Paul had left in him when he went 
into Damascus? Christ, with the demand for 
saving obedience, arouses antagonism in the 
human heart. And so will the Church that 
is faithful to Him. You hear people of the 
type I have been speaking about saying, If 
only the Church had been true to Christ's 
message it would have done wonders for the 
world. If only Christ were preached and practised 
in all His simplicity to the world, how fast 
Christianity would spread. Would it? Do you
\marginpar{21}
really find that the deeper you get into Christ 
and the meaning of His demands Christianity 
spreads faster in your heart? Is it not very 
much the other way? When it comes to close 
quarters you have actually to be got down and 
broken, that the old man may be pulverised and 
the new man created from the dust. Therefore 
when we hear people abusing the Church 
and its history the first thing we have to say 
is, Yes, there is a great deal too much truth 
in what you say, but there is also a greater 
truth which you are not allowing for, and it 
is this. One reason why the Church has 
been so slow in its progress in mankind and 
its effect on human history is because it has 
been so faithful to Christ, so faithful to His 
Cross. You have to subdue the most intractable, 
difficult, and slow thing in the world---man's 
self-will. You cannot expect rapid successes 
if you truly preach the Cross whereon 
Christ died, and which He surmounted not 
simply by leaving it behind but by rising again, 
and converting the very Cross into a power
and glory. 

Christ arouses antagonism in the human heart 
and heroism does not. Everybody welcomes a 
hero. The minority welcome Christ. We do 
resent His absolute command. We do resent 
\marginpar{22}
parting completely with ourselves. We do 
resent Christ. 

\begin{center} \S \end{center} 

4. I go back to the word I spoke about the 
tribute of our shame. The demand is unsparing, 
remorseless. It is not simply that you 
are called on by God for a certain due, a 
change, an amendment, but for the tribute of 
yourself and your shame. When you heard 
about that heroism of my story, when you 
thrilled to it, I wonder did you pat yourself on 
the back a little for being capable of thrilling 
to things so high, so fine? When you thrilled 
to that story you felt a certain satisfaction with 
yourself because there was as much of the God 
in you as allowed you to be capable of thrilling 
to such heroisms. You felt, If I am capable of 
thrilling to such things, I cannot be such a bad 
sort. But when you felt the meaning of 
Christ's death for you, did you ever pat yourself 
on the back? The nearer the Cross came to 
you, the deeper it entered into you, were you 
the more disposed to admire yourself? There is 
no harm in your feeling pleased with yourself 
because you were able to thrill to these human 
heroisms; but if the impression Christ makes 
upon you is to leave you more satisfied with 
yourself, more proud of yourself for being able
\marginpar{23}
to respond, He has to get a great deal nearer to 
you yet. You need to be---I will use a Scottish 
phrase which old ministers used to apply to 
a young minister when he had preached a 
"thoughtful and interesting discourse"---you 
need to be well shaken over the mouth of the 
pit. The great deep classic cases of Christian 
experience bear testimony to that. Christ and 
His Cross come nearer and nearer, and we do 
not realise what we owe Him until we realise 
that He has plucked us from the fearful pit, 
and the miry clay, and set us upon a rock of 
God's own founding. The meaning of Christ's 
death rouses our shame, self-contempt, and 
repentance. And we resent being made to feel 
ashamed of ourselves, we resent being made to 
repent. A great many people are afraid to 
come too near to anything that does that for 
them. That is a frequent reason for not going 
to church. 

\begin{center} \S \end{center} 

5. Again, continuing. You would have gone a 
long way to see this Belgian man. You would 
have gazed upon him with something of reverence, 
certainly with admiration. You would 
have regarded him as one received back from 
the dead. You think, If all men were like that, 
the world would be heaven. Well, there are a 
\marginpar{24}
great many more like that than we think, who 
daily imperil their life for their duty. But 
supposing every man and woman in the world 
were up to that pitch, and supposing you added 
them all together and took the total value of 
their moral heroism (if moral quantities were 
capable of being summed like that), would you 
then have the equivalent of the deed and death 
of Christ? No, indeed! If you took all the 
world, and made heroes of them all, and kept 
them heroic all their lives, instead of only in one 
act, still you would not get the value, the equivalent, 
of Christ's sacrifice. It is not the sum of all 
heroisms. It would be more true to say it is the 
source of all heroisms, the foundation of them 
all. It is the underground something that makes 
heroisms, not something that heroisms make 
up. When Christ did what He did, it was not 
human nature doing it, it was God doing it. 
That is the great, absolutely unique and 
glorious thing. It is God in Christ reconciling. 
It was not human nature offering its very best 
to God. It was God offering His very best to 
man. That is the grand difference between the 
Church and civilisation, even when civilisation 
is religious. We must attend more to those 
great issues between our faith and our world. 
Our religion has been too much a thing done 
\marginpar{25}
in a corner. We must adjust our religion to 
the great currents and movements of the 
world's history. And the great issue of the 
hour is the issue between the Church and 
civilisation. Their essential difference is this. 
Civilisation at its best represents the most 
man can do with the world and with human 
nature; but the Church, centred upon Christ, 
His Cross, and His work, represents the best 
that God can do upon them. The sacrifice 
of the Cross was not man in Christ pleasing 
God; it was God in Christ reconciling man, 
and in a certain sense, reconciling Himself. My 
point at this moment is that the Cross of Christ 
was Christ reconciling man. It was not heroic 
man dying for a beloved and honoured God; it
was God in some form dying for man. God 
dying for man. I am not afraid of that phrase; 
I cannot do without it. God dying for man; 
and for such men---hostile, malignantly hostile 
men. That is a puzzling phrase where we read 
in a gospel: "Greater love hath no man than 
this, that a man lay down his life for his 
friends." There is more love in the phrase 
of the epistle, that a man should lay down 
his life for his bitter enemies. It is not so 
heroic, so very divine, to die for our friends. 
Kindness between the nice people is not so 
\marginpar{26}
very divine---fine and precious as it is. To die 
for enemies that is the divine thing. Christ's 
was grace that died for such---for malignant 
enemies. There is more in God than love. 
There is all that we mean by His holy grace. 
Truly, "God is love." Yes, but the kind of love 
which you must interpret by the whole of 
the New Testament. When John said that, did 
he mean that God was simply the consummation 
of human affection? He knew that he 
was dealing with a holy, gracious God, a God 
who loved His enemies and redeemed them. 
Read with extreme care 1 John iv. 10. 

\begin{center} \S \end{center}

6. Let me gather up the points of difference 
which I have been indicating. 

First, that Belgian hero did not act from love 
so much as from duty. Secondly, he died only 
in one act, not in his whole life, dying daily. 
There have been men capable of acts of sacrifice 
like this hero; loose-living men who, after 
a heroism, were quite capable of returning to 
their looseness of life---heroes of the Bret Harte 
type. There have been many valiant, fearless 
things done on the battlefield by men who in 
the face of bullets never flinched, never turned 
a hair; and when they came home they could 
\marginpar{27}
not stand against a breath of ridicule, they 
could not stand against a little temptation, and 
were soon wallowing in the mire. One act of 
sacrifice is not the same thing as a life gathered 
into one consummate sacrifice, whose value is 
that it has the whole personality put into it 
for ever. 

Third, this man could not take the full 
measure of all that he was doing, and Christ 
could. Christ did not go to His death with 
His eyes shut. He died because He willed to 
die, having counted the cost with the greatest, 
deepest moral vision in the world. 

Fourthly, the hero in the story had nothing 
to do with the moral condition of those whom 
he saved. The scoundrel and the saint in that 
train were both alike to him. 

Again, he had no quarrel with those whom he 
saved. He had nothing to complain of. He had 
nothing from them to try his heroism. They 
were not his bitter enemies. His valour was 
not the heroism of forgiveness, where lies the 
wondrous majesty of God. His act was not 
an act of grace, which is the grand glory of 
the love of Christ. Christ died for people who 
not only did not know Him, but who hated and 
despised Him. He died, not for a trainful of 
people, but for the whole organic world of
\marginpar{28}
people. It was an infinite death, that of His, 
in its range and in its power. It was death 
for enemies more bitter than anything that 
man can feel against man, for such haters as 
only holiness can produce. Here is the singular 
thing: the greater the favour that is done to 
us, the more fiercely we resent it if it does not 
break us down and make us grateful. The 
greater the favour, if we do not respond in its 
own spirit, so much the more resentful and 
antagonistic it makes us. I have already said 
that we speak too often as though the effect 
of Christ's death upon human nature must be 
gratitude as soon as it is understood. It is 
not always gratitude. Unless it is received in 
the Holy Ghost, the effect may just be the 
other way. It is judgment. It is a death unto 
death. 

\begin{center} \S \end{center}

I conclude by saying what I have often said, 
and what often needs saying, that it is not 
possible to hear the gospel and to go away just 
as you came. I wish that were more realised. 
We should not have so many sermon-hunters. 
If people felt that every time they heard the 
gospel they were either better or worse for it, 
they would be more careful about hearing. 
They would not go so often, possibly; better they 
\marginpar{29}
should not, perhaps. I am not speaking about 
hearing of sermons. That is neither here nor 
there. A man may hear sermons and be neither 
the better nor the worse. But a man cannot 
hear the gospel without being either better or 
worse, whether he knows it or not. When you 
come to face the last issues, it is either unto 
salvation or unto condemnation. The great 
central, decisive thing, the last judgment of the 
world, is the Cross of Christ. The reason why 
so many sermons are found uninteresting is not 
always due to the dullness of the preacher. God 
knows how often that is the case, but it is not 
always. It is because the sermons so often turn, 
or ought to turn, upon the miracle of the grace 
of God, which is so great a miracle that it is 
strange, remote, and alien to our natural ways 
of thinking and feeling. It seems foreign to us. 
It is like reading a guide-book if you have never 
been in the country. I take down my Baedeker 
in the winter and read it with the greatest 
delight, because I know the country. If I had 
not been there I should find it the dreariest reading. 
Why do not people read the Bible more? 
Because they have not been in that country. 
There is no experience for it to stir and develop. 
The Cross of Christ, the infinite wonder of it---we 
have got to learn that. We have got to 
\marginpar{30}
learn the deep meaning of that by having been 
there, by the evangelical experience whose lack 
is the cause of all the religious vagrancy of the 
hour. We have got to learn that it was not 
simply magnificent heroism, but that it was God 
in Christ reconciling the world. It was God 
that did that work in Christ. And Christ was 
the living God working upon man, and working 
out the Kingdom of God. 


\chapter{THE GREAT SACRIFICIAL WORK 
IS TO RECONCILE} 
\chaptermark{The Great Sacrificial Work}
%\marginpar{33}

%II 



%THE GREAT SACRIFICIAL WORK IS TO 
%RECONCILE 

\begin{center}
Corinthians v.14--vi.2; Romans v.1--11; Colossians i.10--29;
Ephesians ii.16. 
\end{center}


\textsc{The} great need of the religious world to-day 
is a return to the Bible. That is necessary 
for two reasons, negative and positive. Negatively,
because the most serious feature of the 
hour in the life of the Church is the neglect of 
the Bible for personal use and study by religious 
people. Positively, because we have to-day enormous 
advantages in connection with that return 
to the Bible. Modern scholarship has made of 
the Bible a new Book. It has in a certain sense 
rediscovered it. You might say that the soul 
of the Reformation was the rediscovery of the 
Bible; and in a wider sense that is true to-day 
also. We have, through the labours of more 
than a century of the finest scholarship in all 
\marginpar{34}
the world, come to understand the Bible, in 
its original sense, as it was never understood 
before. These instructed scribes draw forth 
from their treasury things as new as old. It 
is the old Book, and it is a new Book. It 
remains the old Book, and the precious Book, 
because of its power of unceasing self-renovation. 
The spirit that lives within the Bible is 
a spirit of constant self-preservation. One way 
of describing the Reformation is to say that, 
since the early Gnostic centuries, it was the 
greatest effort that ever took place in the 
Church for the self-preservation of Christianity. 
Remember, the Church was not reformed from 
the outside, but from the inside. It was the 
Church reforming the Church. It was the 
Church's faith that arose, under the Holy Spirit, 
and reformed the Church. So it is with the 
Bible. Whatever renovation we find in connection 
with the Bible---I do not here mean renovation 
of ourselves, but renovation of our way of 
understanding the Book---arises out of the Bible 
itself. This remains true to-day, as it was true 
in the Reformation time, although it is now 
true in a somewhat different application. The 
Bible is still the best commentary upon itself. 
I have always done much in my ministry in 
the way of expounding the Bible, and I would 
\marginpar{35}
say to the younger ministers particularly who are 
here, Do not be afraid of that manner of preaching. 
I have known young ministers who were 
over-scrupulous. I have known them say, "If I 
take a long text people will think it is because 
I am lazy and do not want the labour of getting 
a sermon out of a small one." Never mind such 
foolish people. Do not be afraid of long texts, 
long passages. Preach less from verses and 
more from paragraphs. If I had my time over 
again I would do a great deal more in that way 
than I have done. Read but one lesson, and read 
it with elucidatory comments. Of course some 
people can do that better than others. There 
is always the danger that if a person try it who 
has no sort of knack in that direction, the people 
will feel they have been let in for two sermons 
instead of one; and, excellent as these might 
be, people do not like to feel they have been got 
to church upon false pretences. It might even 
give an excuse to certain people for omitting 
one of the services altogether, on the plea they 
had put in the requisite amount of attention at 
one service. I would also admit that if you do 
this it will not reduce your labour. It will really 
add what might amount to another sermon to 
your weekly work. It is no use doing it if you 
do it on the spur of the moment. If you just say 
\marginpar{36}
things that occur to your mind while you are 
reading, you may say some banal, or some nonsensical 
and fantastic things. It means careful 
preparation. The lesson should be prepared as 
truly as the prayer should be prepared, and as 
the sermon should be prepared. You have to 
work your way through the chapter with the 
aid of the best commentary that you can get; 
and you have to exercise continual judgment in 
doing so lest you be dragged away into little 
matters of detail instead of keeping to the 
larger lines of thought in the passage in hand. 
Then, if you do as I say, there is this other 
advantage, that you can take a particular verse 
out of the long passage for your sermon; and 
thus you come to the sermon with an audience 
which you yourself have prepared to listen to 
you. You have created your own atmosphere, 
and you have done it on a Bible basis. 

Now I will confess against myself that sometimes, 
as I preach about here and there, and 
have done as I have been recommending you to 
do, people have come to me afterwards and said, 
as nicely as they could, that the sermon was all 
very well, but in respect of the reading of the 
Scripture, they never heard it after that fashion; 
they had never realised how vivid Scripture 
could become. That simply results from paying 
\marginpar{37}
attention to the chapter with the best help. 
You will find, I am sure, that your congregation 
will welcome it.

\begin{center}
\S
\end{center} 

Supposing, then, we return to the Bible. 
Supposing that the Church did---as I think it 
must do if it is not going to collapse; certainly 
the Free Churches must---supposing we return 
to the Bible, there are three ways of reading the 
Bible. The first way asks, What did the Bible 
say? The second way asks, What can I make 
the Bible say? The third way asks, What does 
God say in the Bible? 

\begin{center}
\S
\end{center} 

The first way is, with the aid of these magnificent 
scholars, to discover the true historic 
sense of the Bible. There is no more signal 
illustration of success here than in the case of 
the Prophets. During the time when theology 
dominated everything and was considered to 
be the Church's one grand concern, about one 
hundred years after the Reformation, when 
its great prophets had passed away, and the 
Church had fallen into different hands, the 
whole of the Old Testament---the Prophets 
amongst the rest---was read for proof passages 
of theological doctrines. Now for books like 
\marginpar{38}
the Prophets that is absolutely fatal---fatal to 
the books and to the Church; and fatal in the 
long run to Christian truth. There is no greater 
service that has been done to the Bible than 
what has been done by the scholars I speak 
of, in making the Prophets live again, putting 
them in their true historical setting and position. 
Dr. George Adam Smith, for example, has done 
inestimable service in this way. And what 
has been done for the Prophets has also been 
done for the New Testament. Immense steps 
onward have been taken; and we are coming 
to know with much exactness what the writer 
actually had in his mind at the moment of 
writing, and what he was understood to have 
had in his mind by those to whom he first 
wrote. In this way we get rid, for example, 
of the idea that Paul was thinking about us 
who live two thousand years after him. He 
was not thinking of us at all. He did not 
expect the world to last a century. It is quite 
another question what the Holy Spirit was 
thinking about. Paul was thinking in a natural 
way about his age and his Churches, about their 
actual situation and needs. That is another 
illustration of the principle that if you want 
to work for immortality you must work in 
the most relevant and faithful way amid the 
\marginpar{39}
circumstances round about you. The present 
duty is the path to immortality. And so also 
I might illustrate in respect to the Gospels. 

\begin{center}
\S
\end{center}

The second way of reading the Bible is reading 
it unto edification. That is to say, we read 
a passage, and we allow ourselves to receive 
any suggestion that may come to us from it, 
and we do not stop to ask whether that was in 
the writer's mind, or whether it was in the 
mind of the people to whom he wrote. That is 
immaterial. We allow the Spirit of God to 
suggest to us whatever lessons or ideas He 
thinks fit out of the words that are under our 
eyes. We read the Bible not for correct 
or historic knowledge, but for religious and 
spiritual purposes, for our own private and 
personal needs. That is, of course, a perfectly 
legitimate thing---indeed, it is quite necessary. 
It is the way of reading the Bible which the 
large mass of the Church must always practise. 
But it has its dangers. You need the other 
ways to correct it. All the three must cooperate 
for the true use and understanding of 
the Bible by the Church at large. But I am 
speaking now about its use by individuals, 
and the danger I mean is that the suggestiveness
\marginpar{40}
may sometimes become fantastic. Some 
preachers fail at times in that way. They get 
to taking what are called fancy texts, texts 
which impress the audience much more with 
the ingenuity of the preacher than with his 
inspiration. For instance, a preacher in the 
North, now dead, was preaching against the 
Higher Criticism and its slicing up of the 
Bible, and he took his text from Nehemiah, 
"He cut it with a penknife"! That is all very 
well, perhaps, for a motto, but for a text it 
is rather a liberty. It is not fair to the Bible 
to indulge in much of that at least. If I remember 
rightly, Dr. Parker had a great gift in 
this way, and more than sometimes it ran away 
with him. It is a temptation of every witty 
man, and every ingenious-minded man. But 
there is a peril in it, the abuse of a right principle. 
We are bound, of course, to vindicate 
for ourselves and for others the right to use the 
Bible in the suggestive way, if we are not to 
make a present of it to the scholars. And that 
would be just as bad as making a present of it 
to a race of priests. But when we read too 
much in that way it is apt to become a minister 
to our spiritual egotism, or, what is equally bad, 
our fanciful subjectivity. 

Now the grand value of the Bible is just the 
\marginpar{41}
other thing---its objectivity. The first thing is 
not how I feel, but it is, How does God feel, 
and what has God said or done for my soul? 
When we get to real close quarters with that 
our feeling and response will look after itself. 
Do not tell people how they ought to feel 
towards Christ. That is useless. It is just 
what they ought that they cannot do. Preach 
a Christ that will make them feel as they ought. 
That is objective preaching. The tendency and 
fashion of the present moment is all in the 
direction of subjectivity. People welcome 
sermons of a more or less psychological kind, 
which go into the analysis of the soul or of 
society. They will listen gladly to sermons on 
character-building, for instance; and in the 
result they will get to think of nothing else 
but their own character. They will be the 
builders of their own character; which is a 
fatal thing. Learn to commit your soul and 
the building of it to One who can keep it 
and build it as you never can. Attend then to 
Christ, the Holy Spirit, the Kingdom, and the 
Cause, and He will look after your soul. A 
consequence of this passion for subjective and 
psychological analysis, for sentimental experience 
and problem-preaching, is that when 
a preacher begins preaching a real, objective, 
\marginpar{42}
New Testament gospel he has raised against 
him what is now the most fatal accusation---even 
within the Christian Church it has come 
to be very fatal---he is accused of being a 
theologian. That is a very fatal charge to 
make now against any preacher. It ought to 
be actionable in the way of libel. We have 
come to this---that if you penetrate into the 
interior of the New Testament you will be 
accused of being a theologian; and then it is 
all over with your welcome. But that state 
of things has to be turned upside down, else 
the Church dries into the sand. There is no 
message in it. 

\begin{center}
\S
\end{center}


The third way of reading the Bible is reading 
it to discover the purpose and thought of God, 
whether it immediately edify us or whether it 
do not. If we did actually become aware of the 
will and thought of God it would edify us as 
nothing else could. No inner process, no discipline 
to which we might subject ourselves, no 
way of cultivating subjective holiness would do 
so much for us as if we could lose ourselves, and 
in some godly sort forget ourselves, because we 
are so preoccupied with the mind of Christ. If 
you want psychological analysis, analyse the 
will, work, and purpose of Christ our Lord. I 
\marginpar{43}
read a fine sentence the other day which puts in 
a condensed form what I have often preached 
about as the symptom of the present age: 
"Instead of placing themselves at the service 
of God most people want a God who is at their 
service." These two tendencies represent in the 
end two different religions. The man who is 
exploiting God for the purposes of his own soul 
or for the race, has in the long run a different 
religion from the man who is putting his own 
soul and race absolutely at the disposal of the 
will of God in Jesus Christ. 

\begin{center}
\S
\end{center}

All this is by way of preface to an attempt to 
approach the New Testament and endeavour to 
find what is really the will of God concerning 
Christ and what Christ did. Doctrine and life 
are really two sides of one Christianity; and 
they are equally indispensable, because Christianity 
is living truth. It is not merely 
truth; it is not simply life. It is living 
truth. The modern man says that doctrine 
which does not pass into life is dead; 
and then the mistake he makes is that he 
wants to turn it into life directly, and to 
politicise it, perhaps; whereas it works indirectly.
The experience of many centuries, 
\marginpar{44}
on the other hand, says that Christian life 
which does not grow out of Christian doctrine 
becomes a failure. If not in individuals, it 
does in the Church. You cannot keep Christian 
piety alive except upon Christian truth. You 
can never get a Catholic Church except by 
Catholic truth. I think perhaps we all here 
agree about that. It is of immense importance 
that we do not think entirely about our individual 
souls, and that we think more about 
the Church, the divine will, the divine Word, 
and the divine Kingdom in the world. It is 
of supreme importance that we should know 
what the Christian doctrine is on the great 
matters. 

Now in connection with the work of Christ 
the great expositor in the Bible is St. Paul. 
And Paul has a word of his own to describe 
Christ's work---the word "reconciliation." But 
he thinks of reconciliation not as a doctrine but 
as an act of God---because he was not a theologian 
but an experience preacher. To view it 
so produces an immense change in your whole 
way of thinking. It secures for you all that 
is worth having in theology, and it delivers 
you from the danger of obsession by theology 
in a one-sided way. Remember, then, that the 
truth we are dealing with is precious not as a 
\marginpar{45}
mere truth but as the means of expressing the 
eternal act of God. The most important thing 
in all the world, in the Bible or out of it, is 
something that God has done---for ever finally 
done. And it is this reconciliation; which is 
only secondarily a doctrine; it is only secondarily 
even a manner of life. Primarily it is an act of 
God. That is to say, it is a salvation before it is 
a religion. For Christianity as a religion stands 
upon salvation. Religion which does not grow 
out of salvation is not Christian religion; it 
may be spiritual, poetic, mystic; but the essence 
of Christianity is not just to be spiritual; it is 
to answer God's manner of spirituality, which 
you find in Jesus Christ and in Him crucified. 
Reconciliation is salvation before it is religion. 
And it is religion before it is theology. All 
our theology in this matter rests upon the 
certain experience of the fact of God's salvation. 
It is salvation upon divine principles 
It is salvation by a holy God. It is bound 
of course, to be theological in its very nature 
Its statement is a theology. The moment 
you begin to talk about the holiness of God 
you are theologians. And you cannot talk 
about Christ and His death in any thorough 
way without talking about the holiness of 
God. 
\marginpar{46}

\begin{center}
\S 
\end{center} 

Christ and Him crucified, that is the historic 
fact. But what do I mean when I say Christ 
and Him crucified? Does it mean that a certain 
personality lived who was recognised in history 
as Jesus Christ, and that He came by His end 
by crucifixion? That in itself is worthless for 
religious purposes. It is useful enough if you 
are writing history; but for religion historical 
fact must have interpretation, and the whole of 
Christianity depends upon the interpretation 
that is put upon such facts. You will find 
people sometimes who say, "Let us have the 
simple historic facts, the Cross and Christ." 
That is not Christianity. Christianity is a 
certain interpretation of those facts. How and 
why did the New Testament come into being? 
Was it simply to convince posterity that those 
facts had taken place? Was it simply to convince 
the world that Christ had risen from the 
dead? If that were the grand object of the 
New Testament we should have a very different 
Bible in our hands, one addressed to the world 
and not to the Church, to critical science and 
not to faith; and there would not be so much 
argument amongst scholars as there is. The 
Bible did not come into being in order to 
provide future historians with a valuable document.
\marginpar{47}
It came for the purposes of interpretation. 
Here is a sentence I came across once: 
"The fact without the word is dumb; the word 
without the fact is empty." It is useful to turn 
it over and over in your mind. 

Paul was almost the creator and the great 
representative of that interpretation. It was 
continued on his lines by Augustine, Anselm, 
Luther, and many another. But what is it 
that we hear about so much to-day? We 
hear a great deal about an undogmatic Christianity. 
And there is a certain plausibility in 
it. If you have no theological training, no 
training in the understanding of the Scripture 
in a serious way, that is, if you do not know 
your business as ministers of the Word, it seems 
natural that undogmatic Christianity should be 
just the thing you want. Leave the dogma 
of it, you will say, to those who devote their 
lives to dogma---just as though theologians were 
irrepressible people who take up theology as a 
hobby and become the bores of the Church! 
It was not a hobby to the apostles. Why, 
there are actually people of a similar stamp 
who look upon missions as a hobby of the 
Church, instead of their belonging to the 
very being and fidelity of the Church. So 
some people think theology is a hobby, and 
\marginpar{48}
that theologians are persons with an uncomfortable 
preponderance of intellect, who are 
trying to destroy the privileges secured by 
our national lack of education and to sacrifice 
Christianity to mind. People say we do not 
want so much intellect in preaching; we want 
sympathy and unction. Now, I am always looking 
afield, and looking forward, and thinking 
about the prospects of the Church in the great 
world. And unction dissociated from Christian 
truth and Christian intelligence has at last the 
sentence of the Church's death within itself. 
You may cherish an undogmatic Christianity 
with a sort of magnetic casing, a purely human, 
mystical, subjective kind of Christ for yourself 
or an audience, but you could not continue to 
preach that in a Church for the ages. The 
Church could not live on that and do its 
preaching in such a world. You could not 
spread a gospel like that. Subjective religion 
is valuable in its place, but its place is limited. 
The only Cross you can preach to the whole 
world is a theological one. It is not the fact 
of the Cross, it is the interpretation of the 
Cross, the prime theology of the Cross, what 
God meant by the Cross, that is everything. 
That is what the New Testament came to 
give. That is the only kind of Cross that 
can make or keep a Church.
\marginpar{49}

\begin{center}
\S
\end{center} 

You will say, perhaps, "Cannot I go out and 
preach my impressions of the Cross?" By all 
means. You will only discover the sooner that 
you cannot preach a Cross to any purpose if you 
preach it only as an experience. If you only 
preach it so you would not be an apostle; and 
you could not do the work of an apostle for the 
Church. The apostles were particular about 
this, and one expressed it quite pointedly: "We 
preach not ourselves [nor our experiences] but 
Christ crucified." "We do not preach religion," 
said Paul, "but God's revelation. We do not 
preach the impression the Cross made upon 
us, but the message that God by His Spirit sent 
through a Christ we experience." And so with 
ourselves. We do not preach our impressions, 
or even our experience. These make but the 
vehicle, as it were. What we preach is something 
much more solid, more objective, with 
more stay in it; something that can suffice when 
our experience has ebbed until it seems to be as 
low as Christ's was in the great desertion and 
victory on the Cross. We want something 
that will stand by us when we cannot feel any 
more; we want a Cross we can cling to, not 
simply a subjective Cross. That is, to put the 
thing in another way, what we want to-day is 
\marginpar{50}
an insight into the Cross. You see I am making 
a distinction between impression and insight. 
It is a useful part of the Church's work, for 
instance, that it should act by means of revival 
services, where perhaps the dominant element 
may be temporary impression. But unless that 
is taken up and turned to account by something 
more, we all know how evanescent a thing it is 
apt to be. We need, not simply to be impressed 
by Christ, but to see into Christ and into His 
Cross. We need to deepen the impression until 
it become new life by seeing into Christ. There 
are certain circumstances in which we may be 
entitled to declare that we do not want so many 
people who glibly say they love Jesus; we want 
more people who can really see into Christ. 
We do, of course, want more people who love 
Jesus; but we want a multitude of more people 
who are not satisfied with that, but whose love 
fills them with holy curiosity and compels them 
habitually to cultivate in the Spirit the power of 
seeing into Christ and into His Cross. More 
than impression, do we need a spirit of divination. 
Insight is what we want for power---less 
of mere interest and more of real insight. 
There are some people who talk as though, 
when we speak of the Cross and the meaning of 
the Cross, we were spinning something out 
\marginpar{51}
of the Cross. Paul was not spinning anything 
out of the Cross. He was gazing into the Cross, 
seeing what was really there with eyes that 
had been unsealed and purged by the Holy 
Ghost. 

\begin{center}
\S
\end{center}


The doctrine of Christ's reconciliation, or His 
Atonement, is not a piece of medieval dogma 
like transubstantiation, not a piece of ecclesiastical 
dogma or Aristotelian subtlety which 
it might be the Bible's business to destroy. If 
you look at the Gospels you will see that from 
the Transfiguration onward this matter of 
the Cross is the great centre of concern; it 
is where the centre of gravity lies. I met a 
man the other day who had come under some 
poor and mischievous pulpit influence, and he 
said, "It is time we got rid of hearing so much 
about the Cross of Christ; there should be 
preached to the world a humanitarian Christ, 
the kind of Christ that occupies the Gospels." 
There was nothing for it but to tell that man 
he was the victim of smatterers, and that he 
must go back to his Gospels and read and study 
for a year or two. It is the flimsiest religiosity, 
and the most superficial reading of the Gospel, 
that could talk like that. What does it mean 
that an enormous proportion of the Gospel 
\marginpar{52}
story is occupied with the passion of Christ? 
The centre of gravity, even in the Gospels, falls 
upon the Cross of Christ and what was done 
there, and not simply upon a humanitarian 
Christ. You cannot set the Gospels against 
Paul. Why, the first three Gospels were much 
later than Paul's Epistles. They were written 
for Churches that were made by the apostolic 
preaching. But how, then, do the first three 
Gospels \textit{seem} so different from the Epistles? Of 
course, there is a superficial difference. Christ 
was a very living and real character for the 
people of His own time, and His grand business 
was to rouse his audiences' faith in His Person 
and in His mission. But in His Person and in 
His mission the Cross lay latent all the time. 
It emerged only in the fullness of time---that 
valuable phrase---just when the historic crisis, 
the organic situation, produced it. Jesus was 
not a professor of theology. He did not lecture 
the people. He did not come with a theology 
of the Cross. He did not come to force events 
to comply with that theology. He did not 
force His own people to work out a theological 
scheme. He did force an issue, but it 
was not to illustrate a theology. It was to 
establish the Kingdom of God, which could 
be established in no other wise than as He 
\marginpar{53}
established it---upon the Cross. And He could 
only teach the Cross when it had happened---which 
He did through the Evangelists with the 
space they gave it, and through the Apostles 
and the exposition they gave it. 

To come back to this work of Christ described 
by Paul as reconciliation. On this 
interpretation of the work of Christ the whole 
Church rests. If you move faith from that 
centre you have driven \textit{the} nail into the Church's 
coffin. The Church is then doomed to death, 
and it is only a matter of time when she 
shall expire. The Apostle, I say, described the 
work of Christ as above all things reconciliation. 
And Paul was the founder of the Church, 
historically speaking. I do not like to speak 
of Christ as the Founder of the Church. It 
seems remote, detached, journalistic. It would 
be far more true to say that He is the foundation 
of the Church. "The Church's one foundation 
is Jesus Christ her Lord." The founder 
of the Church, historically speaking, was Paul. 
It was founded by and through him on this 
reconciling principle---nay, I go deeper than 
that, on this mighty \textit{act} of God's reconciliation. 
For this great act the interpretation was 
given to Paul by the Holy Spirit. In this connection 
read that great word in 1 Corinthians ii.; 
\marginpar{54}
that is the most valuable word in the New 
Testament about the nature of apostolic inspiration. 

\begin{center}
\S
\end{center}

What, then, did Paul mean by this reconciliation 
which is the backbone of the Church? 
He meant the total result of Christ's life-work 
in permanently changing the relation between 
collective man and God. By reconciliation Paul 
meant the total result of Christ's life-work in 
the fundamental, permanent, final changing of 
the relation between man and God, altering 
it from a relation of hostility to one of confidence 
and peace. Remember, I am speaking 
as Paul spoke, about man, and not about 
individual men or groups of men. 

There are two principal Greek words connected 
with the idea of reconciliation, one of 
them being always translated by it, the other 
sometimes. They are \textit{katallassein}, and \textit{hilaskesthai}---reconciliation 
and atonement. Atonement 
is an Old Testament phrase, where the 
idea is that of the covering of sin from God's 
sight. But by whom? Who was that great 
benefactor of the human race that succeeded in 
covering up our sin from God's sight? Who 
was skilful enough to hoodwink the Almighty? 
Who covered the sin? The all-seeing God 
\marginpar{55}
alone. There can therefore be no talk of hoodwinking. Atonement means the covering of 
sin by something which God Himself had 
provided, and therefore the covering of sin by
God Himself. It was of course not the blinding 
of Himself to it, but something very different. 
How could the Judge of all the earth make 
His judgment blind? It was the covering of 
sin by something which makes it lose the power 
of deranging the covenant relation between 
God and man and founds the new Humanity. 
That is the meaning of it. 

If you think I am talking theology, you must 
blame the New Testament. I am simply expounding 
to you the New Testament. Of course, 
you need not take it unless you please. It is 
quite open to you to throw the New Testament 
overboard (so long as you are frank 
about it), and start what you may loosely call 
Christianity on other floating lines. But if you 
take the New Testament you are bound to try to 
understand the New Testament. If you understand 
the New Testament you are bound to 
recognise that this is what the New Testament 
says. It is a subsequent question whether the 
New Testament is right in saying so. Let us 
first find out what the Bible really says, and then 
discuss whether the Bible is right or wrong. 

\marginpar{56}
The idea of atonement is the covering of sin 
by something which God provided, and by the 
use of which sin looses its accusing power, and 
its power to derange that grand covenant and 
relationship between man and God which founds 
the New Humanity. The word \textit{katallassein} (reconcile) 
is peculiar to Paul. He uses both words; 
but the other word, "atonement," you also find in 
other New Testament writings. Reconciliation 
is Paul's great characteristic word and thought. 
The great passages are those I have mentioned 
at the head of this lecture. I cannot take time 
to expound them here. That would mean a long 
course. Read those passages carefully and 
check me in anything I say---particularly, for 
instance, 2 Corinthians v. 14--vi. 2. Out of it we 
gather this whole result. First, Christ's work 
is something described as reconciliation. And 
second, reconciliation rests upon atonement as 
its ground. Do not stop at "God was in Christ 
reconciling the world." You can easily water 
that down. You may begin the process by 
saying that God was in Christ just in the same 
way in which He was in the old prophets. That 
is the first dilution. Then you go on with the 
hom{\oe}pathic treatment, and you say, "Oh yes, 
all He did by Christ was to affect the world, and 
impress it by showing it how much He loved it." 
\marginpar{57}
Now, would that reconcile anybody really in 
need of it? When your child has flown into a 
violent temper with you, and still worse, a sulky 
temper, and glooms for a whole day, is it any 
use your sending to that child and saying, 
"Really, this cannot go on. Come back. I love 
you very much. Say you are sorry." Not 
a bit of use. For God simply to have told 
or shown the evil world how much He loved 
it would have been a most ineffectual thing. 
Something had to be \textit{done}---judging or saving. 
Revelation alone is inadequate. Reconciliation 
must rest on atonement. For, as I say, 
you must not stop at "God was in Christ 
reconciling the world unto Himself," but go on 
"not reckoning unto them their trespasses." 
"He made Christ to be sin for us, who knew 
no sin." That involves atonement. You cannot 
blot out that phrase. And the third thing 
involved in the idea is that this reconciliation, 
this atonement, means change of relation between 
God and man---man, mind you, not two 
or three men, not several groups of men, 
but man, the human race as one whole. And it 
is a change of relation from alienation to communion---not 
simply to our peace and confidence, 
but to reciprocal communion. The grand end of 
reconciliation is communion. I am pressing 
\marginpar{58}
that hard. I am pressing it hard here by 
saying that it is not enough that we should 
worship God. It is not enough that we should 
worship a personal God. It is not enough that 
we should worship and pay our homage to a 
loving God. That does not satisfy the love of 
God. Nothing short of living, loving, holy, 
habitual communion between His holy soul and 
ours can realise at last the end which God 
achieved in Jesus Christ. 

\begin{center}
\S
\end{center}

In this connection let me offer you two 
cautions. First, take care that the direct fact 
of reconciliation is not hidden up by the indispensable 
means---namely, atonement. There 
have been ages in the Church when the 
attention has been so exclusively centred upon 
atonement that reconciliation was lost sight 
of. You found theologians flying at each 
other's throats in the interest of particular 
theories of atonement. That is to say, atonement 
had obscured reconciliation. In the same 
way, after the Reformation period, they dwelt 
upon justification until they lost sight of 
sanctification altogether. Then the great 
pietistic movement had to arise in order to 
redress the balance. Take care that the end, 
\marginpar{59}
reconciliation, is not hidden up by the means, 
atonement. Justification, sanctification, reconciliation 
and atonement are all equally inseparable 
from the one central and compendious 
work of Christ. Various ages need various 
aspects of it turned outward. Let us give 
them all their true value and perspective. If 
we do not we shall make that fatal severance 
which orthodoxy has so often made between 
doctrine and life. 

The second caution is this. Beware of reading 
atonement out of reconciliation altogether. 
Beware of cultivating a reconciliation which is 
not based upon justification. The apostle's 
phrases are often treated like that. They are 
emptied of the specific Christian meaning. 
There are a great many Christian people, 
spiritual people of a sort, to-day, who are 
perpetrating that injustice upon the New 
Testament. They are taking mighty old words 
and giving them only a subjective, arbitrary 
meaning, emptying out of them the essential, 
objective, positive content. They are preoccupied 
with what takes place within their 
own experience, or imagination, or thought; 
and they are oblivious of that which is 
declared to have taken place within the experience 
of God and of Christ. They are 
\marginpar{60}
oblivious and negligent of the essential things 
that Christ did, and God in Christ. That is 
not fair treatment of New Testament terms---to 
empty them of positive Christian meaning 
and water them down to make something 
that might suit a philosophic or mystic or 
subjective or individualist spirituality. There 
is a whole system of philosophy that has 
attempted this dilution at the present day. It 
is associated with a name that has now become 
very well known, the name of the greatest 
philosopher the world ever saw, Hegel. I am 
not now going to expound Hegelianism. But 
I have to allude to one aspect of it. If you 
are paying any attention to what is going on 
around you in the thinking world, you are 
bound to come face to face with some phase 
of it or other. But I see my time is at an end 
for to-day. 

\begin{center}
\S
\end{center}

To-morrow I begin where I now leave off 
and shall say something about this version of 
St. Paul's idea of reconciliation, which is so 
attractive philosophically. I remember the 
appeal it had for me when I came into contact 
with it first. I did feel that it seemed to give 
a largeness to certain New Testament terms, 
which I finally found was a largeness of latitude 
\marginpar{61}
only. If it did seem to give breadth it 
did not give depth. And I close here by reminding 
you of this---that while Christ and 
Christianity did come to make us broad men, 
it did not come to do that in the first instance. 
It came to make us deep men. The living 
interest of Christ and of the Holy Spirit is not 
breadth, but it is depth. Christ said little 
that was wide compared with what He said 
piercing and searching. I illustrate by referring 
you to an interest that is very prominent 
amongst---you the interest of missions. How 
did modern missions arise? I mean the last 
hundred years of them. Modern Protestant 
missions are only one hundred years old. 
Where did they begin? Who began them? 
They began at the close of the eighteenth 
century, the century whose close was dominated 
by philosophers, by scientists, by a 
reasonable, moderate interpretation of religion, 
by broad humanitarian religion. Of course, 
you might expect it was amongst those broad 
people that missions arose. We know better. 
We know that the Christian movement which 
has spread around the world did not arise out 
of the liberal thinkers, the humanitarian philosophers 
of the day, who were its worst enemies, 
but with a few men Carey, Marshman, Ward,
\marginpar{62} 
and the like---whose Calvinistic theology we 
should now consider very narrow. But they did 
have the root of the universal matter in them. 
A gospel deep enough has all the breadth of the 
world in its heart. If we are only deep enough 
the breadth will take care of itself. I would 
ten times rather have one man who was burning 
deep, even though he wanted to burn me 
for my modern theology, than I would have a 
broad, hospitable, and thin theologian who was 
willing to take me in and a nondescript crowd 
of others in a sheet let down from heaven, 
but who had no depth, no fire, no skill to 
search, and no power to break. For the deep 
Christianity is that which not only searches 
us, but breaks us. And a Christianity which 
would exclude none has no power to include 
the world. 


\chapter{RECONCILIATION: PHILOSOPHIC 
AND CHRISTIAN}
\chaptermark{Reconciliation}

%\marginpar{66}



%III 



%RECONCILIATION: PHILOSOPHIC AND 
%CHRISTIAN 



\textsc{I place} on the board before you five points 
as to Christ's reconciling work which I 
think vital:--- 

\begin{enumerate}
\item It is between person and person.
\item Therefore it affects both sides. 
\item It rests on atonement. 
\item It is a reconciliation of the world as 
one whole. 
\item It is final in its nature and effect.
\end{enumerate}

\begin{center}
\S
\end{center}

I was saying yesterday that two cautions 
ought to be observed in connection with this 
matter of reconciliation. First, we should not 
hide up the idea of reconciliation by the idea of 
atonement; we should not obscure the end, or 
the effect, by the great and indispensable means 
to it. Second, at the other extreme we are to 
\marginpar{66}
beware of emptying reconciliation of atonement 
altogether. Two very great thinkers arose last 
century in Germany---where most of the thinking 
on this subject has for the last hundred years 
been done. Much of our work has been to steal. 
That does not matter if it is done wisely and 
gratefully. When a man gives out a great 
thought, get it, work it; it is common property. 
It belongs to the whole world, to be claimed and 
assimilated by whoever shall find. Well, there 
were two very powerful men in Germany much 
opposed to each other, yet at a certain point at 
one---Hegel and Ritschl. While they preached 
the doctrine of reconciliation in different senses, 
they both united to obscure the idea of atonement 
or expiation. Now we are to beware of 
emptying the reconciliation idea of the idea 
of atonement, whether we do it philosophically 
with Hegel or theologically with Ritschl. I 
mention these men because their thought has 
very profoundly affected English thinking, 
whether philosophical or theological. I protested 
yesterday against the practice, so common, 
of taking New Testament words, and 
words consecrated to Christian experience, 
emptying them of their essential content, and 
keeping them in a vapid use. That is done for 
various reasons. It is sometimes done because 
\marginpar{67}
the words are too valuable to be parted with; 
sometimes because a philosophic interpretation 
seems to rescue them from the narrowness of 
an outworn theology; and it is sometimes done 
for lower motives in order to produce a fictitious 
impression upon people that they are still substantially 
hearing the substance of the old truths 
when really they are not. 

Especially I began yesterday to call attention 
to the view which is associated with the philosophical 
position of Hegel. Being a philosopher 
he was great upon the idea. The whole world, 
he said, was a movement or process of the grand, 
divine idea; but it was a \textit{process}. Now please 
to put down and make much use of this fundamental 
distinction between a process and an 
act. A process has nothing moral in it. We 
are simply carried along on the crest of a wave. 
An act, on the other hand, can only be done by 
a moral personality. The act involves the notion 
of will and responsibility, and, indeed, the whole 
existence of a moral world. The process destroys 
that notion. Now the general tendency of 
philosophy is to devote itself to the idea and 
to the process. Science, for example, which is 
the ground floor, not to say the basement, of 
philosophy---science knows nothing about acts, 
it only knows about processes. The chemist 
\marginpar{68}
knows only about processes. The biologist 
knows only about processes. The psychologist 
treats even acts as processes. But the theologian, 
and, indeed, religion altogether, stands 
or falls with the idea of an act. For him an 
infinite process is at bottom an eternal act. The 
philosophical thinker says the world is the process 
of an evolving idea, which may be treated as 
personal or may not. But for Christianity the 
world is the action of the eternal, divine act, 
a moral act, an act of will and of conscience. 
Let us see how this applies to our thoughts 
about reconciliation. I have already indicated 
to you that the grand goal of the divine reconciliation 
is communion with God, not simply 
that we should be in tune with the Infinite, 
as an attractive but thin book has it. The 
object of the divine atonement is something 
much more than bringing us into tune with 
God. It is more than raising our pitch and 
defining our note. It means that we are 
brought into actual, reciprocal communion with 
God out of guilt. We have personal intercourse 
with the Holy, we exchange thoughts and feelings. 
But this Christian idea of reconciliation, 
the idea of communion with the living and 
holy God, is replaced in philosophic theology by 
another idea, that, namely, of adjustment to 
\marginpar{69}
rational Godhead, our adjustment to that 
mighty idea, that mighty rational process, 
which is moving on throughout the world. 
Sometimes the Godhead is conceived as personal, 
sometimes as impersonal; but in any case 
reconciliation would be rather a resigned adjustment 
to this great and overwhelming idea, 
which, having issued everything, is perpetually 
recalling, or exalting, everything into fusion 
with itself. But fusion, however organic and 
concrete, is one thing, communion is another 
thing. An individual might be lost in the great 
sum of being as a drop of water is lost in the 
ocean. That is fusion. Or it might be taken 
up as a cell in the body's organic process. 
That is a certain kind of reconciliation or 
absorption. But moral, spiritual reconciliation  
where we have personal beings to deal with, 
is much more than fusion; more than absorption; 
it is communion. It is more than placing 
us in our niche. When we think in the philosophic 
way it practically means that reconciliation 
is understood almost entirely from man's 
side, without realising the divine initiative as an 
act. But such divine initiative is everything. 
It is in the mercy of our God that all our hopes 
begin. Nothing that confuses that gets at the 
root of our Christian reconciliation. Or, sometimes,
\marginpar{70}
those philosophic ideas are carried so far 
that God's concern for the individual is ignored. 
These great processes work according to general 
laws; and general laws, like Acts of Parliament, 
are bound to do some injustice to individuals. 
You cannot possibly get complete justice by Act 
of Parliament. It is bound to hit somebody 
very hard. And it has often been doubted by 
exponents of philosophical theology such as I 
describe whether the individual as an individual 
was really present to God's mind and affection 
at all. And they think prayer is unreasonable 
except for its reflex effect on us. Thus the 
whole stress comes to be put upon our attitude 
to God, and not upon a reciprocal relationship. 
That is to say, religion becomes, as I described 
yesterday, a subjectivity, a resignation. In 
others it becomes a sense of dependence. People 
are invited to become preoccupied with their 
own attitude, their own relation, their own 
feelings toward the unchangeable, but absorbing, 
and even unfeeling God. Attention is 
directed upon the human side instead of insight 
cultivated into the divine side. The result of 
that practically is that religion comes to consist 
far too much in working up a certain frame of 
feeling instead of dwelling upon the objective 
reality of the act of God. Resignation is, 
\marginpar{71}
then, my act; but it is not resignation to 
a sympathetic act of approach in God, but 
only to His onward movement. But, as I 
have said before, if we are to produce the real 
Christian faith we must dwell upon, we must 
preach and press, that objective act and gift of 
God which in itself produces that faith. We 
cannot produce it. Many try. There are some 
people who actually work at holiness. It is a 
dangerous thing to do, to work at your own 
holiness. The way to cultivate the holiness of 
the New Testament is to cultivate the New Testament 
Christ, the interpretation of Christ in 
His Cross, by His Spirit, which cannot but 
produce holiness, and holiness of a far profounder 
order than anything we may make 
by taking ourselves to pieces and putting 
ourselves together in the best way we can, 
or by adjusting ourselves with huge effort 
to a universal process. Religious subjectivity 
is truly a most valuable phase; and at some 
periods in the Church's history it is urgently 
called for. In the seventeenth century it was 
so called for because Protestantism had degenerated 
into a mere theological orthodoxy, 
a very hard-shell kind of Christianity. It was 
necessary that the great Pietistic movement 
should arise and correct it. But this is itself a 
\marginpar{72}
danger in turn; and we have to rise up in the 
name of the gospel, of the New Testament, and 
demand a more objective religion; and we have 
to declare that if ever divine holiness is to be 
produced in man it can only be produced by 
God's act through Christ in the Holy Spirit.   

\begin{center}
\S
\end{center}

The philosophic kind of theology (which is 
rather theosophy) often ends, you perceive, 
in turning real reconciliation into something 
quite different. It becomes turned into the 
mere forced adjustment of man to his fate; 
and naturally this often ends in a resentful 
pessimism. Supposing the whole universe to 
be a vast rational process unfolding itself like 
an infinite cosmic flower, you cannot have communion 
or any hearty understanding between a 
living, loving soul and that evolutionary process. 
All you can do is to adjust yourself to 
that process, settle down to it and make the 
best of it, square yourself to it in the way that 
seems best for you, and that will cause you and 
others least discomfort. But reconciliation becomes 
debased indeed when it turns to mere 
resignation. Of course, we have to practise 
resignation. But Christianity is not the practice 
of resignation. At least, that is not the
\marginpar{73}
meaning of reconciliation. When two friends 
fall out and are reconciled, it does not simply 
mean that one adjusts himself to the other. 
That is a very one-sided arrangement. There 
must be a mutuality. Theology of the kind I 
have been describing has a great deal to say 
about men changing their way of looking at 
things or feeling about them. If I were 
preaching a theology like that I should say: 
"This mighty process, of which you are all parts, 
is unfolding itself to a grand closing result. It 
is going to be a grand thing for everybody in 
the long run (provided, that is, that they continue 
to exist as individuals and are capable of 
feeling anything, whether grand or mean). It 
is all going to work out to a grand consummation. 
You do not see that, but you must make 
an effort and accept it as the genius and drift of 
things; and that is faith. You must accept the 
idea that the whole world is working out, 
through much suffering and by many roundabout 
ways, to a grand final consummation 
which will be a blessing for everybody, even 
though it might mean their individual extinction. 
What you have to do in these circumstances 
is, by a great act of faith, to believe 
that this is so and to immolate yourself, if 
need be, for the benefit of this grand whole; 
\marginpar{74}
at any rate, accommodate yourself to its evolving 
movement." 

The gospel of Christ speaks otherwise. It 
speaks of a God to whom we are to be reconciled 
in a mutual act which He begins; and not of 
an order or process with which we are to be 
adjusted by our lonely act, or to which we are 
to be resigned. If we have an idea of such a 
Godhead as I have been describing, how does 
it affect our thought of Christ? Christ then 
becomes but one of its grandest prophets, or 
one of the greatest instances and illustrations of 
that adjustment to the mighty order. He first 
realised, and He first declared, this great change 
in the way of reading the situation. What you 
have to do if you accept Him is to change your 
way of reading the situation, to accept His 
interpretation of life, and accept it as rationally, 
spiritually, and resignedly as you best can. 
Accept His principle. Die to live. But what a 
poor use of Christ---to accept His interpretation 
of life, as if He were a mere spiritual Goethe! 
That is a very attenuated Christ compared with 
the Christ that is offered to us in the New 
Testament. That is not the eternal Son of God 
in whom God was reconciling the world unto 
Himself. That is another Christ---from some 
hasty points of view indeed a larger Christ; 
\marginpar{75}
for the philosophers have a larger Christ, apparently,
one more cosmic. But it is a diluted 
Christ, and one that cannot penetrate to the 
centre and depth of our human need or our 
human personality, cannot reach our guilt and 
hell, and therefore cannot be the final Christ 
of God. 

\begin{center}
\S
\end{center}

Whether from the side of the philosophers, as 
I have been showing, or from the side of certain 
theologians like Ritschl, who was so much 
opposed to Hegel, you will often hear this 
said: that only man needed to be reconciled, 
that God did not need any reconciliation. 
Now, I have been asking you to observe that 
we are dealing with persons. That is the first 
point I put upon the board. Our reconciliation 
is between person and person. It is not 
between an order or a process on the one hand 
and a person on the other. Therefore a real 
and deep change of the relation between the 
two means a change on both sides. That is 
surely clear if we are dealing with living persons. 
God is an eternal person; I am a finite 
person; yet we are persons both. There is that 
parity. Any reconciliation which only means 
change on one side is not a real reconciliation 
at all. A real, deep change of relation affects 
\marginpar{76}
both sides when we are dealing with persons. 
That is not the case when we are dealing on 
the one side with ideas, or one vast idea 
or process, and on the other side a person 
only. 

When Christianity is being watered down in 
the way I have described, we have to concentrate 
our attention upon the core of it. All 
round us Christianity is being diluted either 
by thought or by \textit{blague}; we must press to the 
core of the matter. It is true the theology of 
the Christian Church on this head needs a 
certain amount of modification and correction 
at the present day. That will appear presently. 
But I want to make it clear that the view of 
the Church upon the whole, especially the 
great view associated with the Reformation, 
preserves the core of the matter, which we 
are in danger of losing either on one side or 
the other. 

Let me call your attention, then, to these 
five points, which you will find immanent in 
what I have subsequently to say. 

First, you will note that the reconciliation is 
between \textit{two persons} who have fallen out, and 
not between a failing person on the one hand 
and a perfect, imperturbable process on the 
other. 

\marginpar{77}
The second thing is a corollary from the first, 
and is that the reconciliation \textit{affects and alters} 
\textit{both parties} and not only one party. There 
was reconciliation on both sides. 

Thirdly, it is a reconciliation which \textit{rests upon} 
\textit{atonement and redemption}. 

Fourthly, it is a reconciliation of \textit{the world} 
\textit{as a cosmic whole}. The world as one whole; 
not a person here and another there, snatched 
as brands from the burning; not a group here 
and a group there; but the reconciliation of 
the whole world. 

Fifthly, it is a reconciliation \textit{final in Jesus} 
\textit{Christ and His Cross}, done once for all; really 
effected in the spiritual world in such a way 
that in history the great victory is not still to 
be won; it has been won in reality, and has 
only to be followed up and secured in actuality. 
In the spiritual place, in Christ Jesus, 
in the divine nature, the victory has been 
won. That is what I mean by using the 
word "Final" at the close of the list. 

\begin{center}
\S
\end{center}

I will expound these heads as I go along. Let 
me begin almost at the foundation and say this. 
Reconciliation has no moral meaning as between 
finite and infinite---none apart from the 
\marginpar{78}
sense of guilt. The finished reconciliation, the 
setting up of the New Covenant by Christ, 
meant that human guilt was once for all robbed 
of its power to prevent the consummation of the 
Kingdom of God. It is the sense of guilt that 
we have to get back to-day for the soul's sake 
and the kingdom's; not simply the sense of sin. 
There are many who recognise the power of sin, 
the misfortune of it; what they do not recognise 
is the thing that makes it most sinful, which 
makes it what it is before God, namely, guilt; 
which introduces something noxious and not 
merely deranged, malignant and not merely 
hostile; the fact that it is transgression against 
not simply God, not simply against a loving 
God, but against a holy God. Everything 
begins and ends in our Christian theology 
with the holiness of God. That is the idea we 
have to get back into our current religious 
thinking. We have been living for the last two 
or three generations, our most progressive side 
has been living, upon the love of God, God's love 
to us. And it was very necessary that it should 
be appreciated. Justice had not been done to it. 
But we have now to take a step further, and we 
have to saturate our people in the years that are 
to come as thoroughly with the idea of God's 
\textit{holiness} as they have been saturated with the 
\marginpar{79}
idea of God's love. I have sometimes thought 
when preaching that I saw a perceptible change 
come over my audience when I turned from 
speaking about the love of God to speak about 
the holiness of God. There was a certain indescribable 
relaxing of interest, as though their 
faces should say, "What, have we not had 
enough of these incorrigible and obtrusive 
theologians who will not let us rest with the 
love of God but must go on talking about things 
that are so remote and professional as His 
holiness!" All that has to be changed. We 
have to stir the interest of our congregations 
as much with the holiness of God as 
the Church was stirred---first with the justice 
and then latterly with the love of God. It is 
the holiness of God which makes sin guilt. It 
is the holiness of God that necessitates the 
work of Christ, that calls for it, and that provides 
it. What is the great problem? The great 
problem in connection with atonement is not 
simply to show how it was necessary to the 
fatherly love, but how it was necessary to a holy 
love, how a holy love not only must have it but 
must make it. The problem is how Christ can 
be a revelation not of God's love simply, but 
of God's holy love. Without a holy God there 
would be no problem of atonement. It is the 
\marginpar{80}
holiness of God's love that necessitates the 
atoning Cross. 

I say, then, that the reconciliation has no 
meaning apart from guilt which must stir the 
anger of a holy God and produce separation 
from Him. That is, the reconciliation rests 
upon a justification, upon an atonement. Those 
were the great Pauline ideas which were 
rediscovered in the fifteenth and sixteenth 
centuries and became the backbone of the Reformation.
They were practically rediscovered. 
Look at the movement in the history of the 
Church's thought in this respect. You have 
three great points: you might name them---the 
first from Augustine, the second from Luther; 
for the third, our modern time, we have as 
yet no such outstanding name. The first great 
movement towards the rediscovery of Paul 
was by Augustine. Do you know that Paul 
went under after the first century? He went 
under for historic reasons I cannot stay to 
explain. It is a remarkable thing how he was 
kept in the canon of Scripture. Paul went 
under, and for centuries remained under, and 
he had to be rediscovered. That was done by 
Augustine. Again he went under, and Luther 
rediscovered him. And he is being rediscovered 
again to-day. Augustine's rediscovery was this, 
\marginpar{81}
justification by grace alone; Luther's side of the 
rediscovery was justification by faith alone---faith 
in the Cross, that is to say, faith in grace. 
What is our modern point of emphasis? Justification 
by holiness and for it alone. That is to 
say, as I have already pointed out, reconciliation 
is something that comes from the whole holy 
God, and it covers the whole of life, and it is not 
exhausted by the idea of atonement only or 
redemption only. It is the new-created race 
being brought to permanent, vital, life-deep 
communion with the holy God. Only holiness 
can be in communion with the holy God. We 
have to be saved---not indeed from morality, 
because we can only be saved by the moral; that 
is the grand sheet-anchor of our modern theories. 
However we be saved, we can only be saved 
in a way consistent with God's morality---that 
is to say, with holiness. The rescue is not from 
morality; but it is from mere moralism, from 
a religion three parts conduct. We are saved 
through the Spirit of a new life, an indiscerptible 
life in Jesus Christ. That is the grand 
new thing in Christianity (2 Corinthians iii. 6). 

\begin{center}
\S
\end{center}

Reconciliation, then, has no meaning apart 
from a sense of guilt, that guilt which is involved 
\marginpar{82} 
in our justification. I am going to try to 
expound that before I am done. I want to note 
here that it means not so much that God is reconciled,
but that God is the Reconciler. It is the 
neglect of that truth which has produced so much 
scepticism in the matter of the atonement. So 
much of our orthodox religion has come to talk 
as though God were reconciled by a third party. 
We lose sight of this great central verse, 
"God was in Christ reconciling the world unto 
Himself." As we are both living persons, that 
means that there was reconciliation on God's 
side as well as ours; but wherever it was, it was 
effected by God Himself in Himself. In what 
sense was God reconciled within Himself? We 
come to that surely as we see that the first 
charge upon reconciling grace is to put away 
guilt, reconciling by not imputing trespasses. 
Return to our cardinal verse, 2 Corinthians 
v. 19. In reconciliation the ground for 
God's wrath or God's judgment was put away. 
Guilt rests on God's charging up sin; reconciliation 
rests upon God's non-imputation 
of sin; God's non-imputation of sin rests upon 
Christ being made sin for us. You have thus 
three stages in this magnificent verse. God's 
reconciliation rested upon this, that on His 
Eternal Son, who knew no sin in His experience, 
\marginpar{83}
(although He knew more about sin than any 
man who has ever lived), sin's judgment fell. 
Him who knew no sin by experience, God made 
sin. That is to say, God by Christ's own consent 
identified Him with sin in treatment though 
not in feeling. God did not judge Him, but 
judged sin upon His head. He never once 
counted Him sinful; He was always well 
pleased with Him; it was part, indeed, of His 
own holy self-complacency. Christ was made sin 
for us, as He could never have been if He had 
been made a sinner. It was sin that had to be 
judged, more even than the sinner, in a world-salvation; 
and God made Christ sin in this sense, 
that God as it were took Him in the place of sin, 
rather than of the sinner, and judged the sin 
upon Him; and in putting Him there He really 
put Himself there in our place (Christ being 
what He was); so that the divine judgment of 
sin was real and effectual. That is, it fell where 
it was perfectly understood, owned, and praised, 
and had the sanctifying effect of judgment, the 
effect of giving holiness at last its own. God 
made Him to be sin in treatment though not 
in feeling, so that holiness might be perfected in 
judgment, and we might become the righteousness 
of God in Him; so that we might have 
in God's sight righteousness by our living 
\marginpar{84}
union with Christ, righteousness which did 
not belong to us actually, naturally, and 
finally. Our righteousness is as little ours individually 
as the sin on Christ was His. The 
thief on the cross, for instance---I do not suppose 
he would have turned what we call a saint 
if he had survived; though saved, he would 
not have become sinless all at once. And the 
great saint, Paul, had sin working in him long 
after his conversion. Yet by union with Christ 
they were made God's righteousness, they were 
integrated into the New Goodness; God made 
them partakers of His eternal love to the ever-holy 
Christ. That is a most wonderful thing. 
Men like Paul, and far worse men than Paul, 
by the grace of God, and by a living faith, 
become partakers of that same eternal love 
which God from everlasting and to everlasting 
bestowed upon His only-begotten Son. It is 
beyond words. 

It was not a case of wiping a slate. Sin 
is graven in. You cannot wipe off sin. It 
goes into the tissue of the spiritual being. And 
it alters things for both parties. Guilt affected 
both God and man. It was not a case of destroying 
an unfortunate prejudice we had 
against God. It was not a case of putting 
right a misunderstanding we had of God. 
\marginpar{85}
"You are afraid of God," you hear easy people 
say; "it is a great mistake to be afraid of 
God. There is nothing to be afraid of. God is 
love." But there is everything in the love of 
God to be afraid of. Love is not holy without 
judgment. It is the love of holy God that 
is the consuming fire. It was not simply a 
case of changing our method, or thought, our 
prejudices, or the moral direction of our soul. 
It was not a case of giving us courage when we 
were cast down, showing us how groundless 
our depression was. It was not that. If that 
were all it would be a comparatively light 
matter. 

If that were all, Paul could only have spoken 
about the reconciliation of single souls, not 
about reconciliation of the whole world as a 
unity. He could not have spoken about a 
finished reconciliation to which every age of 
the future was to look back as its glorious and 
fontal past. In the words of that verse which 
I am constantly pressing, "God was in Christ 
reconciling the world unto Himself." Observe, 
first, "the world" is the unity which corresponds 
to the reconciled unity of "Himself "; and 
second, that He was not trying, not taking steps 
to provide means of reconciliation, not opening 
doors of reconciliation if we would only walk in 
\marginpar{86}
at them, not labouring toward reconciliation, 
not (according to the unhappy phrase) waiting 
to be gracious, but "God was in Christ reconciling," actually reconciling, finishing the work. 
It was not a tentative, preliminary affair 
(Romans xi. 15). Reconciliation was finished in 
Christ's death. Paul did not preach a gradual 
reconciliation. He preached what the old 
divines used to call the finished work. He did 
not preach a gradual reconciliation which was 
to become the reconciliation of the world only 
piecemeal, as men were induced to accept it, or 
were affected by the gospel. He preached something 
done once for all---a reconciliation which is 
the base of every soul's reconcilement, not an 
invitation only. What the Church has to do is 
to appropriate the thing that has been finally 
and universally done. We have to enter upon 
the reconciled position, on the new creation. 
Individual men have to enter upon that reconciled 
position, that new covenant, that new relation, 
which already, in virtue of Christ's Cross, 
belonged to the race as a whole. I will even 
use for convenience' sake the word totality. 
(People turn up their noses at a word like that, 
and they say it smells of philosophy. Well, 
philosophy has not a bad smell! You cannot 
have a proper theology unless you have a 
\marginpar{87}
philosophy. You cannot accurately express 
the things that theology handles most deeply. 
The misfortune of our ministry is that it comes 
to theology without the proper preliminary 
culture---with a pious or literary culture only.) 
I am going to use this word totality, and say 
that the first bearing of Christ's work was upon 
the race as a totality. The first thing reconciliation 
does is to change man's corporate 
relation to God. Then when it is taken home 
individually it changes our present attitude. 
Christ, as it were, put us into the eternal 
Church; the Holy Spirit teaches us how to 
behave properly in the Church. 

\begin{center}
\S
\end{center}

I go on to show that reconciliation has its 
effect not upon man only, but upon God also. 
That is a difficulty to many people. And, indeed, 
we require to be somewhat discriminating here. 
If you say bluntly that Christ reconciled God, 
it is more false than true. I do not say it 
is untrue. It is the people who want plain 
black and white, false or true, that do so much 
mischief in these matters. It is the thin, 
commonsense rationalists, orthodox or heterodox. 
It is the people who put a pistol to your 
head and say, "I am a plain man and I 
\marginpar{88}
want a plain yes or no," that cause so 
much difficulty. Christ always refused to 
answer with a pistol to His head. It was the 
whole manner of His ministry to refuse to give 
a plain answer when asked a blunt question. 
We see that in Peter's discovery and confession, 
"Thou art the Christ," and in Christ's joyful 
answer, "Blessed Simon." Peter in his confession 
had crowned what Christ had laboured 
to live in upon them, but what He had never 
said plainly in so many words---"I am the 
Christ." He lived it into them and made them 
discover it. Repeatedly He was asked, "Give us 
signs," "Give us yes or no," and He always 
refused. That would be sight, not faith. A 
plain yes or no is sight. But faith is insight 
into Christ. In this region a plain yes or no 
is somewhat out of place. So, therefore, while 
it is not false to say that Christ reconciled 
God, it is more false than true as it is mostly 
put. You do not get it in the Bible. It would 
be a useful exercise to go through the Bible 
and see what proofs you can get of Christ 
reconciling God. If we talk about Christ reconciling 
God in the way some do, we suggest that 
there was some third party coming between us 
and God, reconciling God on the one hand and 
us on the other, like a daysman. That is one 
\marginpar{89}
great mischief that is done by the popular 
theories of atonement. God can never be 
regarded as the object of some third party's 
intervention in reconciling. If it were so, what 
would happen? There would be no grace. It 
would be a bought thing, a procured thing, 
the work of a pardon-broker; and the one 
essential thing about grace is that it is unbought 
and unpurchasable. It is the freest 
thing in heaven or earth. It would not be 
free if procured by some third party. The 
"daysman" metaphor has been much abused. 
It is a Scriptural figure, but we get it in the 
Old Testament, in Job, the idea being that of 
one who, in the case of a dispute, puts one hand 
on one head and the other on another and 
brings two persons together. That is a crude 
version of the Christian idea of reconciliation. 
The grace of God would not then be the prime 
and moving cause. It would not be spontaneous 
and creative, it would be negotiated grace; and 
that is a contradiction in terms. Mediation can 
never mean that. In paganism the gods were 
mollified. God, our God, could never be mollified. 
There is no mollification of God, no placation of 
God. Atonement was not the placating of God's 
anger. Even in the old economy we are told, "I 
have \textit{given} you the blood to make atonement." 
\marginpar{90}
Given! Did you ever see the force of it? "I 
have given you the blood to make atonement. 
This is an institution which I set up for you to 
comply with, set it up for purposes of My own, 
on principles of My own, but it is My gift." The 
Lord Himself provided the lamb for the burnt 
offering. Atonement in the Old Testament was 
not the placating of God's anger, but the sacrament 
of God's grace. It was the expression 
of God's anger on the one hand and the expressing 
and putting in action of God's grace on the 
other hand. The effect of atonement was to 
cover sin from God's eyes, so that it should no 
longer make a visible breach between God and 
His people. The actual ordinance was established, 
they held, by God Himself. He covered 
the sin. Sacrifices were not desperate efforts and 
surrenders made by terrified people in the hope 
of propitiating an angry deity. The sacrifices 
were in themselves prime acts of obedience 
to God's means of grace and His expressed will. 
If you want to follow that out further, perhaps 
I may be forgiven if I were to allude to the 
last chapter in my book, "The Cruciality of 
the Cross" (1909), in which there is a fuller 
discussion of the particular point, and especially 
of what is morally meant by the blood of 
Christ. 

\marginpar{91}
\begin{center}
\S
\end{center}

But some one immediately asks, Is there then 
no objective atonement? It is a question worth 
deep attention. A great many people say 
Christianity wrecks chiefly on the idea of objective 
atonement. How cheap the objection is 
in many cases, how easy and common it is! If 
you find somebody who is making it his mission 
in life to pull to pieces the venerable theology of 
the Catholic Church, and show how poor a thing 
it is in the light of the thirty years in which he 
has lived, you will hear it put likely enough in 
such terms as these: that objective atonement is 
sheer paganism. The Christian idea of atonement 
is identified offhand with the pagan idea 
of atonement, as a Hyde Park lecturer might. 
And when you have done that at the outset, it 
is the simplest thing to show how false and 
absurd and pagan such theology is. It is said 
further, that the whole Church has become 
paganised in this way, and has spoken as though 
God could be mollified by something offered to 
Him. The criticism is sometimes ignorant, 
sometimes ungenerous, sometimes culpable. If 
such language has ever been held, it has only 
been by sections of the Church, sections that 
have gone wrong in the direction of unqualified 
extremes. You have extravagancies, remember, 
\marginpar{92}
even in rational heresy. Has the Church on 
the whole ever really forgotten that it is in 
the mercy of God that all our hopes begin and 
end? And even if the Church had gone further 
wrong than it has done about this, we do not 
live upon the Church, but upon the gospel and 
upon the Bible. We live in and through the 
Church. We cannot do without it. We must 
get back a great deal more respect for it. But 
we do not live on the Church; we live on the 
word of the gospel which is in the Bible. 

\begin{center}
\S
\end{center}

What is the real objective element in the 
Bible's gospel? What is the real objective 
element in atonement? We are tempted, I say, 
to declare that it was the offering of a sacrifice 
to God outside of Him and us, the offering of a 
sacrifice to God by somebody not God yet more 
than a single man. That is the natural, the 
pagan notion of objective atonement. But the 
real meaning of an objective atonement is that 
God Himself made the complete sacrifice. The 
real objectivity of the atonement is not that it 
was made to God, but by God. It was atonement 
made by God, not by man. When I use the 
word objective, I do not mean objective to you or 
to me. You are objective to me, and I to you. 
\marginpar{93}
That is not the idea. Let us learn to think on 
the scale of the whole race. What is objective 
to that? The deadly kind of subjectivity is 
the kind that is engrossed with individuals, 
or with humanity, and does not allow for God. 
It is the egotism of the race. And the real objectivity 
is that which is objective to the whole 
human race, over against it, and not merely 
facing you or me within it. The real objective 
element in the atonement, therefore, is that God 
made it and gave it finished to man, not that 
it was made to God by man. Any atonement 
made by man would be subjective, however 
much it might be made for man by his brother, 
or by a representative of entire Humanity. 

\begin{center}
\S
\end{center}
  
But we have a certain farther difficulty to 
face here. If it was God that made the atonement---which 
it certainly was in Christianity---then 
was it not made to man? Can God reconcile 
Himself? And can the atonement mean 
anything more than the attuning of man to 
God---that is to say, of individual men in their 
subjective experience? God then says to each 
soul, "Be reconciled. See, I have put My anger 
away." Can such attuning of Himself by God 
have for its results anything more than individual 
\marginpar{94}
conversion? Now, conversion means 
much, but it does not mean the whole of 
Christianity. Reconciliation means the life-communion 
of the race. But, if God made the 
atonement, it might seem that the result and 
effect of this atonement could only be reached 
gradually by the attuning of individual men to 
God. It would seem to destroy the totality of 
the race, or (to employ another word even 
more useful) the solidarity of the race. That 
would seem to be the effect; and it is such a 
serious effect, for this reason: that it affects 
the universality of Christ's work. Whatever 
affects the universality of Christ's work cuts 
the ground from under aggressive Christianity, 
from under missions, whether at home or 
abroad. They cannot thrive except upon a 
faith which means the universality of Christ's 
work, which means again the solidarity, the 
organic unity, of the whole human race. And 
the conversion of a race is a work that exceeds 
conversion and is redemption. About that the 
Old Testament and the New Testament are 
at one. 

But, you say, you do not have the solidarity 
of the human race in the Old Testament. Well, 
you do, and you do not. What you have face to 
face with God in the Old Testament is a collective 
\marginpar{95}
nation, Israel. We shall never read the 
Old Testament with true understanding until 
we realise that. That is one of the great things 
modern scholarship has brought home to us---that 
the \textit{vis-\`{a}-vis} of God in the Old Testament is 
Israel and not the individual Jew. Gradually, 
as the Old Testament develops in spiritual intimacy, 
you have this changing and becoming 
intensely individual, as in the later Psalms. In 
Jeremiah it became so especially. The greatest 
prefiguration of Christ's individual solitude in 
the Old Testament is Jeremiah. But both of 
them were representative or collective individuals. 
They condensed the people. The object 
that faced God in the Old Testament in the 
main was not primarily the individual soul, it 
was the soul of the nation of Israel, even 
though it was sometimes reduced to a remnant. 
What took place when Israel made the great 
refusal of Christ? There was set up another 
collective unity, the Church, the new Israel, the 
spiritual Israel, the landless, homeless Israel, 
whose home was in Him, the universal Israel, 
the new Humanity of the new covenant. The 
Church became the prophecy and prefiguration 
of the unity of Humanity. It is through the 
Church alone that the unity of Humanity can 
be consummated, because it is possible only 
\marginpar{96}
through the gospel. And the preacher of this 
gospel in the world is the collective Church. 

We must, therefore, avoid every idea of atonement which seems to reduce it to God's dealing 
with a mass of individuals instead of with the 
race as a whole---instead of a racial, a social, a 
collective salvation, in which alone each individual has his place and part. Our Protestant 
theology has been too individualist, too little 
collectivism And that has had serious social 
consequences as well as theological. The basis 
of a social salvation is the final redemption in 
one act of the total race. And that act was the 
Cross of Christ. 



\chapter{RECONCILIATION, ATONEMENT, 
AND JUDGMENT} 
\chaptermark{Reconciliation, Atonement, \& Judgment}




%IV 



%RECONCILIATION, ATONEMENT, AND 
%JUDGMENT 


%\marginpar{99}
\textsc{The} point at which I broke off yesterday
was this. I was pointing out that 
objective atonement is absolutely necessary. 
Of course, it is quite necessary also that we 
should know what is meant by an objective 
atonement. The real objective element in 
atonement is not that something was offered 
to God, but that God made the offering. 
And in this connection I hinted that my 
remarks to-day and to-morrow would have 
to follow the idea also, that God's atonement 
initially was made on behalf of the race, and 
on behalf of individuals in so far as they were 
members of the race. The first charge upon 
Christ and His Cross was the reconciliation of 
the race, and of its individuals by implication. 

We start to-day, then, from the position that 
God made the atonement. This (we saw) suggests 
\marginpar{100}
a number of questions, not to say difficulties. 
If God made the atonement, but reconciliation 
meant no more than simply the moving and 
attuning of individual men in their subjective 
experience, it might seem as though it destroyed 
the solidarity of mankind and made it 
granular. And the peril there is that whatever 
destroys that, destroys the universality of 
Christ's work. But that atomism is not the 
Gospel. To reduce the reconciliation merely 
to the aggregate of individual conversions 
would be a total misrepresentation of New 
Testament reconciliation, which is both solidary 
and final. 

Then there is another difficulty. If we say 
that the one object of the atonement was not 
the reconciliation of God, but the reconciliation 
of man to God, then it looks as though the 
work of Christ became only the grand heliograph 
from divine heights, the chief word in 
what I might call a language of signs; as though 
it were only the leading expression of God's 
will towards men, instead of something actually 
done, and not merely said or shown, by God, 
something really done from the depth of God 
Who is the action of the world, something eternally 
changing the whole situation, and destiny, 
and responsibility of our race. If God in Christ 
\marginpar{101}
simply said the most powerful word about His 
goodwill, His placability, and His readiness to 
forgive, that would destroy the permanence of 
Christ---the depth of His work, and the height 
of His place. Thus God would be \textit{saying} more 
than He \textit{did}; and we have a natural and proper 
difficulty in thoroughly trusting people who say 
more than they do. If Christ were simply an 
expression of God's love, then His Cross would 
simply be what is called an object-lesson of 
God's love; or it would simply be a witness 
to the serious way in which God takes man's 
sin; or it might even be no more than the expression 
of the strong conviction of Jesus 
about it. We are exposed to the danger there 
always is when we make revelation a word 
rather than a deed, something said instead of 
something done, when we make it manifestation 
only and not redemption. The work of 
Christ would be only something educational, 
or at most impressive. And what happens 
then? If the work of Christ is only impressively 
educational, if the need and value of it 
ceases when we have recognised its meaning, 
when we have taken God's word for it in 
Christ that He does really love us, what 
happens then? Why, as soon as the lesson 
had been learnt, the work of Christ might be 
\marginpar{102}
left behind. There are a great many people 
to-day who are Christian in a way, but have 
very loose ideas as to what is involved centrally 
in their Christianity. Many of them are in this 
position I describe---they think they can ignore 
Christ and the work of Christ since they have 
assimilated the lesson these taught. If the 
Cross is a kind of practical parable which God 
set forth of His love and His willingness to 
save, then when the parable has done its work 
it can be forgotten. When the lesson has been 
taught, the example can be put away into the 
school store-room until we want it again. It 
is exhausted for the time being, until somebody 
else comes who needs the same lesson. In that 
case the work of Christ simply sinks to the 
level of other valuable events in the history 
of religion. It is not fontal but episodic. It 
represents the transition from Judaism to a religion 
of Humanity. It represents a great movement 
in the history of religion, when religion 
ceased to be national and particularist, and 
became universal, when it ceased to be ritual and 
became spiritual. The death of Christ would 
thus be a great monument in the past, which 
fades out of sight as we surmount it and leave 
it behind; and it does not retain a permanent 
meaning and function at the centre of our faith. 

\marginpar{103}
\begin{center}
\S
\end{center}

I said that the work of Christ meant not only 
an action on man, it meant an action on God. 
Yet I pointed out that it was more false than 
true to say that Christ and His death reconciled 
God to man. I said that we must in some way 
construe the matter as God reconciling Himself. 
It was out of the question to think of any 
reconciliation effected upon God by a third 
party standing between God and man. God 
could not be reconciled by man nor by one 
neither God nor man. The only alternative, 
therefore, is that God should reconcile Himself. 
But then is there not something in that which 
seems a little forced and unnatural? Did God 
have to compel Himself to change His feeling 
about us? Did He force Himself to be gracious? 
There is something wrong here surely, something 
that needs adjustment, explanation, restatement 
in some way. 

Are we obliged to suppose that if God did 
reconcile Himself it was in the sense of changing 
His own heart and affection towards us? I 
have pointed out that the heart of God towards 
us, His gracious disposition towards us, was 
from His own holy eternity; that grace is of the 
unchangeable. God in that respect had not to 
be changed. Was He changed at all then? If 
\marginpar{104}
His heart was not changed, what remained in 
Him to be changed, what was changed in 
connection with the work of Christ? 

There was a change. And I am going to ask 
you to recognise here another of those valuable 
distinctions of which the man without the evangelical 
experience and its theological discipline 
is so impatient. As I work my way through 
the difficulties and questions that present themselves, 
over and over again I perceive that many 
of the difficulties that seem so serious to some 
turn entirely upon some valuable distinction 
that has been ignored, often for lack of deep religion 
or due professional education. Of course 
the man in the street says, as soon as he is 
asked to distinguish, that that is getting into 
the region of subtleties. Never mind the man 
in the street. The distinguished person for him 
is the person with the least distinction from 
himself, the person who gives him most satisfaction 
with least trouble, the person who works 
in black and white with no shades. Besides, the 
man in the street is not devoted to his Bible, nor 
to getting into the interior of the Bible, as you 
preachers are. We must take our way, God's 
way, and follow the subtle and searching Holy 
Spirit as He leads and speaks in and through 
the questions that arise to our earnest thought 
\marginpar{104}
concerning Christ's death. And the man in the 
street must be left to the grace which has taken 
us in from the street. 

The distinction I ask you to observe is 
between a change of feeling and a change of 
treatment, between affection and discipline, 
between friendly feeling and friendly relations. 
God's feeling toward us never needed to be 
changed. But God's treatment of us, God's 
practical relation to us---that had to change. I 
have pointed out that the relation between God 
and man in reconciliation is a personal one, and 
that, where you have real personal relation 
and personal communion, if there is change on 
one side there must be change on the other. 
The question is as to the nature of the change. 
We have barred out the possibility of its being a 
change of affection, of hatred into grace. God 
never ceased to love us even when He was most 
angry and severe with us. It will not do to 
abolish the reality of God's anger towards 
us. True love is quite capable of being angry, 
and must be angry and even sharp with its 
beloved children. Let us fix our attention 
more closely upon this distinction of mood and 
manner. 

\marginpar{106}
\begin{center}
\S
\end{center}

Take the parable of the prodigal for illustration. 
There are those who say you have the 
whole of the gospel really in the parable of the 
prodigal son, that that was the culmination of 
Christ's grand revelation of God. Well, if that 
were so the wonder to me is, first, that the 
apostles never seem to have used it; and, second, 
that having delivered this parable Christ did 
not at once consider His mission discharged and 
return to heaven. Or, on the other hand, why 
did He not continue to live to a ripe and useful 
age, reiterating in various forms and in different 
settings this waiting (but inert) love and grace 
of God? We are moved sometimes to think He 
might have done well had He not provoked 
death so early, had He remained, like John, to 
seventy or ninety years of age continually 
publishing, applying, and spreading the message 
which He gave His disciples. But you have not 
the whole gospel in the parable of the prodigal 
son. What is the function of a parable? It is 
one of the great discoveries and lessons taught 
us by modern scholarship, that parables are not 
allegories, because they exist for the sake of 
one central idea. While we may allow ourselves, 
under the suggestion of the Holy Spirit, 
to receive hints of edifying truth from this or 
\marginpar{107}
the other phase or detail of the parable, we 
have chiefly to ask, What was it in the mind of 
Christ for the sake of which He uttered this 
parable? Each parable puts in an ample ambit 
one central idea. Now the one ruling idea in 
the parable of the prodigal son is the idea of 
the centrality, the completeness, the unreservedness, 
the freeness, fullness, whole-heartedness of 
God's grace---the absolute fullness of it, rather 
than the method of its action. But however a 
parable might preach that fullness, it took the 
Cross and all its train to give it effect, to put it 
into action, life, and history, to charge it with 
the Spirit. Those who tell us that the whole 
gospel is embodied in the parable say, You 
observe nothing is suggested in the parable 
about the Cross and the Atonement; therefore 
the Cross and the Atonement are subsequent 
and gratuitous additions, confusing the gospel 
of grace. But that turns Christ into a mere 
preacher, instead of the centre of the world's 
history. Bear in mind also that this parable was 
spoken by the Christ who had the Cross in the 
very structure of His personality as its vocation, 
and at the root, therefore, of all His words. 
That Cross was deep embedded in the very structure 
of Christ's Person, because nowadays you 
cannot separate His Person from His vocation, 
\marginpar{108}
from the work He came to do, and the words He 
came to speak. The Cross was not simply a 
fate awaiting Christ in the future; it pervaded 
subliminally His holy Person. He was born for 
the Cross. It was His genius, His destiny. It 
was quite inevitable that, in a world like this, 
One holy as Jesus was holy should come to the 
Cross. The parable was spoken by One in 
whom the Cross and all it stands for were 
latent in His idea of God; and it became 
patent, came to the surface, became actual, 
and practical, and powerful in the stress of 
man's crisis and the fullness of God's time. 
That is an important phrase. Christ Himself 
came in a fullness of time. The Cross which 
consummated and crowned Christ came in its 
fullness of time. The time was not full during 
Christ's life for preaching an atonement that life 
could never make. Hence as to the \textit{method} of 
God's free and flowing grace the parable has 
nothing to say. It does not even say that 
the father went seeking the prodigal. The 
seeking grace of God we find there as little 
as the redeeming grace. And so also you 
have not the mode of grace's action \textit{on a} 
\textit{world}. But, speaking of what you do have in 
the parable, the father knows no change of 
feeling towards the prodigal; yet could he go 
\marginpar{109}
on making no difference? Could he go on 
treating the prodigal as though he never had 
become a prodigal? He did not certainly when 
he returned; and as little could he before. 
His heart followed the prodigal, but his relations, 
his confidence, his intercourse were 
with his brother. So long as the son is prodigal 
he cannot be treated as though he were 
otherwise. Even repentance needs some guarantee 
of permanence. The father's heart is 
the same, but his treatment must be different. 
Cases have been known where the father had 
to expel the black sheep from the family for 
the sake of the others. Loving the poor 
creature all the same, he yet found it quite 
impossible, in the interests of the whole family, 
to treat him as though he were like the rest. 
So God needed no placation, but He could not 
exercise His kindness to the prodigal world, He 
certainly could not restore communion with its 
individuals, without doing some act which permanently 
altered the relation. And this is what 
set up that world's reconciliation with Him. It 
was set up by an act of crisis, of judgment. 

\begin{center}
\S
\end{center}

Remember always we are dealing with the 
world in the first instance and not with individuals. 
\marginpar{110}
I constantly come back upon that, for 
the orthodox and their critics forget it alike. 
I suppose the prodigal was a slave, I suppose 
he had sold himself to that vile work of swine-feeding. When he returned I suppose he ran 
away from his master. But the prodigal would 
% sic. Original says "world" not "would"
of course, could not run away from its master, 
it could not run away from the power that it 
was enslaved to. "Myself am hell." Supposing 
now the prodigal had not been able to run away. 
Supposing he had been guarded as a convict 
is guarded, then he could only come back by 
being bought off. As soon as you go beyond 
the one theme of the parable, the absolute 
heartiness of grace, and begin to think of grace's 
methods with a world, this point must be faced 
by all who are more than pooh-pooh sentimentalists 
in their religion. We have to deal 
with a world in a bondage it could not break. 
If the prodigal could not have arisen to go to 
his father; if the elder brother had sold up the 
whole farm, reduced himself to poverty, taken 
the sum in his hand, followed the prodigal 
into the far country, and there spent the whole 
amount in buying his brother's manumission 
from his master before a judge; and if it 
was all done by mutual purpose and consent 
of himself and his father; would not that act 
\marginpar{111}
be a great and effective thing, not so much 
in producing repentance but in a harder matter---in 
destroying a lien and making absolute certainty 
of the father's forgiveness? He is sure 
because the father not only says but pays. His 
mere repentance could not make him sure, 
could not place him at home again, could not 
put him where he set out. His mere repentance 
could turn his heart to his father, but it could 
not break the bar and fill him with certainty of 
his father's love and forgiveness. And that is 
what the sinner wants, and what the great and 
classic penitents find it so hard to believe. Now, 
the parable tells us of the freeness of God's 
grace, and its fullness, but the Cross enacts it 
and inserts it in real history. It shows to what 
a length that grace could go in dealing with a 
difficulty otherwise insuperable when we turn 
from a single prodigal to a world. The act 
which I have described by a New Testament 
extension of the parable---the act of Christ's 
Cross---is not simply to produce individual repentance, 
but it has its great effect upon the 
relation of the whole world to God. And the 
judgment, the payment, was on that scale. I 
will show you later that it was not pain that 
was paid but holy obedience. 

What the elder brother does in the supposition
\marginpar{112}
I have made is twofold. First, he secures 
the liberation, he deals with the equitable conditions 
of the release. Secondly, he also acts upon 
the prodigal's heart and confidence. In the first 
case he meets certain judicial conditions, certain 
social conditions, ethical conditions, bound 
up with the existing order, the law of society 
in which the prodigal was living. But it is 
said sometimes that there the analogy fails, 
because the elder son, acting for the father, 
in my extension of the story, has to deal with 
a law which is outside his control and outside 
the father's control; he has to deal with the 
law of society, with the law of the land where 
the prodigal was. Whereas, if you come to 
think about God, there can be no social and 
moral conditions which are outside His control. 
There, it is said, your illustration breaks down. 
God could ignore any such impediments at 
His loving will. Now, that is just the crucial 
mistake that you make, that even Kant does 
not allow us to make. God could do nothing 
of the kind. So far the omnipotence of God is 
a limited omnipotence. He could not trifle with 
His own holiness. He could will nothing against 
His holy nature, and He could not abolish the 
judgment bound up with it. Nothing in the 
compass of the divine nature could enable Him 
\marginpar{113}
to abolish a moral law, the law of holiness. That 
would be tampering with His own soul. It had 
to be dealt with. Is the law of God more loose 
than the law of society? Can it be taken liberties 
with, played with, and put aside at the 
impulse even of love? How little we should 
come to think of God's love if that were possible! 
How essential the holiness of that love is to 
our respect for it and our faith in its unchangeableness! 
If God's love were not essentially 
holy love, in course of time mankind would 
cease to respect it, and consequently to trust 
it. We need not a fond love, but a love we 
can trust, and for ever. What love wants is not 
simply love in response, but respect and confidence. 
In the bringing up of children to-day 
one often wishes they had more training in 
respect, even if less in affection. God's holy 
law is His own holy nature. His love is under 
the condition of eternal respect. It is quite 
unchangeable. It is just as much outside His 
operation, so far as abrogation goes, as was 
the law of the far country to the father of 
the prodigal. 

\begin{center}
\S
\end{center}

What was there in the work of Christ which 
went beyond a mere impressive declaration of a 
God who could not help being gracious, but fell 
\marginpar{114}
on the prodigal's neck without more ado? It 
was solidary judgment. I am urging that the 
difficulty we have in answering that question 
is due to our modern individualism. Individualism has done its work for Christianity for the 
time being, and we are now suffering from 
its after-effects. We do not realise that we are 
each one of us saved in a racial salvation. We 
are each one of us saved in the salvation of the 
race, in a collectivist redemption. What Christ 
saved was the whole human race. What He 
bought, if we may provisionally use the metaphor, 
was the Church, and not any aggregate 
of isolated souls. So great is a soul, and so great 
is its sin, that each man is only saved by an act 
which at the same time saves the whole world. 
If you reduce or postpone Christ's effect upon 
the totality of the world, you are in the long run 
preparing the way for a poor estimate of the 
human soul. The more you abolish the significance 
of Christ's redeeming death once for 
all, the more you are doing to lower Humanity 
morally, and make it a less precious thing than 
the cosmic world around us. My plea is that 
with no atonement, no solidary judgment of sin, 
you reduce reconciliation not only to sentiment 
but to a piecemeal series of individual repentances 
and conversions, leaving it a problem 
\marginpar{115}
whether the race as a whole will be saved at 
last. For the universality of Christianity (so 
dear to Broad Church) you must have that foregone 
finality which the New Testament offers in 
the atonement. 

I pointed out to you that in the Old Testament, 
for the most part, what faced God was not 
this prophet or that saint, this king or that particular 
juncture, but Israel. I said that in the 
subsequent phases of Jewish religion, indeed, 
that idea has its detail filled in; and in the later 
psalms, in many of those psalms which we know 
could only have been written after the captivity, 
you have pious individualism sometimes expressing 
itself very strongly. But there the two warring 
notes were---new individualism and old collectivism; 
and between these there never came 
complete reconcilement until Christ came and 
Christ's work. What have we in that great text, 
John iii. 16? "God so loved \textit{the world}"---the world 
was the prime object of God's love---"God so 
loved the world, that He gave His only-begotten 
Son, that \textit{whosoever} believeth on Him should not 
perish, but have eternal life." Love in the first 
instance directed upon the world, but directed 
upon the world in such a way that it should be 
taken home in every individual experience. 
Mark the two words, "the world" and "whosoever." 
\marginpar{116}
Dwell upon the contrast. God loved 
not this or that individual, or group of individuals, 
only. "God so loved the world" that 
He did something to it in such a way that every 
individual "whosoever" should receive the benefit, 
and receive it in the only way which made 
a world of saved individuals possible. You can 
never compound a saved world out of any 
number of saved individuals. But God did so 
save the world as to carry individual salvation 
in the same act. The Son of God was not an 
individual merely; He was the representative of 
the whole race, and its \textit{vis-\`{a}-vis}, on its own scale. 
So that, in Ephesians, the Church, in rising to 
Christ, had to acquire the fullness of a complete 
and colossal man. No individual prophet of salvation 
could save the world. He could not be 
capable of a pity great enough, or a love. The 
world could only be saved by somebody as large 
as the world, and indeed larger. If he could 
not save the world he could make no \textit{eternal} 
salvation of any individual. It is universal, 
eternal salvation every way universal not by 
the addition of all units, but in a solidary sense. 
What we are tempted to think of in our common 
version of Christianity is a mass of people, great 
or small, a mass of individuals, each one of 
whom makes his own terms with God and gets 
\marginpar{117}
discharge of his sin. It is salvation by private 
bargain. In conversion every individual makes 
his own peace with God through Jesus Christ, 
so that the work of God becomes a mere change 
of attitude, feeling, or temper on the side of man 
after man. That is not the New Testament idea. 
Again, in speaking of the change in God, Christ 
has been represented as enabling God to forgive 
by enabling Him to adjust His two attributes of 
justice and mercy within Himself. Some theologians 
of the Reformation---Melancthon for one---spoke 
of Christ in that fashion. But we have 
entirely outgrown that way of thinking and 
talking about it. It has produced much difficulty 
and scepticism. What does it proceed 
upon? It proceeds upon a certain definition 
of an attribute, as though an attribute were 
something loose within God which He could 
manipulate---as though the attributes of God 
were not God Himself, unchangeable God, in 
certain relations. The attributes of God are 
not things within Himself which He could 
handle and adjust. An attribute of God is God 
Himself behaving, with all His unity, in a particular 
way in a particular situation. God is 
a thinking God, let us say. He has the attribute 
of thought. Does that mean that the 
attribute of thought could be taken away, 
\marginpar{118}
that God could divest Himself of it? No. The 
thought of God is simply God thinking. So 
also the love of God is not an attribute of 
God; it is God loving. The holiness of God 
is not an attribute of God; it is the whole 
God Himself as holy. There is nothing in the 
Bible about the strife of attributes. Rather 
remember 1 John i. 9, "He is faithful and just 
to forgive us our sins." It is in the exercise of 
His faithfulness to Himself and His observance 
of justice that He should forgive. It lies in 
the very holiness that condemns. There is a 
similar text in the Psalms, "Thou art merciful; 
Thou givest to every man according to his 
work." He is the faithful and just to forgive. 
There needed no adjustment of His justice with 
His forgiveness. So also in Isaiah, "A just God 
and a Saviour." There can therefore be no strife 
of attributes. 

\begin{center}
\S
\end{center}

What, then, does it mean when we hear about 
the anger of God being turned away? To begin 
with, the anger of God means a great deal more 
than His passion, His temper, His mode of 
feeling, more than anger as an affection. The 
anger of God in the Bible means much rather 
the judgment of God in the reaction of His 
moral and spiritual order. The judgment of God 
\marginpar{119}
is perfectly compatible with His continued love, 
just as a father's punishment is perfectly compatible 
with his love for his children. The 
father has to discipline his children. He institutes 
certain laws, the children disobey; they 
must be punished, or, using the more dignified 
term, judged. The anger of God: we shall get 
the most meaning out of it when we think of it 
as the judgment of God, the exalted, inflexible 
judgment of God. 

\begin{center}
\S
\end{center}

Taking a step further, it is judgment on the 
world. It seems at first sight as though it 
were meaningless to speak as if God could be 
wroth with the world and yet gracious and 
loving to individuals. But I may be very angry 
with a political party, yet I cherish respect and 
love for individuals belonging to that party. 
We must be on our guard against narrow, individual 
views, against treating individuals according 
to their public and collective condemnation. 
We are created, redeemed, judged as members 
of a race or of a Church. Salvation is personal, 
but it is not individual. (There is another distinction for you, if you have come in off the 
street.) It is personal in its appropriation but 
collective in its nature. What did the Reformation stand for? Not for religious individualism. 
\marginpar{120}
But I hear some one asking in the back of his 
mind, Was not the Reformation the charter of 
private judgment and individual independence? 
It was nothing of the kind. It was the charter 
of personal direct faith and its freedom. What 
the Reformation did was to turn religion from 
being a thing mainly institutional into a thing 
mainly personal. The reformers were as strong 
as their opponents about the necessity of the 
Church for the soul---though as its home, not its 
master. They were not individualists. Individualism 
is fatal to faith. It was the backbone 
of the rationalism and atheism of the French 
Revolution. The Reformation stands for personal 
religion and social religion and not for 
religious individualism. 

There is no such thing as an absolute individual. 
What is the change that takes place 
when we are converted? Our change is really 
from one membership to another, from membership 
of the world to membership of the Church. 
When we become a member of the Church we are 
not really changed from individualism, but from 
membership of the world. It is membership 
either way. The greatest egoist and self-seeker 
is a member of the world. He could not indulge 
his egotism if it were not for the society in the 
midst of which he lives and into which he is 
\marginpar{121}
articulated. He is a member of the world who 
exploits his membership instead of serving with 
it. When we are converted we are not converted 
from a sheer and absolute individual. 
There never was such a person. Certainly 
Robinson Crusoe was not. We are converted 
from membership of the world to membership 
of Christ. Before our conversion and after 
we \textit{belong}. We are not absolute, solitary individuals. 
We are in a society, an organism. 
We are made by the past. And our selfish, 
godless actions and influence go out, radiate, 
affect the organism as they could not do were 
we absolute units. They spread far beyond our 
memory or control. In the same way we are 
acted upon by the other people. We are members 
one of another both for evil and for good. 
When you are told that evil is only selfishness it 
is worth while bearing this in mind. Even as selfish 
men, as egoists, we belong---only to a pagan 
order instead of to Christ. The selfish man is a 
member of a kingdom of evil. There is no such 
thing as an absolute individual. Hence, to save 
us, to reconcile us, involves the whole race we 
belong to. Before God that race is an organic 
unity. It is not a mere mass of atoms joined 
together by various arbitrary relations, sympathies, 
and affinities. Hence, as the race before 
\marginpar{122}
God is one, a personal God is able to do for the 
race some one thing which at the same time is 
good for every person in it. 

\begin{center}
\S
\end{center}

But now, if the race is a unity, where does its 
unity lie? Does it lie in our elementary affections 
for each other, in the palpable relationships of 
natural life with our parents, brothers, lovers, 
and friends? Or is the unity of the race simply 
its capacity for being organised by skilful 
engineers? Is the unity of the race like the 
unity of machines? No. The unity of the 
race is a moral unity. Therefore it is a unity 
of conscience. If you want to find the trunk 
out of which all the loves and practices of 
humanity proceed, you must go to conscience 
at the centre. That is where the unity of 
Humanity lies. It is in the conscience, where 
man is member of a vast moral world. It is 
the one changeless order of the moral world, 
emerging in conscience, that makes man universal. 
What have you to preach if you have 
no gospel that goes to the foundations of human 
conscience? What ground have you for a social 
religion? The most universal God is one that 
goes there, not to the heart in the sense of 
affections, but to the conscience. The great 
\marginpar{123}
motive for missions of every high kind is not 
sentiment, but salvation. It is dangerous to 
take your theology from poets and literary 
people. You quote, "One touch of nature makes 
the whole world kin." Well, if you are going to 
build a religion on that, it will have a very short 
life. In the long run nature means anarchy 
when taken by and for itself. But it was never 
meant to be taken by itself. It was meant to 
go in an eternal context with super-nature. It 
is not the touch of nature that makes us kin 
enough for religion, for eternity, but the touch, 
and more than a touch, of the supernatural---not 
nature, but grace. What makes the world 
God's world is the action and unity of God's 
moral order of which our conscience speaks. 

Now, if that order be broken, how can it be 
healed? If I slit the canvas of this tent it can 
be patched. I make a fissure, but it is not irremediable. 
I simply get some one to stitch it up. 
At the worst I can have a new width put in. But 
if the moral order, and its universal solidarity, its 
holiness, is broken, how can that be healed? 
That cannot be patched up. It is not merely a 
rent in a tissue, a gap in a process, which the 
same process goes on to heal into a scar. The 
moral law differs from all natural law in having 
in it a demand, a claim, an "ought" of a 
\marginpar{124}
universal kind. It is all of one piece. We use the 
word "law" in a loose kind of way. We apply 
the same word to gravitation and to the moral 
law of retribution. It is that ambiguity of terms 
which leads us astray. The moral law differs 
from every other law in having a demand, and a 
universal demand, a claim upon us for ever. 
And that has to be made good as well as the 
rents and bruises in us from our own collision 
with it. It is not a gap that has to be made 
good and sound. It is a claim, because we are 
here in a moral and not a natural world. It is 
one thing to make good a gap and another thing 
to make good a claim. The claim must be met. 
It will not do simply to draw the edges together 
by mere amendment, to have God here and man 
there, and gradually bring them together till 
they unite. It is two moral persons with 
moral passions we have to do with. It is moral 
relationship that is in question, communion, 
trustful mutuality, is the object of the divine 
requirement. It is a case of moral, holy reconcilement. 
It is the expression of God's holy 
personality whenever God makes His claim. 
It is Himself in holy, changeless personality 
that says, "Thou shalt." Then the claim can 
only be honoured by personality of acknowledgment. 
But what does that mean? Some 
\marginpar{125}
confession, some compunction---"I have sinned?" 
That is a poor acknowledgment of God's holiness. 
It was neither in word nor in feeling that 
we wounded that, but in life and deed. It must 
be acknowledged in like fashion---practically. 
The holiness of God is the sum of all His action 
and relation to the world; and the acknowledgment 
of it must be made in like action. Do we 
acknowledge the holiness of God's infinite law 
simply when its penalty wrings from poor us a 
confession of sin? We acknowledge natural 
law in spite of ourselves when we suffer its 
penalty amid our rebellion. But the acknowledgment 
of moral, of holy law is something 
different. It must be actively acknowledged---acknowledged 
not in spite of ourselves but by 
ourselves, with our whole heart; and it cannot 
be acknowledged simply by individual, or, indeed, 
any suffering. For divine judgment it 
must be acknowledged in kind and scale, and 
met by a like holiness. Mere suffering is no 
acknowledgment really; it is a pure sequel; it 
is not a confession of the moral law and its 
righteousness, only of its power. Mere suffering 
is no confession of the holiness of God. God, 
truly, might and does assert His power upon 
our defiance by making us suffer. But do you 
think any holiness, any loving holiness, could 
\marginpar{126}
be satisfied with making the offender suffer? 
There is only one thing that can satisfy the 
holiness of God, and that is holiness---adequate 
holiness. To judge is to secure that at cost of 
\textit{any} pain both to the judge and the culprit. But 
the pain is not the end. Nothing, no penalty, 
no passionate remorse, no verbal acknowledgment, 
no ritual, can satisfy the claim of holy 
law---nothing but holiness, actual holiness, and 
holiness upon the same scale as the one holy 
law which was broken. The confession must be 
adequate. Fix that word in your mind. All 
your repentance, and all the world's repentance, 
would not be adequate to satisfying, establishing 
the broken law of holy God. Confession 
must be adequate---as Christ's was. We do not 
now speak of Christ's sufferings as being the 
\textit{equivalent} of what we deserved, but we speak 
of His confession of God's holiness, his acceptance 
of God's judgment, being \textit{adequate} in a 
way that sin forbade any acknowledgment 
from us to be. For the only adequate confession 
of a holy God is perfectly holy man. 
Wounded holiness can only be met by a 
personal holiness upon the scale of the race, 
upon the universal scale of the sinful race, and 
upon the eternal scale of the holy God who was 
wounded. It is not enough that the eternal 
\marginpar{127}
validity of the holy law should be declared as 
some prophet might arise and declare it, with 
power to make the world admire, as the great 
and sublime Kant did. It must take effect. 
Prophets have arisen who have produced 
tremendous effect by insisting upon the moral 
ultimacy in life and things. The greatest 
prophets of the last century, like George Eliot, 
Carlyle, Ruskin, and Maurice among ourselves 
had that as a chief note. But it is not 
enough that the eternal validity and inflexibility 
of eternal law should be powerfully, 
searchingly declared. It must take effect. Its 
breach must be closed up not merely by recognition, 
but by judgment. It is not enough that 
the whole human race should come confessing, 
"We have offended against Thy holy law." 
That would recognise the holy law and confess 
its place, but it would not give it its own, it 
would not bring to pass that which is essential 
to holiness, namely, judgment. It would not 
actually establish holiness in a kingdom, in 
command of history. You cannot separate the 
idea of holiness and its kingdom from the idea 
of judgment. In the Old Testament the final 
coming of the Great Salvation was always connected 
with a great judgment, which was 
therefore not a terror, as we view it, but the 
\marginpar{128}
grandest hope. If the essence of God is that He 
should be holy, it is equally essential that He 
should judge. If He sets up actual holiness it 
must be by actual adjustment of everything 
to it. It is not enough that we should say, 
"Thou art our Judge, we submit and are willing 
to take the penalty. The wages of sin is 
death." All that is best and greatest in human 
life turns upon something more than that. 
There is a phrase which I never tire of quoting, 
and it is this: "The dignity of man is better 
assured if he were broken upon the maintenance 
of that holiness of God than if it 
were put aside just to give him an existence." 
The dignity, the very dignity of man himself 
is better assured if he were broken upon the 
maintenance of that holiness of God than if it 
were put aside arbitrarily, just to let him off 
with his life. This holy order is as essential 
to man's greatness as it is to God's; and that is 
why the holy satisfaction Christ made to God's 
holiness is in the same act the glorifier of 
the new humanity. Any religion which leaves 
out of supreme count the judging holiness 
of God is making a great contribution to the 
degradation of man. We need a religion which 
decides the eternal destiny of man; and unless 
holiness were practically and adequately established---not
\marginpar{129} % Word not is on p. 129
merely recognised and eulogised, 
but established---there could be no real, deep, 
permanent change in the world or the sinner. 
The change in the treatment of us by eternal 
grace must rest on judgment taking effect. 
Man is not forgiven simply by forgetting and 
mending, by agreeing that no more is to be 
said about it. To make little of sin is to 
belittle the holiness of God; and from a reduced 
holiness no salvation could come, nor 
could human dignity remain. 

\begin{center}
\S
\end{center}

Here, perhaps, you want to ask me what I 
mean exactly by saying that the judgment-death 
of Christ set up a real and actual kingdom of 
holiness. It is a point which it is easier for 
faith to realise than for theology to explain. 
But the answer would lie along this line: What 
Christ presented to God for His complete joy 
and satisfaction was a perfect racial obedience. 
It was not the perfect obedience of a saintly 
unit of the race. It was a racial holiness. God's 
holiness found itself again in the humbled holiness 
of Christ's "public person." He presented 
before God a race He created for holiness. Remember 
that the very nature of our faith in 
Christ is union with Him. The kingdom is set 
\marginpar{130}
by Christians being united with the work, the 
victory, the obedience, the holiness of the King. 
Christ, in His victorious death and risen life, 
has power to unite the race to Himself, and 
to work His complete holiness into its actual 
experience and history. He has power, by 
uniting us with Him in His Spirit, to reduce 
Time to acknowledge in act and fact His 
conclusive victory of Eternity. When you 
think of what He did for the race and its 
history, you must on no account do what the 
Church and its theology has too often done---you 
must not omit our living union with Him. 
It is not enough to believe that He gained a 
victory at a historic point. Christ is the 
condensation of history. You must go on to 
think of His summary reconciliation as being 
worked out to cover the whole of history and 
enter each soul by the Spirit. You must think 
of the Cross as setting up a new covenant and 
a new Humanity, in which Christ dwells as the 
new righteousness of God. "Christ for us" is 
only intelligible as "Christ in us" and we in Him. 
By uniting us to Himself and His resurrection 
in His Spirit He becomes the eternal guarantee 
of the historical consummation of all things 
some great day. I return to this later. 

\marginpar{131}
\begin{center}
\S
\end{center}

Sometimes, when I have been talking about 
this claim of God's holiness, a critic has said: 
"You are treating the holiness of God as 
though it were a power outside God, tying 
His hands." Nothing of the kind. What is 
meant by the holiness of God is the holy 
God. We talk nonsense in a like way about 
the decrees of God. We say they stand for 
the wretched survival of an outworn Calvinism, 
as though they were things that God 
could handle. Do you think that mighty men 
such as the great Reformers were would have 
been led into saying the things they did about 
God if they thought the decrees were simply 
things God could handle, or things like a doom 
on God? The decrees of God were to them God 
decreeing. The holiness of God was God as holy. 
When that holiness is wounded or defied, could 
God be content to take us back with a mere 
censure or other penance and the declaration 
that He was holy? We could not respect a God 
like that. Servants despise indulgent masters. 
Sinners would despise a God who would take us 
back when we wept, and speak thus: "Let us 
say no more about it. You did very wrong, and 
you have suffered for it, and I; but let us forget 
it now you have come back." We should not 
\marginpar{132}
respect that. We should go on, as servants do in 
the case I have named, to take more liberties 
still. He would be a God who only talked His 
holiness and did not put it into force. Now, if 
our repentance were our atonement, and the 
Cross were simply an object-lesson to us of 
God's patient and tender mercy to penitence, 
He would be talking, I said, and not acting. He 
would mention the gravity of our sin very impressively, 
but that would not be establishing 
goodness actually in the history and experience 
of man. The sinner's reconciliation to a God 
of holy love could not take place if guilt 
were not destroyed, if judgment did not take 
place on due scale, if the wrath of God did 
not somehow take real effect. You say, perhaps, 
it did take effect in the unseen world 
of spirits. But the moral world is not a world 
of ghostly spirits. It is the unseen side of the 
world of history and of experience, it is its 
inner reality and centre. The vindication, the 
judgment, must take place within human history 
and experience. It must take place in the 
terms of human history, by human action, in a 
place, at some point, on a due scale and with adequate 
depth. That was what took place in the 
Cross of Christ. The idea of judgment is not 
complete without the idea of a crisis, a day of 
\marginpar{133}
judgment. Now the Cross of Christ was the 
world's great day of judgment, the crisis of all 
crises for history. The holy love of God yearning 
over souls could not deal with individual 
sinners, there was a cloud between God and the 
race, till the holiness was owned and perfectly 
praised by its racial confession, until holiness 
was confessed much more than sin, until on 
man's side there was not only confession of 
sin but confession of holiness from sin's side 
amid the experience of a judgment on the 
scale of the race, until the confessing race 
was thus put in right relation to God's holiness. 
Then judgment had done its perfect 
work. The race's sin was covered and atoned 
by it, \textit{i.e.}, by the God who bore it. Individuals 
could not be reconciled to a holy God until 
He thus reconciled the world. Not until sin had 
been brought to do its very worst, and had in 
that culminating act been foiled, judged, and 
overcome; not till then could individuals receive 
the reconciliation. That was the unitary reconciliation 
they must receive in detail. God there, 
in a racial holiness amid racial curse, sets 
up a racial salvation, which our souls enter 
upon by faith. It is by Himself in His changeless 
love and pity that it is set up. It is not the 
Son's suffering and death, but His holy obedience 
\marginpar{134}
to both that is the satisfying thing to God, the 
holiness of God the Son. In a sense, a great 
solemn sense, it is an exercise of God's absolute 
self-satisfaction, exhibited after a long historic 
process, amidst the dissatisfaction of a world's 
ruin. "In His love and in His pity He redeemed 
them." He set up reconciliation by an act of 
judgment on His Son, cutting off His own right 
hand that we might enter into the Kingdom 
of heaven: "In His love and in His pity He 
redeemed them; and He bare them, and carried 
them all the days of old." The redemption was 
a thing that was coming through the whole of 
Israel's history, and in a remoter sense through 
the whole history of the world. The changeless 
holiness must assert itself in such judgment as 
surely as in the kingdom. You all believe that 
the holiness of God must assert itself in the 
Kingdom of God. But how can there be a final 
kingdom without final judgment? Is not all 
judgment in the name of the king, even in our 
human society? Are not king and judge inseparable, 
as inseparable as king and father? 
We say to-day that king and father are inseparable. 
But king and judge are equally 
inseparable, especially if you take the great Old 
Testament idea. Christ submitted with all His 
heart to God's holy final judgment on the race. 
\marginpar{135}
He did not view it as an unfortunate incident 
in His life. He did not treat it as though it 
happened to drop upon Him. But He treated 
it as the grand will of God, as the effectuation 
in history of God's holiness, which holiness 
must have complete response and practical 
confession both on its negative side of judgment 
and its positive side of obedience. Christ's 
death was atoning not simply because it was 
sacrifice even unto death, but because it was 
sacrifice unto holy and radical judgment. There 
is something much more than being obedient 
unto death. Plenty of men can be obedient unto 
death; but the core of Christianity is Christ's 
being obedient unto judgment, and unto the 
final judgment of holiness. It is being obedient 
to a kind of death prescribed by God, indispensable 
to the holiness of God's love, necessitated 
in such a world by the last moral conditions, and 
not simply inflicted by the wickedness of men. 

Get rid of the idea that judgment is chiefly 
retribution, and directly infliction. Realise that 
it is, positively, the establishing and the securing 
of eternal righteousness and holiness. View 
punishment as an indirect and collateral necessity, 
like the surgical pains that make room for 
nature's curing power. You will then find 
nothing morally repulsive in the idea of judgment
\marginpar{136}
effected in and on Christ, any more than 
in the thought that the kingdom was set up 
in Him. 

\begin{center}
\S
\end{center}

To conclude, then, God could only justify man 
before Him by justifying Himself and His holy 
law before men. If He had not vindicated His 
holiness to the uttermost in that way of judgment, 
it would not be a kind of holiness that 
men could trust. Thus a faith which could 
justify man, which could make a foundation for 
a new humanity, could not exist. We can only 
be eternally justified by faith in a God who 
justifies Himself as so holy that He must set 
up His holiness in human history at any price, 
even at the price of His own beloved and 
eternal Son. 

I close, then, upon that unchangeable word of 
God's self-justifying holiness. Even the sinner 
could not trust a love that could not justify itself 
as holy. It is the holiness in God's love, I urge, 
that alone enables us to trust Him. Without 
that we should only love Him, and the love 
would fluctuate. For we could not be perfectly 
sure that His would not. It is the holiness in 
God's love that is the eternal, stable, unchangeable 
element in it---the holiness secured for 
history and its destiny in the Cross. It is only 
\marginpar{137}
the unchangeable that we could trust; and 
there alone we find it. If we only loved the 
love of God, we should have no stable, eternal, 
universal religion. But we love the \textit{holy} love 
He established in Christ, and therefore we 
are safe with an everlasting salvation. 

\chapter{THE CROSS THE GREAT 
CONFESSIONAL} 
\chaptermark{The Cross the Great Confessional}
%\marginpar{141}

%V 

%THE CROSS THE GREAT CONFESSIONAL 

\textsc{In} the days of our fathers Christian belief 
was more solid within the Church than it 
is now; and the defending and expounding of 
Christianity, more especially the defending of 
it, had to concern itself with outsiders---outside 
the Church, and outside Christianity very often. 
To-day our difficulties have changed; and a 
great part of our exposition must keep in view 
the fact that some of the most dangerous challenges 
of Christianity are found amongst those 
who claim the Christian name. There are those 
who have a very real reverence for the character 
of Jesus Christ, and they can speak, and 
do speak, quite sincerely, with great devotion 
and warmth and beauty, about Christ, and 
about many of the ideas that are associated 
with apostolic Christianity. All the same, they 
\marginpar{141}
are strongly and sometimes even violently, 
antagonistic to that redemption which is the 
very centre of the Christian faith; and they 
make denials and challenges which are bound 
to tell upon the existence of that faith before 
many generations are over. We do not take 
the true measure of the situation unless we 
realise that the thing which is at stake at this 
moment is something that will not affect the 
present generation, but is sure to affect two or 
three generations hence. Those who are concerned 
about Christianity on the largest scale 
to-day are concerned with what may be its 
position and its prospects then. The ideas at 
the centre of the Christian faith are too large, 
too deep and subtle, to show their effects 
in one age; and the challenge of them does 
not show its effect in one generation or even 
in two. Individuals, society, and the Church, 
indeed, are able to go on, externally almost unaffected, 
by the way that they have upon them 
from the past; and it is only within the range 
of several generations that the destruction of 
truths with such a comprehensive range as 
those of Christianity takes effect. Therefore it 
is part of the duty of the Church, in certain 
sections and on certain occasions, to be less 
concerned about the effect of the Gospel upon 
\marginpar{142}
the individual immediately, or on the present 
age, and to look ahead to what may be the 
result of certain changes in the future. God 
sets watchmen in Zion who have to keep their 
eye on the horizon; and it is only a drunken 
army that could scout their warning. We are 
not only bound to attend to the needs and 
interests of the present generation; we are 
trustees for a long future, as well as a long 
past. Therefore it is quite necessary that the 
Church should give very particular attention 
to these central and fundamental points whose 
influence, perhaps, is not so promptly prized, 
and whose destruction would not be so mightily 
felt at once, but would certainly become apparent 
in the days and decades ahead. 

That is why one feels bound to invite attention, 
and to press attention, upon points concerning 
which it may very easily be said, "These 
are matters that do not concern my faith and 
my piety; I can afford to let these things alone." 
Perhaps A, B, and C can, and X, Y, and Z can; 
but the Christian Church cannot afford to let 
these things alone. The Church carries the 
individual amid much failure of his faith; 
there is a vicarious faith; but what is to 
carry the Church if its faith fail? Remove 
concern from these things, and the effect of 
\marginpar{144}
the collective message of the Church to the 
great world becomes undermined. Then the 
world must look somewhere else than to the 
Church for that which is to save it. That is 
some apology for dwelling upon points which 
many people would say were simply theological 
and were outside the interest of the individual 
Christian. Theology simply means thinking in 
centuries. Religion tells on the present, but 
theology tells on the religion of the future 
and the race. 

Moreover, there are always natures among 
Christian people who refuse, and properly 
refuse, to remain satisfied with superficial 
experiences or current views of their faith. 
They are bound by the spirit that moves 
within them---by the kind of temperament God 
has given them they are bound to penetrate 
to the heart, to the depths of things. Their 
work does not immediately pay; and while they 
grind in their mill the Philistines mock and the 
libertines jeer. But it would be a great misfortune 
if the whole of the work of the Church 
were measured by the standard which is so 
necessary in the world---the standard of what 
will immediately pay, or promptly tell. It is, 
of course, a great thing to go back upon the 
history of Christianity, and to point out to ourselves 
\marginpar{145}
and to our people the great things that 
Christianity has done in the course of history. 
But you cannot rest Christianity upon that. 
You can only rest Christianity upon Christ 
Himself, and His living presence in the New 
Humanity. You can put the matter in this 
way. You can ask, On what did the Christianity 
rest of those who believed in the very 
first years of the Church's life? They had no 
results of Christianity before them. They had 
no history of the Church before them. They 
had not the glorious story of Christian 
philanthropy before them, nor the magnificent 
expansion of Christian doctrine, nor the enormous
influence of the Christian Church and its 
effect upon the course of the world's history. 
On what did they rest their faith? That upon 
which they rested their faith must be that upon 
which we rest our faith when we come to a 
real crisis, and are driven into a real corner. 
It thus becomes necessary to go into the deep 
things of God as they are revealed to us by the 
Holy Spirit, through His inspired apostles, in 
Christ and His Cross. 

\begin{center}
\S
\end{center}

From what I have said you will be prepared 
to hear me state that reconciliation is effected 
by the representative sacrifice of Christ crucified;
\marginpar{146}
fied; by Christ crucified as the representative of 
God on the one hand and of Humanity, or the 
Church, on the other hand. Also it was by 
Christ crucified in connection with the divine 
judgment. Judgment is a far greater idea than 
sacrifice. For you see great sacrifices made for 
silly or mischievous causes, sacrifices which show 
no insight whatever into the moral order or the 
divine sanctity. Now this sacrifice of Christ, 
when you connect it with the idea of judgment, 
must in some form or other be described as a 
penal sacrifice. Round that word penal there 
rages a great deal of controversy. And I am 
using the word with some reserve, because there 
are forms of interpreting it which do the idea 
injustice. The sacrifice of Christ was a penal 
sacrifice. In what sense is that so? We can 
begin by clearing the ground, by asking, In 
what sense is it not true that the sacrifice of 
Christ was penal? Well, it cannot be true in 
the sense that God punished Christ. That is an 
absolutely unthinkable thing. How could God 
punish Him in whom He was always well 
pleased? The two things are a contradiction 
in terms. And it cannot be true in the sense 
that Christ was in our stead in such a way as 
to exclude and exempt us. The sacrifice of 
Christ, then, was penal not in the sense of God 
\marginpar{147}
so punishing Christ that there is left us only 
religious enjoyment, but in this sense. There is 
a penalty and curse for sin; and Christ consented 
to enter that region. Christ entered voluntarily 
into the pain and horror which is sin's penalty 
from God. Christ, by the deep intimacy of His 
sympathy with men, entered deeply into the 
blight and judgment which was entailed by 
man's sin, and which must be entailed by man's 
sin if God is a holy and therefore a judging 
God. It is impossible for us to say that God 
was angry with Christ; but still Christ entered 
the wrath of God, understanding that phrase 
as I endeavoured to explain it yesterday. He 
entered the penumbra of judgment, and from 
it He confessed in free action, He praised and 
justified by act, before the world, and on the 
scale of all the world, the holiness of God. You 
can therefore say that although Christ was not 
punished by God, He bore God's penalty upon 
sin. That penalty was not lifted even when the 
Son of God passed through. Is there not a real 
distinction between the two statements? To 
say that Christ was punished by God who was 
always well pleased with Him is an outrageous 
thing. Calvin himself repudiates the idea. But 
we may say that Christ did, at the depth of that 
great act of self-identification with us when He 
\marginpar{148}
became man, He did enter the sphere of sin's 
penalty and the horror of sin's curse, in order 
that, from the very midst and depth of it, His 
confession and praise of God's holiness might 
rise like a spring of fresh water at the bottom 
of the bitter sea, and sweeten all. He justified 
God in His judgment and wrath. He justified 
God in this thing. 

\begin{center}
\S
\end{center}

So the act of Christ had this twofold aspect. 
On the one hand it was God offering, and on the 
other hand it was man confessing. Now, what 
was it that Christ chiefly confessed? I hope you 
have read McLeod Campbell on the Atonement. 
Every minister ought to know that book, and 
know it well. But there is one criticism to be 
made upon the great, fine, holy book. And it 
is this. It speaks too much, perhaps, about 
Christ confessing human sin, about Christ 
becoming the Priest and Confessor before God 
of human sin and exposing it to God's judgment. 
The horror of the Cross expresses the repentance 
of the race before a holy God for its sin. 
But considerable difficulties arise in that connection, 
and critics were not slow to point them 
out. How could Christ in any real sense confess 
a sin, even a racial sin, with whose guilt He 
had nothing in common? Now that is rather a 
\marginpar{149}
serious criticism if the confession of sin were 
the first charge upon either Christ or us, if the 
confession of human sin were the chief thing 
that God wanted or Christ did. I think it is 
certainly a defect in that great book that 
it fixes our attention too much upon Christ's 
vicarious confession \textit{of human sin}. The same 
criticism applies to another very fine book, 
that by the late Canon Moberly, of Christ 
Church, "Atonement and Personality." I once 
had the privilege of meeting Canon Moberly 
in discussion on this subject, and ventured to 
point out that defect in his theory, and I was 
relieved to find that on the occasion the same 
criticism was also made by Bishop Gore. But 
we get out of the difficulty, in part at least, if we 
recognise that the great work of Christ, while 
certainly it did confess human sin, was yet not 
to confess that, but to confess something greater, 
namely, God's holiness in His judgment upon 
sin. His confession, indeed, was not in so many 
words, but in a far more mighty way, by act and 
deed of life and death. The great confession is 
not by word of mouth---it is by the life, in the 
sense, not of mere conduct, but in the great 
personal sense in which life contains conduct 
and transcends death. Christ confessed not 
merely human sin---which in a certain sense, 
\marginpar{150}
indeed, He could not do---but He confessed God's 
holiness in reacting mortally against human sin, 
in cursing human sin, in judging it to its very 
death. He stood in the midst of human sin 
full of love to man, such love as enabled Him 
to identify Himself in the most profound, sympathetic 
way with the evil race; fuller still of 
love to the God whose name He was hallowing; 
and, as with one mouth, as if the whole race 
confessed through Him, as with one soul, as 
though the whole race at last did justice to God 
through His soul, He lifted up His face unto 
God and said, "Thou art holy in all Thy judgments, 
even in this judgment which turns not 
aside even from Me, but strikes the sinful spot if 
even I stand on it." The dereliction upon the 
Cross, the sense of love's desertion by love, was 
Christ's practical confession of the holy God's 
repulsion of sin. He accepted the divine situation---the 
situation of the race before God. By 
God's will He did so. By His own free consent 
He did so. Remember the distinction between 
God's changeless love and God's varying treatment 
of the soul. God made Him sin, treated 
Him as if He were sin; He did not view Him as 
sinful. That is quite another matter. God made 
Him to be sin---it does not say He made Him sinful. 
God lovingly treated Him as human sin, and 
\marginpar{151}
with His consent judged human sin in Him and 
on Him. Personal guilt Christ could never confess. 
There is that in guilt which can only be 
confessed by the guilty. "I did it." That kind 
of confession Christ could never make. That is 
the part of the confession that we make, and we 
cannot make it effectually until we are in union 
with Christ and His great lone work of perfectly 
and practically confessing the holiness 
of God. There is a racial confession that can 
only be made by the holy; and there is a personal 
confession that can only be made by the 
guilty. That latter, I say, is a confession Christ 
could never make. In that respect Christ did 
not die, and did not suffer, did not confess, in 
our stead. We alone, the guilty, can make 
that confession; but we cannot make it with 
Christian effect without the Cross and the 
confession there. We say then not only "I did 
this," but "I am guilty before the holiness 
confessed in the Cross." The grand sin is 
not to sin against the law but against the 
Cross. The sin of sins is not transgression, 
but unfaith. 

So also of holiness, there is a confession of 
holiness which can only be made by God, the 
Holy. If God's holiness was to be fully confessed, 
in act and deed, in life, and death, and 
\marginpar{152}
love transcending both, it can only be done by 
Godhead itself. 

\begin{center}
\S
\end{center}

Therefore we press the words to their fullness 
of meaning: "God was in Christ reconciling," 
not reconciling through Christ, but actually 
present as Christ reconciling, doing in Christ 
His own work of reconciliation. It was done by 
Godhead itself, and not by the Son alone. The 
old theologians were right when they insisted 
that the work of redemption was the work of 
the whole Trinity---Father, Son, and Holy 
Spirit; as we express it when we baptize into 
the new life of reconcilement in the threefold 
name. The holiness of God was confessed in 
man by Christ, and this holy confession of 
Christ's is the source of the truest confession of 
our sin that we can make. Our saving confession 
is not merely "I did so and so," but "I did 
it against a holy, saving God." "I have sinned 
against heaven and in thy sight," sinned before 
infinite holiness and forgiving grace. God could 
not forgive until man confessed, and confessed 
not only his own sin but confessed still more---God's 
holiness in the judgment of sin. The 
confession also had to be made in life and action, 
as the sin was done. That is to say, it had to be 
made religiously and not theologically, by an 
\marginpar{153}
experience and not an utterance. A verbal 
confession, however sincere, could not fully own 
an actual sin. If we sin by deed we must so 
confess. It is made thus religiously, spiritually, 
experimentally, practically by Jesus Christ's 
life, its crown of death, and His life eternal. 
The more sinful man is, the less can he thus 
confess either his own sin or God's holiness. 
Therefore God did it in man by a love which 
was as great as it was holy, by an infinite love. 
That is to say, by a love which was as closely 
and sympathetically identified with man as it 
was identified with the power of the holy God. 

So we have arrived at this. The great confession 
was made not alone in the precise hour 
of Christ's death, although it was consummated 
there. It had to be made in life and act, and 
not in a mere feeling or statement; and for this 
purpose death must be organically one with 
the whole life. You cannot sever the death 
of Christ from the life of Christ. When you 
think of the self-emptying which brought 
Christ to earth, His whole life here was a living 
death. The death of Christ must be organic 
with His whole personal life and action. And 
that means not only His earthly life previous to 
the Cross, but His whole celestial life from the 
beginning, and to this hour, and to all eternity. 
\marginpar{154}
The death of Christ is the central point of 
eternity as well as of human history. His own 
eternal life revolves on it. And we shall never 
be so good and holy at any point, even in 
eternity, that we shall not look into the Cross 
of Christ as the centre of all our hope in earth 
or heaven. It is Christ that works out His own 
redemption and reconciliation, from God's right 
hand, throughout the course of history. I 
would gather that up in one phrase. Christ is 
the perpetual providence of His own salvation. 
Christ, acting through His Spirit, is the eternal 
providence of His own salvation. The apostles 
never separated reconciliation in any age from 
the Cross and blood of Jesus Christ. If ever we 
do that (and many are doing it to-day) we 
throw the New Testament overboard. The bane 
of so much that claims to be more spiritual 
religion at the present day is that it simply 
jettisons the New Testament, and with it historic 
Christianity. The extreme critics, people that 
live upon monism and immanence, rationalist 
religion and spiritual impressionism, are people 
who are deliberately throwing overboard the 
New Testament as a whole, deeply as they prize 
it in parts. They say that the apostolic views 
and interpretations of Christ's work may have 
been all very well for people who knew no 
\marginpar{155}
better than men did at so early a period, but 
we are now a long way beyond that, and 
we must re-edit the New Testament theology, 
especially as to Christ's death. I keep urging, 
whatever we do let us do it frankly, let us do it 
with our eyes open and with eyes competent to 
take the measure of what we are doing. The 
trying thing is that tremendous renunciations 
should be blandly made, without, apparently, 
any sense of their appalling dimensions, and 
of the huge thing that is being so ignorantly 
done. (See note at the end of this lecture.) 

\begin{center}
\S
\end{center}


The apostles, I say, never separated reconciliation 
from the Cross and the blood of Jesus 
Christ. The historic Church has never done so, 
with all its divisions. And what the Cross 
meant for the apostles as Jews, with their history 
and education, was something like this. If 
you go back to the Old Testament, you find 
that the whole kingdom of God and destiny of 
man turns on the treatment of sin. And either 
the sin was atoned or the sinner was punished. 
But there were some sins that never could be 
atoned for, what are described as sins with 
a high hand, presumptuous sins, deliberate, 
defiant sins, as distinct from sins of ignorance 
\marginpar{156}
or weakness, when a man so identified himself 
with his sin that he became inseparable from 
it. The man guilty of them was put outside 
the camp, his communication was cut with the 
saved community of Israel. He was committed 
to the outer darkness. There remained only 
punishment and death. The punishment was 
expulsion from the covenant, and so from life. 
And as there is little about immortality in 
the Old Testament, it was death for good and 
all. But in the Cross of Christ there is no 
sin excluded from atonement. I know of 
course what you are thinking about---the sin 
against the Holy Ghost. That is far too large 
a subject to enter on. I can only say that 
I am not keeping it out of my survey. 
And I repeat, there is no sin excluded from 
atonement. Death as punishment of sin was 
absorbed in Christ's sacrifice. Such was its 
atoning work that the judgment due to all 
mankind was absorbed, and the sin of sins 
now was fixed refusal of that Grace. The 
Cross bought up all other debts, so to say. 

\begin{center}
\S
\end{center}

To return to my old point. The objection to 
speaking of Christ's death as penalty is twofold. 
God could not punish One with whom 
\marginpar{157}
He was always well pleased. Consequently 
Christ could not suffer punishment in the true 
sense of the word without having a guilty 
conscience. If the bearing of punishment were 
the whole of Christ's work, there was something 
in that way which He did not and could not do---He 
could not bear the penalty of remorse. But 
the whole of His work, was not the bearing of 
punishment; it was not the acceptance of suffering. 
It was the recognition and justification of 
it, the "homologation" of God's judgment and 
God's holiness in it. 

The death and suffering of Christ was something 
very much more than suffering---it was 
atoning action. At various stages in the history 
of the Church---not the Roman Catholic Church 
only but Protestantism also---exaggerated stress 
has been laid upon the sufferings of Christ. 
But it is not a case of what He suffered, but 
what He did. Christ's suffering was so divine 
a thing because He freely transmuted it into a 
great act. It was suffering accepted and transfigured 
by holy obedience under the conditions 
of curse and blight which sin had brought upon 
man according to the holiness of God. The 
suffering was a sacrifice to God's holiness. In 
so far it was penalty. But the atoning thing 
was not its amount or acuteness, but its 
obedience, its sanctity. 


\marginpar{158}
These pathetic ways of thinking about Christ 
regard Him too much as a mere individual 
before God. They do not satisfy if Christ's 
relation with man was a racial one and He 
represented Humanity. Especially they do not 
hold good if that relationship was no mere 
blood relationship, natural relationship, but a 
supernatural relationship---blood relationship 
only in the mystic Christian sense. We are 
blood relations of Christ, but not in the natural 
sense of that term, only in the supernatural 
sense, as those who are related to Him in His 
blood, in His death, and in His Spirit. The 
value of Christ's unity and sympathy with us 
was not simply that He was continuous with 
the race at its head. It was not a relation 
of \textit{identity}. The race was not prolonged into 
Him. The value consists in that life-act of \textit{self-identification} 
by which Christ the eternal Son of 
God became man. We hear much about Christ's 
essential identity with the human race. That is 
not true in the sense in which other great men, 
like Shakespeare, for instance, were identical 
with the human race, gathering up in consummation 
its natural genius. Christ's identity was 
not natural or created identity, but the self-identification 
of the Creator. Everything turns 
upon this---whether Christ was a created being, 
\marginpar{159}
however grand, or whether He was of increate 
Godhead. 

\begin{center}
\S
\end{center}

As Head of the human race by this voluntary 
self-identification with it, Christ took the 
curse and judgment, which did not belong to 
Himself as sinless. And what He owned was 
not so much the depth of our misery as the depth 
of our guilt; and He did it sympathetically, 
by the moral sympathy possible only to the 
holy. Nor did He simply take the full measure 
of our guilt. His owning it means very much 
more than that His moral perceptions were so 
deep and piercing that He could measure our 
guilt as a bystander of acute moral penetration 
could. He carried it in His own moral experience 
as only divine sympathy could. And 
in dumb action He spread it out as it is 
before God. He felt sin and its horror as 
only the holy could, as God did. We learn 
in our measure to do that when we escape 
from the indifference of our egotism and 
come under His Cross and near His heart; 
we learn to do as Christ did as we enter into 
living union with Christ. And we then rise 
above purity---for purity is only shamed by sin---we 
rise to holiness, which is burdened with sin 
and all its load. How much more than pure 
\marginpar{160}
Christ was! How much fuller of meaning is 
such a word as "holy" or "holiness" than either 
"pure" or "purity." Purity is shamed by human 
sin. Holiness carries it as a load, and carries it 
to its destruction. In the great desertion Christ 
could not feel Himself a sinner whom God rejects. 
For the sinner cannot carry sin; he 
collapses under it. Christ felt Himself treated 
as the sin which God recognises and repels by 
His very holiness. It covered and hid Him from 
God. He was made sin (not sinful, as I say). 
The holiness of God becomes our salvation not 
by slackness of demand but by completeness 
of judgment; not because He relaxes His 
demand, not because He spends less condemnation 
on sin, lets us off or lets sin off, or lets 
Christ off ("spared not"); but because in Christ 
judgment becomes finished and final, because 
none but a holy Christ could spread sin out in 
all its sinfulness for thorough judgment. I 
have a way of putting it which startles some of 
my friends. The last judgment is past. It took 
place on Christ's Cross. What we talk about as 
the last judgment is simply the working out of 
Christ's Cross in detail. The final judgment, 
the absolute judgment, the crucial judgment 
for the race took place in principle on the 
Cross of Christ. Sin has been judged finally 
\marginpar{161}
there. All judgment is given to the Son in 
virtue of His Cross. All other debts are bought 
up there. 

\begin{center}
\S
\end{center}

It is not simply that in the Cross of Christ 
all punishment was shown to be corrective. A 
favourite theme on the part of many of those 
who challenge the apostolic position about the 
death of Christ is that it was only the crowning 
exposition of the great principle that all punishment 
is really corrective and educative. We 
cannot say that. There is plenty of punishment 
that hardens and hardens. That is why we are 
obliged to leave such questions as universal 
restoration unsolved. Even when we recognise 
the absolute power of God's salvation, we also 
recognise that it is in the power of the human 
soul to harden itself until it become shrunk into 
such a tough and irreducible mass as it seems 
the very grace of God could do nothing with. 
Certainly there are people here, in this life, who 
become so tough in their sin that the grace of 
God is in vain. And I am not sure that among 
those who are toughest are not some who are 
much comforted by their religion. You can do 
something with a hardened sinner. He can be 
broken to pieces. But I do not know what 
you can do with a viscous saint, with those 
\marginpar{162}
who are wrapped in the wool, soaked in the 
comfort of their religion, and tanned to leather, 
soft and tough as a glove, by its bitterest baptisms. 
I once used an expression of these people 
which was somewhat criticised. I called them 
"moral tabbies." Is there anything more comfortable, 
and selfish, and hopeless than a really 
accomplished tabby? When religion becomes 
perverted to be a means of mere comfort and 
dense self-satisfaction, it becomes an integument 
so tough that even the grace of God 
cannot get through it, or a substance so flaccid 
that it cannot be handled. 

\begin{center}
\S
\end{center}

I find it convenient, you observe, to distinguish 
between punishment and penalty. A 
man who loses his life in the fire-damp, where 
he is looking for the victims of an accident, 
pays the penalty of sacrifice, but he does not 
receive its punishment. And I think it useful 
to speak of Christ as taking the penalty of 
sin, while I refuse to speak of His taking 
its punishment. I would avoid every word 
that would suggest that He was punished 
in connection with His salvation. It robs 
the whole act of ethical value to say so. 
Penalty is made to honour God in the Cross of 
\marginpar{163}
Christ, and thus it becomes a blessing to us. 
Not that our punishment is turned to good 
account in its subjective results upon us, but 
that Christ's judgment has objective value to 
the honour of God's holiness. He turned the 
penalty He endured into sacrifice He offered. 
And the sacrifice He offered was the judgment 
He accepted. His passive suffering became 
active obedience, and obedience to a holy doom. 
He did not steel His face to the suffering He 
had to endure, as though it were a fate to 
which He had to set His teeth and go through 
it in a stoic way. He never regarded it as a 
mere infliction. For Him, whoever inflicted it, it 
was the holiest thing in all the world---it was 
the will and judgment of God. All the Old 
Testament told Him that the Kingdom of God 
could never come without the prior judgment 
of God; and He was prepared to force that 
judgment in His impatience for the Kingdom.\footnote{
See Schweitzer's very remarkable "Quest of the 
Historical Jesus" (A. and C. Black)---the last two chapters---where 
a dogmatic and atoning motive in Jesus is declared by an advanced critic to have been the explanation 
of His death.} 
He answered the judgment of God with a grand 
affirmative act. The willing acceptance of final 
judgment was for Jesus the means presented by 
God for effecting human reconciliation and the 
\marginpar{164}
Divine Kingdom. The essence of all sacrifice, 
which is self-surrender to God, was lifted out of 
the Old Testament garb of symbolism, and was 
made a moral reality in Christ's holy obedience. 
In the Old Testament we have the lamb and the 
various other things brought for offering; but 
where did their essential value lie? In the 
obedience of the offerer; in the fact that those 
institutions were given and prescribed by holy 
God, however their details were due to man. And 
the presentation of the victim was valuable, not 
because of anything in the victim, but because 
of the obedience and surrender of the will with 
which the offerer presented it. This is the bearing 
of sin---the holy bearing of its judgment. 
This is the taking of sin away---the acknowledgment 
of judgment as holy, wise, and good, and 
its conversion into blessing; the absorption 
and conversion of judgment into confession and 
praise, the removal of that guilt which stood 
between God and man's reconciliation the 
robbing sin of its power to prevent communion 
with God. 

I should, therefore, express the difference 
between the old view and the new by saying 
that one emphasises substitutionary expiation 
and the other emphasises solidary reparation, 
consisting of due acknowledgment of God's 
\marginpar{165}
holiness, and the honouring of that and not of 
His honour. 

\begin{center}
\S
\end{center}

Now let me pass as I close to-day to two or 
three points I want specially to emphasise. 

There is one quotation which I wanted to 
make at a particular point and did not. The 
Reformers are still, on the whole, the masters of 
the great verities of experience in connection 
with the work of Christ. They had an amazing 
insight into the morbid psychology of the conscience. 
They did understand what sin meant, 
and they said this---the sinner, beginning with 
indifference, must keep flying from God until he 
actually hate God as a persecutor, unless he 
grasp the pursuit as God's mercy. Indifference 
could not stop at indifference, but goes on 
through aversion to hate. Even if a man die 
indifferent in this life, he comes into circumstances 
where he ceases to be indifferent. If we 
believe about a future at all, it will be impossible 
for an indifferent man to remain indifferent 
when he has passed on there. Indifference is 
an unstable position. It changes either upward 
or downward---downward into antagonism, into 
deadly hate against God, something Satanic; or 
upwards it passes into acceptance of God's mercy 
by faith, and all its blossom and fruit, its joy 
\marginpar{166}
and peace in the Holy Ghost. The Reformers 
were perfectly right. It is only our dull experience 
and preoccupied vision which prevent 
us seeing that it is so. 

\begin{center}
\S
\end{center}

Then I should like to call attention to this 
value in such a cross. It is only the judgment 
sacrifice of the Son of God that assures the 
sinner of the deep changelessness of grace. 
Forgiving is not forgetting. Popular theology 
too often tends to pacify us by reducing the 
offence. But the Reformers put the matter quite 
otherwise in saying that a justifying faith only 
goes with a full sense of guilt. You cannot get 
a full, justifying faith without a full sense and 
confession of guilt. We always have mistrust 
in the background of our own self-extenuations. 
When conscience begins to work and you begin 
to extenuate, when you try your hand earnestly 
at justifying yourself to yourself, you have some 
idea of how much more vast must be God's 
justification of you before Himself. You cannot 
cease to ask what charge conscience has 
against you. Then you magnify that to God's 
charge. If your heart condemn you, His condemnation 
is greater than that of your condemning 
heart. Do you consider His conscience? 
\marginpar{167}
His conscience has to be pacified as well as His 
heart indulged. And if His conscience be not 
met, ours is not sure. Has His conscience been 
met? Conscience has always mistrust in the 
background if grace is mere remission. Mere 
remission of sin does not satisfy even us. If 
conscience witnesses, against our extenuations, 
to the holy majesty of moral claim, is it to 
be less severe and less changeless than the 
claim of God Himself? Conscience has in trust 
God's law and its majesty, which must be made 
good, as mere remission does not make it. Suppose 
I transgress and I hear the message of 
grace, does it tell me the accusing, irrepressible 
demand of conscience, the haunting fear of 
judgment, was an illusion? It is doing me 
very ill service if it does. True, there is now 
no condemnation for faith; but if the message 
of grace ever teaches us that the judgment of 
conscience is exaggeration, is illusion, it is not 
the true grace of God. If a message of grace 
tell us there was and is no judgment any more, 
and that God has simply put judgment on 
one side and has not exercised it, that cannot 
be the true grace of God. Surely the grace of 
God cannot stultify our human conscience like 
that! So we are haunted by mistrust, unless 
conscience be drowned in a haze of heart. We 
\marginpar{168}
have always the feeling and fear that there is 
judgment to follow. How may I be sure that I 
may take the grace of God seriously and finally, 
how be sure that I have complete salvation, that 
I may entirely trust it through the worst my 
conscience may say? Only thus, that God is the 
Reconciler, that He reconciles in Christ's Cross 
that the judgment of sin was there for good and 
all. We are judged now by the Cross, and by 
the Cross we stand or fall. The great sin is not 
something we do, but it is refusing to make ourselves 
right with God in Christ's Cross. We are 
judged in the end by our relation to the Cross of 
Christ. It is the principle of our moral world. 
All judgment is committed to that Son. We 
stand before God at last according as we are 
owned by Christ. We are confessed by Him 
according to our confession of Him. Nemesis 
on us is hallowed as a part of the judgment 
on Him to whose death we are joined. There is 
no such thorough assertion of God's holy, loving 
law anywhere as there, where in the Cross it 
was given its own, and was perfected in judgment 
in Him who became a curse for us. His 
prayer for His murderers, or the closing sigh of 
victory in the midst of that judgment, vouches 
for ever to this, that it is the same holy will 
which judges man's wickedness and also loves us 
\marginpar{169}
and gives His Son for a propitiation for us. 
Only that holiness which is changeless in its 
judgment could be changeless also in grace. 
His grace was so little to be foiled that He 
graciously took His own judgment. Thus the 
severity of conscience becomes the certainty of 
salvation. 


But, changeless in judgment! Does that mean 
exacting the uttermost farthing of penalty, of 
suffering? Does it mean that in the hour of 
His death Christ suffered, compressed into one 
brief moment, all the pains of hell which the 
human race deserved. We cannot think about 
things in that way. God does not work by such 
equivalents. What is required is not an equivalent 
penalty, but an adequate confession of His 
holiness. Let us get rid of that materialist 
idea of equivalents. What Christ gave to God 
was not an equivalent penalty, but an adequate 
confession of God's holiness, rising from amid 
extreme conditions of sin. God's holiness, then, 
was so little to be mocked, that He actually 
took His own judgment to save it. He spared 
not His own Son---His own self. His severity 
of conscience becomes at the same moment our 
security of salvation. And the more conscience 
preaches the changelessness of the judging God, 
\marginpar{170}
the more it preaches the same changelessness in 
the grace of Christ. 

\begin{center}
\S
\end{center}

There is another consequence. Only the 
eternal Reconciler, the High Priest, can 
guarantee us our full redemption. "Take, my 
soul, thy full salvation." You cannot do it 
except you do it in such a Cross. It is not 
enough to have in the Cross a great demonstration 
of God's love, a forgiveness of the 
past which leaves us to fend for ourselves in 
the future. Is my moral power so great after 
all, then, that, supposing I believe past things 
were settled in Christ's Cross, I may now feel 
I can run in my own strength? Can I be 
perfectly confident about meeting temptation? 
Nay, we must depend daily upon the continued 
energy of the crucified and risen One. We 
must depend daily upon the action of that 
same Christ whose action culminated there 
but did not end there. His death is as organic 
with His heavenly life as it was with His 
earthly. What is the meaning of His perpetual 
intercession if it does not mean that---the 
exhaustless energy of His saving act? It is 
by His work from heaven that we appropriate 
His work upon earth. He guarantees our perfection 
as well as our redemption. 

\marginpar{171}
\begin{center}
\S
\end{center}

The last step. It is only the atoning reconciliation 
of a whole world that guarantees the 
final perfecting of that world by its Creator. 
How do we know that creation is going to 
be perfected? How do we know that it is 
to be to the glory of God who made it and 
called it good? How do we know the world 
will not be a failure for God with all but the 
group of people saved in an ark of some kind? 
We only know because we believe in the 
reconciliation of the whole world in Christ's 
Cross. There is a great deal of pessimism 
to-day, much doubt as to whether perfection 
really remains for the whole world; and you 
find people in the burdened West drawn to 
the Buddhistic idea of the human soul's extinction. 
Some Christians content themselves 
with individual salvation out of a world which 
is left in the lurch, or they are satisfied with 
personal union with Christ securing their 
own future. But the gospel deals with the 
world of men as a whole. It argues the restoration 
of all things, a new heaven and a 
new earth. It intends the regeneration of 
human society as a whole. Christ is the 
Saviour of the world, who was also the agent 
of its creation. The Creator has not let His 
\marginpar{172}
world get out of hand for good and all. That 
is to say, our faith is social and communal in 
its nature. We must have a social gospel. And 
this you cannot get upon the basis of mere 
individual or sectional salvation. You can only 
have a social gospel upon one basis, namely, that 
Christ saved, reconciled the whole world as a 
unity, the whole of society and history. The 
Object of our faith, Jesus Christ, is what our 
fathers used to call a federal Person, a federal 
Saviour, in a federal act. All humanity is in 
Him and in His act. It is quite true every 
man must believe for himself, but no man 
can believe by himself or unto himself. The 
Christian faith fades away if it is not nourished 
and built up in a community, in a Church. 
And the Church fades away if it do not hold 
this faith in trust for the whole world. Each 
one of us is saved only by the act and by the 
Person that saved the whole world. 

\begin{center}
\textsc{Note to p. 153.} 
\end{center}

In some cases it seems due to congenital defect. An 
able member of the "New Theology" group was conversing 
with my informant, who said, "For me all 
Christianity turns on the unspeakable mercy of God to my 
soul in the Cross of Christ." The reply was blankly, "I 
do not understand it." 



\chapter{THE PRECISE PROBLEM TO-DAY} 
\chaptermark{The Precise Problem To-day}
%\marginpar{175}

%VI 
%THE PRECISE PROBLEM TO-DAY* 
% This footnote is actually placed at the end of the chapter title.
\textsc{There}\footnote{This chapter owes much to Kirn, \textit{Herzog}, xx., Art. "Vers\"{o}hnung."} 
is a popular impression about both 
philosophy and theology that the history 
of their problems is very sterile; that it is 
not a long development, carrying the discussion 
on with growing insight from age to age, and 
passing from thinker to thinker with growing 
depth, but rather a scene in which each newcomer 
demolishes the work of his predecessor 
in order to put in its place some theory doomed 
in turn to the same fruitless fate. Truly, as 
Hegel says, if that were so with philosophy, 
its history would become one of the saddest 
and sorriest things, and it would have no 
right to go on. And if it were so with theology, 
we should not only be distressed for Humanity, 
but we should be sceptical about the Holy Spirit 
\marginpar{176} 
in the Church. It could be the Church of 
no Holy Spirit if those who translated its 
life into thought did not offer to posterity a 
spectacle higher than dragons that tare each 
other in the slime, or lions that bit and 
devoured one another. 

As a matter of truth and fact, both philosophy 
and theology have not only a chronicle but a 
history. They register the highest spiritual 
evolution of the race. The wave behind rolls on 
the wave before. The past is not devoured but 
lives on, and comes to itself in the future. The 
new arrivals do not consume their predecessors, 
and do not ignore them; they interpret them 
and carry them forwards. They take their 
fertile place in the great organic movement. 
They modulate what is behind upwards into 
what is to come. They correct the past and 
enrich it; and they hand on their corrected past 
to be a foundation for the workers yet to be. 

The amateur, or the self-taught, therefore is 
at a great disadvantage. He does not take up 
the problem where the scientific succession laid 
it down. He does not come in where his great 
co-workers left off. He must start \textit{ab ovo}. He 
must do over again for himself what they have 
conspired to do better. He risks "being a fool 
at first hand." He wastes himself criticising 
\marginpar{177}
what has long been dropped, and slaying the 
long-time slain. He throws away effort in 
establishing what the competent have agreed 
to accept. And he misses the right points to 
attack or to strengthen, because he has not surveyed 
the ground. Every now and then one 
meets the capable amateur, whose misfortune 
it has been to have no schooling in the scientific 
history or method of the subject, who applied 
to it a shrewd mother-wit or an earnest but 
uninstructed conscience, and who perhaps publishes 
a theory of Incarnation or Atonement 
which, for all its hints and glimpses of truth, 
makes no real contribution either to the history 
or the merits of the case. This is the misfortune 
of the self-taught who goes straight 
to his Bible for the materials of his theology, 
and ignores most of the treatment the problem 
has received from the greatest minds in the 
history of the Church or the soul. The Bible 
is enough for our saving faith, but it is not 
enough for our scientific theology. 

To make the most therefore of godly and 
able men, who would else be wasted more or 
less, it is well that we should teach them at 
the outset to take up the question where they 
find it, to begin where their best predecessors 
left off, to work upon results, and to carry 
\marginpar{178}
forward the subject in the train of its evolution 
from the great and growing past. Let us couple 
up with the past, and repay its gifts by fructifying 
them for the future. Let us call in our 
thought, and concentrate it upon the precise 
question which previous thinkers have left us 
to solve. 

\begin{center}
\S
\end{center}

There is, thus, another thing we have to do. 
We have to try to find a due place for those views 
which, however one-sided, yet do compel attention 
to aspects that the Church from time to time 
ignores. We have to meet, satisfy, and exceed 
such views. Much, for instance, has been done in 
the lifetime of most of us to correct and extend 
those views of Christ's work which were so 
rigidly objective that they became external. It 
has been urged that the Church long thought 
too much of Christ's action on God and not 
enough of His action on man. And what is 
called the moral theory of the Atonement has 
therefore been pressed upon us, to replace the 
ultra-objective and satisfactionary view. And 
the pressure has often been so hard that an objective 
theory has been entirely denied as immoral, 
and denied sometimes with a scorn unjustified 
by either the mental acumen or moral dignity 
of the critic. 

\marginpar{179}
But in spite of this over-pressure, and the 
occasional insolence that goes with ignorance, 
it remains our duty to find a proper place in 
our view of the whole great subject for that 
effect of Christ upon men which has meant so 
much for the sanctity of the Church. We have 
to meet, satisfy, and transcend those pleas which 
have been called into existence to redress the 
balance of theological neglect, and to fill out 
that which was behind in our grasp of the 
manifold work. Especially we have to adjust 
our theology of Christ's work to those who 
observe that the repentance of the guilty is an 
essential condition of forgiveness, and who go 
on to ask how we can speak of a finished 
reconciliation or atonement by a sinless Christ, 
who could not possibly present before God a 
repentance of that kind. 

\begin{center}
\S
\end{center}

There are certain results which, it may be 
said, we have definitely reached in correction 
of what has long been known as the popular 
view of Christ's death and work. They are 
modern, and they owe much to Schleiermacher, 
Ritschl, McLeod Campbell, Maurice and others; 
but they have also been shown to be scriptural, 
by a new, objective and scientific investigation 
\marginpar{180}
of what the Bible has to say on the subject. 
When we have brought the long history of 
the question up to date, balanced the books, 
and taken account of the general agreement 
on the modern side, we can then go on to ask 
where exactly the question now stands. 

The modifications on which the best authorities 
are substantially at one we have seen to be 
such as these:--- 

1. Reconciliation is not the result of a change 
in God from wrath to love. It flows from the 
changeless will of a loving God. No other view 
could make the reconciliation sure. If God 
changed \textit{to} it, He might change \textit{from} it. And 
the sheet-anchor of the soul for Eternity would 
then have gone by the board. Forgiveness 
arose at no point in time. Grace was there 
before even creation. It abounded before sin 
did. The holiness which makes sin sin, is one 
with the necessity to destroy sin in gracious 
love. 

2. Reconciliation rests on Christ's person, 
and it is effected by His entire work, doing, and 
suffering. This work does three things. (1) It 
reveals and puts into historic action the changeless 
grace of God. (2) It reveals and establishes 
His holiness, and therein also the sinfulness of 
sin. And (3) it exhibits a Humanity in perfect 
\marginpar{181}
tune with that will of God. And it does more 
than exhibit these things---it \textit{sets them up}, grace, 
holiness, and the new Humanity in its Head. 

3. This reconciling and redeeming work of 
Christ culminates in His suffering unto death, 
which is indeed more of an act than an experience. 
Here, in the Cross, is the summit of 
His revelation of grace, of sin, and of Humanity. 
And the central feature of this threefold revelation 
in the Cross is the holiness of God's love. 
It is this holiness that deepens error into sin, 
sin into guilt, and guilt into repentance; without 
which any sense of forgiveness would be 
but an anodyne and not a grace, a self-flattering 
unction to the soul and not the peace 
of God. 

4. In this relation to God's holiness and its 
satisfaction, nobody now thinks of the transfer 
of our punishment to Christ in its entirety---including 
the worst pains of hell in a sense 
of guilt. Christ experienced the world's hate, 
and the curse of the Law in the sense of the 
suffering entailed on man by sin; but a direct 
infliction of men's total deserts upon Him 
by God is unthinkable. His penalty was not 
punishment, because it was dissociated from 
the sense of desert. Whatever we mean by 
atonement must be interpreted in that sense. 
\marginpar{182}
And judgment is a much better word than 
either penalty or punishment. 

5. What we have in Christ's work is not 
the mere pre-requisite or condition of reconciliation, but the actual and final effecting of it 
in principle. He was not making it possible, 
He was doing it. We are spiritually in a reconciled 
world, we are not merely in a world in 
process of empirical reconciliation. Our experience 
of religion is experience of a thing done 
once for all, for ever, and for the world. That 
is, it is more than even experience, it is a faith. 
The same act as put God's forgiveness on a 
moral foundation also revolutionised Humanity. 
Hence we are not disposed to speak of substitution\footnote{
Because substitution does not take account of the 
moral results on the soul, and for a full account of the 
cause we must include all the effects. To do justice 
to the whole of Christ's work we must include the 
Church, and in justification include sanctification.} 
so much as of representation. But it 
is representation by One who creates by His act 
the Humanity He represents, and does not 
merely sponsor it. \textit{The same act} as disburdens us 
of guilt commits us to a new life. Our Saviour 
in His salvation is not only our comfort but 
our power; not merely our rescuer but 
our new life. His work is in the same act 
reclamation as well as rescue. 

\marginpar{183}
6. Another thing may perhaps be taken as 
recognised in some form by the main line of 
judicious advance in our subject. The work of 
Christ was moral and not official. It was the 
energy and victory of His own moral personality, 
and not simply the filling of a position, the 
discharge of an office He held. His victory was 
not due to His rank, but to His will and 
conscience. It lay in His faithfulness to the 
uttermost amid temptations morally real and 
psychologically relevant to what He was. It 
was a work that drew on His whole personality, 
and was built into the nature of that personality 
as a moral necessity of it. What He did He 
did not do simply in the room and stead of 
others, He did it as a necessity of His own 
person also---though its effect for them was 
not what it was for Him. He fulfilled an 
obligation under which His own personality 
lay; He did not simply pay the debts of other 
people. He fulfilled a personal vocation. 

And His faithfulness was not only to a vocation. 
It was to a special vocation, that of a 
Redeemer, not merely a saint. The immediate 
source of His suffering was not the sight of 
human sin, and it was not a general holiness 
in Him. It was not the quivering of the saint's 
purity at the touch of evil. But it was the 
\marginpar{184}
suffering of One who touched sin \textit{as the Redeemer}. 
He would not have suffered for sin 
as He did, had He not faced it as its destroyer. 
Not only was this His vocation as a moral 
hero, but His special vocation as Saviour. It 
was the work of a moral personality at the 
heart of the race, of One who concentrated on a 
special yet universal task that of Redemption. 

His perfection was not that of a paragon, one 
who could do better what every soul and genius 
of the race could do well. He was not all the 
powers and excellencies of mankind rolled into 
one superman. But His perfection was that of 
the race's Redeemer. It was interior to all other 
powers and achievements. It was central both 
for God and man. He made man's centre and 
God's coincide. He took mankind at its centre 
and laid it on the centre of God. His identification 
with man was not extensive but intensive, 
it was not discursive and parallel, so to say. 
It was morally central and creative. He was 
not Humanity on its divine side; He was its 
new life from the inside. The problem He had 
to solve was the supreme and central moral 
problem of guilt; and the work could only be 
done by the native action of a personality moral 
in its nature and methods, moral to the pitch 
of the Holy. 

\marginpar{185}
It is an immense gain thus to construe 
Christ's work as that of a moral personality 
instead of a heavenly functionary. It brings 
it into line with the modern mind and into 
organic union with the moral problem of 
the race. It enables us to realise that every 
step of the moral victory in His life was a step 
also in the Redemption of the whole human 
conscience. And we grasp with new power the 
idea that His crowning victory of the Cross was 
the victory in principle of the whole race in 
Him---that Justification is really one with Reconciliation, 
and what He did before God contained 
all He was to do on man. It makes 
possible for us what my last lecture will attempt 
to indicate a---unitary view of His whole work 
and person. 

\begin{center}
\S
\end{center}

7. After these great modifications and gains, 
we have cleared the ground to ask with some 
exactness just where the question at the moment 
stands. What was the divinest thing, the atoning, 
satisfying thing, the thing offered to God, in 
Christ; the thing, therefore, final and precious in 
what He did? The permanent thing in Christianity 
must be that which gives it its chief 
value to God. We are now beyond the crude 
alternative that so easily besets us, "Did Christ's 
\marginpar{186}
work bear upon God or on man?" Neither alone 
would be a true Reconciliation. Neither Orthodoxy 
nor Socinianism has it. But we have to ask 
this: "Can we combine the truth in each alternative? 
Can we reach the value of Christ's 
saving work to God (\textit{i.e.}, its true and final value) 
if we exclude its effect within man? Must we 
not take that in? \textit{Nihil in effectu quod non prius 
in causa}. Must we not include the effect to get 
the full value of the cause, and give a full 
account of it?" 

Now, let us own at the outset that the first, 
things we must be sure about are the objective 
reality of our religion, its finality, and its initiative 
in God's free grace independent of act or 
desert of ours. But if we start there, it looks 
as if we were shut up to the first of the crude 
alternatives, as if the idea of Christ's work as 
acting on God only gave the best effect to these 
conditions. It looks as if the old theory alone 
guaranteed a salvation finished on the Cross, one 
wholly God's in His grace, one that ensures a 
full and objective release of the conscience. 
These things are not secured by what we do, but 
by Christ's work on the Cross. Moreover, that 
work was done for the whole of mankind, and 
was complete even for those who as yet make no 
response. And, besides, that first alternative is 
\marginpar{187}
a view that seems to have the letter of Scripture 
with it. It does look as if we could not have full 
security except by trust of an objective something, 
done over our heads, and complete without 
any reference to our response or our 
despite. 

But the difficulties begin when we ask what 
the objective something was. How describe it? 
For that purpose the old doctrine used juridical 
forms. But these are not large enough for 
the dimensions of a modern world, or for its 
deepened ethical insight. How exactly could the 
obedience of Christ stand for the obedience of 
all? It was the fulfilment of His own personal 
vocation; how does it stand for the obedience 
of every other person? Or how does the suffering 
of Christ restore the moral order, especially 
one He never broke? If you treat it as punishment, 
that punishment alone does not restore 
the moral order. And, if we say He did not do 
that, He did not restore a moral order, so much 
as acknowledge and confess the holiness of God 
in His judgment, is not the value of that recognition 
still greatly impaired by the fact that it 
is not made by the guilty but the Guiltless, who 
is not directly affected by the connection between 
sin and suffering. A finished religion 
would then be set up without the main thing---the 
\marginpar{188}
acknowledgment by the guilty. That acknowledgment, 
that repentance, would then be 
outside the complete act, and would be at best 
but a sequel of it; whereas we ought to give 
a real place in a complete work of Reconciliation 
to our repentance (which some extremists say is 
all that is required), or to Christ's moral action 
on us. Do we not need to include in some way 
the effect in the cause, in order to give the cause 
its full and final value, \textit{i.e.}, its value to God. 
The thing of price done by Christ for God, must 
it not already include the thing done upon men?. 
Does not Christ's confession of God's holiness 
include man's confession of his sin? 

\begin{center}
\S
\end{center}

Let us return to that idea of the moral order 
which is at the bottom of this objective theory. 
We ask whether the moral order is what the 
Bible means by the idea of the righteousness 
of God. The righteousness of God is not only 
holy but gracious, not only regulative and retributory, 
but also forgiving and restoring. It 
seems, indeed, in the Gospels to need no other 
condition of forgiveness than repentance. This 
is so; and it is all very well, we have seen, for 
individual cases. But we have to deal, as Christ 
at last had to deal, with the forgiveness of a 
\marginpar{189}
world, the pardon of solidary sin. And we 
need to be sure, as Christ alone with His insight 
could be sure, that the repentance is true and 
deep. There it is that we are carried into 
questions which the Cross alone can answer. 
How shall I know how much repentance is deep 
enough? Where find a repentance wide enough 
to cover the sin of a guilty world? Could Christ 
offer that? No; directly, He could not. He 
could not offer it as a pathos, a personal experience, 
for He had no guilt. But, then, guilt 
is much more than a sense of guilt. And the 
essence of repentance is not its intensity or 
passion but the thing confessed. It is therefore 
the holiness more even than the sin that 
holiness makes so sinful. It is the due and 
understanding acknowledgment of the holiness 
offended. And this only a sinless Christ could 
really do, who was also sympathetic enough 
with men to do it from their side. And only 
the sinless could realise what sin meant for 
God. 

Farther, this acknowledgment is not simply 
verbal, nor simply a matter of profound moral 
conviction and admission, but it must be a 
practical confession, as practical as the sin. It 
must place itself as if it were active sin under 
the reaction of the Divine holiness; it must be 
\marginpar{190}
made sin. That is, it must accept judgment as 
the only adequate acknowledgment of the holy 
God in a sinful world; it must allow His holy 
law to assert itself in the Saviour's person in 
the form forced on the sinner's Friend. He bore 
this curse as God's judgment, praised it, hallowed 
it, absorbed it; and His resurrection 
showed that He exhausted it. 

But would His acceptance of judgment for us 
be possible, would it stand to our good, would 
it be of value in God's sight for us, if He 
were not in moral solidarity with us? How 
could it? What God sought was nothing so 
pagan as a mere victim outside our conscience 
and over our heads. It was a Confessor, a 
Priest, one taken from among men. But then 
this moral solidarity is the very thing that 
also gives, and must give, Him His mighty and 
revolutionary power on us. What makes it 
possible for Him to be a Divine victim or a 
Divine priest for us also makes Him a new 
Creator in us. His offering of a holy obedience 
to God's judgment is therefore valuable to God 
for us just because of that moral solidarity 
with us which also makes Him such a moral 
power upon us and in us. His creative regenerative 
action on us is a part of that same 
moral solidarity which also makes His acceptance
\marginpar{191}
of judgment stand to our good, and His 
confession of God's holiness to be the ground 
of ours. The same stroke on the one Christ 
went upward to God's heart and downward 
to ours. 

\begin{center}
\S
\end{center}

Is this not clear? Christ could make no due 
confession of holiness for us in judgment if 
He were outside Humanity, if He were a third 
party satisfying God over our head. The acknowledgment 
would not be really from the 
side of the culprit, certainly not from his interior, 
his conscience. The judgment would not 
really be the judgment of \textit{our} sin, which would 
therefore be still due. To be of final value 
the atoning judgment must be also within the 
conscience of the guilty. But how is the judgment, 
the self-condemnation, the confession within 
our guilty conscience to be offered to God 
as an ingredient of Christ's reconciling work 
and not its mere sequel? It is not yet there. 
Or else it is nothing worth offering by way of 
atonement when it is there. Is there any way of 
offering our self-condemnation as a meritorious 
contribution to forgiveness? Can it be included 
in the Divine ground of forgiveness in a guiltless 
Christ? Repentance is certainly a condition 
of forgiveness. But Christ could not repent. 
\marginpar{192}
How then could He perfectly meet the conditions 
of salvation? The answer is that our 
repentance was latent in that holiness of His 
which alone could and must create it, as the 
effect is really part of the cause---that part of 
the cause which is prolonged in a polar unity 
into the sequential conditions of time. 

Not only, generally, is there an organic moral 
connection and a spiritual solidarity between 
Christ and us, but also, more particularly, there 
is such a moral effect on Humanity included in 
the work of Christ, who causes it, that that antedated 
action on us, judging, melting, changing 
us, is also part of His offering to God. He comes 
bringing His sheaves with Him. In presenting 
Himself He offers implicitly and proleptically 
the new Humanity His holy work creates. The 
judgment we brought on Him becomes our 
worst judgment when we arraign ourselves; 
and it makes it so impossible for us to forgive 
ourselves that we are driven to accept forgiveness 
from the hands of the very love which 
our sins doomed to a curse. 

\begin{center}
\S
\end{center}

What Christ offers to God is, therefore, not 
simply an objective satisfaction outside His 
revolutionary effect on the soul of man in the 
\marginpar{193}
way of faith, repentance, and our whole sanctification. 
As the very judgment He bore for 
us is relevant to our sin by His moral solidarity 
with us, so the value of His work to God includes 
also that value which it has in acting 
on us through that same solidarity, and in presenting 
us to God as the men it makes us to be. 
He represents before God not a natural Humanity 
that produces Him as its spiritual classic, but 
the new penitent Humanity that His influence 
creates. He calls things that are not yet as 
though they were. In Him a goodness of ours 
that is not yet, rising from its antenatal spring, 
brings to naught the sin that is. There was 
presented to God, in Christ's holiness, also that 
repentance in us which it alone has power to 
create. He stretches a hand through time and 
seizes the far-off interest of our tears. The 
faith which He alone has power to wake is 
already offered to God in the offering of 
all His powers and of His finished work. 
That obedience of ours which Christ alone 
is able to create, is already set out in Him 
before God, implicit in that mighty and subduing 
holiness of His in which God is always 
well-pleased. All His obedience and holiness 
is not only fair and beloved of God, but it is 
also great with the penitent holiness of the race 
\marginpar{194}
He sanctifies. Our faith is already present in His 
oblation. Our sanctification is already presented 
in our justification. Our repentance is already 
acting in His confession. The effect of His 
Cross is to draw us into a repentance which 
is a dying with Him, and therefore a part of 
the offering in His death; and then it raises 
us in newness of life to a fellowship of His 
resurrection. 

\begin{center}
\S
\end{center}

He is thus not only the pledge to us of God's 
love but the pledge to God of our sure response 
to it in a total change of will and life. We see 
now how organic, how central to Christ's gospel 
of Atonement is Paul's idea of dying and rising 
with Him, how vital to His work is this effect 
of it, this function of it. For such a process, 
such an experience, is not a mere moral sequel 
or echo of ours to the story of the Cross, it is 
no mere imitation or repetition of its moral 
greatness; nor is it a sensitive impression of 
its touching splendour. To die and rise with 
Christ does not belong to Christian ethic, to 
the method of Jesus, but it has a far deeper 
and more religious meaning. It is to be taken 
into His secret life. It is a mystic incorporation 
into Christ's death and resurrection as the 
standing act of spiritual existence. We are 
\marginpar{195}
baptized into His death, and not merely into 
dying like Him. We do not echo His resurrection, 
we share it. As His trophies we become 
part of Christ's offering to God; just as the 
captives in his procession were part of the 
victor's self-presentation to the divinity of 
Rome. God leadeth us in triumph in Christ 
(2 Cor. ii. 14). It is, indeed, for Christ's sake 
we are forgiven, but for the sake of a Christ 
who is the Creator of our repentance and not 
only the Proxy of our curse. And it is \textit{to} our 
faith the grace is given, yet not \textit{because of} our 
faith, which is no more perfect than our repentance. 
It is to nothing so poor as our faith or 
our repentance that new life is given, but only 
to Christ on His Cross, and to us for His sake 
who is the Creator and Fashioner of both. Our 
justification rests on this atoning creative Christ 
alone. And when the matter is so viewed, the 
objection some have to the phrase "for Christ's 
sake" should disappear. 

No martyrdom could do what the death of 
Christ does for faith. No martyrdom could 
offer God in advance the souls of a changed 
race. For no martyr as such is sure of the 
future. It is easier to forget all the martyrs 
than the Saviour; and their power fades with 
time, while His grows with the ages. With the 
\marginpar{196}
martyr's death we can link many admirable 
reflections, exhortations, and even inspirations. 
What it does not give us is the new and Eternal 
Life. It is not the consummation of God's saving 
purpose for the world. 

\chapter{THE THREEFOLD CORD}
\chaptermark{The Threefold Cord}

%\marginpar{199}
%VII 
%THE THREEFOLD CORD 

\textsc{There} are three great aspects of the work 
of Christ which have in turn held the 
attention of the Church, and come home with 
special force to its spiritual situation at a special 
time. These are--- 

\begin{enumerate}
\item Its triumphant aspect; 
\item Its satisfactionary aspect; 
\item Its regenerative aspect. 
\end{enumerate}

The first emphasises the finality of our Lord's 
victory over the evil power or devil; the second, 
the finality of His satisfaction, expiation, or 
atonement presented to the holy power of God; 
and the third the finality of His sanctifying or 
new-creative influence on the soul of man. The 
first marked the Early Church, the second the 
Medieval and Reformation Church, while the 
third marks the Modern Church. 

And if you fall back upon the New Testament, 
where all the subsequent development of the 
Church is in the germ, as a philosophy might be 
\marginpar{200}
packed in a phrase, you will find those three 
strands wonderfully and prophetically entwined 
in 1 Cor. i. 30, where it is said that Christ is 
made unto us (2) justification; (3) sanctification; 
and (1) redemption. The whole history of the 
doctrine in the Church may be viewed as the 
exegesis by time of this great text of the Spirit. 

Now, it is not meant that in the period 
specially marked by one of these aspects the 
other two were absent. In various of the 
medieval theologians you find all three. And 
it is a good test of the native aptitude of any 
theologian, and of his evangelical grasp, that he 
should find them all necessary to express the 
fullness of the vast work, and its adequacy to 
anything so great and manifold as the soul. 
But what we do not find in the classic theologians 
of the past is the co-ordination of the 
three aspects under one comprehensive idea, 
one organic principle, corresponding to the complete 
unity of Christ's person, who did the work. 
We do not find such a unitary view of the 
work as we should expect when we reflect that 
it was the work of a personality so complete 
as Christ, and so absolute as the God who acted 
in Christ. Yet we must strive after such a 
view, by the very nature of our faith. A mere 
composite or eclectic theology means a distracted 
\marginpar{201}
faith. A creed just nailed together means 
Churches that cannot draw together. We cannot, 
at least the Church cannot, rest healthily 
upon medley and mortised aspects of the one 
thing which connects our one soul with the 
one God in one moral world. We cannot rest 
in unresolved views of reconciliation. As the 
reconciliation comes to pervade our whole being, 
and as we answer it with heart and strength 
and mind, we become more and more impatient 
of fragmentary ways of understanding it. We 
crave, and we move, to see that the first aspect 
is the condition of the second, and the second of 
the third, and that they all condition each other 
in a living interaction. 

Now the object I have in view in this lecture 
is to press a former point as furnishing this 
unity---that the active and effective principle in 
the work of Christ was the perfect obedience 
of holy love which He offered amidst the conditions 
of sin, death, and judgment. The potent 
thing was not the suffering but the sanctity, 
and not the sympathetic confession of our sin so 
much as the practical confession of God's holiness. 
This principle (I hope to show) co-ordinates 
the various aspects which have been 
distorted by isolation. This one action of the 
holy Saviour's total person was, on its various 
\marginpar{202}
sides, the destruction of evil, the satisfaction of 
God, and the sanctification of men. And it is in 
this moral medium of holiness (if I may so say) 
that these three effects pass and play into each 
other with a spiritual interpenetration. 

Thus Christ's complete victory over the evil 
power or principle. His redemption (1), is the 
obverse of His regenerating and sanctifying effect 
on us (3). To deliver us from evil is not simply to 
take us out of hell, it is to take us into heaven. 
Christ does not simply pluck us out of the 
hands of Satan, He does so by giving us to God.
He does not simply release us from slavery, He 
commits us in the act to a positive liberty. He 
does not simply cancel the charge against us in 
court and bid us walk out of jail, He meets us 
at the prison-door and puts us in a new way of 
life. His forgiveness is not simply retrospective, 
it is, in the same act, the gift of eternal life. 
Our evil is overcome by good. We are won 
from sin by an act which at the same time 
makes us not simply innocent but holy. 

So also we must see that the third---our 
regenerate sanctification---is the condition of 
the second the complete satisfaction of God. 
The only complete satisfaction that can be made 
to a holy God from the sinful side is the sinner's 
restored obedience, his return to holiness. Now, 
\marginpar{203}
the cheap and superficial way of putting that is 
to say that penitent amendment is the only 
satisfaction we can give to a grieved God. But 
future amendment does no more than the duty 
of the future hour. And rivers of water from 
our eyes will not wash out the guilt of the past; 
nor will they undo the evil we have set afloat 
in souls far gone beyond our reach or control. 
Yet it remains true that nothing can atone to 
holiness but holiness. And it must be the 
holiness of the sinner. It must also be an 
obedience of the kind required by the whole 
situation, moral and spiritual. It must be the 
obedience not of improvement but of reconciliation, 
not of laborious amendment but of 
regenerated faith. But faith in what? Faith 
in One who alone contains in Himself a holy 
obedience so perfect as to meet the holiness 
of God on the scale of our sin; but One also 
who, by the same obedience, has the power 
to reproduce in man the kind of holiness which 
alone can please God after all that has come and 
gone. No suffering can atone. No pain can 
satisfy a holy God; no death, as death. Yet 
satisfied He must be; else the freedom of grace 
becomes but an arbitrary and non-holy thing, 
a thing instinctive to the divine nature instead 
of a victory of the divine will. Also consider 
\marginpar{204}
this: much of your difficulty in connection with 
satisfaction will yield if you keep in view that 
what we are concerned with is not the satisfaction 
of a demand but of a Person, not of a claim 
by God but of the heart and soul of God. I 
know it is easier to discuss and adjust statutory 
claims than to grasp the manifold action of 
a living and eternal Person. So I am afraid 
I must be very theological for a moment and 
tax you accordingly. The chief reason why so 
many hate theology is because it taxes; and 
there is nothing we shrink from like spiritual 
toil. But let the choice and earnest spirit 
consider this. 

The essence of holiness is God's perfect 
satisfaction, His perfect repose in eternal fullness. 
And the Christian plea is that this is 
Self-satisfaction, in the sublimest sense of the 
phrase. For us, mostly, the word has an ignoble 
sense. But that is only because what we meet 
most is an exclusive self-satisfaction, an individual 
self-sufficiency. But when we have 
an entirely inclusive self-satisfaction, an eternal 
and complete adequacy to Himself in the most 
critical situation, we have the whole native fullness 
of God blessed for ever, with men beneath 
the shadow of His wing. The perpetual act of 
holy God is a perpetual satisfaction or accord 
\marginpar{205}
between His nature and His will at every juncture, 
and a satisfaction from His own infinite 
holy resource---a Self-satisfaction. God is always 
the author of His own satisfaction: that is to say, 
His holiness is always equal to its own atonement. 
God in the Son is the perfect satisfaction 
and joy of God in the Father; and God holy 
in the sinful Cross is the perfect satisfaction of 
God the holy in the sinless heavens. Satisfaction 
there must be in God's own nature, whether 
under the conditions of perfect obedience in a 
harmonious world, or under those of obedience 
jarred and a world distraught. God has power 
to secure that the perfect holy obedience of 
heaven shall not be eternally destroyed by the 
disobedience of earth. He has power to satisfy 
Himself, and maintain His holiness infrangible, 
even in face of a world in arms. But satisfied He 
must be. For an unsatisfied God, a dissatisfied 
God, would be no God. He would but reflect the 
distraction of the world, and so succumb to it. 

But a holy God could be satisfied by neither 
pain nor death, but by holiness alone. The 
atoning thing is not obedient suffering but 
suffering obedience. He could be satisfied and 
rejoiced only by the hallowing of His name, 
by perfect and obedient answer to His holy 
heart from amid conditions of pain, death, and 
\marginpar{206}
judgment. Holy obedience alone, unto death, 
can satisfy the Holy Lord. 

\begin{center}
\S
\end{center}

Now as to this obedience mark two things. 

1. It includes (we saw) the idea that in obedience 
Christ accepted the judgment holiness 
must pass upon sin, and did so in a way that 
confessed it as holy from amidst the deepest 
experience of it, the experience not of a spectator 
but a victim. His obedience was not 
merely a fine, perfect, and mighty harmony 
of His own with God's blessed will; but it 
was the acceptance on man's behalf of that 
judgment which sin had entailed, and the confession 
on man's behalf in a tremendous act 
that the judgment was good and holy. For the 
holiness of God makes two demands: first, for 
an answering holiness in love, and second, for a 
judgment on those who do not answer but defy. 
And Christ met both, in one and the same act. 
He was judged as one who, being made sin, was 
never sinful, but absolutely well-pleasing to God. 

2. And the second point is this: The satisfactory 
obedience must be obedience from the race 
that rebelled. Its holiness must atone for its sin. 
But how can that possibly be? Can it be by mere 
amendment from us? Can we bring any amendment 
\marginpar{207}
to atone for the past and secure its 
remission? Could the race do it? Solidary in 
its sin by its moral unity, could the race earn a 
solidary salvation? Could you conceive of mankind 
as one vast sinful soul repenting with a 
like unity, turning like the prodigal, and deputing 
the most illustrious spiritual hero of its number 
to offer its repentance to God in Jesus Christ? If 
the supposition were possible, that might indeed 
be a certain welcome offering made \textit{to} God's 
holiness; but it would not be made \textit{by} it. It 
would be something beyond the resources of 
holiness, and God would not be the Saviour. 
He would accept more sacrifice than He had 
power to make. And it would make the action 
of Christ a power conferred on Him by self-saved 
man instead of inherent in Him from 
God. His commission would be but to God, not 
from God. And how should a sinful race offer 
from its own damaged resources what would 
satisfy the holiness of God? Or, if repentance 
could satisfy holiness, how are we to know how 
much, how deep, repentance would do it, and 
leave us sure it was done? 

\begin{center}
\S
\end{center}

The holiness that atones, though it return 
from the race that rebelled, must therefore be 
\marginpar{208}
the gift of the holiness atoned. For if holiness 
could be satisfied by anything outside itself it 
would not be absolutely holy. So if holiness can 
be satisfied with nothing but holiness it can only 
be with a holiness which itself creates. God 
alone can create in us the holiness that will please 
Him. And this He has done in Jesus Christ incarnate. 
But it is in Jesus Christ as the creator 
of man's holiness, not as the organ of it, as man's 
sanctifier, and not merely man's delegate. Christ 
is our reconciler because on the Cross He was our 
redeemer from sin's power into no mere independence 
or courage or safety, but into real holiness; 
because the same act that redeems us produces 
holiness, and presents us in this holiness to God 
and His communion. The holiness of Christ is 
the satisfying thing to God, yet not because of 
the beauty of holiness offered to His sight in the 
perfect character of Christ. We are not saved 
either by Christ's ethical character or our own, 
but by His person's creative power and work on 
us. Christ's holiness is the satisfying thing 
to God, because it is not only the means but also 
the anticipation of our holiness, because it 
carries all our future holiness latent in it and to 
God's eye patent; because in His saving act He 
is the creative power of which our new life is the 
product. It is not only that Christ conquered 
\marginpar{209}
for Himself and emerged with His soul for a 
prey, but, He being what He was, His victory 
contained ours. If He died all died. It was 
not only that all the sin of the world, pointed 
to its worst, could not make Him a sinner. 
It was that by all the holiness of eternity 
He had power to make the worst sinners 
saints. Of course, there is no way to sanctification 
but by deliverance from sin, by being "unsinned." But no sinful man can "unsin" himself, 
however he amend. 

It can only be done by the creation in him 
of a new life. It can only be done by the 
sinless Son of God, who lived from eternity 
in God's holiness, entered man, lived that 
holiness out in the face of sin, and thus not 
only broke the evil power by living it down 
but created that holiness in us by living it 
in. What is our redemption is thus also our 
reconciliation. If the atoning thing is holiness 
(which it is), and not suffering (which it is not), 
then Christ atoned by an act which created 
a new holiness in us and not a new suffering. 
The act which overcame the world intensively 
for good and all was also the act which 
slowly masters the world in the extensive sense. 
His moral and spiritual victory was so deep 
and thorough that it gives Him power to subdue 
\marginpar{210}
other consciences to His holy self, world 
without end. 

\begin{center}
\S
\end{center}

There is an old word used in this connection 
which there is much disposition at the present 
to recall and reclaim. It is the word \textit{surety}, of 
which some of our fathers were so fond. The 
word substitute has unfortunate and misleading 
suggestions, and it has practically been dropped 
in favour of a word more ethical and more constitutional, 
like representative. But even that 
word misleads us to think of Christ as the 
spiritual protagonist of a democracy, drawing 
His power from those He represents; and it 
muffles the truth that His relation to us is royal 
and not elective, that it is creative and not 
merely expository. He does not express the 
natural repentance of the old humanity but 
creates the penitent faith of the new---"the 
new man created unto holiness." It is not 
easy to find a word that has no defect, since 
all words, even the greatest, are made from 
the dust and spring from our sandy passions, 
earthly needs, and fleeting thoughts; and they 
are hard to stretch to the measure of eternal 
things without breaking under us somewhere. 
The word surety itself gives way at a great 
strain---as does guarantee. Christ's function for 
\marginpar{211}
us was not simply an assurance to God, from one 
who knew us well, that for all our aberrations 
we were sound and could be trusted at bottom. 
His confession of us was not simply His 
expression of His conviction, as deep as life, 
that man, though tough and slow, would in 
the long-run turn, obey, and confess if properly 
treated from above. It was not a pledge to God, 
or an encouragement to man, that Humanity 
would come right when experience had done its 
work on his native goodness and his spiritual 
nature, so much deeper than his sin. It was not 
a warranty to God that human nature would 
at last recover its spiritual balance, of which 
recovery Christ might point to Himself as being 
an earnest, a prelude, a classic illustration. It 
was not that Christ staked His insight into the 
deep nature of this most excellent creature man 
that he would one day rise from his swine, and return 
from his rebellion, and fall into the Father's 
arms. Such poor suggestions as these spring 
from our common and commercial use of a word 
like surety or guarantee. As if Christ were a 
third party between two who did not quite believe 
in each other. As if God by this aid might be 
led to foresee that man would come to himself 
in a faith and repentance distant but certain, 
might credit it to him in advance, and might 
\marginpar{212}
pardon on that ground. That would destroy 
grace. And it would give man the satisfaction 
of satisfying God if He would but give him time 
to collect the wherewithal. 

Christ is no third party, no arbitrator, no 
moral broker. And He is not the first instalment 
of man's return to God, its harbinger. In 
no such sense is He our surety before God. 
Because His work is not one of insight but 
of regeneration. It did not turn on His genius 
for reading us, but His power to create us 
anew. He Himself is the creator in us of 
what He promises for us. Any surety that 
Christ gives to God for man is really God swearing 
by Himself; it is the Creator's self-assurance 
of His own regenerative power. Christ, as the 
Eternal Son of Holy God, can offer Him a holiness 
which creates and includes that of the race, 
and does not simply prophesy it. 

\begin{center}
\S
\end{center}

We might put it thus: Christ alone in His 
sinless perfection can feel all God's holiness in 
judging sin; and therefore He alone could 
confess and honour it. No sinful man could do 
that; and therefore no sinful man could duly 
repent. The value of repentance is always in 
proportion to the sense of God's holiness. To 
\marginpar{213}
confess that holiness is the great postulate in 
order to confess sin. And the race could duly 
confess its sin and repent only if there arose in it 
One who by a perfect and impenitent holiness in 
Himself, and by His organic unity with us, could 
create such holiness in the sinful as should make 
the new life one long repentance transcended by 
faith and thankful joy. This was and is Christ's 
work. And the satisfaction to God, as it was 
certainly not His suffering, was also more than 
the spectacle of His own holy soul presented to 
God. It was that holy soul (the holier as He 
faced and conquered evil ever growing more 
black and bitter)---it was that holy soul seen 
by God as the cause and creator of the race's 
confession, both of holiness and of sin, in a 
Church of the reborn. The satisfaction to God 
was Christ, not as an isolated character, or in 
an act wholly outside us and our responsive 
union with Him; but it was Christ as the 
author of our sanctification and repentance. 
Our repentance and our sanctity are of saving 
value before God only as produced by the creative 
holiness of Christ. Christ creates our holiness 
because of His own sanctification of Himself---John 
xvii. 19---and His complete victory over 
the evil power in a life-experience of moral 
conflict. 

\marginpar{214}
You wish perhaps here to ask me this question: 
Is then the sanctity of a Unitarian who rejects 
any satisfaction by Christ, any atonement, as the 
ground of man's holiness, is that sanctity of no 
account before God? Is the true repentance of 
those who do not know of an atoning Christianity 
of little price with Him? Far from it. But from 
our point of view we must regard them as incomplete 
stages, which draw their value with 
God from a subliminal union with that completed 
and holy offering of Christ which He never ceases 
to see, however far it be beneath our conscious, 
light. 

\begin{center}
\S
\end{center}

When therefore we speak of Christ as our 
Surety, we mean much more than would be 
meant by a mere sponsorship. We suppose a 
solidary union of faith created by the Saviour in 
the sinner, which not only impresses him but 
incorporates him with Christ. All turns upon 
that spiritual solidarity. All turns upon the 
reality of that new life for which Paul had to 
invent a new phrase---"in Christ." A tremendous 
phrase, like that other, "the New Creation"---and 
hardly intelligible to a youthful or impressionist 
Christianity. The real ground of our forgiveness 
is not our confession of sin, and not even 
Christ's confession of our sin, but His agonised 
\marginpar{215}
confession of God's holiness, and its absorbing 
effect on us. To be in grace we must be 
found in Him. Our new penitent life is His 
creation. He contains the principle and power 
of our forgiveness. And it comes home to us 
only as we abide in Him. In Him, and only 
in Him, the normal holy man, the man holy 
with all the holiness of God, have we the 
living power of release from guilt, escape from 
sin, repentance, faith, and newness of life. 
We are justified only as we are incorporate 
(not clothed) in the perfect righteousness of 
Christ, our Regenerator, and not in proportion as the righteousness of Christ has made 
palpable way in us. It is not as Christ is in 
us that we are saved, but as we are in Christ. 
It is this being in Christ for our justification 
that makes justification necessarily work out 
to sanctification, and forgiveness be one with 
eternal life. 

We shall be misled even by what is true in the 
representative aspect of Christ unless we grasp 
how much more He is, how creative He is, how 
the solidarity involved in His representation is 
due to His own act of self-identification and 
not to natural identity with us. We must take 
quite seriously that supreme word of a "new 
creation in Jesus Christ." We need not get 
\marginpar{216}
lost in discussing the metaphysic of it; but 
we must have so tasted the new life that 
nothing but the strongest word possible is just 
to it. 

\begin{center}
\S
\end{center}

Christ our New Creator! He was not simply 
a new departure in the history of \textit{ethical civilisation}, 
by the introduction of an exalted morality. 
If that was what He came with, He brought much 
less than the conscience needs; and on countless 
points He has left us without guidance to-day. 
Nor was He simply a great new departure in the 
history of \textit{religious ideas}. He did much more 
than bring us a new idea of God. If that was 
all, again it was not what we need. For we 
have more and higher ideas of God than we 
know what to do with, more than we have 
power to realise. But He stands for a new 
departure in the history of \textit{Creation}. His work 
in so far is cosmic. It is a new storey added to 
the world. It is a new departure in the action 
which made the universe. It is an entirely new 
stage in the elevation of human nature, so 
imperfect in our first creation, to its divine 
height in holiness. By His moral treatment of 
our sinful case, which is our actual historic case, 
we are taken into a share of His superhuman 
life. That is our salvation. It is life and power 
\marginpar{217}
we need. It is to be made over again by the 
Maker's redeeming hand. We are redeemed 
\textit{from} the ban of sin's magic circle by the only 
One who has the secret of the unseen powers; 
we are joined with the sin-destroying life of 
Christ. And we are redeemed, by the very 
nature of that redemption, \textit{into} the fellowship of 
His eternal and blessed peace. And that is our 
Reconciliation. The act that justified sanctifies 
and reconciles. And that totality of Christ in 
His Church is what God looks on and is satisfied. 
We are, as a believing race, in the Son in whom 
He is always well pleased. 

\begin{center}
\S
\end{center}

Now what is it that has created so much difficulty 
for the old Protestant doctrine? I mean 
difficulty in the mind of Christian believers, and 
still more in their experience. For we need not 
trouble here about difficulty from the side of the 
worldlings or the ethical sentimentalists. But 
difficulty arose within the pale of the most 
devout and devoted evangelical experience. 
Perhaps it has arisen in your own minds. Well, 
the old Protestantism, as you know, was greatly 
exercised about the true relation between faith 
and works. And it had to insist so strongly on 
the sole value of faith in order to cope with 
\marginpar{218}
Rome that its later years fell into an excessive 
dread of good works, lest there should be 
ascribed to them saving effect. As a result faith 
was credited with a merely receptive power, or 
no more beyond that than a power of assent. 
Men lost hold of the great Lutheran fact that 
faith is the most mighty and active thing in the 
soul, that our faith is our all before God, that it 
is an energy of the whole person, that good 
works are done by this whole believing person, 
and that faith by its very nature, as trust in God's 
love, is bound to work out in love. They misread 
the moral impulse in faith, its power to 
recast personality and refashion life. They did 
not, of course, overlook the necessity of such 
renovation; but they put it down to a subsequent 
action of the Spirit over and above faith 
almost as if the Spirit and His sanctification were 
a second revelation, a new dispensation. Which 
indeed many of the mystics thought it was---like 
many rationalist mystics to-day, who think we 
have outgrown historic Christianity and the 
historic Christ through our modern light. The 
old Protestant orthodoxy did not realise that the 
real source of the Spirit is the Cross. It therefore 
detached faith from life in a way that has produced 
the most unfortunate results, both in an 
antinomianism within the Church, and in a 
\marginpar{219}
Socinian protest without, which was inevitable, 
and so far valuable, but was equally extreme. 
Faith was treated by the positive school then as a 
mystic power, or an intellectual, but not as a 
moral. It was not the renovating power in life, 
but only prepared the ground for the renovating 
power to come in. It had not in itself the transforming 
power either individually or socially. 
Its connection with love was accidental and not 
necessary---as it must be, being faith in love. 

\begin{center}
\S
\end{center}

Now, if we translate this experimental language 
into theological, it means that they did 
not connect up justification and sanctification. 
Forgiveness of sin was not identified closely 
enough with eternal life. Eternal life was detached 
from identity with that which was the 
true eternal in life, from faith's practical (\textit{i.e.}, 
experimental) godliness. Forgiveness did not 
go, as it should, with renewal of heart and conduct 
in one act. It delivered from an old world 
without opening a new and planting us in its 
revolutionised principles. Faith had, indeed, 
the power to do works of love, but it was not 
driven to them so that it could do no other. 
And this flaw in faith corresponded to a like 
flaw in the reading of Christ's act which was 
\marginpar{220}
the object of faith. They treated the work of 
Christ in a way far too objective. It was something 
done wholly over our heads. There was 
not a solidary connection between Christ's work 
and the Church it created. Attention was concentrated 
upon one aspect of Christ's work---its 
action on God. That is quite an essential aspect 
(perhaps the chief), but it must not be isolated. 
No aspect of that work must be isolated, as I 
began by saying. It is the service an accomplished 
theology does for the Church to keep 
all aspects in one purview, in the proportion of 
a great and comprehensive faith. We have 
to-day gone to another extreme, and isolated 
another aspect---the moral effect of Christ on 
man. So we need not give ourselves any airs 
of superiority to the old orthodoxy in that 
respect of onesidedness. And we must also remember 
that the whole secret of truth in this 
matter is not what we are sometimes told---a 
change of emphasis. We have changed the 
emphasis, and yet we are short of the truth; 
and the state of the Church's piety shows it. 
We have moved the accent from the objective 
to the subjective work of Christ; and we fall 
victims more and more to a weak religious subjectivism 
which has the ethical interest but not 
the moral note. We fall into a subjectivism 
\marginpar{221} 
which is reflected in one aspect of Pragmatism 
and overworks the principle contained in the 
words, "By their fruits shall ye know them" 
(know \textit{them}, whether they are true to the Gospel, 
not the Gospel and whether it is true to God 
and reality). So that people say, "I will believe 
whatever I feel does me good. My soul will 
eat what I enjoy, and drink what makes me 
happy." They are their own test of truth, and 
"their own Holy Ghost." The secret, therefore, 
is not change of accent but balance of aspects. 
And the true and competent theology is not 
only one which regards the Church's whole 
history and outlook (thinking in centuries, I 
called it), but it is one disciplined to think in 
proportion, to think together the various 
aspects of the Cross, and make them enrich 
and not exclude one another. 

\begin{center}
\S
\end{center}

The defect of the old view was, then, as I have 
said, that it could not couple up justification 
and sanctification. It could not show how the 
same act of Christ which delivered from the 
guilt of sin delivered also from its power. And 
this was because in the justification too much 
stress was laid upon the suffering; and suffering 
in itself has no sanctifying power. You see how 
\marginpar{222}
our practical experience, when it is well noted, 
provides our theological principles. We do 
find that suffering by itself debases, and even 
imbrutes, instead of purifying; that pain is an 
occasion rather than a cause of profit. That is 
a moral principle of spiritual experience. Consequently 
when excessive attention was given to 
the suffering of Christ, and the atoning value 
was supposed to reside there instead of in the 
holy obedience, the work of Christ lost in purifying 
and sanctifying effect, whatever it may 
have done in pacifying or converting. The 
atoning thing being the holy obedience to the 
Holy, the same holiness which satisfied God 
sanctifies us. That is the idea the Reformers 
did not grasp, through their preoccupation 
with Christ's sufferings. But it is the only 
idea which unites justification and sanctification 
and both with redemption. For the holiness 
which satisfied God and sanctifies us also 
destroyed the evil power in the world and 
its hold on us. It was the moral conquest 
of the world's evil, amid the extreme conditions 
of sin and suffering, by a Victor who 
had a capital solidarity with the race, and not 
merely an individual connection with it as a 
member. So that it has been said that we 
must explain and correct current ideas of substitutionary 
\marginpar{223}
expiation by the idea of solidary 
reparation. The curse on man was the guilty 
power of sin and its train---hitherto invincible. 
There was but one way in which this could be 
mastered. A moral curse could be mastered 
only in a purely moral way, the world-curse 
by the world-conscience. It could be mastered 
but by One whose sinlessness was not only 
negatively proof against all that sin could do, 
but positively holy; and He was thus deadly to 
sin, satisfactory to God's loving judgment, and 
creative of a new humanity in the heart of the 
old. This was a task beyond mere substitutionary 
penal suffering as that phrase is now so 
poorly understood. For that would have been 
just and effectual only if it had fallen on the 
arch-rebel, who, with the nobility of Milton's 
Satan in his first stage, assumed himself all the 
worst consequences of his revolt to spare the 
other souls whom he had misled. 

\begin{center}
\S
\end{center}

The truth is that Anselm, in spite of the 
unspeakable service he did both to the faith 
and thought of his time and all time, yet put 
theology on a false track in this matter. He 
had too much to say of a superethical tribute 
paid to God's \textit{honour} by the composition of a 
\marginpar{224}
voluntary suffering. Our sin was compounded 
rather than really atoned. He did not grasp the 
sacrifice of Christ as made to God's \textit{holiness}; as 
one therefore which could only be ethical in its 
nature, by way of holy obedience. This obedience 
was the Holy Father's joy and satisfaction. He 
found Himself in it. And it was also the foiling 
and destruction of the evil power. And it was 
farther the creative source of holiness in a race 
not only impressed by the spectacle of its tragic 
hero victorious, but regenerate by the solidarity 
of a new life from its creative Head. The work 
of Christ was thus in the same act triumphant 
on evil, satisfying to the heart of God, and 
creative to the conscience of man by virtue of 
His solidarity with God on the one side, and on 
the other with the race. He subdued Satan, 
rejoiced the Father, and set up in Humanity the 
kingdom---all in one supreme and consummate 
act of His one person. He destroyed the kingdom 
of evil, not by way of preparation for the 
kingdom of God, but by actually establishing 
God's kingdom in the heart of it. And He rejoiced, 
filled, and satisfied the heart of God, not 
by a statutory obedience, or by one private to 
Himself, which spectacle disposed God to bless 
and sanctify man; but by presenting in the 
compendious compass of His own person a 
\marginpar{225}
Humanity presanctified by the irresistible 
power of His own creative and timeless work. 

The holy demand of God is always couched in 
a false form when it is made to call for the 
expiation of an equivalent suffering instead of a 
confession of God's holiness, adequately holy, 
from the side of the sinner under judgment. 
Heaven and its happiness are wrongly conceived 
as immunity from judgment instead of joy in 
the consummation of judgment in righteousness 
and holiness for ever. It was not clear to the 
old view that the very nature of justification 
was sanctification, that the Justifier was so only 
as One who always perfectly sanctified Himself, 
and was organic, in the act, with the race in its 
new life. It appeared to our fathers as if sanctification 
were only a facultative sequel of justification. 

Whatever we mean, therefore, by substitution, 
it is something more than merely vicarious. It 
is certainly not something done over our heads. 
It is representative. Yet not by the will of 
man choosing Christ, but by the will of 
Christ choosing man, and freely identifying 
Himself with man. It is a matter not so much 
of substitutionary expiation (which, as these 
words are commonly understood, leaves us too 
little committed), but of solidary confession and 
\marginpar{226} 
praise from amid the judgment fires, where the 
Son of God walks with the creative sympathy 
of the holy among the sinful sons of men. It is 
not as if Christ were our changeling, as if His lot 
and ours were transposed on the Cross. But He 
was our self-appointed plenipotentiary, and 
what He engaged for we must implement by an 
organic spiritual entail. So far His work was as 
objective as our creation, as independent of our 
leave; and it committed us without reference 
to our consent but to our need. When He died 
for all, all implicitly died. The great transaction 
was done for the race. But objective as it 
was, gift as it was to us from pure grace, it 
was so in its initiative rather than in its 
method. Essentially it was a new creation of 
us, but practically the new creator was in us, 
and the word was flesh. In such a way that He 
and His are one by faith in a solidarity corresponding 
from beneath, \textit{mutatis mutandis}, to 
the solidarity between Father and Son from 
above. 

He and His form an organic spiritual unity---one 
will in two parties or persons. Mere substitution 
is mere exchange of parts, in which one 
is excluded and immune. But the work of 
Christ is inclusive and committal, by our continuity 
of life with Him through the spirit in a 
\marginpar{227}
Church.\footnote{
In His saving act He so became one with the race that the 
new Humanity He set up arises in history as the company of 
those who answer and seal His incarnate act with their faith. 
By his incarnation and redemption Christ did not simply deify 
Humanity, as a pagan Christianity had it in the fourth century, 
nor manifest the essential deity of Humanity as a pagan Christianity 
has it in the twentieth. But He so took a Humanity 
predestined for Him that those who take Him should become 
the new Humanity in the true Church.}
The suffering of Christ is but the under 
and seamy side of that solidarity whose upper 
side is the beauty of our corporate holiness in 
Him. The same law, the same act, which laid 
our sin on Him lays His holiness on us, and 
absorbs us into His satisfaction to God. In the 
same act God made Him to be sin for us and 
made us righteousness in Him. In the empirical 
sense we are no more made righteous than He 
was made sinful. But we are as closely incorporated 
in the holy world as He was in the 
sinful. And our holiness is not ours, in the 
same sense as our sin was not His---in the 
sense of initiative and individual responsibility 
for it. 

It was as our self-appointed representative 
that Christ died. He died as the result, as the 
finale, of the act by which He identified Himself 
with us and emptied Himself from heaven. He 
is our Head by divine right and not by election of 
\marginpar{228}
ours. Our representative, our surety He was---not 
our choice illustration, not our mandatory 
champion, not our moral deputy, not our friendly 
sponsor promising that we should one day pay 
our debt because of His optimistic faith in us. 
It was not in us that He had faith so much as in 
Himself as the power and grace of God. He did 
not promise that we would pay (if the metaphor 
may be allowed); He paid for us, knowing that in 
Himself alone could we raise the vast advance. 
What was presented to God was not only 
Christ's perfection, nor was it His confidence in 
us, but also His antedated action on us, His confidence 
in Himself for us. That was what stood 
to our good. There was offered to God a racial 
obedience which was implicit in the creative 
power of His, and not merely parallel with His, 
as if He were our firstfruits instead of our Sun. 
%Note that firstfruits is a single word in the original text.

\begin{center}
\S
\end{center}

The juristic aspect is a real element in Christ's 
death. It has a moral core; and we cannot 
discard it without discarding the moral order of 
the world as one revelation of that irrefragable 
holiness of God which must be expressed in 
judgment and confessed from its midst. The 
chief defect of the great revolution which began 
in Schleiermacher and ended in Ritschl has 
\marginpar{228}
been that it allowed no place to that side of 
Christ's work. And it is a defect that much 
impoverishes the current type of religion, beclouds 
it, and robs it of the power of moral conviction 
by reducing the idea of sin and dismissing 
the note of guilt. It makes grace not so much 
free as arbitrary, because it does not regard in 
its revelation what is due to the holiness of God. 
It banishes from our Christian faith the one 
note which more than any other we have to-day 
come to need restored---the note of judgment. 
When properly construed the juristic element is 
a great power to lift faith from the mere 
ethicism to which Ritschl tends into the mystic 
region which is so essential to make a moral 
power a religious, to provide a home for the 
soul as well as a lamp to our feet, and to secure 
for believers a hidden communion with Christ. 
It also saves the grace of God from being a 
mere favouritism to believers, or a mere concession 
to misery. 

There is no doubt we are in reaction from a 
time when that side of things was overdone. 
The juristic aspect taken alone, and taken in 
relation to legal demand rather than personal 
holiness---such \textit{satisfaction}, when isolated, does 
not do justice to the aspect in which Christ was 
triumphant over evil (\textit{redemption}) nor to the 
\marginpar{230}
aspect in which His work is regenerative for 
mankind (\textit{sanctification}). And it tended to promote 
the fatal notion that holiness could be 
satisfied with suffering and death, or with anything 
short of an answering holiness effected 
and guaranteed. The satisfaction in it was 
offered to a distributive justice rather than 
to a personal holiness, to a claim rather than 
a person, to a regulative law rather than to 
a constitutive life. All that and more is quite 
true. 

But I must ask you to deal sympathetically 
with those juristic views, to treat them with 
spiritual insight. It was the vice of Socinianism, 
and it is the vice of the Rationalism which is its 
legatee, that it criticised orthodoxy by the fierce 
light of the natural conscience instead of by the 
inner nature and better knowledge of the revelation 
on which orthodoxy founded all. It criticised 
theology by the natural reason and not by 
the supernatural Gospel. There is nothing more 
vulgar than slashing criticism in such a matter. 
You cannot slash here without cutting the face 
of some great and true saints to whom these 
views are dearer than life because bound up 
with their entrusted Gospel and their life 
eternal. One of the most damnatory features 
of popular theological liberalism is the violent 
\marginpar{231}
handling of what it calls orthodoxy, and its 
total lack of spiritual flexibility and interpretative 
sympathy---caused largely by the prior 
lack of theological knowledge and culture. 
That some orthodoxy is also shallow and insolent 
is no justification for those whose plea 
is that they know better. I pray you to listen 
to the old theology not as fools but as wise, 
as evolutionists and reformers, not as dynamitards. 
Consider what was gained for us in 
it. True, it sometimes presented its gain in 
false forms, as when it spoke of the equivalence 
of Christ's suffering to what we all deserved. 
That was but the form, and the Socinians did good 
work in the correction of such things. But this 
at least had been gained the conviction that it 
was not the touchy honour of a feudal monarch 
that was to be dealt with at the head of the 
world, but the love of a just God. The conviction 
behind all was the grandest moral conviction 
possible---that all things are by Christ in the 
hands of infinite righteousness and holy love. 
This vast moral step had been taken. Men had 
come to realise that the result of Christ's work 
was eternal \textit{right}; and especially that it was 
right, not in reference to the claims of an evil 
will, but in regard to those of a will perfectly 
good. The days were certainly outgrown by this 
\marginpar{232} 
juristic theology when there could be any such 
talk as filled the early Church about dealing 
with the rights Satan had won over man. 
Evil has no rights in the soul. From that, indeed, 
it was a great advance even to Anselm's 
apotheosis of God's honour. And it was a 
further advance still beyond feudal dignity 
when the great and noble categories of jurisprudence 
were invoked to replace the notion 
of courtly or military honour which made 
God and man duellists rather than aught else. 
It was a vast step in the moralising of theology 
when its grand concern came to be the establishment 
of men before a righteous and social 
judge. Do not speak contemptuously of that 
step. It is one of our own stages. It gave us 
rest and uplifting on our journey to where we 
now stand. We have only had to carry further 
that moralising of the nature of justice. The 
whole idea was ethical and social compared 
with what went before it---at least as much so 
as ours now marks a farther advance. It was 
ethical as regards claims by an evil power which 
can have no moral rights. And it was social 
in that it brought Christian belief into line 
with the ruling principles of society as it then 
was. It is a view, moreover, which has shown 
itself capable of inspiring some of the deepest, 
\marginpar{233}
sweetest, and most beneficent piety the world 
has ever seen. Moreover, it had in it active 
conditions of moral growth which broke 
through the packthreads of its own time. 
We to-day have only had to carry forward 
that process of moralising the idea of our relation 
to God which the jurists began. Their 
theology had a moral passion which shed the 
% sic. "had" above seems to be "bad" in the text.
features in it that were ethically defective, and 
assimilated the moral idea of the Gospel as 
we are now taught to read it in a Bible rediscovered 
and reconstrued by the Spirit's action 
both in the faith and the criticism of the day. 

\begin{center}
\S
\end{center}

Among these three aspects of Christ's work 
some minds will be drawn by preference to 
one, some to another, just as different ages 
have been. Some souls, according to their experience, 
will gravitate to the great Deliverance, 
some to the great Atonement, and some to the 
great Regeneration. Some ministries will be 
marked by the influence of one, some of 
another. That is all within the free affinities 
of the spiritual life, and the preferential sympathies 
of the moral idiosyncrasy. And the 
Church is enriched by the complementary 
action of such diversities of ministry. But 
\marginpar{234}
what ought not to be encouraged is any tendency 
on the part of those who prefer the 
one line to deny the equal right of the others. 
And what ought not to be tolerated is the 
habit of denunciation, by those who see the 
one side, of the sides they find nothing in; and 
especially the habit of assuming that the sides 
they are blind to represent a lower Christian 
level. Where this is possible there has really 
been little done for the conscience by the view 
that is adopted. And it is both absurd and overweening 
to ask us to believe that those sections 
of the Church, and those lights of piety, who 
held to views at present in the background were 
all theological bigots and moral inepts; that 
real moral aptitude and theological faculty did 
not arise till now; that a like devotion obscures 
such questions; that babes and sucklings perfect 
theological praise; that wisdom is justified 
by children; and that it is now the monopoly 
of those who detach theology from religion, and 
dismiss it to a historical museum. 

If Christ be the Saviour of the world in 
any sense, the thing He did must be at least as 
great as the world. And if as great, then no 
less manifold, and no less the object for first-rate 
% "First-rate" is on the line break so I'm leaving the "-" in it. 
intelligence than the lower objects of 
experience. Faith in such a Saviour cannot 
\marginpar{235}
continue to live for either heart or conscience 
if it is detached from mind. Nor can mind 
submit to be warned off the supreme object of 
the soul's concern if that object is loved and 
sought with all our heart and soul and 
strength. The very type of prayer in the 
non-theological forms which claim to be Christian 
shows to what we can sink when faith is 
stripped of mind and strength. It is only a poor 
Christ that can be housed in a poor creed, and 
a feeble prophet that is canonised when a 
sentimentalised ethic is offered as religion. 


\chapter*{ADDENDUM} \addcontentsline{toc}{chapter}{ADDENDUM}
\chaptermark{Addendum}



%ADDENDUM 

%\marginpar{239}
\begin{center}
\textsc{Note to Lecture IV.} 
\end{center}

There is a point in pp. 118-9 where, in speaking freely, 
I have spoken loosely, and I have expressed myself with 
some want of caution likely to cause misunderstanding 
of my full meaning. I there say that the wrath of God 
is not to be taken as a pathos or affection, but as the 
working out of His judgment in a moral order. My 
intention was to discourage the idea that it was a 
mood or temper, and to connect it with the sure changelessness 
of God's moral nature. But on reviewing the 
passage I find I have so put it that I might easily suggest 
that the anger of God was simply the automatic recoil 
of His moral order upon the transgressor, the nemesis 
which dogs him and makes hard his way, his self-hardening; 
as if there were no personal reaction of a Holy God 
Himself upon the sin, and no infliction of His displeasure 
upon the sinner. This is an impression I should be sorry 
to leave; for it is one that would take much of its most 
holy significance and solemn mystery out of the work of 
Christ. 

Was Christ's bearing of God's wrath just His exposure 
to the action of the vast moral machine? Did He just 
become involved, as our rescuer, in the mechanism which 
regulates ethical Humanity, using at times man's anger as 
its agent? This mechanism might be there possibly without 
\marginpar{240}
the ordinance of a God that it should be so, or possibly 
as the institution of a deist and distant God who calmly 
watches His world spin with the motion He gave it. But 
is God not personally immanent and active in His own 
moral order? Did Christ just incur the automatic penalty 
of that order as He strove to save its victims? Was He 
just caught in the works? Or was there implied, and 
felt, also the element of personal displeasure acting 
through that order---the element that would differentiate 
wrath from mere nemesis, and infliction from mere recoil? 

Granting then that there was in Christ's suffering the 
element of personal displeasure and infliction, was it man's 
or God's? Was His treatment simply the reaction of 
sinful man against holiness, or was it the reaction of a 
holy God against sin? Did He Himself feel He was 
yielding to man's dark will, or God's will, darker, but 
higher and surer? Did He suffer, just as the holiest 
saint might in a wicked world, the extreme hate of 
men; or was God's displeasure also upon Him? We 
have abundantly seen that this could not be upon Him as 
His own desert, not as it lies upon a guilty conscience. 
If He was made sin He was not made sinful; if He was 
made a curse He was not accursed. And have we not 
also seen that He who acted in our stead could act with 
no fitness and no precision if He took on Him the mere 
equivalent of what the guilty would have paid had they 
never been redeemed (that would have needed a generous 
arch-rebel), but only if he paid what was appointed as the 
price of their redemption? The uttermost farthing is not 
the last mite of their desert but of God's ransom price. 
But the curse of sin's sequel is most real whatever the 
amount. And it was certainly on Christ, by His freely 
putting Himself under it beside the men on whom it lay. 
That curse then---was it an infliction from God, which did 
not lift, did not cease to be inflicted, even when the Son 
put Himself in its way; or was it something that struck 
\marginpar{241}
Him only from men below and not from God above 
at all? 

Surely as it falls on man at least it is God's infliction. 
We do not only grieve God but we provoke His anger. 
There is nothing we need more to recall into our sense of 
sin at present than this (though we must extend it, as we 
must extend our redemption, to a racial and solidary 
wrath of God in which we share). Its absence has 
slackened and flattened the whole tone and level of 
Christian life. The love of God becomes real anger to 
our sin, and to us as we identify ourselves with the sin, 
to us while, outside Christ, we are no more than members 
of a sinful race. Is not our satisfaction and increase in 
well-doing the personal blessing of God? Then surely our 
misery and infatuation on the other path is His personal 
anger. If a true evolution carries with it the personal 
and joyful action of God in blessing its results, is the 
result of degeneration a mere natural process in the 
moral region, secluded from God's displeased action and 
infliction? Is it all His will only as a thing willed, and 
not as His action in willing it? 

Weigh, as men of real moral experience, what is involved 
in the hardening of the sinner. That is the worst 
penalty upon sin, its cumulative and deadening history. 
Well, is it simply self-hardening? Is it simply the 
reflex action of sin upon character, sin going in, settling 
in, and reproducing itself there? Is it no part of God's 
positive procedure in judging sin, and bringing it, for salvation, 
to a crisis of judgment grace? When Pharaoh 
hardens his heart, is that in no sense God hardening 
Pharaoh's heart? When a man hardens himself against 
God, is there nothing in the action and purpose of God 
that takes part in that induration? Is that anger not as 
real as the superabounding grace? Are not both bound 
up in one complex treatment of the moral world? When 
a man piles up his sin and rejoices in iniquity, is God 
\marginpar{242}
simply a bystander and spectator of the process? Does 
not God's pressure on the man blind him, urge him, 
stiffen him, shut him up into sin, if only that he might be 
shut up to mercy alone? Is it enough to say that this is 
but the action of a process which God simply watches in a 
permissive way? Is He but passive and not positive to 
the situation? Can the Absolute be passive to anything? 
If so where is the inner action of the personal God whose 
immanence in things is one of His great modern revelations? 
Everything you call absolute is in active relation 
to the whole creation. Go into the psychology of sin as 
it is understood, not indeed to-day, but by those in the 
long, deep history of the moral soul whose experience 
coincided with a real genius for reading it---true sons of 
him of Romans vii. Ask such experients if it is never 
thus---that the anger of God promotes a sin, cherished in 
the private imagination, to actual transgression; which 
then shocks, appals, the dallier into the horrified loss of 
all confidence in the flesh; that out of the collapse may 
rise a totally new man? God never put sin in the 
world; but, sin being in the world, with its spreading 
power, does God never bring it to such a head as 
precipitates its destruction? Does He never drive the 
lunatic over a precipice into water where he can be saved 
and divert him from the quarry edge where he would be 
dashed to pieces? Did God not so act with Israel (John 
xii. 39)? When sin has once begun, is there no such thing 
morally possible as the provocative action of God's law? 
With God's law sin gains life (Romans vii. 10) and becomes 
more sinful. Every law deepens the guilt of defying it. 
That is the curse of the law. And is that law detached 
from God, and cut adrift to do its own mechanic work 
under His indifference? Is it not His curse and anger still, 
if God be in His law, as we now do believe Him to pervade 
His world? 

The love of God is not more real than the wrath of God. 
\marginpar{243}
For He can be really angry only with those He loves. 
And how can Absolute Love love without acting to save? 

Well, if it be so, that God's direct displeasure and 
infliction is the worst thing in sin's penalty, did the displeasure 
totally vanish from the infliction when Christ 
stood under it? Would He have really borne the true 
judgment on sin if it had? Was Christ's great work not 
the meeting of that judgment and hallowing it? Did 
the complete obedience and reparation not include the 
complete acceptance of God's displeasure as an essential 
factor in the curse? A holy God could not look on sin 
without acting on it; nor could He do either but to abhor 
and curse it, even when His Son was beneath it. Wherein 
is guilt different from sin but in this---that it is sin, not 
cut adrift from God and let go its own way and go to 
pieces, but sin placed under the anger of God, under the 
personal reaction of that Absolute Holy God which no 
creature, no situation, can escape? And could Christ bear 
our guilt and take it away if He did not carry it there, and 
bear it there, and hallow its judgment there? Did He 
just throw it down there, leave it, and rid Himself of it? 
Does not the best of sons suffer from the angry gloom that 
spreads from the father over the whole house at the 
prodigal's shameless shame? Did God not lay on Him the 
iniquity of us all, and inflict that veiling of His face which 
darkened to dereliction even the Redeemer's soul? It is 
not desert that is the worst thing in judgment, but desertion---the 
sense of desert forsaken by God. The forsakenness 
is the worst judgment. For with God's presence 
my sense of desert may be my sanctification. What 
Christ bore was not simply a sense of the connection 
between the sinner and the impersonal consequences of 
sin, but a sense of the sinner's relation to the personal 
\textit{vis-\`{a}-vis} of an angry God. God never left Him, but He 
did refuse Him His face. The communion was not broken, 
but its light was withdrawn. He was forsaken but not 
\marginpar{244}
disjoined. He was insolubly bound to the very Father 
who turned away and could not look on sin but to abhor 
and curse it even when His Son was beneath it. How 
could He feel the grief of being forsaken by God if He 
was not at bottom one with Him? Neglect by one to 
whom we have no link makes no trouble. 

Even a theologian so little orthodox as Weizs\"{a}cker 
says:---

"The moral experience of guilt is too strong to let me 
say that it can be met by any mere manifestation of grace 
or of love from God to man even when that manifestation 
carries in it the sympathetic suffering of sin's curse, borne 
merely in the way of confirming the manifestation and 
pressing the object-lesson." " When repentance helps the 
believer to peace it is not \textit{ex opere operato}, because he has 
repented and may now trust grace; but it is because in 
his repentance he has part and lot in the infinite pain and 
confession of Christ." 



\end{document}