\documentclass[12pt,letterpaper,oneside]{book}
\usepackage[utf8]{inputenc}
\author{Peter Taylor Forsyth}
\title{The Work of Christ}

\usepackage [english]{babel}
\usepackage [autostyle, english = american]{csquotes}
\MakeOuterQuote{"}


\begin{document}
\marginpar{i}

THE WORK OF CHRIST 


\marginpar{ii}

\marginpar{iii}

THE WORK OF CHRIST 



BY 

PETER TAYLOR FORSYTH, D.D. 

PRINCIPAL OF HACKNEY COLLEGE, HAMPSTEAD 



LONDON: HODDER AND STOUGHTON 

TORONTO: THE WESTMINSTER COMPANY LIMITED 


\marginpar{iv}


\chapter*{PREFACE} 

% Note that \textsc{} gives small caps. TK still need to figure out how to do dropcaps.

\textsc{These} chapters need to have it said that 
they were given as extempore lectures 
from rough notes to a gathering, largely of 
young ministers, in connection with Rev. 
Dr. Campbell Morgan's annual conference at 
Mundesley, Norfolk. They were taken down 
in shorthand and then carefully revised. They 
took place in July, 1909, immediately after 
the delivery of my Congregational Lecture on 
the \textit{Person and Place of Christ}, which they 
supplement---especially when taken with my 
\textit{Cruciality of the Cross} a few months before. 
It will be seen from the conditions that the 
book cannot pretend to be more than a 
higher kind of popularisation, though this is 
less true of the two last chapters, which 
have been more worked over. The style approaches 
in parts a conversational familiarity 
which would have been out of place in addressing 
theological experts. And as some of the 
ideas are unfamiliar I have not been too careful 
\marginpar{vi}
to avoid repetition. My hope is to be of some 
use to those ministers who are still at a 
stage when they are seeking more footing on 
such matters than they have been provided 
with in mere Biblical or Historical Theology. 
There is no region where religion becomes so 
quickly theology as in dealing with the work 
of Christ. No doctrine takes us so straight 
to the heart of things, or so forces on us a 
discussion of the merits of the case, the dogmatic 
of it, as distinct from its scriptural or 
its ecclesiastical career. No doctrine draws so 
directly on the personal religion of sinful men, 
and none, therefore, is open to so much change 
in the course of the Church's thought upon its 
growing faith and life. Thus when we consider 
that here we are at once where the form may 
change most in time and yet the feet be most 
firmly set for eternity, we realise how difficult 
and delicate our task must be. And we are 
made to feel as if the due book on such a 
theme could only be written from behind the 
veil with the most precious blood that ever 
flowed in human veins. 

\begin{center}
\noindent\rule{4cm}{0.4pt}
\end{center}

We are in a time when a spirituality without 
positive content seems attractive to many minds. 
And the numbers may grow of those favouring 
\marginpar{vii}
an undogmatic Christianity which is without 
apostolic or evangelical substance, but cultivates 
a certain emulsion of sympathetic mysticism, 
intuitional belief, and benevolent action. Among 
lay minds of a devout and social but impatiently 
practical habit, this is not unlikely to spread; 
and particularly among those whose public interests 
get the upper hand of ethical and 
historical insight and denude their religion of 
most of the reflection it demands. 

Upon undogmatic, undenominational religion 
no Church can live. With mere spirituality the 
Church has not much directly to do; it is but 
a subjective thing; and its favour with many 
may be but another phase of the uncomprehending 
popular reverence (not to say superstition) 
for the recluse religionist, the mysterious 
ecstatic, and the ascetic pietist. What Christian 
faith and the Christian Church have to do with 
is \textit{holy} spirituality the spirituality of the Holy 
Spirit of our Redemption. The Christian revelation 
is not "God is a spirit," nor is it "God 
is love." Each of these great words is now 
much used to discredit the more positive faith 
from whose midst John wrote them down. 
Herein is love, not in affection but in propitiation 
(1 John iv. 10). Would Paul ever 
have written 1 Cor. xiii. if it had been revealed 
\marginpar{viii}
to him that it was going to be turned against 
Rom. iii. 25? And what would his language 
have been to those who abused that chapter 
so? Christian faith is neither spirituality nor 
charity. Its revelation is the holiness in judgment 
of the spiritual and loving God. Love 
if only divine as it is holy; and spirituality is 
Christian only as it meets the conditions of 
Holy Love in the way the Cross did, as the 
crisis of holy judgment and holy grace. If 
the Cross is not simply a manner of religion 
but the object of our religion and the site 
of revelation, then it stands there above all 
to effect God's holiness, and not to concentrate 
man's self-sacrifice. And except in the 
Cross we have no guarantee for the supreme 
thing, the divine thing, in God, which is the 
changeless reality and irresistible sovereignty 
of His Holy Love. 

It is upon such faith alone, given by the Cross 
alone, that a Church can live---upon the faith 
that founded it---upon a positive New Testament 
Gospel. Of that Gospel the Church is 
the trustee. And the Church betrays its trust 
and throws its life and its Lord away when it 
says, "Be beautifully spiritual and believe as 
you like," or "Do blessed good and think as you 
please." 
\marginpar{ix}
There is a timely saying of that searching 
Christian genius Kierkegaard---the great and 
melancholy Dane in whom Hamlet was mastered 
by Christ: 

"For long the tactics have been: use every 
means to move as many as you can---to move 
everybody if possible to enter Christianity. 
Do not be too curious whether what they enter 
\textit{is} Christianity. My tactics have been, with 
God's help, to use every means to make it clear 
what the demand of Christianity really is---if 
not one entered it." 

The statement is extreme; but that way lies 
the Church's salvation in its anti-Nicene relation 
to the world, its pre-Constantinian, non-established, relation to the world, and devotion 
to the Word. Society is hopeless except for the 
Church. And the Church has nothing to live on 
but the Cross that faces and overcomes the 
world. It cannot live on a cross which is on 
easy terms with the world as the apotheosis of 
all its aesthetic religion, or the classic of all its 
ethical intuition. The work of Christ, rightly 
understood, is the final spiritual condition of all 
the work we may aspire to do in converting 
society to the kingdom of God. 


\marginpar{x}
\marginpar{xi}
\chapter*{CONTENTS}
TK


PAGE 

THE DIFFERENCE BETWEEN GOD'S SACRIFICE 

AND MAN'S . . . . .1 



II 

THE GREAT SACRIFICIAL WORK IS TO RECONCILE 31 



III 

RECONCILIATION: PHILOSOPHIC AND CHRISTIAN 63 



IV 

RECONCILIATION, ATONEMENT, AND JUDGMENT 97 

xi 



xii CONTENTS 

V 

PAGE 

THE CROSS THE GREAT CONFESSIONAL . . 139 



VI 

THE PRECISE PROBLEM TO-DAY .... 173 



VII 

THE THREEFOLD CORD 197 



ADDENDUM 237 



\marginpar{1}

\chapter{THE DIFFERENCE BETWEEN 
GOD'S SACRIFICE AND MAN'S} 

%\marginpar{3}  



%I

%THE DIFFERENCE BETWEEN GOD'S 
%SACRIFICE AND MAN'S 

% STill need dropcaps and the title sorted out TK

\textsc{What} I am going to say is not directly unto 
edification, but indirectly it is so most 
certainly. Directly it is rather for that instruction 
which is a need in our Christian life as 
essential as edification. We cannot do without 
either. On the one hand instruction with no idea 
of edification at all becomes mere academical 
discourse. It may begin anywhere and it may 
end anywhere. On the other hand, edification 
without instruction very soon becomes a feeble 
and ineffective thing. I think a great many of 
us would be agreed that part of the poverty 
and weakness of the Church at the present 
moment is due to the fact that edification has 
been pursued to the neglect of instruction. We 
have been a little too prone to dwell upon the 
simple side of the gospel. All our capital is in
\marginpar{4}
small circulation. We have not put by a reserve, 
as it were. And therefore the simplicity 
itself has become unsettled and ineffectual, confused 
and confusing. 

I ask your attention to certain aspects of 
our Christian faith which perhaps do not lie 
immediately upon the surface, but which are 
yet the condition of the Church's continued 
energy and success in the world. I suppose there 
is nobody here who does not believe in the 
Church. At any rate, what I propose to say 
will be said entirely from that standpoint. We 
believe in the Holy Catholic Church. My contention 
would be that, apart from such a position 
as I desire to bring to your notice---some 
real apostolic belief in the real work of Jesus 
Christ---apart from that no Church can continue 
to exist. That is the point of view which I take 
at the outset. The Church is precious, not in 
itself, but because of God's purpose with it. It 
is there because of what God has done for it. 
It is there, more particularly, because of what 
Christ has done, and done in history. It is 
there solely to serve the Gospel. 

It is impossible not to observe at the 
present day that the Church is under a cloud. 
You cannot take any division of it, in any 
country of the world, without feeling that that 
\marginpar{5}
is so. Therefore I will begin by making quite a 
bold statement; and I should be quite prepared, 
given time and opportunity, to devote a whole 
week to making it good. The statement is 
that the Church of Christ is the greatest and 
finest product of human history. It is the 
greatest thing in the universe. That is in complete 
defiance of the general view and tendency 
of society at the present moment. I say the 
Church is the greatest and finest product of 
human history; because it is not really a 
product of human history, but the product of 
the Holy Spirit within history. It stands for the 
new creation, the New Humanity, and it has that 
in trust. The man who has a slight acquaintance 
with history is ready to bridle at a statement 
like that. He says: "Consider what the Roman 
Church has done; consider how obscurantist 
many sections of the Protestant Church are; 
consider the ineffectual position of the Church 
in modern civilisation---and what nonsense to 
talk about the Church as the greatest and 
finest product of human history!" True enough, 
the authority of the Church is failing in many 
quarters. And that does not mean only the 
external authority of what you might call a 
statutory Church, a great institutional Church, 
a great organised Church like Rome, for example.
\marginpar{6}
It means much more than that. It 
means that the authority of the whole Church is 
weakened in respect of the inward and spiritual 
matter which it contains and preaches, and 
which makes it what it is. The Church is 
there as the vehicle of the power of the Holy 
Ghost and of the authority of the saving 
God---a God, that is, who is saving not groups 
here and there, but the whole of human society. 
But a spiritual authority for man altogether 
is at a discount. Perhaps we have brought that 
in some measure upon ourselves. Perhaps, too, 
it was historically necessary. But, necessary 
or not, it is a matter of fact that our Protestantism 
has developed often into a masterless 
individualism which is as deadly to Christian 
life as an over-organised institution like Rome. 
Many spiritual people to-day find it difficult to 
make their choice between the two extremes. 
Without going into the historic causes of the 
situation, let us recognise the situation. 
Spiritual authority, especially that of the 
Church, is for the time being at a great 
discount.

\begin{center} \S \end{center}

The Church is valuable as the organ of 
Christian grace, and truth, and power. But 
what do we find offered us in place of the 
\marginpar{7}
Church? Those who attack the Church most 
seriously, and disbelieve in it most thoroughly, 
are not proposing simply to level the Church 
to the ground in the sense of destroying any 
religious society. What they want to do is to 
put some other kind of society in the place of 
the Church. For they say, as we all say, that it 
is impossible for religion, certainly impossible 
for Christianity, to exist without a social body 
in which it is cultivated and has its effect. 
Therefore, those who are opposed to the Church 
most bitterly are yet not prepared to make a 
total desert. But they put all kinds of organisations, 
fancy organisations and fancy religions, 
in its place. Take the great movement in the 
direction of Socialism. Take the Socialist programmes 
that you find so plentifully everywhere. 
What do these various organisations 
mean? What do all these organisations mean 
which profess to embody human brotherhood, 
and are represented by Trades Unions, Cooperation, Fraternities, Guilds, Socialisms? 
What is it they all confess? That some social 
vehicle there must be. You cannot promote 
Anarchy itself without associations for the 
purpose. So that the very existence of these 
rival organisations is a confession of the one 
fundamental principle of the Church, namely, 
\marginpar{8}
that the human ideal, that religion in the 
true sense of the word, cannot do without a 
social habitation. They put in their own way 
what we put in our way (and we think a better 
way), that there must be a Church builded 
together for a habitation of God in the Spirit. 
Our individualisms have been troubling and 
weakening us so much that everybody is looking 
away to some form of human life which 
shall have the advantages of individualism 
without its perils. The pietistic form of individualism 
did in its day great service. But it is 
out of date. Rationalistic individualism, again, 
taking shape in political radicalism, has done 
good work in its day. That also seems going 
out of date. The value of the new movement 
is its---shall I say---solidarity; which is a confession 
of that social, fraternal principle which 
finds its consummation really, and its power 
only, in the Church of Christ. 

When we look at these rival organisations 
(and they are many, and some will occur to 
you which I have not named), we can, I think, 
gather most of them under one head. In contrast 
with the Church the various social forms 
that are offered to us to-day would build society 
upon a natural basis, the basis of natural 
brotherhood, natural humanity, natural goodness\marginpar{9}---on human nature. And the issue between 
the Church and the chief rivals of the Church 
is an issue between society upon this natural 
basis, and society upon a supernatural basis. 
Our Christian belief is based upon the work 
of Christ; and we hold that human society can 
only continue to exist in final unity upon that 
same supernatural basis. It is an issue, therefore, 
between human nature deified and human 
nature saved; between mere sympathy and 
faith---faith taken in a quite positive and definite 
sense. We think that a brotherhood of 
mere sympathy, however warm it can be at a 
particular moment, has no stay in it, no eternal 
promise. The eternal promise is with supernatural 
faith. Do you ever believe otherwise? 
I hope you have been so tempted; because 
having got over it you will be very much better 
for having gone through it. I wish much more 
of our belief had gone through troubled scenes 
and come to its rest; we should make far 
greater impression upon men if we gave them 
to feel we had fought our way to the peace and 
power we have. Well, were you ever tempted 
to believe that Christianity is just human 
nature at its best? That is the most powerful 
and dangerous plea that is put forward just 
now in challenge of our Christian position 
\marginpar{10}
and Church. Is the Kingdom of God just our 
natural spirituality and altruism developed? 
Is it just the spirit of religion or self-sacrifice, 
which you often find in human nature, developed 
to its highest? Is that the Kingdom of 
God? I trust you believe not---that human 
nature is not capable, by all the finest sacrifices 
it might develop, of saving, of ensuring itself, 
and setting up the Kingdom of God. Take 
the best side of human nature, that side which 
moves men to unselfishness and sacrifice, the side 
that comes out in many a heroic battle, in the 
silent battles of our civilisation, where the 
victims get no applause and no reputation for 
their heroism whatever. Take the best side of 
human nature, illustrated in every coalpit 
accident and every such thing, in countless 
quiet homes of poverty, where lives are being 
worked down to the bone and ground to death 
toiling and slaving for others. Take the vast 
mass of fatherhood and motherhood living for 
the children only. Take that best side of human 
nature, make the most of it, and then put this 
question: "How does man's noblest work differ 
from Christ's great work?" That is the 
question to which I desire to attract your 
attention to-day. How does man's best work 
differ from Christ's great work? 
\marginpar{11}

\begin{center} \S \end{center}

Let me begin with a story which was reported 
in the Belgian papers some years ago. 

Two passenger trains were coming in opposite 
directions at full speed. As they approached the 
station, it was found the levers would not work, 
owing to the frost, and the points could not be 
set to clear the trains of each other. A catastrophe 
seemed to be inevitable; when a signalman threw himself flat between the rails, and 
with his hands held the tie-rod in such a way 
that the points were properly set and kept; and 
he remained thus while the train thundered 
over him, in great danger of having his head 
carried away by the low-hung gear of the 
Westinghouse brake. When the train had 
passed, he quietly rose and returned to his 
work. 

I offer you some reflections on this incident. 
It is the kind of incident that may be multiplied
indefinitely. I offer you certain reflections, 
first, on some of its analogies with Christ's 
work, and secondly, on some of its differences. 

\begin{center} \S \end{center} 

1. This man, in a very true sense, died and 
rose again. His soul went through what he 
would have gone through if he had never risen 
\marginpar{12}
from the track. He gave himself; and that is 
all a man can give at last. His deed had the 
moral value which it would have had if he had 
lost his life. He laid it down, but it did not 
please God to take it. Like Abraham's sacrifice 
of Isaac, it was complete and acceptable, even 
though not accepted. The man's rising from 
the ground---was it not really a resurrection 
from the dead? It was not simply a return to 
his post. He went back another man. He went 
back a heavenlier man. He had died and risen, 
just as if he had been called, and had gone, to 
God's presence could he but remain there. 
This is a death and rising again possible to us 
all. If the death and resurrection of Jesus 
Christ do not end in producing that kind of 
thing amongst us, then it is not the power of 
God unto salvation. These moral deaths and 
resurrections are what make men of us. "In 
deaths oft." That is the first point. 

\begin{center} \S \end{center} 

2. The second point is this. Not one of the 
passengers in either of those trains knew until 
they read it what had been done for them, nor 
to whom they owed their lives. It is so with 
the whole world. To-day it owes its existence, 
in a way it but poorly understands, to the 
\marginpar{13}
death and resurrection of Jesus Christ. That 
is the permanent element in Christianity---the 
Cross and resurrection of Jesus Christ. And 
yet it is nothing to all them that pass by. 
Under the feet of those travellers in Belgium 
there had taken place one of those deeds that 
are the very soul and glory of life, and they had 
no idea of it. Perhaps some of them were at 
the very moment grumbling at the staff of the 
railway for some small grievance or other. It 
is useful to remember, when we are inclined to 
grumble thus, what an amount of devotion to 
duty goes to make it possible for us to travel as 
safely as we do---far more than can be acknowledged 
by the payment of a wage. These 
people were ploughing along in safety over one 
of the railway staff lying in a living grave. I 
say it is so with the whole civilised world. Its 
progress is like that of the train; it seldom 
stops to think that its safety is owing to a 
divine death and resurrection, much more than 
heroic. The safety of that train was not due 
to the mechanism. The mechanism had gone 
wrong. It was not due to organisation, or to 
work done from fear of punishment. Heroic 
duty raised to martyrdom saved the whole 
train. And the world's progress is saved to-day 
because of a death and resurrection of which
\marginpar{14}
it knows little and mostly cares to know less. 
"\textit{Propter Jesum non qu{\ae}rimus Jesum}." The 
success of Christ hides Him. It is the death 
of Christ that is the chief condition of modern 
progress. It is not civilisation that keeps 
civilisation safe and progressive. It is that 
power which was in Jesus Christ and culminated 
in His death and resurrection. When 
people read the Bible, and get behind the 
Bible, and that principle comes home to them, 
it may sometimes be like the shock that those 
travellers would receive when they read in the 
newspaper of their risk and deliverance. 

\begin{center} \S \end{center}

3. Another point. And I am now coming on to 
the difference. This man died for people who 
would thrill with the sense of what they owed 
him as soon as they read about it. His act appeals 
to the instinct which is ready to spring to life 
in almost every breast. You felt the response 
at once when I told you the story. Some of you 
may have even felt it keenly. Do you ever feel 
as keenly about the devoted death of Christ? 
Perhaps you never have. You have believed it, 
of course, but it never came home to you and 
gripped you as the stories of the kind I instance 
do. You see the difference between Christ's 
\marginpar{15}
death and every case of human heroism. I am 
moving to answer that question I put a moment 
ago as to whether the development of the best 
in human nature would ever give us the work 
of Christ and the Kingdom of God. I have been 
illustrating one of the finest things in human 
nature, and I am asking whether, if that were 
multiplied indefinitely, we should yet have the 
effect which is produced by the death of Christ, 
or which is still to be produced by it in God's 
purpose. No, there is a difference between 
Christ's death and every case of heroism. 
Christ's was a death on behalf of people within 
whom the power of responding had to be 
created. Everybody thrills to that story I told 
you, and to every similar story. The power of 
response is lying there in the human heart 
ready it only needs to be touched. There is in 
human nature a battery charged with admiration 
for such things; you have only to put your 
knuckle to it and out comes the spark. But 
when we are dealing with the death of Christ 
we are in another position. Christ's was a 
death on behalf of people in whom the power of 
responding had to be created. We are all 
afraid of death, and rise to the man who delivers 
us from it. But we are not afraid of that worse 
thing than death from which Christ came to 
\marginpar{16}
deliver us. Christ's death was not a case of 
heroism simply, it was a case of redemption. 
It acted upon dull and dead hearts. It was 
a death which had to evoke a feeling not 
only latent but paralysed, not only asleep but 
dead. What does Paul say? "While we were 
yet without strength, Christ died for us"---without 
power, without feeling, as the full 
meaning is. 

Let me illustrate. Take a poet like Wordsworth. 
When he began to publish his poetry 
he was received, just as Browning was received 
later, with ridicule and contempt. The greatest 
critic of the time began an article in the leading 
critical organ of the day by saying, "This will 
never do." But it has done; and it has done for 
Jeffrey's critical reputation. Lord Jeffrey wrote 
himself down as one who was incapable of 
gauging the future, however much he might 
be capable of understanding the literature of 
the past. Some of you may remember---I 
remember perfectly well---the same kind of 
thing in the penny papers about Browning 
when he was fighting for recognition. I remember, 
when I was a student, reading articles 
in luminaries like *The Standard* which sneered 
and jeered at Browning, just as smaller men to-day 
would sneer at men of like originality. But 
\marginpar{17}
Wordsworth and Browning have conquered. I 
take another case. Turner was assailed with 
even more ridicule when he exposed his works 
to the British public. What would have happened 
to Turner if Ruskin had not arisen to be 
his prophet I do not know. His pictures might 
not even have been mouldering in the cellars of 
the National Gallery. They might have been 
selling at little second-hand shops in back streets 
for ten shillings to any one who had eyes in his 
head. Wordsworth, Browning, and Turner were 
all people of such original and unprecedented 
genius that there was no taste and interest for 
them when they appeared; they had to create 
the very power of understanding themselves. 
A poet of less original genius, a great genius 
but less of a genius, like Tennyson, comes along, 
and he writes about the "May Queen" and 
"The Northern Farmer," and all those simple, 
elementary things which immediately fetch the 
handkerchiefs out. Now no doubt to do that 
properly takes a certain amount of genius. But 
it taps the prompt and fluent emotions; and the 
misfortune is that kind of work is easily counterfeited 
and abused by those who wish to 
exploit our feelings rather than exalt them. 
It is a more easy kind of thing than was done 
by those great geniuses I first named. Original
\marginpar{18}
poets like Wordsworth and Browning had to 
create the taste for their work. 

Now in like manner Christ had to make the soul 
which should respond to Him and understand 
Him. He had to create the very capacity for 
response. And that is where we are compelled to 
recognise the doctrine of the Holy Spirit as well 
as the doctrine of the Saviour. We are always 
told that faith is the gift of God and the work 
of the Holy Spirit. The reason why we are told 
that, and must be told it, lies in the direction 
I have indicated. The death of Christ had not 
simply to touch like heroism, but it had to 
redeem us into power of feeling its own worth. 
Christ had to save us from what we were too far 
gone to feel. Just as the man choked with 
damp in a mine, or a man going to sleep in 
arctic cold, does not realise his danger, and the 
sense of danger has to be created within him, 
so the violent action of the Spirit takes men by 
force. The death of Christ must call up more 
than a responsive feeling. It is not satisfied 
with affecting our heart. That is mere impressionism. 
It is very easy to impress an audience. 
Every preacher knows that there is nothing 
more simple than to produce tears. You have 
only to tell a certain number of stories about 
dying children, lifeboats, fire eseapes, and so on, 
\marginpar{19}
and you can make people thrill. But the thrill 
is neither here nor there. What is the thrill 
going to end in? What is the meaning of the 
thrill for life? If it is not ending as it should, 
and not ending for life, it is doing harm, not 
good, because it is sealing the springs of feeling 
and searing the power of the spiritual life. 

What the work of Christ requires is the 
tribute not of our admiration or even gratitude, 
not of our impressions or our thrills, but 
of ourselves and our shame. Now we are coming 
to the crux of the matter---the tribute of our 
shame. That death had to make new men of 
us. It had to turn us not from potential friends 
to actual, but from enemies into friends. It 
had not merely to touch a spring of slumbering 
friendship. There was a new creation. The 
love of God---I quote Paul, who did understand 
something of these things---the love of God is 
not merely evoked within us, it is "shed abroad 
in our hearts by the Holy Spirit which is given 
to us." That is a very different thing from 
simply having the reservoir of natural feeling 
tapped. The death of Christ had to do with 
our sin and not with our sluggishness. It had 
to deal with our active hostility, and not simply 
with the passive dullness of our hearts. Our 
hostility---that is what the easy-going people
\marginpar{20}
cannot be brought to recognise. That is what 
the shallow optimists, who think we can now 
dispense with emphasis on the death of Christ, 
feel themselves able to do---to ignore the fact 
that the human heart is enmity against God, 
against a God who makes demands upon it; 
who goes so far as to make demands for 
the whole, the absolute obedience of self. 
Human nature puts its back up against that. 
That is what Paul means when he speaks 
about human nature, the natural man the 
carnal man is a bad translation---being enmity 
against God. Man will cling to the last rag of 
his self-respect. He does not part with that 
when he thrills, admires, sympathises; but he
does when he has to give up his whole self in 
the obedience of faith. How much self-respect 
do you think Paul had left in him when he went 
into Damascus? Christ, with the demand for 
saving obedience, arouses antagonism in the 
human heart. And so will the Church that 
is faithful to Him. You hear people of the 
type I have been speaking about saying, If 
only the Church had been true to Christ's 
message it would have done wonders for the 
world. If only Christ were preached and practised 
in all His simplicity to the world, how fast 
Christianity would spread. Would it? Do you
\marginpar{21}
really find that the deeper you get into Christ 
and the meaning of His demands Christianity 
spreads faster in your heart? Is it not very 
much the other way? When it comes to close 
quarters you have actually to be got down and 
broken, that the old man may be pulverised and 
the new man created from the dust. Therefore 
when we hear people abusing the Church 
and its history the first thing we have to say 
is, Yes, there is a great deal too much truth 
in what you say, but there is also a greater 
truth which you are not allowing for, and it 
is this. One reason why the Church has 
been so slow in its progress in mankind and 
its effect on human history is because it has 
been so faithful to Christ, so faithful to His 
Cross. You have to subdue the most intractable, 
difficult, and slow thing in the world---man's 
self-will. You cannot expect rapid successes 
if you truly preach the Cross whereon 
Christ died, and which He surmounted not 
simply by leaving it behind but by rising again, 
and converting the very Cross into a power
and glory. 

Christ arouses antagonism in the human heart 
and heroism does not. Everybody welcomes a 
hero. The minority welcome Christ. We do 
resent His absolute command. We do resent 
\marginpar{22}
parting completely with ourselves. We do 
resent Christ. 

\begin{center} \S \end{center} 

4. I go back to the word I spoke about the 
tribute of our shame. The demand is unsparing, 
remorseless. It is not simply that you 
are called on by God for a certain due, a 
change, an amendment, but for the tribute of 
yourself and your shame. When you heard 
about that heroism of my story, when you 
thrilled to it, I wonder did you pat yourself on 
the back a little for being capable of thrilling 
to things so high, so fine? When you thrilled 
to that story you felt a certain satisfaction with 
yourself because there was as much of the God 
in you as allowed you to be capable of thrilling 
to such heroisms. You felt, If I am capable of 
thrilling to such things, I cannot be such a bad 
sort. But when you felt the meaning of 
Christ's death for you, did you ever pat yourself 
on the back? The nearer the Cross came to 
you, the deeper it entered into you, were you 
the more disposed to admire yourself? There is 
no harm in your feeling pleased with yourself 
because you were able to thrill to these human 
heroisms; but if the impression Christ makes 
upon you is to leave you more satisfied with 
yourself, more proud of yourself for being able
\marginpar{23}
to respond, He has to get a great deal nearer to 
you yet. You need to be---I will use a Scottish 
phrase which old ministers used to apply to 
a young minister when he had preached a 
"thoughtful and interesting discourse"---you 
need to be well shaken over the mouth of the 
pit. The great deep classic cases of Christian 
experience bear testimony to that. Christ and 
His Cross come nearer and nearer, and we do 
not realise what we owe Him until we realise 
that He has plucked us from the fearful pit, 
and the miry clay, and set us upon a rock of 
God's own founding. The meaning of Christ's 
death rouses our shame, self-contempt, and 
repentance. And we resent being made to feel 
ashamed of ourselves, we resent being made to 
repent. A great many people are afraid to 
come too near to anything that does that for 
them. That is a frequent reason for not going 
to church. 

\begin{center} \S \end{center} 

5. Again, continuing. You would have gone a 
long way to see this Belgian man. You would 
have gazed upon him with something of reverence, 
certainly with admiration. You would 
have regarded him as one received back from 
the dead. You think, If all men were like that, 
the world would be heaven. Well, there are a 
\marginpar{24}
great many more like that than we think, who 
daily imperil their life for their duty. But 
supposing every man and woman in the world 
were up to that pitch, and supposing you added 
them all together and took the total value of 
their moral heroism (if moral quantities were 
capable of being summed like that), would you 
then have the equivalent of the deed and death 
of Christ? No, indeed! If you took all the 
world, and made heroes of them all, and kept 
them heroic all their lives, instead of only in one 
act, still you would not get the value, the equivalent, 
of Christ's sacrifice. It is not the sum of all 
heroisms. It would be more true to say it is the 
source of all heroisms, the foundation of them 
all. It is the underground something that makes 
heroisms, not something that heroisms make 
up. When Christ did what He did, it was not 
human nature doing it, it was God doing it. 
That is the great, absolutely unique and 
glorious thing. It is God in Christ reconciling. 
It was not human nature offering its very best 
to God. It was God offering His very best to 
man. That is the grand difference between the 
Church and civilisation, even when civilisation 
is religious. We must attend more to those 
great issues between our faith and our world. 
Our religion has been too much a thing done 
\marginpar{25}
in a corner. We must adjust our religion to 
the great currents and movements of the 
world's history. And the great issue of the 
hour is the issue between the Church and 
civilisation. Their essential difference is this. 
Civilisation at its best represents the most 
man can do with the world and with human 
nature; but the Church, centred upon Christ, 
His Cross, and His work, represents the best 
that God can do upon them. The sacrifice 
of the Cross was not man in Christ pleasing 
God; it was God in Christ reconciling man, 
and in a certain sense, reconciling Himself. My 
point at this moment is that the Cross of Christ 
was Christ reconciling man. It was not heroic 
man dying for a beloved and honoured God; it
was God in some form dying for man. God 
dying for man. I am not afraid of that phrase; 
I cannot do without it. God dying for man; 
and for such men---hostile, malignantly hostile 
men. That is a puzzling phrase where we read 
in a gospel: "Greater love hath no man than 
this, that a man lay down his life for his 
friends." There is more love in the phrase 
of the epistle, that a man should lay down 
his life for his bitter enemies. It is not so 
heroic, so very divine, to die for our friends. 
Kindness between the nice people is not so 
\marginpar{26}
very divine---fine and precious as it is. To die 
for enemies that is the divine thing. Christ's 
was grace that died for such---for malignant 
enemies. There is more in God than love. 
There is all that we mean by His holy grace. 
Truly, "God is love." Yes, but the kind of love 
which you must interpret by the whole of 
the New Testament. When John said that, did 
he mean that God was simply the consummation 
of human affection? He knew that he 
was dealing with a holy, gracious God, a God 
who loved His enemies and redeemed them. 
Read with extreme care 1 John iv. 10. 

\begin{center} \S \end{center}

6. Let me gather up the points of difference 
which I have been indicating. 

First, that Belgian hero did not act from love 
so much as from duty. Secondly, he died only 
in one act, not in his whole life, dying daily. 
There have been men capable of acts of sacrifice 
like this hero; loose-living men who, after 
a heroism, were quite capable of returning to 
their looseness of life---heroes of the Bret Harte 
type. There have been many valiant, fearless 
things done on the battlefield by men who in 
the face of bullets never flinched, never turned 
a hair; and when they came home they could 
\marginpar{27}
not stand against a breath of ridicule, they 
could not stand against a little temptation, and 
were soon wallowing in the mire. One act of 
sacrifice is not the same thing as a life gathered 
into one consummate sacrifice, whose value is 
that it has the whole personality put into it 
for ever. 

Third, this man could not take the full 
measure of all that he was doing, and Christ 
could. Christ did not go to His death with 
His eyes shut. He died because He willed to 
die, having counted the cost with the greatest, 
deepest moral vision in the world. 

Fourthly, the hero in the story had nothing 
to do with the moral condition of those whom 
he saved. The scoundrel and the saint in that 
train were both alike to him. 

Again, he had no quarrel with those whom he 
saved. He had nothing to complain of. He had 
nothing from them to try his heroism. They 
were not his bitter enemies. His valour was 
not the heroism of forgiveness, where lies the 
wondrous majesty of God. His act was not 
an act of grace, which is the grand glory of 
the love of Christ. Christ died for people who 
not only did not know Him, but who hated and 
despised Him. He died, not for a trainful of 
people, but for the whole organic world of
\marginpar{28}
people. It was an infinite death, that of His, 
in its range and in its power. It was death 
for enemies more bitter than anything that 
man can feel against man, for such haters as 
only holiness can produce. Here is the singular 
thing: the greater the favour that is done to 
us, the more fiercely we resent it if it does not 
break us down and make us grateful. The 
greater the favour, if we do not respond in its 
own spirit, so much the more resentful and 
antagonistic it makes us. I have already said 
that we speak too often as though the effect 
of Christ's death upon human nature must be 
gratitude as soon as it is understood. It is 
not always gratitude. Unless it is received in 
the Holy Ghost, the effect may just be the 
other way. It is judgment. It is a death unto 
death. 

\begin{center} \S \end{center}

I conclude by saying what I have often said, 
and what often needs saying, that it is not 
possible to hear the gospel and to go away just 
as you came. I wish that were more realised. 
We should not have so many sermon-hunters. 
If people felt that every time they heard the 
gospel they were either better or worse for it, 
they would be more careful about hearing. 
They would not go so often, possibly; better they 
\marginpar{29}
should not, perhaps. I am not speaking about 
hearing of sermons. That is neither here nor 
there. A man may hear sermons and be neither 
the better nor the worse. But a man cannot 
hear the gospel without being either better or 
worse, whether he knows it or not. When you 
come to face the last issues, it is either unto 
salvation or unto condemnation. The great 
central, decisive thing, the last judgment of the 
world, is the Cross of Christ. The reason why 
so many sermons are found uninteresting is not 
always due to the dullness of the preacher. God 
knows how often that is the case, but it is not 
always. It is because the sermons so often turn, 
or ought to turn, upon the miracle of the grace 
of God, which is so great a miracle that it is 
strange, remote, and alien to our natural ways 
of thinking and feeling. It seems foreign to us. 
It is like reading a guide-book if you have never 
been in the country. I take down my Baedeker 
in the winter and read it with the greatest 
delight, because I know the country. If I had 
not been there I should find it the dreariest reading. 
Why do not people read the Bible more? 
Because they have not been in that country. 
There is no experience for it to stir and develop. 
The Cross of Christ, the infinite wonder of it---we 
have got to learn that. We have got to 
\marginpar{30}
learn the deep meaning of that by having been 
there, by the evangelical experience whose lack 
is the cause of all the religious vagrancy of the 
hour. We have got to learn that it was not 
simply magnificent heroism, but that it was God 
in Christ reconciling the world. It was God 
that did that work in Christ. And Christ was 
the living God working upon man, and working 
out the Kingdom of God. 


\chapter{THE GREAT SACRIFICIAL WORK 
IS TO RECONCILE} 

%\marginpar{33}

%II 



%THE GREAT SACRIFICIAL WORK IS TO 
%RECONCILE 

\begin{center}
Corinthians v.14--vi.2; Romans v.1--11; Colossians i.10--29;
Ephesians ii.16. 
\end{center}


\textsc{The} great need of the religious world to-day 
is a return to the Bible. That is necessary 
for two reasons, negative and positive. Negatively,
because the most serious feature of the 
hour in the life of the Church is the neglect of 
the Bible for personal use and study by religious 
people. Positively, because we have to-day enormous 
advantages in connection with that return 
to the Bible. Modern scholarship has made of 
the Bible a new Book. It has in a certain sense 
rediscovered it. You might say that the soul 
of the Reformation was the rediscovery of the 
Bible; and in a wider sense that is true to-day 
also. We have, through the labours of more 
than a century of the finest scholarship in all 
\marginpar{34}
the world, come to understand the Bible, in 
its original sense, as it was never understood 
before. These instructed scribes draw forth 
from their treasury things as new as old. It 
is the old Book, and it is a new Book. It 
remains the old Book, and the precious Book, 
because of its power of unceasing self-renovation. 
The spirit that lives within the Bible is 
a spirit of constant self-preservation. One way 
of describing the Reformation is to say that, 
since the early Gnostic centuries, it was the 
greatest effort that ever took place in the 
Church for the self-preservation of Christianity. 
Remember, the Church was not reformed from 
the outside, but from the inside. It was the 
Church reforming the Church. It was the 
Church's faith that arose, under the Holy Spirit, 
and reformed the Church. So it is with the 
Bible. Whatever renovation we find in connection 
with the Bible---I do not here mean renovation 
of ourselves, but renovation of our way of 
understanding the Book---arises out of the Bible 
itself. This remains true to-day, as it was true 
in the Reformation time, although it is now 
true in a somewhat different application. The 
Bible is still the best commentary upon itself. 
I have always done much in my ministry in 
the way of expounding the Bible, and I would 
\marginpar{35}
say to the younger ministers particularly who are 
here, Do not be afraid of that manner of preaching. 
I have known young ministers who were 
over-scrupulous. I have known them say, "If I 
take a long text people will think it is because 
I am lazy and do not want the labour of getting 
a sermon out of a small one." Never mind such 
foolish people. Do not be afraid of long texts, 
long passages. Preach less from verses and 
more from paragraphs. If I had my time over 
again I would do a great deal more in that way 
than I have done. Read but one lesson, and read 
it with elucidatory comments. Of course some 
people can do that better than others. There 
is always the danger that if a person try it who 
has no sort of knack in that direction, the people 
will feel they have been let in for two sermons 
instead of one; and, excellent as these might 
be, people do not like to feel they have been got 
to church upon false pretences. It might even 
give an excuse to certain people for omitting 
one of the services altogether, on the plea they 
had put in the requisite amount of attention at 
one service. I would also admit that if you do 
this it will not reduce your labour. It will really 
add what might amount to another sermon to 
your weekly work. It is no use doing it if you 
do it on the spur of the moment. If you just say 
\marginpar{36}
things that occur to your mind while you are 
reading, you may say some banal, or some nonsensical 
and fantastic things. It means careful 
preparation. The lesson should be prepared as 
truly as the prayer should be prepared, and as 
the sermon should be prepared. You have to 
work your way through the chapter with the 
aid of the best commentary that you can get; 
and you have to exercise continual judgment in 
doing so lest you be dragged away into little 
matters of detail instead of keeping to the 
larger lines of thought in the passage in hand. 
Then, if you do as I say, there is this other 
advantage, that you can take a particular verse 
out of the long passage for your sermon; and 
thus you come to the sermon with an audience 
which you yourself have prepared to listen to 
you. You have created your own atmosphere, 
and you have done it on a Bible basis. 

Now I will confess against myself that sometimes, 
as I preach about here and there, and 
have done as I have been recommending you to 
do, people have come to me afterwards and said, 
as nicely as they could, that the sermon was all 
very well, but in respect of the reading of the 
Scripture, they never heard it after that fashion; 
they had never realised how vivid Scripture 
could become. That simply results from paying 
\marginpar{37}
attention to the chapter with the best help. 
You will find, I am sure, that your congregation 
will welcome it.

\begin{center}
\S
\end{center} 

Supposing, then, we return to the Bible. 
Supposing that the Church did---as I think it 
must do if it is not going to collapse; certainly 
the Free Churches must---supposing we return 
to the Bible, there are three ways of reading the 
Bible. The first way asks, What did the Bible 
say? The second way asks, What can I make 
the Bible say? The third way asks, What does 
God say in the Bible? 

\begin{center}
\S
\end{center} 

The first way is, with the aid of these magnificent 
scholars, to discover the true historic 
sense of the Bible. There is no more signal 
illustration of success here than in the case of 
the Prophets. During the time when theology 
dominated everything and was considered to 
be the Church's one grand concern, about one 
hundred years after the Reformation, when 
its great prophets had passed away, and the 
Church had fallen into different hands, the 
whole of the Old Testament---the Prophets 
amongst the rest---was read for proof passages 
of theological doctrines. Now for books like 
\marginpar{38}
the Prophets that is absolutely fatal---fatal to 
the books and to the Church; and fatal in the 
long run to Christian truth. There is no greater 
service that has been done to the Bible than 
what has been done by the scholars I speak 
of, in making the Prophets live again, putting 
them in their true historical setting and position. 
Dr. George Adam Smith, for example, has done 
inestimable service in this way. And what 
has been done for the Prophets has also been 
done for the New Testament. Immense steps 
onward have been taken; and we are coming 
to know with much exactness what the writer 
actually had in his mind at the moment of 
writing, and what he was understood to have 
had in his mind by those to whom he first 
wrote. In this way we get rid, for example, 
of the idea that Paul was thinking about us 
who live two thousand years after him. He 
was not thinking of us at all. He did not 
expect the world to last a century. It is quite 
another question what the Holy Spirit was 
thinking about. Paul was thinking in a natural 
way about his age and his Churches, about their 
actual situation and needs. That is another 
illustration of the principle that if you want 
to work for immortality you must work in 
the most relevant and faithful way amid the 
\marginpar{39}
circumstances round about you. The present 
duty is the path to immortality. And so also 
I might illustrate in respect to the Gospels. 

\begin{center}
\S
\end{center}

The second way of reading the Bible is reading 
it unto edification. That is to say, we read 
a passage, and we allow ourselves to receive 
any suggestion that may come to us from it, 
and we do not stop to ask whether that was in 
the writer's mind, or whether it was in the 
mind of the people to whom he wrote. That is 
immaterial. We allow the Spirit of God to 
suggest to us whatever lessons or ideas He 
thinks fit out of the words that are under our 
eyes. We read the Bible not for correct 
or historic knowledge, but for religious and 
spiritual purposes, for our own private and 
personal needs. That is, of course, a perfectly 
legitimate thing---indeed, it is quite necessary. 
It is the way of reading the Bible which the 
large mass of the Church must always practise. 
But it has its dangers. You need the other 
ways to correct it. All the three must cooperate 
for the true use and understanding of 
the Bible by the Church at large. But I am 
speaking now about its use by individuals, 
and the danger I mean is that the suggestiveness
\marginpar{40}
may sometimes become fantastic. Some 
preachers fail at times in that way. They get 
to taking what are called fancy texts, texts 
which impress the audience much more with 
the ingenuity of the preacher than with his 
inspiration. For instance, a preacher in the 
North, now dead, was preaching against the 
Higher Criticism and its slicing up of the 
Bible, and he took his text from Nehemiah, 
"He cut it with a penknife"! That is all very 
well, perhaps, for a motto, but for a text it 
is rather a liberty. It is not fair to the Bible 
to indulge in much of that at least. If I remember 
rightly, Dr. Parker had a great gift in 
this way, and more than sometimes it ran away 
with him. It is a temptation of every witty 
man, and every ingenious-minded man. But 
there is a peril in it, the abuse of a right principle. 
We are bound, of course, to vindicate 
for ourselves and for others the right to use the 
Bible in the suggestive way, if we are not to 
make a present of it to the scholars. And that 
would be just as bad as making a present of it 
to a race of priests. But when we read too 
much in that way it is apt to become a minister 
to our spiritual egotism, or, what is equally bad, 
our fanciful subjectivity. 

Now the grand value of the Bible is just the 
\marginpar{41}
other thing---its objectivity. The first thing is 
not how I feel, but it is, How does God feel, 
and what has God said or done for my soul? 
When we get to real close quarters with that 
our feeling and response will look after itself. 
Do not tell people how they ought to feel 
towards Christ. That is useless. It is just 
what they ought that they cannot do. Preach 
a Christ that will make them feel as they ought. 
That is objective preaching. The tendency and 
fashion of the present moment is all in the 
direction of subjectivity. People welcome 
sermons of a more or less psychological kind, 
which go into the analysis of the soul or of 
society. They will listen gladly to sermons on 
character-building, for instance; and in the 
result they will get to think of nothing else 
but their own character. They will be the 
builders of their own character; which is a 
fatal thing. Learn to commit your soul and 
the building of it to One who can keep it 
and build it as you never can. Attend then to 
Christ, the Holy Spirit, the Kingdom, and the 
Cause, and He will look after your soul. A 
consequence of this passion for subjective and 
psychological analysis, for sentimental experience 
and problem-preaching, is that when 
a preacher begins preaching a real, objective, 
\marginpar{42}
New Testament gospel he has raised against 
him what is now the most fatal accusation---even 
within the Christian Church it has come 
to be very fatal---he is accused of being a 
theologian. That is a very fatal charge to 
make now against any preacher. It ought to 
be actionable in the way of libel. We have 
come to this---that if you penetrate into the 
interior of the New Testament you will be 
accused of being a theologian; and then it is 
all over with your welcome. But that state 
of things has to be turned upside down, else 
the Church dries into the sand. There is no 
message in it. 

\begin{center}
\S
\end{center}


The third way of reading the Bible is reading 
it to discover the purpose and thought of God, 
whether it immediately edify us or whether it 
do not. If we did actually become aware of the 
will and thought of God it would edify us as 
nothing else could. No inner process, no discipline 
to which we might subject ourselves, no 
way of cultivating subjective holiness would do 
so much for us as if we could lose ourselves, and 
in some godly sort forget ourselves, because we 
are so preoccupied with the mind of Christ. If 
you want psychological analysis, analyse the 
will, work, and purpose of Christ our Lord. I 
\marginpar{43}
read a fine sentence the other day which puts in 
a condensed form what I have often preached 
about as the symptom of the present age: 
"Instead of placing themselves at the service 
of God most people want a God who is at their 
service." These two tendencies represent in the 
end two different religions. The man who is 
exploiting God for the purposes of his own soul 
or for the race, has in the long run a different 
religion from the man who is putting his own 
soul and race absolutely at the disposal of the 
will of God in Jesus Christ. 

\begin{center}
\S
\end{center}

All this is by way of preface to an attempt to 
approach the New Testament and endeavour to 
find what is really the will of God concerning 
Christ and what Christ did. Doctrine and life 
are really two sides of one Christianity; and 
they are equally indispensable, because Christianity 
is living truth. It is not merely 
truth; it is not simply life. It is living 
truth. The modern man says that doctrine 
which does not pass into life is dead; 
and then the mistake he makes is that he 
wants to turn it into life directly, and to 
politicise it, perhaps; whereas it works indirectly.
The experience of many centuries, 
\marginpar{44}
on the other hand, says that Christian life 
which does not grow out of Christian doctrine 
becomes a failure. If not in individuals, it 
does in the Church. You cannot keep Christian 
piety alive except upon Christian truth. You 
can never get a Catholic Church except by 
Catholic truth. I think perhaps we all here 
agree about that. It is of immense importance 
that we do not think entirely about our individual 
souls, and that we think more about 
the Church, the divine will, the divine Word, 
and the divine Kingdom in the world. It is 
of supreme importance that we should know 
what the Christian doctrine is on the great 
matters. 

Now in connection with the work of Christ 
the great expositor in the Bible is St. Paul. 
And Paul has a word of his own to describe 
Christ's work---the word "reconciliation." But 
he thinks of reconciliation not as a doctrine but 
as an act of God---because he was not a theologian 
but an experience preacher. To view it 
so produces an immense change in your whole 
way of thinking. It secures for you all that 
is worth having in theology, and it delivers 
you from the danger of obsession by theology 
in a one-sided way. Remember, then, that the 
truth we are dealing with is precious not as a 
\marginpar{45}
mere truth but as the means of expressing the 
eternal act of God. The most important thing 
in all the world, in the Bible or out of it, is 
something that God has done---for ever finally 
done. And it is this reconciliation; which is 
only secondarily a doctrine; it is only secondarily 
even a manner of life. Primarily it is an act of 
God. That is to say, it is a salvation before it is 
a religion. For Christianity as a religion stands 
upon salvation. Religion which does not grow 
out of salvation is not Christian religion; it 
may be spiritual, poetic, mystic; but the essence 
of Christianity is not just to be spiritual; it is 
to answer God's manner of spirituality, which 
you find in Jesus Christ and in Him crucified. 
Reconciliation is salvation before it is religion. 
And it is religion before it is theology. All 
our theology in this matter rests upon the 
certain experience of the fact of God's salvation. 
It is salvation upon divine principles 
It is salvation by a holy God. It is bound 
of course, to be theological in its very nature 
Its statement is a theology. The moment 
you begin to talk about the holiness of God 
you are theologians. And you cannot talk 
about Christ and His death in any thorough 
way without talking about the holiness of 
God. 
\marginpar{46}

\begin{center}
\S 
\end{center} 

Christ and Him crucified, that is the historic 
fact. But what do I mean when I say Christ 
and Him crucified? Does it mean that a certain 
personality lived who was recognised in history 
as Jesus Christ, and that He came by His end 
by crucifixion? That in itself is worthless for 
religious purposes. It is useful enough if you 
are writing history; but for religion historical 
fact must have interpretation, and the whole of 
Christianity depends upon the interpretation 
that is put upon such facts. You will find 
people sometimes who say, "Let us have the 
simple historic facts, the Cross and Christ." 
That is not Christianity. Christianity is a 
certain interpretation of those facts. How and 
why did the New Testament come into being? 
Was it simply to convince posterity that those 
facts had taken place? Was it simply to convince 
the world that Christ had risen from the 
dead? If that were the grand object of the 
New Testament we should have a very different 
Bible in our hands, one addressed to the world 
and not to the Church, to critical science and 
not to faith; and there would not be so much 
argument amongst scholars as there is. The 
Bible did not come into being in order to 
provide future historians with a valuable document.
\marginpar{47}
It came for the purposes of interpretation. 
Here is a sentence I came across once: 
"The fact without the word is dumb; the word 
without the fact is empty." It is useful to turn 
it over and over in your mind. 

Paul was almost the creator and the great 
representative of that interpretation. It was 
continued on his lines by Augustine, Anselm, 
Luther, and many another. But what is it 
that we hear about so much to-day? We 
hear a great deal about an undogmatic Christianity. 
And there is a certain plausibility in 
it. If you have no theological training, no 
training in the understanding of the Scripture 
in a serious way, that is, if you do not know 
your business as ministers of the Word, it seems 
natural that undogmatic Christianity should be 
just the thing you want. Leave the dogma 
of it, you will say, to those who devote their 
lives to dogma---just as though theologians were 
irrepressible people who take up theology as a 
hobby and become the bores of the Church! 
It was not a hobby to the apostles. Why, 
there are actually people of a similar stamp 
who look upon missions as a hobby of the 
Church, instead of their belonging to the 
very being and fidelity of the Church. So 
some people think theology is a hobby, and 
\marginpar{48}
that theologians are persons with an uncomfortable 
preponderance of intellect, who are 
trying to destroy the privileges secured by 
our national lack of education and to sacrifice 
Christianity to mind. People say we do not 
want so much intellect in preaching; we want 
sympathy and unction. Now, I am always looking 
afield, and looking forward, and thinking 
about the prospects of the Church in the great 
world. And unction dissociated from Christian 
truth and Christian intelligence has at last the 
sentence of the Church's death within itself. 
You may cherish an undogmatic Christianity 
with a sort of magnetic casing, a purely human, 
mystical, subjective kind of Christ for yourself 
or an audience, but you could not continue to 
preach that in a Church for the ages. The 
Church could not live on that and do its 
preaching in such a world. You could not 
spread a gospel like that. Subjective religion 
is valuable in its place, but its place is limited. 
The only Cross you can preach to the whole 
world is a theological one. It is not the fact 
of the Cross, it is the interpretation of the 
Cross, the prime theology of the Cross, what 
God meant by the Cross, that is everything. 
That is what the New Testament came to 
give. That is the only kind of Cross that 
can make or keep a Church.
\marginpar{49}

\begin{center}
\S
\end{center} 

You will say, perhaps, "Cannot I go out and 
preach my impressions of the Cross?" By all 
means. You will only discover the sooner that 
you cannot preach a Cross to any purpose if you 
preach it only as an experience. If you only 
preach it so you would not be an apostle; and 
you could not do the work of an apostle for the 
Church. The apostles were particular about 
this, and one expressed it quite pointedly: "We 
preach not ourselves [nor our experiences] but 
Christ crucified." "We do not preach religion," 
said Paul, "but God's revelation. We do not 
preach the impression the Cross made upon 
us, but the message that God by His Spirit sent 
through a Christ we experience." And so with 
ourselves. We do not preach our impressions, 
or even our experience. These make but the 
vehicle, as it were. What we preach is something 
much more solid, more objective, with 
more stay in it; something that can suffice when 
our experience has ebbed until it seems to be as 
low as Christ's was in the great desertion and 
victory on the Cross. We want something 
that will stand by us when we cannot feel any 
more; we want a Cross we can cling to, not 
simply a subjective Cross. That is, to put the 
thing in another way, what we want to-day is 
\marginpar{50}
an insight into the Cross. You see I am making 
a distinction between impression and insight. 
It is a useful part of the Church's work, for 
instance, that it should act by means of revival 
services, where perhaps the dominant element 
may be temporary impression. But unless that 
is taken up and turned to account by something 
more, we all know how evanescent a thing it is 
apt to be. We need, not simply to be impressed 
by Christ, but to see into Christ and into His 
Cross. We need to deepen the impression until 
it become new life by seeing into Christ. There 
are certain circumstances in which we may be 
entitled to declare that we do not want so many 
people who glibly say they love Jesus; we want 
more people who can really see into Christ. 
We do, of course, want more people who love 
Jesus; but we want a multitude of more people 
who are not satisfied with that, but whose love 
fills them with holy curiosity and compels them 
habitually to cultivate in the Spirit the power of 
seeing into Christ and into His Cross. More 
than impression, do we need a spirit of divination. 
Insight is what we want for power---less 
of mere interest and more of real insight. 
There are some people who talk as though, 
when we speak of the Cross and the meaning of 
the Cross, we were spinning something out 
\marginpar{51}
of the Cross. Paul was not spinning anything 
out of the Cross. He was gazing into the Cross, 
seeing what was really there with eyes that 
had been unsealed and purged by the Holy 
Ghost. 

\begin{center}
\S
\end{center}


The doctrine of Christ's reconciliation, or His 
Atonement, is not a piece of medieval dogma 
like transubstantiation, not a piece of ecclesiastical 
dogma or Aristotelian subtlety which 
it might be the Bible's business to destroy. If 
you look at the Gospels you will see that from 
the Transfiguration onward this matter of 
the Cross is the great centre of concern; it 
is where the centre of gravity lies. I met a 
man the other day who had come under some 
poor and mischievous pulpit influence, and he 
said, "It is time we got rid of hearing so much 
about the Cross of Christ; there should be 
preached to the world a humanitarian Christ, 
the kind of Christ that occupies the Gospels." 
There was nothing for it but to tell that man 
he was the victim of smatterers, and that he 
must go back to his Gospels and read and study 
for a year or two. It is the flimsiest religiosity, 
and the most superficial reading of the Gospel, 
that could talk like that. What does it mean 
that an enormous proportion of the Gospel 
\marginpar{52}
story is occupied with the passion of Christ? 
The centre of gravity, even in the Gospels, falls 
upon the Cross of Christ and what was done 
there, and not simply upon a humanitarian 
Christ. You cannot set the Gospels against 
Paul. Why, the first three Gospels were much 
later than Paul's Epistles. They were written 
for Churches that were made by the apostolic 
preaching. But how, then, do the first three 
Gospels \textit{seem} so different from the Epistles? Of 
course, there is a superficial difference. Christ 
was a very living and real character for the 
people of His own time, and His grand business 
was to rouse his audiences' faith in His Person 
and in His mission. But in His Person and in 
His mission the Cross lay latent all the time. 
It emerged only in the fullness of time---that 
valuable phrase---just when the historic crisis, 
the organic situation, produced it. Jesus was 
not a professor of theology. He did not lecture 
the people. He did not come with a theology 
of the Cross. He did not come to force events 
to comply with that theology. He did not 
force His own people to work out a theological 
scheme. He did force an issue, but it 
was not to illustrate a theology. It was to 
establish the Kingdom of God, which could 
be established in no other wise than as He 
\marginpar{53}
established it---upon the Cross. And He could 
only teach the Cross when it had happened---which 
He did through the Evangelists with the 
space they gave it, and through the Apostles 
and the exposition they gave it. 

To come back to this work of Christ described 
by Paul as reconciliation. On this 
interpretation of the work of Christ the whole 
Church rests. If you move faith from that 
centre you have driven \textit{the} nail into the Church's 
coffin. The Church is then doomed to death, 
and it is only a matter of time when she 
shall expire. The Apostle, I say, described the 
work of Christ as above all things reconciliation. 
And Paul was the founder of the Church, 
historically speaking. I do not like to speak 
of Christ as the Founder of the Church. It 
seems remote, detached, journalistic. It would 
be far more true to say that He is the foundation 
of the Church. "The Church's one foundation 
is Jesus Christ her Lord." The founder 
of the Church, historically speaking, was Paul. 
It was founded by and through him on this 
reconciling principle---nay, I go deeper than 
that, on this mighty \textit{act} of God's reconciliation. 
For this great act the interpretation was 
given to Paul by the Holy Spirit. In this connection 
read that great word in 1 Corinthians ii.; 
\marginpar{54}
that is the most valuable word in the New 
Testament about the nature of apostolic inspiration. 

\begin{center}
\S
\end{center}

What, then, did Paul mean by this reconciliation 
which is the backbone of the Church? 
He meant the total result of Christ's life-work 
in permanently changing the relation between 
collective man and God. By reconciliation Paul 
meant the total result of Christ's life-work in 
the fundamental, permanent, final changing of 
the relation between man and God, altering 
it from a relation of hostility to one of confidence 
and peace. Remember, I am speaking 
as Paul spoke, about man, and not about 
individual men or groups of men. 

There are two principal Greek words connected 
with the idea of reconciliation, one of 
them being always translated by it, the other 
sometimes. They are \textit{katallassein}, and \textit{hilaskesthai}---reconciliation 
and atonement. Atonement 
is an Old Testament phrase, where the 
idea is that of the covering of sin from God's 
sight. But by whom? Who was that great 
benefactor of the human race that succeeded in 
covering up our sin from God's sight? Who 
was skilful enough to hoodwink the Almighty? 
Who covered the sin? The all-seeing God 
\marginpar{55}
alone. There can therefore be no talk of hoodwinking. Atonement means the covering of 
sin by something which God Himself had 
provided, and therefore the covering of sin by
God Himself. It was of course not the blinding 
of Himself to it, but something very different. 
How could the Judge of all the earth make 
His judgment blind? It was the covering of 
sin by something which makes it lose the power 
of deranging the covenant relation between 
God and man and founds the new Humanity. 
That is the meaning of it. 

If you think I am talking theology, you must 
blame the New Testament. I am simply expounding 
to you the New Testament. Of course, 
you need not take it unless you please. It is 
quite open to you to throw the New Testament 
overboard (so long as you are frank 
about it), and start what you may loosely call 
Christianity on other floating lines. But if you 
take the New Testament you are bound to try to 
understand the New Testament. If you understand 
the New Testament you are bound to 
recognise that this is what the New Testament 
says. It is a subsequent question whether the 
New Testament is right in saying so. Let us 
first find out what the Bible really says, and then 
discuss whether the Bible is right or wrong. 

\marginpar{56}
The idea of atonement is the covering of sin 
by something which God provided, and by the 
use of which sin looses its accusing power, and 
its power to derange that grand covenant and 
relationship between man and God which founds 
the New Humanity. The word \textit{katallassein} (reconcile) 
is peculiar to Paul. He uses both words; 
but the other word, "atonement," you also find in 
other New Testament writings. Reconciliation 
is Paul's great characteristic word and thought. 
The great passages are those I have mentioned 
at the head of this lecture. I cannot take time 
to expound them here. That would mean a long 
course. Read those passages carefully and 
check me in anything I say---particularly, for 
instance, 2 Corinthians v. 14--vi. 2. Out of it we 
gather this whole result. First, Christ's work 
is something described as reconciliation. And 
second, reconciliation rests upon atonement as 
its ground. Do not stop at "God was in Christ 
reconciling the world." You can easily water 
that down. You may begin the process by 
saying that God was in Christ just in the same 
way in which He was in the old prophets. That 
is the first dilution. Then you go on with the 
hom{\oe}pathic treatment, and you say, "Oh yes, 
all He did by Christ was to affect the world, and 
impress it by showing it how much He loved it." 
\marginpar{57}
Now, would that reconcile anybody really in 
need of it? When your child has flown into a 
violent temper with you, and still worse, a sulky 
temper, and glooms for a whole day, is it any 
use your sending to that child and saying, 
"Really, this cannot go on. Come back. I love 
you very much. Say you are sorry." Not 
a bit of use. For God simply to have told 
or shown the evil world how much He loved 
it would have been a most ineffectual thing. 
Something had to be \textit{done}---judging or saving. 
Revelation alone is inadequate. Reconciliation 
must rest on atonement. For, as I say, 
you must not stop at "God was in Christ 
reconciling the world unto Himself," but go on 
"not reckoning unto them their trespasses." 
"He made Christ to be sin for us, who knew 
no sin." That involves atonement. You cannot 
blot out that phrase. And the third thing 
involved in the idea is that this reconciliation, 
this atonement, means change of relation between 
God and man---man, mind you, not two 
or three men, not several groups of men, 
but man, the human race as one whole. And it 
is a change of relation from alienation to communion---not 
simply to our peace and confidence, 
but to reciprocal communion. The grand end of 
reconciliation is communion. I am pressing 
\marginpar{58}
that hard. I am pressing it hard here by 
saying that it is not enough that we should 
worship God. It is not enough that we should 
worship a personal God. It is not enough that 
we should worship and pay our homage to a 
loving God. That does not satisfy the love of 
God. Nothing short of living, loving, holy, 
habitual communion between His holy soul and 
ours can realise at last the end which God 
achieved in Jesus Christ. 

\begin{center}
\S
\end{center}

In this connection let me offer you two 
cautions. First, take care that the direct fact 
of reconciliation is not hidden up by the indispensable 
means---namely, atonement. There 
have been ages in the Church when the 
attention has been so exclusively centred upon 
atonement that reconciliation was lost sight 
of. You found theologians flying at each 
other's throats in the interest of particular 
theories of atonement. That is to say, atonement 
had obscured reconciliation. In the same 
way, after the Reformation period, they dwelt 
upon justification until they lost sight of 
sanctification altogether. Then the great 
pietistic movement had to arise in order to 
redress the balance. Take care that the end, 
\marginpar{59}
reconciliation, is not hidden up by the means, 
atonement. Justification, sanctification, reconciliation 
and atonement are all equally inseparable 
from the one central and compendious 
work of Christ. Various ages need various 
aspects of it turned outward. Let us give 
them all their true value and perspective. If 
we do not we shall make that fatal severance 
which orthodoxy has so often made between 
doctrine and life. 

The second caution is this. Beware of reading 
atonement out of reconciliation altogether. 
Beware of cultivating a reconciliation which is 
not based upon justification. The apostle's 
phrases are often treated like that. They are 
emptied of the specific Christian meaning. 
There are a great many Christian people, 
spiritual people of a sort, to-day, who are 
perpetrating that injustice upon the New 
Testament. They are taking mighty old words 
and giving them only a subjective, arbitrary 
meaning, emptying out of them the essential, 
objective, positive content. They are preoccupied 
with what takes place within their 
own experience, or imagination, or thought; 
and they are oblivious of that which is 
declared to have taken place within the experience 
of God and of Christ. They are 
\marginpar{60}
oblivious and negligent of the essential things 
that Christ did, and God in Christ. That is 
not fair treatment of New Testament terms---to 
empty them of positive Christian meaning 
and water them down to make something 
that might suit a philosophic or mystic or 
subjective or individualist spirituality. There 
is a whole system of philosophy that has 
attempted this dilution at the present day. It 
is associated with a name that has now become 
very well known, the name of the greatest 
philosopher the world ever saw, Hegel. I am 
not now going to expound Hegelianism. But 
I have to allude to one aspect of it. If you 
are paying any attention to what is going on 
around you in the thinking world, you are 
bound to come face to face with some phase 
of it or other. But I see my time is at an end 
for to-day. 

\begin{center}
\S
\end{center}

To-morrow I begin where I now leave off 
and shall say something about this version of 
St. Paul's idea of reconciliation, which is so 
attractive philosophically. I remember the 
appeal it had for me when I came into contact 
with it first. I did feel that it seemed to give 
a largeness to certain New Testament terms, 
which I finally found was a largeness of latitude 
\marginpar{61}
only. If it did seem to give breadth it 
did not give depth. And I close here by reminding 
you of this---that while Christ and 
Christianity did come to make us broad men, 
it did not come to do that in the first instance. 
It came to make us deep men. The living 
interest of Christ and of the Holy Spirit is not 
breadth, but it is depth. Christ said little 
that was wide compared with what He said 
piercing and searching. I illustrate by referring 
you to an interest that is very prominent 
amongst---you the interest of missions. How 
did modern missions arise? I mean the last 
hundred years of them. Modern Protestant 
missions are only one hundred years old. 
Where did they begin? Who began them? 
They began at the close of the eighteenth 
century, the century whose close was dominated 
by philosophers, by scientists, by a 
reasonable, moderate interpretation of religion, 
by broad humanitarian religion. Of course, 
you might expect it was amongst those broad 
people that missions arose. We know better. 
We know that the Christian movement which 
has spread around the world did not arise out 
of the liberal thinkers, the humanitarian philosophers 
of the day, who were its worst enemies, 
but with a few men Carey, Marshman, Ward,
\marginpar{62} 
and the like---whose Calvinistic theology we 
should now consider very narrow. But they did 
have the root of the universal matter in them. 
A gospel deep enough has all the breadth of the 
world in its heart. If we are only deep enough 
the breadth will take care of itself. I would 
ten times rather have one man who was burning 
deep, even though he wanted to burn me 
for my modern theology, than I would have a 
broad, hospitable, and thin theologian who was 
willing to take me in and a nondescript crowd 
of others in a sheet let down from heaven, 
but who had no depth, no fire, no skill to 
search, and no power to break. For the deep 
Christianity is that which not only searches 
us, but breaks us. And a Christianity which 
would exclude none has no power to include 
the world. 


\chapter{RECONCILIATION: PHILOSOPHIC 
AND CHRISTIAN}



\end{document}